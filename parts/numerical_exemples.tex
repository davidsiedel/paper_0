\section{Numerical examples}
\label{sec_numerical_examples}

In this section, we evaluate the proposed axi-symmetric HHO method on classical test cases taken from the literature to emphasize robustness to volumetric locking.
We consider both the small and large strains framework, and for elasto-plastic behaviors.
The first test case is that of a elasto-perfect plastic swelling sphere. The second one consists in the necking of a notched bar.
In this section, we denote by HHO($k,l$) the HHO element of order $k$ on faces, and order $l$ in the cell.

\paragraph{Stabilization parameter}

To ensure coercivity of the HHO method, the stabilisation parameter needs be chosen according to the material under study. In the literature \cite{di_pietro_discontinuous-skeletal_2015}, a value of order $2 \mu$ is advocated, where $\mu$ denotes the shear modulus of the material.
We use this values for all test cases in the present section.
% where $\mu = \frac{E}{2(1+\nu)}$ with 

% \subsection{Perfect plastic swelling sphere in small strains}

% In this section, we specify the thermodynamical framework in which falls the considered test case, \textit{i.e.} that of elasto-plasticity under the small strain hypothesis.
% Then we describe the chosen behavior law for the numerical example, as well as the mechanical setting of the test case. Results are finally displayed and discussed at the end of the present section.

\subsubsection{Perfect plastic swelling sphere}
\label{sec_swelling_sphere}

\paragraph{Specimen and loading}

This benchmark consists in a quasi-incompressible sphere under uniform internal loading.
This test case has an analytical solution and the state of the specimen is known when the plastic region has reached the external border of the sphere.
The sphere has an inner radius $r_{int} = 0.8$ mm and an outer
radius $r_{ext} = 1$ mm. An internal radial displacement $u$ is imposed. The mesh is composed of XXX quadrangles (see Figure \ref{fig_sphereall}).
The simulation is performed until the limit load corresponding to an internal displacement of $0.2$ mm is reached.

\paragraph{Behaviour law}

An isotropic hardening energy potential $\mecPotential{}_{\bodyLag{}}^p$ is chosen for the description of the plastic evolution of the material such that

\begin{equation}
    \mecPotential{}_{\bodyLag{}}^p(
        % \tensorii{\varepsilon}{}_{\cell}^p
        % ,
        p
    )
    % =
    % \frac{K}{2} \tensorii{\varepsilon}{}_{\cell}^p : \tensorii{\varepsilon}{}_{\cell}^p + \frac{K}{2} p^2
    =
    \sigma_0 p + \frac{1}{2} H p^2 + (\sigma_{\infty} - \sigma_0)(p - \frac{1 - e^{-\delta p}}{\delta})
\end{equation}
%
%
%
where the parameter $p$ denotes the equivalent plastic strain and a Von Mises yields function $f$ describes the flow rule
%
%
%
\begin{equation}
    % f(\tensorii{\sigma}{}_{\cell}^p, q) = \sqrt{\frac{3}{2}} \rVert \text{dev} (\tensorii{\sigma} - K \tensorii{\varepsilon}{}^p) \lVert - \sigma_0 - H p
    f = \sqrt{\frac{3}{2}} \rVert \text{dev} (\tensorii{\sigma}) \lVert - p
\end{equation}
%
%
%
Moreover, the small strain hypothesis is assumed for this test case.

\paragraph{Material parameters}

Perfect plasticity is considered for this test case , where the saturation parameter $\delta = 0$, the yield stresses $\sigma_0 = \sigma_{\infty} = 6$ MPa, the hardening parameter $H = 0$ and the elastic potential parameters are the Young modulus $E = 28.85$ MPa and the Poisson ratio $\nu = 0.499$, such that the material is quasi-incompressible.

\begin{figure}[H]
    \centering
    \includegraphics[width=12.cm]{img_calcs/sphere_mesh.png}
    \caption{the swelling sphere test case. Geometry, loadings, final displacement along the radius of the sphere, and final equivalent plastic strain map at quadrature points}
    \label{fig_sphereall}
\end{figure}

\paragraph{Displacement along the radius}

Since an analytical solution is known for this test case, we compare it to the proposed HHO method. The displacement of the section of the sphere at cell nodes
is plotted in Figure \ref{fig_sphereall}, along with the analytical one, and we observe that the obtained results are in agreement with the analytical response.
Figure \ref{fig_sphereall} mentions the label HHO without specifying approximation orders for all computations deliver the same result.

\paragraph{Trace of the Cauchy stress}

As for the displacement, the analytical solution for the trace of the Cauchy stress tensor is compared to the one computed using the proposed HHO method for three approximation orders.
A sign of volumetric locking is the presence of strong oscillations in the trace of the Cauchy stress (or, equivalently, the hyrostatic pressure) within elements.
We observe that numerical results at quadrature points fit the analytical curve, and display no sign of volumetric locking. The computed solution is however less smooth
at the borders of the specimen for higher orders, a phenomenon that was pointed out in \cite{abbas_hybrid_2019-1} for the three dimensional case, and attributed to the fact that planar faces are considered.

\begin{figure}[H]
    \centering
    \includegraphics[width=15.cm]{img_calcs/sphere_pressures.png}
    \caption{trace of the Cauchy stress tensor along the radius of the sphere at quadrature points}
    \label{fig_sphereall}
\end{figure}

% We compare
% our results to the analytical solution whenever available or to numerical results obtained using the industrial code \texttt{Cast3M}. In this case, we consider a linear, respectively, quadratic, cG formulation, referred to as
% Q1, respectively, Q2, when full integration is used, or, Q2-RI when reduced integration is used, depending on the
% mesh, and a three-field mixed formulation in which the unknowns are the displacement, the pressure, and the volumetric
% strain fields referred to as UPG; in the UPG method, the displacement field is quadratic, whereas both the pressure and
% the volumetric strain fields are linear. The conforming Q1 and Q2 methods with full integration, contrary to the
% Q2-RI method with reduced integration in most of the situations, are known to present volumetric locking due to plastic
% incompressibility, whereas the UPG method is known to be robust but costly. Numerical results obtained using the UPG
% method are used as a reference solution whenever an analytical solution is not available.
% A nonlinear isotropic plasticity model with a von Mises yield criterion is used for the test
% cases. For the first three test cases, strain-hardening plasticity is considered with the following material parameters: Young
% modulus $E = 206.9$ GPa, Poisson ratio $\nu = 0.29$, hardening parameter $H = 129.2$ MPa, initial yield stress $\sigma_0 = 450$ MPa,
% infinite yield stress $\sigma_\infty = 715$ MPa, and saturation parameter $\delta = 16.93$. For the fourth case, perfect plasticity is considered
% with the following material parameters: Young modulus $E = 28.85$ MPa, Poisson ratio $\nu = 0.499$, hardening
% parameter $H = 0$ MPa, initial and infinite yield stresses $\sigma_\infty = 6$ MPa, and saturation parameter $\delta = 0$.
% In the numerical experiments reported in this section, the stabilization parameter is taken to be $1$, and all the quadratures
% use positive weights. In particular, for the HHO method, we employ a quadrature of order $Q = 2k$ for the behavior cell
% integration.

% The goal of this section is to evaluate the proposed HHO method on two and three-dimensional benchmarks from the
% literature: (i) a necking of a 2D rectangular bar subjected to uniaxial extension, (ii) a quasi-incompressible sphere under internal pressure.
% We compare
% our results to the analytical solution whenever available or to numerical results obtained using the industrial open-source
% FEM software code aster. In this case, we consider a linear, respectively, quadratic, cG formulation, referred to as
% Q1, respectively, T2 or Q2, when full integration is used, or, Q2-RI when reduced integration is used, depending on the
% mesh, and a three-field mixed formulation in which the unknowns are the displacement, the pressure, and the volumetric
% strain fields referred to as UPG 6 ; in the UPG method, the displacement field is quadratic, whereas both the pressure and
% the volumetric strain fields are linear. The conforming Q1, T2, and Q2 methods with full integration, contrary to the
% Q2-RI method with reduced integration in most of the situations, are known to present volumetric locking due to plastic
% incompressibility, whereas the UPG method is known to be robust but costly. Numerical results obtained using the UPG
% method are used as a reference solution whenever an analytical solution is not available.
% The nonlinear isotropic plasticity model with a von Mises yield criterion described in Section ABOVE is used for the test
% cases. For the first three test cases, strain-hardening plasticity is considered with the following material parameters: Young
% modulus E = 206.9 GPa, Poisson ratio nu = 0.29, hardening parameter H = 129.2 MPa, initial yield stress sig = 450 MPa,
% infinite yield stress sigunf = 715 MPa, and saturation parameter $\delta$ = 16.93. For the fourth case, perfect plasticity is considered
% with the following material parameters: Young modulus E = 28.85 MPa, Poisson ratio nu = 0.499, hardening
% parameter H = 0 MPa, initial and infinite yield stresses 6 MPa, and saturation parameter 0. Moreover, for the two-dimensional test cases (i) and (ii), we assume additionally a plane strain condition. In the numerical
% experiments reported in this section, the stabilization parameter is taken to be 1, and all the quadratures
% use positive weights. In particular, for the HHO method, we employ a quadrature of order k Q = 2k for the behavior cell
% integration. We employ the notation HHO(k, l) when using face polynomials of order k and cell polynomials of order l.
% In Section 5, we perform further numerical investigations to test other aspects of HHO methods such as the support of
% general meshes with possibly nonconforming interfaces, the possibility of considering the lowest-order case k = 0, and
% the dependence on the stabilization parameter $\beta$.

% \subsection{Necking of a 2D rectangular bar}

% In this first benchmark, we consider a 2D rectangular bar with an initial imperfection. The bar is subjected to uniaxial
% extension. This example has been studied previously by many authors as a necking problem 3,5,7,8,22 and can be used to
% test the robustness of the different methods. The bar has a length of 53.334 mm and a variable width from an initial width
% value of 12.826 mm at the top to a width of 12.595 mm at the center of the bar to create a geometric imperfection. A vertical
% displacement u y = 5mm is imposed at both ends, as shown in Figure 2A. For symmetry reasons, only one-quarter of the
% bar is discretized, and the mesh is composed of 400 quadrangles (see Figure 2B). The load-displacement curve is plotted
% in Figure 2C. We observe that, except for Q1, all the other methods give very similar results. Moreover, the equivalent
% plastic strain p, respectively, the trace of the Cauchy stress tensor sigma, are shown in Figure 3, respectively, in Figure 4, at
% the quadrature points on the final configuration. A sign of locking is the presence of strong oscillations in the trace of
% the Cauchy stress tensor sigma. We notice that the cG formulations Q1 and Q2 lock, contrary to the HHO, Q2-RI, and UPG
% methods that deliver similar results. We remark, however, that the results for HHO(1;1), HHO(1;2), and Q2-RI are slightly
% less smooth than for HHO(2;2), HHO(2;3), and UPG. The reason is that, on a fixed mesh, the three former methods have
% less quadrature points than the three latter ones (see Table 1) (HHO(2;2), HHO(2;3), and UPG have the same number of
% quadrature points). Therefore, the stress is evaluated using less points in HHO(1;1), HHO(1;2), and Q2-RI. It is sufficient
% to refine the mesh or to increase the order of the quadrature by two in HHO(1;1) and HHO(1;2) to retrieve similar results
% to those for the three other methods (not shown for brevity).
% %
% %
% %
% \begin{figure}[H]
%     \centering
%     \includegraphics[width=10.cm]{img_calcs/ssna_concat.png}
%     \caption{schematic representation of the model problem}
%     \label{fig_ssnaall}
% \end{figure}

% \subsection{Quasi-incompressible sphere under internal pressure}

% This last benchmark 6 consists of a quasi-incompressible sphere under internal pressure for which an analytical solution
% is known when the entire sphere has reached a plastic state. This benchmark is particularly challenging compared to
% the previous ones since we consider here perfect plasticity. The sphere has an inner radius R in = 0.8 mm and an outer
% radius R out = 1 mm. An internal radial pressure P is imposed. For symmetry reasons, only one-eighth of the sphere is
% discretized, and the mesh is composed of 1580 tetrahedra (see Figure 10A). The simulation is performed until the limit
% load corresponding to an internal pressure 2.54 MPa is reached.
% The equivalent plastic strain p is plotted for
% HHO(1;2) in Figure 10B, and the trace of the Cauchy stress tensor simga is compared for HHO, UPG, and T2 methods in
% Figure 11 at all the quadrature points on the final configuration for the limit load. We notice that the quadratic element T2
% locks, whereas HHO and UPG do not present any sign of locking and produce results that are very close to the analytical
% solution. However, the trace of the Cauchy stress tensor simga is slightly more dispersed around the analytical solution for
% HHO(2;2) and HHO(2;3) than for HHO(1;1) and HHO(1;2) near the outer boundary. For this test case, we do not expect
% that HHO(2;2) and HHO(2;3) will deliver more accurate solutions than HHO(1;1) and HHO(1;2) since the geometry is
% discretized using tetrahedra with planar faces.
% We next investigate the influence of the quadrature order k Q on the accuracy of the solution. The trace of the Cauchy
% stress tensor simga is compared for HHO(1;1), HHO(2;2), and UPG methods in Figure 12 at all the quadrature points on the
% final configuration for the limit load, and for a quadrature order k Q higher than the one employed in Figure 11 (HHO(1;2)
% and HHO(2;3) give similar results and are not shown for brevity). We remark that, when we increase the quadrature
% order, UPG locks for quasi-incompressible finite deformations, whereas HHO does not lock, and the results are (only) a
% bit more dispersed around the analytical solution. Moreover, HHO(2;2) is less sensitive than HHO(1;1) to the choice of
% the quadrature order k Q . Note that this problem is not present for HHO methods with small deformations. Furthermore,
% this sensitivity to the quadrature order seems to be absent for finite deformations when the elastic deformations are
% compressible (the plastic deformations are still incompressible). To illustrate this claim, we perform the same simulations
% as before but for a compressible material. The Poisson ratio is taken now as nu = 0.3 (recall that we used nu = 0.499
% in the quasi-incompressible case), whereas the other material parameters are unchanged. Unfortunately, an analytical
% solution is no longer available in the compressible case. We compare again the trace of the Cauchy stress tensor simga for
% HHO(1;1), HHO(2;2), and UPG methods in Figure 13 at all the quadrature points on the final configuration and for
% different quadrature orders k Q . We observe a quite marginal dependence on the quadrature order for HHO methods (as in
% the quasi-incompressible case), whereas the UPG method still locks if the order of the quadrature is increased. Moreover,
% in the compressible case, HHO(2;2) gives a more accurate solution than HHO(1;1).

% We employ the notation $HHO(k, l)$ when using face polynomials of order $k$ and cell polynomials of order $l$.



\subsection{Necking of a notched bar}

\paragraph{Specimen and loading}

We consider a notched bar that is subjected to uniaxial
extension.
% This example has been studied previously by many authors as a necking problem 3,5,7,8,22 and can be used to
% test the robustness of the different methods.
The bar has a length of $30$ mm, a top section of radius $5$ mm and a bottom section of radius $3$ mm.
A vertical
displacement $u_z = 0.8$ mm is imposed at the top, as shown in Figure \ref{fig_ssnaallmesh}.
For symmetry reasons, only one-quarter of the
bar is discretized, and the mesh is composed of XXX quadrangles.

\paragraph{Behaviour law}

The same behavior law as that in \ref{sec_swelling_sphere} is considered for the present test case. 
However, the finite strain hypothesis is chosen, based on a logarithmic decomposition of the stress \cite{miehe_anisotropic_2002}.

\paragraph{Material parameters}

Materials parameters are taken as
$\sigma_0 = 450$ MPa, $\sigma_{\infty} = 715$ MPa with a saturation parameter $\delta = 16.93$. The Young modulus is $E = 206.9$ GPa, and the Poisson ratio is $\nu = 0.29$.

\paragraph{Load deflection curve}

The load-displacement curve is plotted
in Figure \ref{fig_ssnaallmesh}, and gives similar results to that obtained with quadratic reduced integration elements.

\paragraph{Equivalent plastic strain}

Moreover, the equivalent
plastic strain $p$ at quadrature points and at the final load is plotted Figure \ref{fig_ssnaallplastic}.
It has been observed that the equivalent plastic strain might suffer some oscillations at a certain limit load with UPG methods.
One notices through the present example, that the proposed HHO method displays no oscillations of the equivalent plastic strain.

%
%
%
\begin{figure}[H]
    \centering
    \includegraphics[width=12.cm]{img_calcs/ssna_plastic.png}
    \caption{
        final equivalent plastic strain map at quadrature points in the notch region
    }
    \label{fig_ssnaallplastic}
\end{figure}

\paragraph{Hydrostatic pressure}

The hyrostatic pressure map at quadrature points and at the final load is shown Figure \ref{fig_ssnaallmesh} for three HHO element orders (respectively $1, 2$ and $3$).
As for the swelling sphere test case, one notices that the hydrostatic pressure map is
fairly smooth over the whole structure at all approximation orders, even at the bottom left corner where plasticity is confined.
% Though some feeble signs of oscillations are noticed at the bottom right corner of the specimen for HHO(1,1),
%
%
%
\begin{figure}[H]
    \centering
    \includegraphics[width=12.cm]{img_calcs/ssna_mesh.png}
    \caption{
        the notched specimen test case. Geometry, loadings, load deflection curve, and final hydrostatic pressure map at quadrature points in the notch region
    }
    \label{fig_ssnaallmesh}
\end{figure}
%
%
%



% Moreover, the equivalent
% plastic strain $p$, respectively, the trace of the Cauchy stress tensor sigma, are shown in Figure 3, respectively, in Figure 4, at
% the quadrature points on the final configuration. A sign of locking is the presence of strong oscillations in the trace of
% the Cauchy stress tensor sigma. We notice that the cG formulations Q1 and Q2 lock, contrary to the HHO, Q2-RI, and UPG
% methods that deliver similar results. We remark, however, that the results for HHO(1;1), HHO(1;2), and Q2-RI are slightly
% less smooth than for HHO(2;2), HHO(2;3), and UPG. The reason is that, on a fixed mesh, the three former methods have
% less quadrature points than the three latter ones (see Table 1) (HHO(2;2), HHO(2;3), and UPG have the same number of
% quadrature points). Therefore, the stress is evaluated using less points in HHO(1;1), HHO(1;2), and Q2-RI. It is sufficient
% to refine the mesh or to increase the order of the quadrature by two in HHO(1;1) and HHO(1;2) to retrieve similar results
% to those for the three other methods (not shown for brevity).

% \subsection{Quasi-incompressible sphere under internal pressure}

% This last benchmark 6 consists of a quasi-incompressible sphere under internal pressure for which an analytical solution
% is known when the entire sphere has reached a plastic state. This benchmark is particularly challenging compared to
% the previous ones since we consider here perfect plasticity. The sphere has an inner radius R in = 0.8 mm and an outer
% radius R out = 1 mm. An internal radial pressure P is imposed. For symmetry reasons, only one-eighth of the sphere is
% discretized, and the mesh is composed of 1580 tetrahedra (see Figure 10A). The simulation is performed until the limit
% load corresponding to an internal pressure 2.54 MPa is reached.
% The equivalent plastic strain p is plotted for
% HHO(1;2) in Figure 10B, and the trace of the Cauchy stress tensor sigma is compared for HHO, UPG, and T2 methods in
% Figure 11 at all the quadrature points on the final configuration for the limit load. We notice that the quadratic element T2
% locks, whereas HHO and UPG do not present any sign of locking and produce results that are very close to the analytical
% solution. However, the trace of the Cauchy stress tensor sigma is slightly more dispersed around the analytical solution for
% HHO(2;2) and HHO(2;3) than for HHO(1;1) and HHO(1;2) near the outer boundary. For this test case, we do not expect
% that HHO(2;2) and HHO(2;3) will deliver more accurate solutions than HHO(1;1) and HHO(1;2) since the geometry is
% discretized using tetrahedra with planar faces.
% We next investigate the influence of the quadrature order k Q on the accuracy of the solution. The trace of the Cauchy
% stress tensor sigma is compared for HHO(1;1), HHO(2;2), and UPG methods in Figure 12 at all the quadrature points on the
% final configuration for the limit load, and for a quadrature order k Q higher than the one employed in Figure 11 (HHO(1;2)
% and HHO(2;3) give similar results and are not shown for brevity). We remark that, when we increase the quadrature
% order, UPG locks for quasi-incompressible finite deformations, whereas HHO does not lock, and the results are (only) a
% bit more dispersed around the analytical solution. Moreover, HHO(2;2) is less sensitive than HHO(1;1) to the choice of
% the quadrature order k Q . Note that this problem is not present for HHO methods with small deformations. Furthermore,
% this sensitivity to the quadrature order seems to be absent for finite deformations when the elastic deformations are
% compressible (the plastic deformations are still incompressible). To illustrate this claim, we perform the same simulations
% as before but for a compressible material. The Poisson ratio is taken now as nu = 0.3 (recall that we used nu = 0.499
% in the quasi-incompressible case), whereas the other material parameters are unchanged. Unfortunately, an analytical
% solution is no longer available in the compressible case. We compare again the trace of the Cauchy stress tensor simga for
% HHO(1;1), HHO(2;2), and UPG methods in Figure 13 at all the quadrature points on the final configuration and for
% different quadrature orders k Q . We observe a quite marginal dependence on the quadrature order for HHO methods (as in
% the quasi-incompressible case), whereas the UPG method still locks if the order of the quadrature is increased. Moreover,
% in the compressible case, HHO(2;2) gives a more accurate solution than HHO(1;1).