The goal of this section is to evaluate the proposed HHO method on benchmarks from the
literature: (i) a necking of a 2D notched bar subjected to uniaxial extension, (ii) a quasi-incompressible sphere under internal pressure.
We compare
our results to the analytical solution whenever available or to numerical results obtained using the industrial code \texttt{Cast3M}. In this case, we consider a linear, respectively, quadratic, cG formulation, referred to as
Q1, respectively, Q2, when full integration is used, or, Q2-RI when reduced integration is used, depending on the
mesh, and a three-field mixed formulation in which the unknowns are the displacement, the pressure, and the volumetric
strain fields referred to as UPG; in the UPG method, the displacement field is quadratic, whereas both the pressure and
the volumetric strain fields are linear. The conforming Q1 and Q2 methods with full integration, contrary to the
Q2-RI method with reduced integration in most of the situations, are known to present volumetric locking due to plastic
incompressibility, whereas the UPG method is known to be robust but costly. Numerical results obtained using the UPG
method are used as a reference solution whenever an analytical solution is not available.
A nonlinear isotropic plasticity model with a von Mises yield criterion is used for the test
cases. For the first three test cases, strain-hardening plasticity is considered with the following material parameters: Young
modulus $E = 206.9$ GPa, Poisson ratio $\nu = 0.29$, hardening parameter $H = 129.2$ MPa, initial yield stress $\sigma_0 = 450$ MPa,
infinite yield stress $\sigma_\infty = 715$ MPa, and saturation parameter $\delta = 16.93$. For the fourth case, perfect plasticity is considered
with the following material parameters: Young modulus $E = 28.85$ MPa, Poisson ratio $\nu = 0.499$, hardening
parameter $H = 0$ MPa, initial and infinite yield stresses $\sigma_\infty = 6$ MPa, and saturation parameter $\delta = 0$.
In the numerical experiments reported in this section, the stabilization parameter is taken to be $1$, and all the quadratures
use positive weights. In particular, for the HHO method, we employ a quadrature of order $Q = 2k$ for the behavior cell
integration.

% The goal of this section is to evaluate the proposed HHO method on two and three-dimensional benchmarks from the
% literature: (i) a necking of a 2D rectangular bar subjected to uniaxial extension, (ii) a quasi-incompressible sphere under internal pressure.
% We compare
% our results to the analytical solution whenever available or to numerical results obtained using the industrial open-source
% FEM software code aster. In this case, we consider a linear, respectively, quadratic, cG formulation, referred to as
% Q1, respectively, T2 or Q2, when full integration is used, or, Q2-RI when reduced integration is used, depending on the
% mesh, and a three-field mixed formulation in which the unknowns are the displacement, the pressure, and the volumetric
% strain fields referred to as UPG 6 ; in the UPG method, the displacement field is quadratic, whereas both the pressure and
% the volumetric strain fields are linear. The conforming Q1, T2, and Q2 methods with full integration, contrary to the
% Q2-RI method with reduced integration in most of the situations, are known to present volumetric locking due to plastic
% incompressibility, whereas the UPG method is known to be robust but costly. Numerical results obtained using the UPG
% method are used as a reference solution whenever an analytical solution is not available.
% The nonlinear isotropic plasticity model with a von Mises yield criterion described in Section ABOVE is used for the test
% cases. For the first three test cases, strain-hardening plasticity is considered with the following material parameters: Young
% modulus E = 206.9 GPa, Poisson ratio nu = 0.29, hardening parameter H = 129.2 MPa, initial yield stress sig = 450 MPa,
% infinite yield stress sigunf = 715 MPa, and saturation parameter $\delta$ = 16.93. For the fourth case, perfect plasticity is considered
% with the following material parameters: Young modulus E = 28.85 MPa, Poisson ratio nu = 0.499, hardening
% parameter H = 0 MPa, initial and infinite yield stresses 6 MPa, and saturation parameter 0. Moreover, for the two-dimensional test cases (i) and (ii), we assume additionally a plane strain condition. In the numerical
% experiments reported in this section, the stabilization parameter is taken to be 1, and all the quadratures
% use positive weights. In particular, for the HHO method, we employ a quadrature of order k Q = 2k for the behavior cell
% integration. We employ the notation HHO(k, l) when using face polynomials of order k and cell polynomials of order l.
% In Section 5, we perform further numerical investigations to test other aspects of HHO methods such as the support of
% general meshes with possibly nonconforming interfaces, the possibility of considering the lowest-order case k = 0, and
% the dependence on the stabilization parameter $\beta$.

% \subsection{Necking of a 2D rectangular bar}

% In this first benchmark, we consider a 2D rectangular bar with an initial imperfection. The bar is subjected to uniaxial
% extension. This example has been studied previously by many authors as a necking problem 3,5,7,8,22 and can be used to
% test the robustness of the different methods. The bar has a length of 53.334 mm and a variable width from an initial width
% value of 12.826 mm at the top to a width of 12.595 mm at the center of the bar to create a geometric imperfection. A vertical
% displacement u y = 5mm is imposed at both ends, as shown in Figure 2A. For symmetry reasons, only one-quarter of the
% bar is discretized, and the mesh is composed of 400 quadrangles (see Figure 2B). The load-displacement curve is plotted
% in Figure 2C. We observe that, except for Q1, all the other methods give very similar results. Moreover, the equivalent
% plastic strain p, respectively, the trace of the Cauchy stress tensor sigma, are shown in Figure 3, respectively, in Figure 4, at
% the quadrature points on the final configuration. A sign of locking is the presence of strong oscillations in the trace of
% the Cauchy stress tensor sigma. We notice that the cG formulations Q1 and Q2 lock, contrary to the HHO, Q2-RI, and UPG
% methods that deliver similar results. We remark, however, that the results for HHO(1;1), HHO(1;2), and Q2-RI are slightly
% less smooth than for HHO(2;2), HHO(2;3), and UPG. The reason is that, on a fixed mesh, the three former methods have
% less quadrature points than the three latter ones (see Table 1) (HHO(2;2), HHO(2;3), and UPG have the same number of
% quadrature points). Therefore, the stress is evaluated using less points in HHO(1;1), HHO(1;2), and Q2-RI. It is sufficient
% to refine the mesh or to increase the order of the quadrature by two in HHO(1;1) and HHO(1;2) to retrieve similar results
% to those for the three other methods (not shown for brevity).
% %
% %
% %
% \begin{figure}[H]
%     \centering
%     \includegraphics[width=10.cm]{img_calcs/ssna_concat.png}
%     \caption{schematic representation of the model problem}
%     \label{fig_ssnaall}
% \end{figure}

% \subsection{Quasi-incompressible sphere under internal pressure}

% This last benchmark 6 consists of a quasi-incompressible sphere under internal pressure for which an analytical solution
% is known when the entire sphere has reached a plastic state. This benchmark is particularly challenging compared to
% the previous ones since we consider here perfect plasticity. The sphere has an inner radius R in = 0.8 mm and an outer
% radius R out = 1 mm. An internal radial pressure P is imposed. For symmetry reasons, only one-eighth of the sphere is
% discretized, and the mesh is composed of 1580 tetrahedra (see Figure 10A). The simulation is performed until the limit
% load corresponding to an internal pressure 2.54 MPa is reached.
% The equivalent plastic strain p is plotted for
% HHO(1;2) in Figure 10B, and the trace of the Cauchy stress tensor simga is compared for HHO, UPG, and T2 methods in
% Figure 11 at all the quadrature points on the final configuration for the limit load. We notice that the quadratic element T2
% locks, whereas HHO and UPG do not present any sign of locking and produce results that are very close to the analytical
% solution. However, the trace of the Cauchy stress tensor simga is slightly more dispersed around the analytical solution for
% HHO(2;2) and HHO(2;3) than for HHO(1;1) and HHO(1;2) near the outer boundary. For this test case, we do not expect
% that HHO(2;2) and HHO(2;3) will deliver more accurate solutions than HHO(1;1) and HHO(1;2) since the geometry is
% discretized using tetrahedra with planar faces.
% We next investigate the influence of the quadrature order k Q on the accuracy of the solution. The trace of the Cauchy
% stress tensor simga is compared for HHO(1;1), HHO(2;2), and UPG methods in Figure 12 at all the quadrature points on the
% final configuration for the limit load, and for a quadrature order k Q higher than the one employed in Figure 11 (HHO(1;2)
% and HHO(2;3) give similar results and are not shown for brevity). We remark that, when we increase the quadrature
% order, UPG locks for quasi-incompressible finite deformations, whereas HHO does not lock, and the results are (only) a
% bit more dispersed around the analytical solution. Moreover, HHO(2;2) is less sensitive than HHO(1;1) to the choice of
% the quadrature order k Q . Note that this problem is not present for HHO methods with small deformations. Furthermore,
% this sensitivity to the quadrature order seems to be absent for finite deformations when the elastic deformations are
% compressible (the plastic deformations are still incompressible). To illustrate this claim, we perform the same simulations
% as before but for a compressible material. The Poisson ratio is taken now as nu = 0.3 (recall that we used nu = 0.499
% in the quasi-incompressible case), whereas the other material parameters are unchanged. Unfortunately, an analytical
% solution is no longer available in the compressible case. We compare again the trace of the Cauchy stress tensor simga for
% HHO(1;1), HHO(2;2), and UPG methods in Figure 13 at all the quadrature points on the final configuration and for
% different quadrature orders k Q . We observe a quite marginal dependence on the quadrature order for HHO methods (as in
% the quasi-incompressible case), whereas the UPG method still locks if the order of the quadrature is increased. Moreover,
% in the compressible case, HHO(2;2) gives a more accurate solution than HHO(1;1).

% We employ the notation $HHO(k, l)$ when using face polynomials of order $k$ and cell polynomials of order $l$.

\subsection{Plasticity small defs}

Dans le cadre de la thermodynamique des milieux continus, la combinsaison de l'application des deux premiers principes de la themodynamique donne lieu à l'équation de Clausius-Duhem qui postule la positivité de l'énergie de dissipation
%
%
%
\begin{equation}
    \label{eq_clausius_duhem_0}
    \begin{aligned}
        \mathcal{D}
        =
        (\tensorii{\sigma}{}_{\cell} - \frac{\partial \mecPotential{}_{\bodyLag{}}}{\partial \dot{\tensorii{\varepsilon}}{}_{\cell}}) : \dot{\tensorii{\varepsilon}}{}_{\cell}
        -
        \rho \frac{\partial \mecPotential{}_{\bodyLag{}}}{\partial v_{int}} \dot{v}_{int}
        \geq
        0
        % =
        % (\tensorii{\sigma}{}_{\cell} - \rho \frac{\partial \mecPotential{}_{\bodyLag{}}}{\partial \tensorii{\varepsilon}{}_{\cell}}) : \dot{\tensorii{\varepsilon}}{}_{\cell}
        % % -
        % % \rho (s + \frac{\partial \mecPotential{}_{\bodyLag{}}}{\partial T}) \dot{T}
        % -
        % \rho \frac{\partial \mecPotential{}_{\bodyLag{}}}{\partial v_{int}} \dot{v}_{int}
        % \geq
        % 0
    \end{aligned}
\end{equation}
%
%
%
en l'absence de dépendence du problème à la température. Dans le cadre de l'hyper-élasticité qui est un processus de transformation réversible, comme évoque Section \ref{sec_model_problem}, l'ensemble $V_{int}$ des variables internes $v_{int}$ est supposé vide, de sorte que l'inégalité \eqref{eq_clausius_duhem_0} revient à l'équation d'égalité \eqref{eq_model_problem_0:eq2}.
En revanche, pour des comportements dissipatifs de nature élasto-visco-plastique, on introduit un certains nombre de variables internes, qui sont liées à l'expression de l'énergie dissipée et à l'irréversibilité de la transformation.
Pour des déformations infinitésimales, on suppose la décomposition additive de la déformation
%
%
%
\begin{equation}
    \tensorii{\varepsilon}{}_{\cell} = \tensorii{\varepsilon}{}_{\cell}^e + \tensorii{\varepsilon}{}_{\cell}^p
\end{equation}
%
%
%
En une partie élastique $\tensorii{\varepsilon}{}_{\cell}^e$ et une partie plastique $\tensorii{\varepsilon}{}_{\cell}^p$.
En particulier, dans le cadre des matériaux standards généralisés, on suppose l'existence d'un potentiel également décomposable en une partie élastique et en une partie plastique tel que
%
%
%
\begin{equation}
    \label{eq_plast_2}
    \begin{aligned}
        \mecPotential{}_{\bodyLag{}} = \mecPotential{}_{\bodyLag{}}^e(\tensorii{\varepsilon}{}_{\cell}^e)
        +
        \mecPotential{}_{\bodyLag{}}^p(
            % \tensorii{\varepsilon}{}_{\cell}^p
            % ,
            v_{int})
    \end{aligned}
\end{equation}
%
%
%
Comme évoque Section \ref{sec_model_problem}, le potentiel d'énergie libre de Helmoltz $\mecPotential{}_{\bodyLag{}}$ dépend éventuellement d'un ensemble de variables internes $v_{int}$ dans $V_{int}$, qui a été supposé vide jusque là.
Dans le cadre d'un comportement élasto-visco-plastique, on introduit au moins une variable interne, de manière à assurer la positivité de l'énergie dissipée. Par injection de \eqref{eq_plast_2} dans \eqref{eq_clausius_duhem_0}, il vient que le tenseur des contraintes $\tensorii{\sigma}{}_{\cell}$ est la la force duale assosciées aux défromations élastiques $\tensorii{\varepsilon}{}_{\cell}^e$. On définit également les forces thermodynamiques $V_{\cell}$ duales des variables internes $v_{int}$ telles que
%
%
%
\begin{equation}
    \label{eq_plast_1}
    \begin{aligned}
        \mathcal{D}
        =
        \tensorii{\sigma}{}_{\cell} : \dot{\tensorii{\varepsilon}}{}_{\cell}^p
        -
        \rho \frac{\partial \mecPotential{}_{\bodyLag{}}}{\partial v_{int}} \dot{v}_{int}
        =
        \begin{Bmatrix}
            \tensorii{\sigma}{}_{\cell}
            \\
            % \tensori{V}{}_{\cell}
            V_{\cell}
        \end{Bmatrix}
        \cdot
        \begin{Bmatrix}
            \dot{\tensorii{\varepsilon}}{}_{\cell}^p
            \\
            \dot{v}_{int}
        \end{Bmatrix}
        \geq
        0
        &&
        \text{with}
        &&
        V_{\cell} = - \rho \frac{\partial \mecPotential{}_{\bodyLag{}}}{\partial v_{int}}
    \end{aligned}
\end{equation}
%
%
%
Par ailleurs, le cadre des matériaux standards généralisés stipule l'existence d'un convex potential $\phi$ containing the origin, together with a threshold function $f$, that define the evolution of the generalized strains such that
%
%
%
\begin{equation}
    \label{eq_plast_1}
    \begin{aligned}
        \dot{v_{int}} = \frac{\partial \phi}{\partial f} \frac{\partial f}{\partial V_T}
    \end{aligned}
\end{equation}
%
%
%
Le potentiel $\phi$ dépend de la fonction de charge $f$ telle que celle-ci contient l'origine, est différentiable en tout point de ...

En particulier, on introduit l'ensemble des variables internes $V_{int} = \{ \tensorii{\varepsilon}{}_{\cell}^p, p \}$ avec $p$ la déformation plastique cumulée, et le potentiel plastique 
%
%
%
\begin{equation}
    \mecPotential{}_{\bodyLag{}}^p(
        \tensorii{\varepsilon}{}_{\cell}^p
        ,
        p
    )
    =
    \frac{K}{2} \tensorii{\varepsilon}{}_{\cell}^p : \tensorii{\varepsilon}{}_{\cell}^p + \frac{K}{2} p^2
\end{equation}
%
%
%
où $K$ est le module d'écrouissage cinématique, et $H$ le module d'écrouissage isotrope. Les forces thermodynamiques assosciées aux variables internes $\tensorii{\varepsilon}{}_{\cell}^p$ et $p$ sont respectivement $K \tensorii{\varepsilon}{}_{\cell}^p$ et $Hp$.
%
%
%
\begin{equation}
    f(\tensorii{\sigma}{}_{\cell}^p, q) = \sqrt{\frac{3}{2}} 
\end{equation}


%
%
%
We introduce the discrete logarithmic stress tensor $\tensorii{E}{} = 1/2 \text{ln}(\tensorii{F}{}_{\cell}^{\text{t}} \cdot \tensorii{F}{}_{\cell})$

\subsection{Necking of a notched bar}

In this first benchmark, we consider a 2D rectangular bar with an initial imperfection. The bar is subjected to uniaxial
extension. This example has been studied previously by many authors as a necking problem 3,5,7,8,22 and can be used to
test the robustness of the different methods. The bar has a length of 53.334 mm and a variable width from an initial width
value of 12.826 mm at the top to a width of 12.595 mm at the center of the bar to create a geometric imperfection. A vertical
displacement u y = 5mm is imposed at both ends, as shown in Figure 2A. For symmetry reasons, only one-quarter of the
bar is discretized, and the mesh is composed of 400 quadrangles (see Figure 2B). The load-displacement curve is plotted
in Figure 2C. We observe that, except for Q1, all the other methods give very similar results. Moreover, the equivalent
plastic strain p, respectively, the trace of the Cauchy stress tensor sigma, are shown in Figure 3, respectively, in Figure 4, at
the quadrature points on the final configuration. A sign of locking is the presence of strong oscillations in the trace of
the Cauchy stress tensor sigma. We notice that the cG formulations Q1 and Q2 lock, contrary to the HHO, Q2-RI, and UPG
methods that deliver similar results. We remark, however, that the results for HHO(1;1), HHO(1;2), and Q2-RI are slightly
less smooth than for HHO(2;2), HHO(2;3), and UPG. The reason is that, on a fixed mesh, the three former methods have
less quadrature points than the three latter ones (see Table 1) (HHO(2;2), HHO(2;3), and UPG have the same number of
quadrature points). Therefore, the stress is evaluated using less points in HHO(1;1), HHO(1;2), and Q2-RI. It is sufficient
to refine the mesh or to increase the order of the quadrature by two in HHO(1;1) and HHO(1;2) to retrieve similar results
to those for the three other methods (not shown for brevity).
%
%
%
\begin{figure}[H]
    \centering
    \includegraphics[width=12.cm]{img_calcs/ssna_mesh.png}
    \caption{ssna}
    \label{fig_ssnaallmesh}
\end{figure}
%
%
%
\begin{figure}[H]
    \centering
    \includegraphics[width=12.cm]{img_calcs/ssna_plastic.png}
    \caption{ssna}
    \label{fig_ssnaallplastic}
\end{figure}

\subsection{Quasi-incompressible sphere under internal pressure}
%
%
%
\begin{figure}[H]
    \centering
    \includegraphics[width=12.cm]{img_calcs/sphere_mesh.png}
    \caption{sphere}
    \label{fig_sphereall}
\end{figure}
%
%
%
\begin{figure}[H]
    \centering
    \includegraphics[width=15.cm]{img_calcs/sphere_pressures.png}
    \caption{sphere}
    \label{fig_sphereall}
\end{figure}

This last benchmark 6 consists of a quasi-incompressible sphere under internal pressure for which an analytical solution
is known when the entire sphere has reached a plastic state. This benchmark is particularly challenging compared to
the previous ones since we consider here perfect plasticity. The sphere has an inner radius R in = 0.8 mm and an outer
radius R out = 1 mm. An internal radial pressure P is imposed. For symmetry reasons, only one-eighth of the sphere is
discretized, and the mesh is composed of 1580 tetrahedra (see Figure 10A). The simulation is performed until the limit
load corresponding to an internal pressure 2.54 MPa is reached.
The equivalent plastic strain p is plotted for
HHO(1;2) in Figure 10B, and the trace of the Cauchy stress tensor sigma is compared for HHO, UPG, and T2 methods in
Figure 11 at all the quadrature points on the final configuration for the limit load. We notice that the quadratic element T2
locks, whereas HHO and UPG do not present any sign of locking and produce results that are very close to the analytical
solution. However, the trace of the Cauchy stress tensor sigma is slightly more dispersed around the analytical solution for
HHO(2;2) and HHO(2;3) than for HHO(1;1) and HHO(1;2) near the outer boundary. For this test case, we do not expect
that HHO(2;2) and HHO(2;3) will deliver more accurate solutions than HHO(1;1) and HHO(1;2) since the geometry is
discretized using tetrahedra with planar faces.
We next investigate the influence of the quadrature order k Q on the accuracy of the solution. The trace of the Cauchy
stress tensor sigma is compared for HHO(1;1), HHO(2;2), and UPG methods in Figure 12 at all the quadrature points on the
final configuration for the limit load, and for a quadrature order k Q higher than the one employed in Figure 11 (HHO(1;2)
and HHO(2;3) give similar results and are not shown for brevity). We remark that, when we increase the quadrature
order, UPG locks for quasi-incompressible finite deformations, whereas HHO does not lock, and the results are (only) a
bit more dispersed around the analytical solution. Moreover, HHO(2;2) is less sensitive than HHO(1;1) to the choice of
the quadrature order k Q . Note that this problem is not present for HHO methods with small deformations. Furthermore,
this sensitivity to the quadrature order seems to be absent for finite deformations when the elastic deformations are
compressible (the plastic deformations are still incompressible). To illustrate this claim, we perform the same simulations
as before but for a compressible material. The Poisson ratio is taken now as nu = 0.3 (recall that we used nu = 0.499
in the quasi-incompressible case), whereas the other material parameters are unchanged. Unfortunately, an analytical
solution is no longer available in the compressible case. We compare again the trace of the Cauchy stress tensor simga for
HHO(1;1), HHO(2;2), and UPG methods in Figure 13 at all the quadrature points on the final configuration and for
different quadrature orders k Q . We observe a quite marginal dependence on the quadrature order for HHO methods (as in
the quasi-incompressible case), whereas the UPG method still locks if the order of the quadrature is increased. Moreover,
in the compressible case, HHO(2;2) gives a more accurate solution than HHO(1;1).