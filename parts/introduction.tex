\section{Introduction}

% The origin of DG methods dates back to the pioneering work of
% \cite{reed_triangular_1973}, where an hyperbolic formualtion is used to
% solve the neutron transport equation.

The Hybird High Order method (HHO) is a discontinuous discretization
method, that takes root in the Discontinuous Galerkin method (DG). From
the physical standpoint, DG methods ensure the continuity of the flux
across interfaces, by seeking the solution element-wise, hence allowing
jumps of the potential across elements. They can be seen as a
generalization of Finite Volume methods, and are able to capture
physically relevant discontinuities without producing spurious
oscillations.

The origin of DG methods dates back to the pioneering work of
\cite{reed_triangular_1973}, where an hyperbolic formualtion is used to
solve the neutron transport equation. The first application of the
method to elliptic problems originates in \cite{babuska_finite_1973}
where Nitsche's method \cite{nitsche_uber_1970} is used to weakly impose
continuity of the flux across interfaces. \textcolor{blue} { In 2002,
  Hansbo and Larson \cite{hansbo_discontinuous_2002-1} were the first to
  consider the Nitsche's classical DG method for nearly incompressible
  elasticity. They showed, theoretically and numerically, that this
  method is free from volumetric locking. } However, the bilinear form
arising from this formulation is not symmetric. A so called interior
penalty term has been introduced in \cite{wheeler_elliptic_1978},
leading to the Symmetric Interior Penalty (SIP) DG method. A first study
of the method to linear elasticity has been devised by
\cite{riviere_optimal_2000}, where optimal error estimate has been
proved. \textcolor{blue} { \cite{lew_optimal_2004} generalized the
  Symmetric Interior Penalty method to linear elasticity. }
\textcolor{blue} {
  % In 2002, Hansbo and Larson \cite{hansbo_discontinuous_2002-1} were the first to
 % consider the Nitsche's classical DG method for nearly incompressible
  % elasticity. They showed, theoretically and numerically, that this method
 % is free from volumetric locking. % \cite{lew_optimal_2004}
  % generalized the Symmetric
  % Interior Penalty method to linear elasticity. In about the same
  period of time, DG methods were proposed for other linear problems in
  solid mechanics, such as Timoshenko beams
  \cite{celiker_locking-free_2006}, Bernoulli-Euler beam and the
  Poisson-Kirchhoff plate \cite{brenner_balancing_1999,
    engel_continuousdiscontinuous_2002} and Reissner-Mindlin plates
  \cite{arnold_family_2005}. In the mid 2000's, the first applications
  of DG methods to nonlinear elasticity problems was undertaken by
  \cite{ten_eyck_discontinuous_2006, noels_general_2006}, and in 2007,
  Ortner and Süli \cite{ortner_discontinuous_2007} carried out the a
  priori error analysis of DG methods for nonlinear elasticity.
  % This pioneering work
  % shed light on how to calculate a lower bound on the stability parameters.
 }

DG methods then sollicitated a vigourus interest, mostly in fluid dynamics \cite{shahbazi_high-order_2007, persson_discontinuous_2009} due to their local conservative property and stability in convection domniated problems. However, except some applications for instance in fracture mechanics using XFEM methods \cite{gracie_blending_2008, shen_stability_2010}, or gradient plasticity \cite{djoko_discontinuous_2007,djoko_discontinuous_2007-1} DG methods did not break through in computational solid mechanics because of their numerical cost, since nodal unknowns need be duplicated to define local basis functions in each element.

To adress this problem, in the early 2010's, \cite{cockburn_unified_2009, soon_hybridizable_2009} introduced additional faces unknowns on element interfaces for linear elastic problem, hence leading to the hybridization of DG methods, or Hybridizable Discontinuous Galerkin method (HDG). By adding supplementary boundary unknowns, the authors actually allowed to eliminate original cell unknowns by a static condensation process, in order to express the global problem on faces ones only. Extension of HDG methods to non-linear elasticity were first undertaken in \cite{soon_hybridizable_2008} and have then fueled intense reaserch works for various applications such as linear and non-linear convection-diffusion problems \cite{nguyen_implicit_2009,nguyen_implicit_2009-1,nguyen_hybridizable_2010}, incompressible stokes flows \cite{nguyen_hybridizable_2010, nguyen_implicit_2011} and non-linear mechanics \cite{nguyen_hybridizable_2012}.

In \cite{di_pietro_hybrid_2015, di_pietro_arbitrary-order_2014}, the authors introduced a higher order potential reconstruction operator in the classical HDG formulation for elliptic problems, providing a $h^{k+1} H^1$-norm convergence rate as compared to the ususal $h^k$-rate. This higher order term coined the name for the so called HHO method.
Recent developments of HHO methods in
computational mechanics include the incompressible Stokes
equations (with possibly large irrotational forces) \cite{di_pietro_discontinuous_2016}, the
incompressible Navier–Stokes equations \cite{di_pietro_hybrid_2018}, Biot’s consolidation problem \cite{boffi_nonconforming_2016}, and nonlinear elasticity with small
deformations \cite{botti_hybrid_2017}