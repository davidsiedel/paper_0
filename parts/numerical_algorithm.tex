\section{Numerical examples for the algorithm}
\label{sec_numerical_algorithm}

In this section, we evaluate the response of the cell equilibrium algorithm.

\paragraph{Specimen and loading}

We consider a the cook membrane specimen that is subjected to uniaxial
traction.
% This example has been studied previously by many authors as a necking problem 3,5,7,8,22 and can be used to
% test the robustness of the different methods.
The membrane has a width of $48$ mm and a height of $60$ mm.
A vertical traction $t_y = 1000$ N/m is imposed at the top, as shown in Figure \ref{fig_ssnaallmesh}.
The HHO computation is compared with a standard Q1 and Q2 (\textit{i.e.} linear and quadratic approximations)

\paragraph{Behaviour law}

The same behavior law as that in \ref{sec_swelling_sphere} is considered for the present test case. 
However, the finite strain hypothesis is chosen, based on a logarithmic decomposition of the stress \cite{miehe_anisotropic_2002}.

\paragraph{Material parameters}

Materials parameters are taken as
$\sigma_0 = 450$ MPa, $\sigma_{\infty} = 715$ MPa with a saturation parameter $\delta = 16.93$. The Young modulus is $E = 206.9$ GPa, and the Poisson ratio is $\nu = 0.29$.

% \begin{figure}[H]
%     \centering
%     \includegraphics[width=12.cm]{img_calcs/sphere_mesh.png}
%     \caption{the swelling sphere test case. Geometry, loadings, final displacement along the radius of the sphere, and final equivalent plastic strain map at quadrature points}
%     \label{fig_sphereall}
% \end{figure}

% In this section, we evaluate the proposed axi-symmetric HHO method on classical test cases taken from the literature to emphasize robustness to volumetric locking.
% We consider both the small and large strains framework, and for elasto-plastic behaviors.
% The first test case is that of a elasto-perfect plastic swelling sphere. The second one consists in the necking of a notched bar.
% In this section, we denote by HHO($k,l$) the HHO element of order $k$ on faces, and order $l$ in the cell.

\begin{figure}[H]
    \centering
    \includegraphics[width=12.cm]{img_calcs/cook_comp.png}
    \caption{Hydrostatic pressure map one the reference configuration at the limit load}
    \label{fig_sphereall}
\end{figure}

\paragraph{Algorithm performance}

We compare the performance of 

\paragraph{Prediction decondensation step}

Using a decondensation setp for the cell algorithm

\begin{figure}[H]
    \centering
    \includegraphics[width=12.cm]{img_calcs/algo_comp.png}
    \caption{Comparison in terms of performance for different algorithms}
    \label{fig_sphereall}
\end{figure}