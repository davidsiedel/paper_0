Paragraph~\ref{sec:Hu_Washizu_functional}
introduces the classical Hu–Washizu functional to describe the
quasi-static equilibrium of a body submitted to external load and the
main notations used in this paper. For the sake of simplicity, the body
is assumed hyper-elastic in this section.

Paragraph~\ref{sec:Hu_Washizu_functional}
introduces the classical Hu–Washizu functional to describe the
quasi-static equilibrium of a body submitted to external load and the
main notations used in this paper. For the sake of simplicity, the body
is assumed hyper-elastic in this section.

Paragraph~\ref{sec:HHO} introduces the key idea
of the HHO method, which is to divide the domain in arbitrary subdomains
connected by cohesive interfaces and to apply the Hu–Washizu functional
to each sub-domains.

\subsection{The standard Hu–Washizu functional}
\label{sec_Hu_Washizu_functional}

Let $\bodyEul$ a solid body with boundary $\dBodyEul$, that deforms in the current configuration at some time $t > 0$ under a body force $\loadEul$ acting in $\bodyEul$, a prescribed displacement $\dirichletEul$ on the Dirichlet boundary $\dirichletBoundaryEul$, and a contact load $\neumannEul{}$ on the Neumann boundary $\neumannBoundaryEul$.
% 
On donne $\tensori{\Phi}$ le gradient de la transformation qui passe de $\tensori{x} \in \bodyLag$ en configuration de référence à $\tensori{x}{}_t \in \bodyEul$ dans la configuration actuelle.
On note $\tensorii{F} = \nabla \tensori{\Phi}$ le gradient de la transformation et $J = \text{det}(\tensorii{F})$ la variation de volume, et les forces externes $\loadLag = J \loadEul{}$ et $\neumannLag{} = \tensorii{F} \neumannEul{}$.
Le solide $\bodyLag{}$ est décrit par un potentiel hyperélsatique $\psi$
% 
% \textcolor{blue}{formules de Nanson pour les passages des forces externes en config init}
% At the time $t = 0$, the solid is in the reference configuration $\bodyLag$ with respective Dirichlet and Neumann boundaries $\dirichletBoundaryLag$ and $\neumannBoundaryLag$.
% It is subjected to the body force $\loadLag$ in $\bodyLag{}$, the prescribed displacement $\dirichletLag$ on $\dirichletBoundaryLag{}$, and the contact load $\neumannLag{}$ on $\neumannBoundaryLag{}$.
% Let $\tensori{\Phi}$ the mapping that takes a point $\tensori{x} \in \bodyLag$ from the reference configuration to its coordinates $\tensori{x}{}_t \in \bodyEul$ in the current configuration, and $\tensori{u}$ the displacement field in $\bodyLag{}$ such that $\tensori{\Phi} = \tensori{I}{}_d + \tensori{u}$ where $\tensori{I}{}_d$ denotes the identity function.
% %
% %
% %
% % From a thermodynamical standpoint
% The internal energy of the solid $\mecPotential{}$ depends on the gradient of the transformation $\tensorii{F} = \nabla \tensori{\Phi} = \tensorii{1} + \nabla \tensori{u}$ and on a set of state variables $V_{int}$ that relate to the positivity requirement on the dissipation (in the case of an elasto-visco-plastic behaviour for instance). In the absence of dissipative process (\textit{e.g.} for (hyper)-elasticity), there is no internal state variable and $V_{int} = \emptyset$. In the following, in order to simplify developments and without loss of generality, let consider the latter case.
% \textcolor{blue}{for the sake of simplicty and without loss of generality, let consider a hyperelastic body}
% 
%
%
%
The three fields Hu–Washizu principle characterizes the equilibrium of the body by the minimization of the functional $J_{\bodyLag}^{HW}$ with respect to the displacement field $\tensori{u}{} \in U(\bodyLag)$ verifying $\tensori{u}{} \vert_{\dirichletBoundaryLag{}} = \dirichletLag$, the displacement gradient field $\tensorii{G} \in G(\bodyLag)$, and the first Piola-Kirchoff stress tensor field $\tensorii{P} \in S(\bodyLag)$, such that
%
%
%
\begin{equation}
    \label{eq_hu_washizu_0}
        J_{\bodyLag}^{HW}
        % (\tensori{u}{}_{\cell}, \tensorii{G}{}_{\cell}, \tensorii{P}{}_{\cell})
        =
        \int_{\bodyLag{}} \mecPotential{} + (\nabla \tensori{u}{} - \tensorii{G}{}) : \tensorii{P}{}
        -
        \int_{\bodyLag{}} \loadLag \cdot \tensori{u}{}
        -
        \int_{\neumannBoundaryLag{}} \neumannLag{} \cdot \tensori{u}
        \vert_{\neumannBoundaryLag{}}
\end{equation}
%
%
%
where $U(\bodyLag)$ denotes the space of all kinematically admissible displacement fields, and $G(\bodyLag)$ (respectively $P(\bodyLag)$) denotes the space of all statically admissible strain (respectively stress) fields.
%
%
%
Deriving \eqref{eq_hu_washizu_0} with respect to all variables yields the weak formulation of the problem
%
% 
% 
\begin{subequations}
    \label{eq_hu_washizu_derivative_0}
        \begin{alignat}{3}
            \frac{\partial J_{\bodyLag}^{HW}}{\partial \tensori{u}{}} \delta \tensori{u}{}
            = & \int_{\bodyLag} \tensorii{P}{} : \nabla \delta \tensori{u}{}
            -
            \int_{\bodyLag} \tensori{f}{}_V \cdot \delta \tensori{u}{}
            -
            \int_{\neumannBoundaryLag} \neumannLag{} \cdot \delta \tensori{u}{} \vert_{\neumannBoundaryLag}
            &&
            \ \ \ \ \ \ \ \ 
            &&
            \forall \delta \tensori{u}{}
            \in \virtualDisplacementSpaceCell
        \label{eq_hu_washizu_derivative_0:eq0}
        \\
            \frac{\partial J_{\bodyLag}^{HW}}{\partial \tensorii{G}{}} \delta \tensorii{G}{}
            = &
            \int_{\bodyLag} (\frac{\partial \mecPotential}{\partial \tensorii{G}{}} - \tensorii{P}{}) : \delta \tensorii{G}{}
            &&
            \ \ \ \ \ \ \ \ 
            &&
            \forall \delta \tensorii{G}{}
            \in \gradSpaceCell
        \label{eq_hu_washizu_derivative_0:eq2}
        \\
            \frac{\partial J_{\bodyLag}^{HW}}{\partial \tensorii{P}{}} \delta \tensorii{P}{}
            = & \int_{\bodyLag} (\nabla \tensori{u}{} - \tensorii{G}{} ) : \delta \tensorii{P}{}
            &&
            \ \ \ \ \ \ \ \ 
            &&
            \forall \delta \tensorii{P}{}
            \in \stressSpaceCell
        \label{eq_hu_washizu_derivative_0:eq3}
    \end{alignat}
\end{subequations}
%
%
%
where $U_0(\bodyLag{})$ is the space of all kinematically admissible virtual displacement fields. The strong formulation of problem \eqref{eq_hu_washizu_0} yields
%
% 
% 
\begin{subequations}
\label{eq_model_problem_0}
    \begin{alignat}{2}
    % \tensorii{F} - \nabla_{X} \tensori{u} & = \tensorii{1} \quad && \text{in } \Omega_{0} \label{eq_model_problem:eq1}
    \tensorii{G}{} & = \nabla \tensori{u}{} 
    % \quad
    % &&
    % \text{in } \bodyLag{}
    \label{eq_model_problem_0:eq1}
    \\
    % \tensorii{P} - \frac{\partial \mecPotential}{\partial \tensorii{F}} & = 0 \quad && \text{in } \Omega_{0} \label{eq_model_problem:eq2}
    \tensorii{P}{} & = \frac{\partial \mecPotential{}}{\partial \tensorii{G}{}}
    % \quad
    % &&
    % \text{in } \bodyLag{}
    \label{eq_model_problem_0:eq2}
    \\
    \nabla_{X} \cdot \tensorii{P}{} & = - \loadLag
    % \quad
    % &&
    % \text{in } \bodyLag{}
    \label{eq_model_problem_0:eq3}
    \\
    \tensori{u}{} \vert_{\dirichletBoundaryLag{}} & = \dirichletLag
    % \quad
    % &&
    % \text{on } \dirichletBoundaryLag{}
    \label{eq_model_problem_0:eq4}
    \\
    \tensorii{P}{} \vert_{\neumannBoundaryLag{}} \cdot \tensori{n} & = \neumannLag{}
    % \quad
    % &&
    % \text{on } \neumannBoundaryLag{}
    \label{eq_model_problem_0:eq5}
\end{alignat}
\end{subequations}
%
% 
% 
where $\tensori{n}$ denotes the unit outward normal vector on $\neumannBoundaryLag{}$.
One readily identifies \eqref{eq_model_problem_0:eq1} with the displacement strain relation, \eqref{eq_model_problem_0:eq2} with the constitutive equation, \eqref{eq_model_problem_0:eq3} with the conservation of momentum, \eqref{eq_model_problem_0:eq5} with the traction force continuity and \eqref{eq_model_problem_0:eq4} with Dirichlet boundary conditions.

\subsection{The Principle of virtual works}
\label{sec_Virtual_Works_functional}

%
%
%
Considering now that $\tensorii{G}{}$ and $\tensorii{P}{}$ are not unknowns of the problem, \textit{i.e.} by explicitly eliminating both \eqref{eq_hu_washizu_derivative_0:eq2} and \eqref{eq_hu_washizu_derivative_0:eq3} from \eqref{eq_hu_washizu_derivative_0}, \eqref{eq_hu_washizu_0} simplifies as : find the displacement field $\tensori{u}{} \in \displacementSpaceCell$ verifying $\tensori{u}{} \vert_{\dirichletBoundaryLag{}} = \dirichletLag$ that minimizes the energy functional $J_{\bodyLag{}}^{VW}$
%
%
%
\begin{equation}
    \label{eq_virtual_works_0}
    \begin{aligned}
        J_{\bodyLag{}}^{VW}
        % (\tensori{u}{}_{\cell})
        =
        \int_{\bodyLag{}} \mecPotential{}
        -
        \int_{\bodyLag{}} \loadLag \cdot \tensori{u}{}
        -
        \int_{\neumannBoundaryLag{}} \neumannLag{} \cdot \tensori{u}{}
    \end{aligned}
\end{equation}
%
%
%
Deriving \eqref{eq_virtual_works_0} with respect to the primal unknown $\tensori{u}{}$ yields the weak form \eqref{eq_hu_washizu_derivative_0:eq0}, which is the notorious principle of virtual work
%
%
%
% \begin{equation}
%     \label{eq_virtual_works_02}
%     \begin{aligned}
%         % \frac{\partial J_{\cell}}{\partial \tensori{u}{}_{\cell}} \delta \tensori{u}{}_{\cell}
%         \delta J_{\bodyLag{}}^{VW}
%         % =
%         % \frac{\partial J_{\cell}^{VW}}{\partial \tensori{u}{}_{\cell}} \delta \tensori{u}{}_{\cell}
%         & =
%         \int_{\bodyLag{}}
%         % \frac{\partial \mecPotential{}_{\cell}}{\partial \nabla \tensori{u}{}_{\cell}} : \nabla \delta \tensori{u}{}_{\cell}
%         \tensorii{P}{} : \nabla \delta \tensori{u}{}
%         -
%         \int_{\bodyLag{}} \loadLag \cdot \delta \tensori{u}{}
%         -
%         \int_{\neumannBoundaryLag{}} \neumannLag{} \cdot \delta \tensori{u}{} \vert_{\neumannBoundaryLag{}}
%         &&
%         \ \ \ \ \ \ \ \ 
%         &&
%         \forall \delta \tensori{u}{}
%         \in \virtualDisplacementSpaceCell
%     \end{aligned}
% \end{equation}
%
%
%
that relates to the strong problem defined by equation \eqref{eq_model_problem_0:eq3}, \eqref{eq_model_problem_0:eq4}, \eqref{eq_model_problem_0:eq5} only.

%
%
%
% The initial configuration of the body at the time $t = 0$ (see Figure \ref{fig_setting}) is denoted $\bodyLag$ with respective Dirichlet and Neumann boundaries $\dirichletBoundaryLag$ and $\neumannBoundaryLag$. Let the body force $\loadLag$ the expression of $\loadEul$ in $\bodyLag{}$, the prescribed displacement $\dirichletLag$ that of $\dirichletEul$ onto $\dirichletBoundaryLag$ and the contact load $\neumannLag$ that of $\neumannEul$ onto $\dirichletBoundaryLag$.
% The transformation mapping $\tensori{\Phi}$ takes a point $\tensori{x} \in \bodyLag$ from the initial configuration to $\tensori{x}{}_t \in \bodyEul$ in the current configuration



% Let $\cell \subset \bodyLag$ an arbitrary open subset of the solid body, with boundary $\dCell \subset \mathbb{R}^{d - 1}$
% which is split into an eventual Dirichlet boundary $\dirichletCell \subset \dirichletBoundaryLag$ subjected to an imposed displacement $\dirichletLag$ if $\cell$ shares a boundary with $\dirichletBoundaryLag$ and into the Neumann Boundary $\neumannCell \subset \neumannBoundaryLag \cup \bodyLag$ with contact load $\neumannCellLoad$ such that

% Let us consider the equilibrium of a an (hyperelastic) solid body. The
% reference and current configurations of the body are denoted
% respectively $\bodyLag$ and $\bodyEul$ with respective boundaries $\dBodyLag$ and
% $\dBodyEul$.
% %
% %
% %
% In the reference configuration, the body $\bodyLag$ is subjected to body forces $\loadLag$, to contact loads $\neumannLag{}$ on a the Neumann boundary $\neumannBoundaryLag \subset \dBodyLag{}$ and to prescribed displacements $\dirichletLag$ on the Dirichlet boundary $\dirichletBoundaryLag \subset \dBodyLag{}$, where the Neumann and Dirichlet boundaries form a partition of $\dBodyLag$.
% % $\neumannBoundaryEul$ and
% % $\dirichletBoundaryEul$ form a partition of
% % $\dBodyLag$, i.e.
% % $\dBodyLag = \dirichletBoundaryLag \cup
% % \neumannBoundaryLag$ and $\dirichletBoundaryLag{} \cap
% % \neumannBoundaryLag = \emptyset$.
% %
% %
% %
% %
% %
% %
% where $U(\bodyLag{})$ denotes the space of all kinematically admissible displacement fields in $\bodyLag{}$, $G(\bodyLag{})$ that of all statically admissible displacement gradient fields, and $S(\bodyLag{})$ that of all statically admissible stress fields.
% %
% %
% %

% \begin{itemize}
% \item $\vec{u}$ is the displacement field
% \item ....
% \end{itemize}

% % Petit developpement pour identifier G et du_dX, etc.. Du coup, on retrouve le PTV.

% Let $\bodyEul$ a solid body with boundary $\dBodyEul$, that deforms in the current configuration at some time $t > 0$ under a body force $\loadEul$, a prescribed displacement $\dirichletEul$ on the Dirichlet boundary $\dirichletBoundaryEul \subset \dBodyEul$, and a contact load $\neumannEul{}$ on the Neumann boundary $\neumannBoundaryEul \subset \dBodyEul$.
% % , such that $\dBodyEul = \dirichletBoundaryEul \cup \neumannBoundaryEul$ and $\dirichletBoundaryEul{} \cap \neumannBoundaryEul = \emptyset$.
% %
% %
% %
% The initial configuration of the body at time $t = 0$ (see Figure \ref{fig_setting}) is denoted $\bodyLag$ with respective Dirichlet and Neumann boundaries $\dirichletBoundaryLag$ and $\neumannBoundaryLag$, such that the transformation mapping $\tensori{\Phi}$ takes a point $\tensori{x} \in \bodyLag$ from the initial configuration to $\tensori{x}{}_t \in \bodyEul$ in the current configuration. Let the body force $\loadLag$ the expression of $\loadEul$ in $\bodyLag{}$, the prescribed displacement $\dirichletLag$ that of $\dirichletEul$ onto $\dirichletBoundaryLag$ and the contact load $\neumannLag$ that of $\neumannEul$ onto $\dirichletBoundaryLag$.



% Let $\cell \subset \bodyLag$ an arbitrary open subset of the solid body, with boundary $\dCell$
% which is split into an eventual Dirichlet boundary $\dirichletCell \subset \dirichletBoundaryLag$ subjected to an imposed displacement $\dirichletLag$ if $\cell$ shares a boundary with $\dirichletBoundaryLag$ and into the Neumann Boundary $\neumannCell \subset \neumannBoundaryLag \cup \bodyLag$ with contact load $\neumannCellLoad$ such that
% %
% % 
% % 
% \begin{equation}
%     \label{eq_contact_force}
%     \begin{aligned}
%         \neumannCellLoad = 
%         \left\{
%             \begin{array}{ll}
%                 \tensori{t}{}_{\bodyLag \backslash \cell \rightarrow \cell} & \mbox{on } \neumannCell \cap \bodyLag \backslash \cell
%                 \\
%                 \neumannLag & \mbox{on } \neumannCell \cap \neumannBoundaryLag
%             \end{array}
%         \right.
%     \end{aligned}
% \end{equation}
% % 
% % 
% %
% with $\tensori{t}{}_{\bodyLag \backslash \cell \rightarrow \cell}$ the contact force applied by the surrounding part of the body $\bodyLag$ onto the subset $\cell$, and $\dirichletCell \cap \neumannCell = \emptyset$.
%
% 
% 
% \begin{figure}[H]
%     \centering
%     \includegraphics[width=7.cm]{img/setting.png}
%     \caption{schematic representation of the model problem}
%     \label{fig_setting}
% \end{figure}
%
% 
% 

% split into two distinct media; an open bulk medium $\Bulk \subset \cell$ with boundary $\dBulk$, and an open interface medium $\Crown \subset \cell$ between the bulk $\Bulk$ and the boundary $\dCell$, with boundary $\dCrown = \dBulk \cup \dCell$ and of some width $\ell > 0$ that is supposed to be small compared to $h_{\cell} = \max_{(\tensori{x}{}_a, \tensori{x}{}_b) \in \cell} \lVert \tensori{x}{}_a - \tensori{x}{}_b \rVert$ the diameter of $\cell$ (see Figure \ref{fig_02}).
% %
% % 
% % 
% \begin{figure}[H]
%     \centering
%     \includegraphics[width=12.cm]{img/hu_washizu.png}
%     \caption{schematic representation of the composite region}
%     \label{fig_02}
% \end{figure}
% %
% % 
% % 
% The boundary $\dCell$ moves with a boundary displacement field $\tensori{u}{}_{\dCell} \in V_{}(\dCell)$, where $V_{}(\dCell)$ denotes the space of kinematically admissible boundary displacements. The displacement at the boundary $\dCell$ results from the interactions of $\dCell$ with neighbouring media, \textit{i.e.} from the action of $\bodyLag \backslash \cell$ onto $\dCell$ or from some boundary condition.
% % Assuming the bulk $\Bulk$ is \textit{a priori} not influenced by the movements of $\dCell$, such that it only morphs through the action of the body load $\loadLag$, hence producing a displacement gradient $\tensorii{G}{}_{\Bulk} \in \gradSpaceBulk$ and a stress $\tensorii{P}{}_\Bulk \in \stressSpaceBulk$ under the mechanical potential $\mecPotential{}_{\bodyLag}$, one needs to restrict 
% The bulk $\Bulk$ is not directly attached to $\dCell$, and morphs by a displacement $\tensori{u}{}_{\Bulk} \in \displacementSpaceBulk$ under the action of the body load $\loadLag$, hence producing a displacement gradient $\tensorii{G}{}_{\Bulk} \in \gradSpaceBulk$ and a stress $\tensorii{P}{}_\Bulk \in \stressSpaceBulk$ depending on the mechanical potential $\mecPotential{}_{\bodyLag}$.
% % , and communicates with the surrounding medium $\bodyLag \backslash \cell$.
% It is however bound to the interface $\Crown$ that acts as a patch between the bulk $\Bulk$ and the boundary $\dCell$, such that its displacement $\tensori{u}_{\Crown} \in \displacementSpaceCrown$ verifies

% Let $\tensori{\Phi}{}_{\cell}$ the restriction of $\tensori{\Phi}$ to ${\cell}$, and $\tensori{u}{}_{\cell} \in \displacementSpaceCell$ the displacement field in $\cell$ such that $\tensori{\Phi}{}_{\cell} = \tensori{I}{}_d + \tensori{u}{}_{\cell}{}$ with $\tensori{I}{}_d$ the identity function, where the notation $\displacementSpaceCell$ denotes the space of all kinematically admissible displacement fields in $\cell$.
% % 
% Let $\tensorii{G}{}_{\cell}
% % \in \gradSpaceCell
% $ the displacement gradient field in $\gradSpaceCell$ the space of all statically admissible displacement gradient fields in $\cell$,
% and $\tensorii{F}{}_{\cell} := \nabla \tensori{\Phi}{}_{\cell} = \tensorii{1} + \tensorii{G}{}_{\cell}$ the transformation gradient, where $\nabla$ denotes the Lagrangian nabla operator.
% % and $\tensorii{F}{}_{\cell} := \nabla \tensori{\Phi}{}_{\cell} = \tensorii{1} + \nabla \tensori{u}$ the transformation gradient, where $\nabla$ denotes the Lagrangian nabla operator.
% %
% Let $\mecPotential{}_{\cell} (\tensorii{F}{}_{\cell}, \internaleStateVariables)$ the mechanical energy potential in $\cell$ that depends on the transformation gradient $\tensorii{F}{}_{\cell}$ and possibly on a set of internal state variables $\internaleStateVariables$.
% % where $\nabla$ denotes the Lagrangian nabla operator.
% %
% % Let $\tensorii{P}{}_{\cell} \in \stressSpaceCell$ the first Piola-Kirchoff stress tensor deriving from the expression of the mechanical energy potential, with $\stressSpaceCell$ the space of all statically admissible stress fields in $\cell$.
% Let $\tensorii{P}{}_{\cell}
% % \in \stressSpaceCell
% $ the first Piola-Kirchoff stress tensor in $\stressSpaceCell$ the space of all statically admissible stress fields in $\cell$, that depends on
% % $\mecPotential{}_{\cell, \tensori{F}{}}$
% the derivative of the mechanical energy potential $\mecPotential{}_{\cell}$ with respect to $\tensorii{F}{}_{\cell}$ (or equivalently, with respect to $\tensorii{G}{}_{\cell}$).
% % % where
% % % %
% % % %
% % % %
% % % \begin{equation}
% % %     \label{eq_stress_def}
% % %     \begin{aligned}
% % %         \mecPotential{}_{\cell, \tensori{F}{}}
% % %         :=
% % %         \frac{\partial \mecPotential{}_{\cell}}{\partial \tensorii{F}{}_{\cell}}
% % %         =
% % %         \frac{\partial \mecPotential{}_{\cell}}{\partial \tensorii{G}{}_{\cell}} 
% % %         \frac{\partial \tensorii{G}{}_{\cell}}{\partial \tensorii{F}{}_{\cell}}
% % %         =
% % %         \frac{\partial \mecPotential{}_{\cell}}{\partial \tensorii{G}{}_{\cell}}
% % %     \end{aligned}
% % \end{equation}
% %
% %
% %
% % 
% The equilibrium of the body $\cell$ is reached for the displacement gradient field $\tensorii{G}{}_{\cell} \in \gradSpaceCell$, the first Piola-Kirchoff stress field $\tensorii{P}{}_{\cell} \in \stressSpaceCell$ and the displacement field $\tensori{u}{}_{\cell} \in \displacementSpaceCell$ verifying $\tensori{u}{}_{\cell} \vert_{\dirichletCell} = \dirichletLag$ on $\dirichletCell$ minimizing the three field Hu–Washizu energy functional $J_{\cell}^{HW}$
% %
% %
% %
% \begin{equation}
%     \label{eq_hu_washizu}
%         J_{\cell}^{HW}
%         % (\tensori{u}{}_{\cell}, \tensorii{G}{}_{\cell}, \tensorii{P}{}_{\cell})
%         =
%         \int_{\cell} \mecPotential{}_{\cell} + (\nabla \tensori{u}{}_{\cell} - \tensorii{G}{}_{\cell}) : \tensorii{P}{}_{\cell}
%         -
%         \int_{\cell} \loadLag \cdot \tensori{u}{}_{\cell}
%         -
%         \int_{\neumannCell} \neumannCellLoad \cdot \tensori{u}{}_{\cell} \vert_{\dCell}
% \end{equation}
% %
% %
% %
% Deriving \eqref{eq_hu_washizu} with respect to all variables yields the weak formulation of the problem
% %
% % 
% % 
% \begin{subequations}
%     \label{eq_hu_washizu_derivative}
%         \begin{alignat}{3}
%             \frac{\partial J_{\cell}^{HW}}{\partial \tensori{u}{}_{\cell}} \delta \tensori{u}{}_{\cell}
%             = & \int_{\cell} \tensorii{P}{}_{\cell} : \nabla \delta \tensori{u}{}_{\cell}
%             -
%             \int_{\cell} \tensori{f}{}_V \cdot \delta \tensori{u}{}_{\cell}
%             -
%             \int_{\dCell} \tensori{t}{}_{\neumannCell} \cdot \delta \tensori{u}{}_{\cell} \vert_{\dCell}
%             &&
%             \ \ \ \ \ \ \ \ 
%             &&
%             \forall \delta \tensori{u}{}_{\cell}
%             \in \virtualDisplacementSpaceCell
%         \label{eq_hu_washizu_derivative:eq0}
%         \\
%             \frac{\partial J_{\cell}^{HW}}{\partial \tensorii{G}{}_{\cell}} \delta \tensorii{G}{}_{\cell}
%             = &
%             \int_{\cell} (\frac{\partial \mecPotential_{\cell}}{\partial \tensorii{G}{}_{\cell}} - \tensorii{P}{}_{\cell}) : \delta \tensorii{G}{}_{\cell}
%             &&
%             \ \ \ \ \ \ \ \ 
%             &&
%             \forall \delta \tensorii{G}{}_{\cell}
%             \in \gradSpaceCell
%         \label{eq_hu_washizu_derivative:eq2}
%         \\
%             \frac{\partial J_{\cell}^{HW}}{\partial \tensorii{P}{}_{\cell}} \delta \tensorii{P}{}_{\cell}
%             = & \int_{\cell} (\nabla \tensori{u}{}_{\cell} - \tensorii{G}{}_{\cell} ) : \delta \tensorii{P}{}_{\cell}
%             &&
%             \ \ \ \ \ \ \ \ 
%             &&
%             \forall \delta \tensorii{P}{}_{\cell}
%             \in \stressSpaceCell
%         \label{eq_hu_washizu_derivative:eq3}
%     \end{alignat}
% \end{subequations}
% %
% %
% %
% where $\cdot \ \vert_{\dCell}$ denotes the trace operator on $\dCell$, and $\virtualDisplacementSpaceCell$ the space of all kinematically admissible virtual displacements in $\cell$.
% Since the subset $\cell$ is arbitrary, \eqref{eq_hu_washizu_derivative} holds true in a distributional sense, and by applying the divergence theorem on \eqref{eq_hu_washizu_derivative:eq0}, one obtains the strong formulation of problem \eqref{eq_hu_washizu}
% %
% % 
% % 
% \begin{subequations}
% \label{eq_model_problem}
%     \begin{alignat}{2}
%     % \tensorii{F} - \nabla_{X} \tensori{u} & = \tensorii{1} \quad && \text{in } \Omega_{0} \label{eq_model_problem:eq1}
%     \tensorii{G}{}_{\cell} - \nabla \tensori{u}{}_{\cell} & = 0 \quad
%     &&
%     \text{in } \cell
%     \label{eq_model_problem:eq1}
%     \\
%     % \tensorii{P} - \frac{\partial \mecPotential}{\partial \tensorii{F}} & = 0 \quad && \text{in } \Omega_{0} \label{eq_model_problem:eq2}
%     \tensorii{P}{}_{\cell} - \frac{\partial \mecPotential{}_{\cell}}{\partial \tensorii{G}{}_{\cell}} & = 0 \quad
%     &&
%     \text{in } \cell
%     \label{eq_model_problem:eq2}
%     \\
%     \nabla_{X} \cdot \tensorii{P}{}_{\cell} + \loadLag & = 0 \quad
%     &&
%     \text{in } \cell
%     \label{eq_model_problem:eq3}
%     \\
%     \tensori{u}{}_{\cell} \vert_{\dirichletCell} & = \dirichletLag \quad
%     &&
%     \text{on } \dirichletCell
%     \label{eq_model_problem:eq4}
%     \\
%     \tensorii{P}{}_{\cell} \cdot \tensori{n} & = \neumannCellLoad \quad
%     &&
%     \text{on } \neumannCell
%     \label{eq_model_problem:eq5}
% \end{alignat}
% \end{subequations}
% %
% % 
% % 
% where $\tensori{n}$ denotes the unit ouward normal vector on $\dCell$.
% One readily identifies \eqref{eq_model_problem:eq1} with the displacement strain relation, \eqref{eq_model_problem:eq2} with the constitutive equation, \eqref{eq_model_problem:eq3} with the conservation of momentum, \eqref{eq_model_problem:eq5} with the traction force continuity and \eqref{eq_model_problem:eq4} with Dirichlet boundary conditions.
% %
% %
% %
% Considering now that $\tensorii{G}{}_{\cell}$ and $\tensorii{P}{}_{\cell}$ are not unknowns of the problem, \textit{i.e.} by explicitly eliminating both \eqref{eq_hu_washizu_derivative:eq2} and \eqref{eq_hu_washizu_derivative:eq3} from \eqref{eq_hu_washizu_derivative}, \eqref{eq_hu_washizu} simplifies as : find the displacement field $\tensori{u}{}_{\cell} \in \displacementSpaceCell$ verifying $\tensori{u}{}_{\cell} \vert_{\dirichletCell} = \dirichletLag$ on $\dirichletCell$ that minimizes the energy functional $J_{\cell}^{VW}$
% %
% %
% %
% \begin{equation}
%     \label{eq_virtual_works}
%     \begin{aligned}
%         J_{\cell}^{VW}
%         % (\tensori{u}{}_{\cell})
%         =
%         \int_{\cell} \mecPotential{}_{\cell}
%         -
%         \int_{\cell} \loadLag \cdot \tensori{u}{}_{\cell}
%         -
%         \int_{\neumannCell} \neumannCellLoad \cdot \tensori{u}{}_{\cell}
%     \end{aligned}
% \end{equation}
% %
% %
% %
% Deriving \eqref{eq_virtual_works} with respect to the primal unknown $\tensori{u}{}_{\cell}$ yields the weak form of the problem, which is the notorious principle of virtual work
% %
% %
% %
% \begin{equation}
%     \label{eq_virtual_works_1}
%     \begin{aligned}
%         % \frac{\partial J_{\cell}}{\partial \tensori{u}{}_{\cell}} \delta \tensori{u}{}_{\cell}
%         \delta J_{\cell}^{VW}
%         % =
%         % \frac{\partial J_{\cell}^{VW}}{\partial \tensori{u}{}_{\cell}} \delta \tensori{u}{}_{\cell}
%         & =
%         \int_{\cell}
%         % \frac{\partial \mecPotential{}_{\cell}}{\partial \nabla \tensori{u}{}_{\cell}} : \nabla \delta \tensori{u}{}_{\cell}
%         \tensorii{P}{}_{\cell} : \nabla \delta \tensori{u}{}_{\cell}
%         -
%         \int_{\cell} \loadLag \cdot \delta \tensori{u}{}_{\cell}
%         -
%         \int_{\neumannCell} \neumannCellLoad \cdot \delta \tensori{u}{}_{\cell} \vert_{\dCell{}}
%         &&
%         \forall \delta \tensori{u}{}_{\cell}
%         \in \virtualDisplacementSpaceCell
%     \end{aligned}
% \end{equation}
% %
% %
% %
% that relates to the strong problem defined by equation \eqref{eq_model_problem:eq3}, \eqref{eq_model_problem:eq4}, \eqref{eq_model_problem:eq5} only, where \eqref{eq_model_problem:eq1}, \eqref{eq_model_problem:eq2} are taken to be granted.
% %
% %
% %
% % ,and $\cdot \ \vert_{\dCell}$ is the trace operator on $\dCell$.
% % The equilibrium of the body $\cell$ corresponding to problem \eqref{eq_model_problem} where equations \eqref{eq_model_problem:eq1} and \eqref{eq_model_problem:eq2} are enforced strongly is reached for the displacement field $\tensori{u}{}_{\cell} \in \displacementSpaceCell$ verifying $\tensori{u}{}_{\cell} \vert_{\dirichletCell} = \dirichletLag$ on $\dirichletCell$ and minimizing the energy functional $J_{\cell}$:
% % %
% % % 
% % % 
% % \begin{equation}
% %     \label{eq_virtual_works}
% %     \begin{aligned}
% %         J_{\cell}
% %         % (\tensori{u}{}_{\cell})
% %         =
% %         \int_{\cell} \mecPotential{}_{\cell}
% %         -
% %         \int_{\cell} \loadLag \cdot \tensori{u}{}_{\cell}
% %         -
% %         \int_{\neumannCell} \neumannCellLoad \cdot \tensori{u}{}_{\cell}
% %     \end{aligned}
% % \end{equation}
% % %
% % % 
% % % 
% % The energy functional \eqref{eq_virtual_works} of the equilibrium of $T$ depends on the single displacement unknown, and is the one at the foundation of the notorious principle of virtual works; indeed, deriving \eqref{eq_virtual_works} with respect to $\tensori{u}{}_{\cell}$ and using both \eqref{eq_model_problem:eq1} and \eqref{eq_model_problem:eq2} yields :
% % %
% % % 
% % % 
% % \begin{equation}
% %     \label{eq_virtual_works_1}
% %     \begin{aligned}
% %         % \frac{\partial J_{\cell}}{\partial \tensori{u}{}_{\cell}} \delta \tensori{u}{}_{\cell}
% %         d J_{\cell} = \frac{\partial J_{\cell}}{\partial \tensori{u}{}_{\cell}} \delta \tensori{u}{}_{\cell}
% %         & =
% %         \int_{\cell}
% %         % \frac{\partial \mecPotential{}_{\cell}}{\partial \nabla \tensori{u}{}_{\cell}} : \nabla \delta \tensori{u}{}_{\cell}
% %         \tensorii{P}{}_{\cell} : \nabla \delta \tensori{u}{}_{\cell}
% %         -
% %         \int_{\cell} \loadLag \cdot \delta \tensori{u}{}_{\cell}
% %         -
% %         \int_{\neumannCell} \neumannCellLoad \cdot \delta \tensori{u}{}_{\cell} \vert_{\dCell{}}
% %     \end{aligned}
% % \end{equation}
% % %
% % % 
% % % 
% % with $\displacementSpaceCell{} = H^1(\cell, \mathbb{R}^{d})$. If $\tensorii{G}{}_{\cell}$ and $\tensorii{P}{}_{\cell}$ are unknowns of the problem, one obtains the three-field Hu–Washizu functional $J_{\cell}$, for the displacement field $\tensori{u}{}_{\cell} \in \displacementSpaceCell$ verifying $\tensori{u}{}_{\cell} \vert_{\dirichletCell} = \dirichletLag$ on $\dirichletCell$ such that :
% % % 
% % % 
% % %
% % % \begin{equation}
% % % \label{eq_hu_washizu}
% % %     J_{\cell}
% % %     % (\tensori{u}{}_{\cell}, \tensorii{G}{}_{\cell}, \tensorii{P}{}_{\cell})
% % %     =
% % %     \int_{\cell} \mecPotential{}_{\cell} + (\nabla \tensori{u}{}_{\cell} - \tensorii{G}{}_{\cell}) : \tensorii{P}{}_{\cell}
% % %     -
% % %     \int_{\cell} \loadLag \cdot \tensori{u}{}_{\cell}
% % %     -
% % %     \int_{\neumannCell} \neumannCellLoad \cdot \tensori{u}{}_{\cell}
% % % \end{equation}
% % %
% % % 
% % % 
% % Deriving \eqref{eq_hu_washizu} with respect to all variables of the problem expresses problem \eqref{eq_model_problem} in a weak sense :
% % %
% % % 
% % % 
% % \begin{subequations}
% %     \label{eq_hu_washizu_derivative}
% %         \begin{alignat}{3}
% %             \frac{\partial J_{\cell}}{\partial \tensori{u}{}_{\cell}} \delta \tensori{u}{}_{\cell}
% %             = & \int_{\cell} \tensorii{P}{}_{\cell} : \nabla \delta \tensori{u}{}_{\cell}
% %             -
% %             \int_{\cell} \tensori{f}{}_V \cdot \delta \tensori{u}{}_{\cell}
% %             -
% %             \int_{\dCell} \tensori{t}{}_{\neumannCell} \cdot \delta \tensori{u}{}_{\cell} \vert_{\dCell}
% %             &&
% %             \ \ \ \ \ \ \ \ 
% %             &&
% %             \forall \delta \tensori{u}{}_{\cell}
% %             \in \displacementSpaceCell
% %         \label{eq_hu_washizu_derivative:eq0}
% %         \\
% %             \frac{\partial J_{\cell}}{\partial \tensorii{G}{}_{\cell}} \delta \tensorii{G}{}_{\cell}
% %             = &
% %             \int_{\cell} (\frac{\partial \mecPotential_{\cell}}{\partial \tensorii{G}{}_{\cell}} - \tensorii{P}{}_{\cell}) : \delta \tensorii{G}{}_{\cell}
% %             &&
% %             \ \ \ \ \ \ \ \ 
% %             &&
% %             \forall \delta \tensorii{G}{}_{\cell}
% %             \in \gradSpaceCell
% %         \label{eq_hu_washizu_derivative:eq2}
% %         \\
% %             \frac{\partial J_{\cell}}{\partial \tensorii{P}{}_{\cell}} \delta \tensorii{P}{}_{\cell}
% %             = & \int_{\cell} (\nabla \tensori{u}{}_{\cell} - \tensorii{G}{}_{\cell} ) : \delta \tensorii{P}{}_{\cell}
% %             &&
% %             \ \ \ \ \ \ \ \ 
% %             &&
% %             \forall \delta \tensorii{P}{}_{\cell}
% %             \in \stressSpaceCell
% %         \label{eq_hu_washizu_derivative:eq3}
% %     \end{alignat}
% % \end{subequations}
% % %
% % % 
% % % 
% % where the two supplementary equations \eqref{eq_hu_washizu_derivative:eq2} and \eqref{eq_hu_washizu_derivative:eq3} account for the weak formulation of \eqref{eq_model_problem:eq1} and \eqref{eq_model_problem:eq2}, and $\displacementSpaceCell{} = H^1(\cell, \mathbb{R}^{d})$ and $\gradSpaceCell = \stressSpaceCell = L^2(\cell, \mathbb{R}^{d \times d})$.

% %
% %
% %

% Assuming that $\cell$ is made out of a partition of $N > 0$ distinct media $\matI \subset \cell$ with respective energy potentials $\mecPotential{}_{\matI}$, the problem writes : for each medium $\matI$, find $\tensori{u}{}_{\matI} \in \displacementSpaceMatI$ verifying
% $\tensori{u}{}_{\matI} \vert_{\dirichletMatI} = \dirichletLag$ on $\dirichletMatI$,
% the displacement gradient field $\tensorii{G}{}_{\matI} \in \gradSpaceMatI$ and the first Piola-Kirchoff stress field $\tensorii{P}{}_{\matI} \in \stressSpaceMatI$, that minimize the functional
% % 
% % 
% % 
% \begin{equation}
% \label{eq_hu_washizu_composite}
% \begin{aligned}
%     J_{\cell}^{HW}
%     = & \sum_{1 \leq i \leq N} \int_{\matI} \mecPotential{}_{\matI} + (\nabla \tensori{u}{}_{\matI} - \tensorii{G}{}_{\matI}) : \tensorii{P}{}_{\matI}
%     -
%     \int_{\matI} \loadLag \cdot \tensori{u}{}_{\matI}
%     -
%     \int_{\neumannMatI \cap \neumannCell} \neumannCellLoad \cdot \tensori{u}{}_{\matI}
% \end{aligned}
% \end{equation}
% %
% % 
% % 
% where for all $1 \leq i \neq j \leq N$, the external forces corresponding to the traction applied by $\matI$ onto $\partial T_j \cap \dMatI$ and to that of $T_j$ onto $\dMatI \cap \partial T_j$ are direclty eliminated by continuity of the traction force across $\partial T_j \cap \dMatI$.
% %
% %
% %

% If the unknown fields are continuous in $\cell$, one has :
% $
% \tensori{u}{}_{\matI} = \tensori{u}{}_{\cell} \vert_{\matI},
% \tensorii{G}{}_{\matI} = \tensorii{G}{}_{\cell} \vert_{\matI},
% \tensorii{P}{}_{\matI} = \tensorii{P}{}_{\cell} \vert_{\matI}
% $,
% and the problem simplifies in : find $\tensori{u}{}_{\cell} \in \displacementSpaceCell$
% verifying $\tensori{u}{}_{\cell} \vert_{\dirichletCell} = \dirichletLag$ on $\dirichletCell$,
% the displacement gradient field $\tensorii{G}{}_{\cell} \in \gradSpaceCell$
% and the first Piola-Kirchoff stress field $\tensorii{P}{}_{\cell} \in \stressSpaceCell$ that minimize
% %
% % 
% % 
% \begin{equation}
% \label{eq_hu_washizu_composite_continuous}
% \begin{aligned}
%     J_{\cell}^{HW}
%     = & \sum_{1 \leq i \leq N} \int_{\matI} \mecPotential{}_{\matI} + (\nabla \tensori{u}{}_{\cell} - \tensorii{G}{}_{\cell}) : \tensorii{P}{}_{\cell}
%     -
%     \int_{\matI} \loadLag \cdot \tensori{u}{}_{\cell}
%     -
%     \int_{\neumannCell{}} \neumannCellLoad \cdot \tensori{u}{}_{\cell} \vert_{\dCell{}}
% \end{aligned}
% \end{equation}

% In particular, assuming $\cell = \bodyLag$, one obtains the mechanical problem to solve for the whole body $\bodyLag$.