\section{Appendix}
\label{sec_appendix}

\subsection{From the continuous Hu-Washizu Lagrangian to the HDG one}

Let $\tensori{\Psi}(\tensori{X})$ the linear mapping consisting in a change of euclidean frame such that
%
%
%
\begin{equation}
    \tensori{\Psi} : \tensori{X} \mapsto \tensori{x} = \tensorii{Q}{} \tensori{X} + \tensori{c}
\end{equation}

%
% 
% 
\begin{equation}
    \label{eq22}
    \begin{aligned}
        L_{\Crown{}, \text{int}}^{HW}
        := &
        \int_{\Crown{}} \mecPotential{}_{\Crown} + (\nabla \tensori{u}{}_{\Crown} - \tensorii{G}{}_{\Crown}) : \tensorii{P}{}_{\Crown}
        % -
        % \int_{\Crown{}} \loadLag \cdot \tensori{u}{}_{\Crown}
        \\
        = &
        (1 - \frac{\alpha}{2} \ell)
        \int_{\dBulk{}} \frac{\beta}{2 h_{\cell}} \lVert \tensori{u}{}_{\dCell{}} - \tensori{u}{}_{\Bulk{}} \vert_{\dBulk{}} \rVert^2
        +
        (1 - \frac{\alpha}{2} \ell)
        \int_{\dBulk} (\tensori{u}{}_{\dCell{}} - \tensori{u}{}_{\Bulk{}} \vert_{\dBulk{}}) \cdot \tensorii{P}{}_{\Bulk{}} \vert_{\dBulk{}} \cdot \tensori{n}{}
        -
        \int_{\Crown{}} \tensorii{G}{}_{\Crown{}} : \tensorii{P}{}_{\Crown{}}
        % -
        % \int_{\Crown{}} \loadLag \cdot \tensori{u}{}_{\Crown}
    \end{aligned}
\end{equation}
%
% 
%
The development of \eqref{eq22} is given in Appendix. Injecting \eqref{eq22} in \eqref{eq_hu_washizu_split} yields
%
% 
% 
\begin{equation}
    \label{eq_0014}
    \begin{aligned}
        L_{\cell}^{HW}
        = &
        \int_{\Bulk} \mecPotential{}_{\bodyLag{}} + (\nabla \tensori{u}{}_{\Bulk} - \tensorii{G}{}_{\Bulk}) : \tensorii{P}{}_{\Bulk}
        % \\
        % &
        +
        (1 - \frac{\alpha}{2} \ell)
        % \Biggl(
        \int_{\dBulk{}} (\tensori{u}{}_{\dCell{}} - \tensori{u}{}_{\Bulk} \vert_{\dBulk}) \cdot \tensorii{P}{}_{\Bulk} \vert_{\dBulk} \cdot \tensori{n}{}
        % \\
        % &
        \\
        &
        +
        (1 - \frac{\alpha}{2} \ell)
        \int_{\dBulk{}} \frac{\beta}{2 h_T} \lVert \tensori{u}{}_{\dCell{}} - \tensori{u}{}_{\Bulk} \vert_{\dBulk{}} \rVert^2
        % \Biggr)
        % \\
        % &
        -
        \int_{\Crown{}} \tensorii{G}{}_{\Crown{}} : \tensorii{P}{}_{\Crown{}}
        % \\
        % &
        -
        \int_{\Bulk} \loadLag \cdot \tensori{u}{}_{\Bulk}
        -
        \int_{\Crown{}} \loadLag \cdot \tensori{u}{}_{\Crown{}}
        -
        \int_{\neumannCell{}} \neumannCellLoad{} \cdot \tensori{u}{}_{\dCell{}}
    \end{aligned}
\end{equation}
%
% 
% ------------------------------------------------------- DEVELOPMENT
\textcolor{blue}{
%
\begin{development}[Interafce simplification]
%
Let $C_\Crown = \{ v \in L^2(\Crown) \ \vert \ v \cdot \tensori{n} = \text{cste} \}$ the set of $L^2$-functions which are constant along the normal axis in $\Crown$. For any function in $C_\Crown$, the following equality holds true:
%
% 
% 
\begin{equation}
    \label{eq_virtual_works0}
        \int_{\Crown} v \ dV
        =
        \int_{\dBulk{}} \int_{\epsilon = 0}^{\ell} v (1 - \alpha \epsilon) \ dS d \epsilon
        =
        \ell (1 - \frac{\alpha}{2} \ell) \int_{\dBulk{}} v \ dS
\end{equation}
%
% 
% 
Noticing that $\nabla \tensori{u}{}_{\Crown} \in C_\Crown$, one has :
%
% 
% 
\begin{equation}
    \begin{aligned}
        \int_{\Crown{}} \mecPotential{}_{\Crown}
        % = &
        % \int_{\Crown{}} \frac{1}{2} \beta \frac{\ell}{h_{\cell}} \nabla \tensori{u}{}_{\Crown} : \nabla \tensori{u}{}_{\Crown}
        % \\
        = & 
        \ell (1 - \frac{\alpha}{2} \ell)
        \int_{\dBulk{}} \frac{1}{2} \beta \frac{\ell}{h_{\cell}} \nabla \tensori{u}{}_{\Crown} : \nabla \tensori{u}{}_{\Crown}
        \\
        = & 
        \ell (1 - \frac{\alpha}{2} \ell)
        \int_{\dBulk{}} \frac{\beta}{2 \ell h_{\cell}} (\tensori{u}{}_{\dCell} - \tensori{u}{}_{\Bulk} \vert_{\dBulk}) \otimes
        \tensori{n} : (\tensori{u}{}_{\dCell} - \tensori{u}{}_{\Bulk} \vert_{\dBulk}) \otimes
        \tensori{n}
        \\
        = & 
        \ell (1 - \frac{\alpha}{2} \ell)
        \int_{\dBulk{}} \frac{\beta}{2 \ell h_{\cell}} \sum_{i,j} (\tensoro{u}{}_{\dCell}{}_{i}- \tensoro{u}{}_{\Bulk}{}_{i} \vert_{\dBulk}){}^2
        \tensoro{n}_{j}{}^2
        \\
        = & 
        \ell (1 - \frac{\alpha}{2} \ell)
        \int_{\dBulk{}} \frac{\beta}{2 \ell h_{\cell}} \sum_{j} \tensoro{n}_{j}{}^2 \sum_{i} (\tensoro{u}{}_{\dCell}{}_{i}- \tensoro{u}{}_{\Bulk}{}_{i} \vert_{\dBulk}){}^2
        \\
        = & 
        \ell (1 - \frac{\alpha}{2} \ell)
        \int_{\dBulk{}} \frac{\beta}{2 \ell h_{\cell}} \sum_{i} (\tensoro{u}{}_{\dCell}{}_{i}- \tensoro{u}{}_{\Bulk}{}_{i} \vert_{\dBulk}){}^2
        \\
        = & 
        \ell (1 - \frac{\alpha}{2} \ell)
        \int_{\dBulk{}} \frac{\beta}{2 \ell h_{\cell}} \lVert \tensori{u}{}_{\dCell} - \tensori{u}{}_{\Bulk}{} \vert_{\dBulk} \lVert {}^2
        \\
        = & 
        (1 - \frac{\alpha}{2} \ell)
        \int_{\dBulk{}} \frac{\beta}{2 h_{\cell}} \lVert \tensori{u}{}_{\dCell} - \tensori{u}{}_{\Bulk}{} \vert_{\dBulk} \lVert {}^2
    \end{aligned}
\end{equation}
%
% 
% 
Moreover, for $\tensorii{P}{}_{\Crown}$ in $C_\Crown{}$ :
%
% 
% 
\begin{equation}
    \begin{aligned}
        \int_{\Crown{}} \nabla \tensori{u}{}_{\Crown} : \tensorii{P}{}_{\Crown}
        = &
        \ell (1 - \frac{\alpha}{2} \ell)
        \int_{\dBulk{}} \nabla \tensori{u}{}_{\Crown} : \tensorii{P}{}_{\Crown}
        \\
        = &
        \ell (1 - \frac{\alpha}{2} \ell)
        \int_{\dBulk{}}
        \frac{1}{\ell}
        (\tensori{u}{}_{\dCell} - \tensori{u}{}_{\Bulk}{} \vert_{\dBulk}) \otimes \tensori{n} : \tensorii{P}{}_{\Bulk{}} \vert_{\dBulk{}}
        \\
        = &
        \ell (1 - \frac{\alpha}{2} \ell)
        \int_{\dBulk{}}
        \frac{1}{\ell}
        \sum_{i,j}
        (\tensoro{u}{}_{\dCell}{}_{i} - \tensoro{u}{}_{\Bulk}{}{}_{i} \vert_{\dBulk}) \tensoro{n}{}_{j} \tensoro{P}{}_{\Bulk{}}{}_{ij} \vert_{\dBulk{}}
        \\
        = &
        \ell (1 - \frac{\alpha}{2} \ell)
        \int_{\dBulk{}}
        \frac{1}{\ell}
        (\tensori{u}{}_{\dCell} - \tensori{u}{}_{\Bulk}{} \vert_{\dBulk}) \cdot \tensorii{P}{}_{\Bulk{}} \vert_{\dBulk{}} \cdot \tensori{n}
        \\
        = &
        (1 - \frac{\alpha}{2} \ell)
        \int_{\dBulk{}}
        (\tensori{u}{}_{\dCell} - \tensori{u}{}_{\Bulk}{} \vert_{\dBulk}) \cdot \tensorii{P}{}_{\Bulk{}} \vert_{\dBulk{}} \cdot \tensori{n}
    \end{aligned}
\end{equation}
% 
% 
% 
And Finally :
%
% 
% 
\begin{equation}
    \begin{aligned}
        L_{\Crown{}, \text{int}}^{HW}
        =
        (1 - \frac{\alpha}{2} \ell)
        \int_{\dBulk{}} \frac{\beta}{2 h_{\cell}} \lVert \tensori{u}{}_{\dCell{}} - \tensori{u}{}_{\Bulk{}} \vert_{\dBulk{}} \rVert^2
        +
        (1 - \frac{\alpha}{2} \ell)
        \int_{\dBulk} (\tensori{u}{}_{\dCell{}} - \tensori{u}{}_{\Bulk{}} \vert_{\dBulk{}}) \cdot \tensorii{P}{}_{\Bulk{}} \vert_{\dBulk{}} \cdot \tensori{n}{}
        -
        \int_{\Crown{}} \tensorii{G}{}_{\Crown{}} : \tensorii{P}{}_{\Crown{}}
    \end{aligned}
\end{equation}
%
\end{development}
}
% ------------------------------------------------------- DEVELOPMENT
%
%
%
\textcolor{blue}{
    \begin{development}[Elliptic projection]
        %
        %
        %
        Let $\discreteDisplacementSpaceCell \subset \displacementSpaceCell$ and $U^\perp(\cell) \subset \displacementSpaceCell$ such that $\displacementSpaceCell = \discreteDisplacementSpaceCell \oplus U^\perp(\cell)$, and set $\tensori{u}{}_{\cell} = \tensori{u}{}_{\cell}^h + \tensori{u}{}_{\cell}^\perp$ with
        $\tensori{u}{}_{\cell}^h \in U^h(\cell)$ and $\tensori{u}{}_{\cell}^\perp \in U^\perp(\cell)$ the orthogonal projections of $\tensori{u}{}_{\cell}$ onto $U^h(\cell)$ and $U^\perp(\cell)$ respectively.
        Let $V^h(\dCell) \subset \displacementSpaceDCell$ and $\tensori{u}{}_{\dCell}^h \in V^h(\dCell)$ the orthogonal projection of $\tensori{u}{}_{\cell}$ onto $V^h(\dCell)$.
        The orthogonal projection of $\tensori{u}{}_{\cell}$ onto $\discreteHybridDisplacementSpaceCell = U^h(\cell) \times V^h(\dCell)$ is then the displacement pair $(\tensori{u}{}_{\cell}^h, \tensori{u}{}_{\dCell}^h)$.
        Let $S^h(\cell) = \{ \tensorii{\tau}{}_{\cell}^h \in \stressSpaceCell \ \ \vert \ \ \nabla \cdot  \tensorii{\tau}{}_{\cell}^h \in U^h(\cell) \ \ \vert \ \  \tensorii{\tau}{}_{\cell}^h \vert_{\dCell} \cdot \tensori{n}{} \in V^h(\dCell) \}$,
        and $\tensorii{G}{}_{\cell}^h \in S^h(\cell)$ the solution of \eqref{eq_grad} for $(\tensori{u}{}_{\cell}^h, \tensori{u}{}_{\dCell}^h)$ such that
        %
        %
        %
        \begin{equation}
            \begin{aligned}
                \int_{\cell} \tensorii{G}{}_{\cell}^h(\tensori{u}{}_{\cell}^h, \tensori{u}{}_{\dCell}^h) : \tensorii{\tau}{}_{\cell}^h
                =
                \int_{\cell} \nabla \tensori{u}{}_{\cell}^h : \tensorii{\tau}{}_{\cell}^h
                +
                \int_{\dCell} (\tensori{u}{}_{\dCell}^h - \tensori{u}{}_{\cell}^h \vert_{\dCell}) \cdot \tensorii{\tau}{}_{\cell}^h \vert_{\dCell} \cdot \tensori{n}{}
                &&
                \ \ \ \ \ \ \ \ 
                &&
                \forall \tensorii{\tau}{}_{\cell}^h \in S^h(\cell)
            \end{aligned}
        \end{equation}
        %
        %
        %
        using the fact that $\tensori{u}{}_{\dCell}^h$ is the projection of $\tensori{u}{}_{\cell}$ onto $V^h(\dCell)$ and that $\tensorii{\tau}{} \vert_{\dCell} \cdot \tensori{n}{} \in V^h(\dCell)$:
        %
        %
        %
        \begin{equation}
            \begin{aligned}
                \int_{\cell} \tensorii{G}{}_{\cell}^h(\tensori{u}{}_{\cell}^h, \tensori{u}{}_{\dCell}^h) : \tensorii{\tau}{}_{\cell}^h
                = &
                \int_{\cell} \nabla \tensori{u}{}_{\cell}^h : \tensorii{\tau}{}_{\cell}^h
                +
                \int_{\dCell} (\tensori{u}{}_{\cell} \vert_{\dCell} - \tensori{u}{}_{\cell}^h \vert_{\dCell}) \cdot \tensorii{\tau}{}_{\cell}^h \vert_{\dCell} \cdot \tensori{n}{}
                &&
                \ \ \ \ \ \ \ \ 
                &&
                \forall \tensorii{\tau}{}_{\cell}^h \in S^h(\cell)
                \\
                = &
                \int_{\cell} \nabla \tensori{u}{}_{\cell}^h : \tensorii{\tau}{}_{\cell}^h
                +
                \int_{\dCell} \tensori{u}{}_{\cell}^\perp \vert_{\dCell} \cdot \tensorii{\tau}{}_{\cell}^h \vert_{\dCell} \cdot \tensori{n}{}
                &&
                \ \ \ \ \ \ \ \ 
                &&
                \forall \tensorii{\tau}{}_{\cell}^h \in S^h(\cell)
            \end{aligned}
        \end{equation}
        %
        %
        %
        using the divergence theorem and the fact that $\nabla \cdot  \tensorii{\tau}{}_{\cell}^h \in U^h(\cell)$, one has :
        %
        %
        %
        \begin{equation}
            \begin{aligned}
                \int_{\cell} \nabla \tensori{u}{}_{\cell}^\perp :  \tensorii{\tau}{}_{\cell}^h
                = &
                % -
                % \int_{\cell} \tensori{u}{}_{\cell}^\perp \cdot \nabla \cdot \tensorii{\tau}{}
                % +
                % \int_{\dCell} \tensori{u}{}_{\cell}^\perp \vert_{\dCell} \cdot \tensorii{\tau}{} \vert_{\dCell} \cdot \tensori{n}{}
                % \\
                % = &
                \int_{\dCell} \tensori{u}{}_{\cell}^\perp \vert_{\dCell} \cdot  \tensorii{\tau}{}_{\cell}^h \vert_{\dCell} \cdot \tensori{n}{}
            \end{aligned}
        \end{equation}
        %
        %
        %
        such that :
        %
        %
        %
        \begin{equation}
            \begin{aligned}
                \int_{\cell} \tensorii{G}{}_{\cell}^h(\tensori{u}{}_{\cell}^h, \tensori{u}{}_{\dCell}^h) : \tensorii{\tau}{}_{\cell}^h
                = &
                \int_{\cell} \nabla \tensori{u}{}_{\cell}^h : \tensorii{\tau}{}_{\cell}^h
                +
                \int_{\cell} \nabla \tensori{u}{}_{\cell}^\perp : \tensorii{\tau}{}_{\cell}^h
                &&
                \ \ \ \ \ \ \ \ 
                &&
                \forall \tensorii{\tau}{}_{\cell}^h \in S^h(\cell)
                \\
                = &
                \int_{\cell} \nabla \tensori{u}{}_{\cell} : \tensorii{\tau}{}_{\cell}^h
                &&
                \ \ \ \ \ \ \ \ 
                &&
                \forall \tensorii{\tau}{}_{\cell}^h \in S^h(\cell)
            \end{aligned}
        \end{equation}
        %
        %
        %
        which states that $\tensorii{G}{}_{\cell}^h(\tensori{u}{}_{\cell}^h, \tensori{u}{}_{\dCell}^h)$ is the orthogonal projection of $\nabla \tensori{u}{}_{\cell}$ onto $S^h(\cell)$.
        % IL FAUDRAIT MONTRER QUE $S^h(\cell)$ EST SUFFISANT POUR EMPECHER LE LOCKING DANS $U^h(\cell) \times V^h(\dCell)$ ??
        % ET FAIRE LE LIEN ENTRE $S^h(\cell)$ et $G^h(\cell)$.
        % Trouver aussi une justification pour $\stressSpaceCell \subset \gradSpaceCell$, la contrainte est plus régulière que le gradient, ce qui semble vrai (avec les décompositions sphérique dévaitoriques par exemples, etc)
        %
        %
        %
    \end{development}
}

\paragraph{Reconstructed gradient}

% $\tensoro{G}{}_{\cell \theta \theta}$ does not define by the same equation as those in the other directions. In particular,
For any displacement pair $(\tensori{v}{}_{\cell}^l, \tensori{v}{}_{\dCell}^k) \in \discreteDisplacementSpaceCell{} \times \discreteDisplacementSpaceDCell{}$, the component $\tensoro{G}{}_{\cell \theta \theta}(\tensoro{v}{}_{\cell r}, \tensoro{v}{}_{\dCell r})$ solves
%
%
%
\begin{equation}
    \label{axi_symmetric_gradient_theta}
    \begin{aligned}
        \int_{\cell} 2 \pi r \tensoro{G}{}_{\cell \theta \theta}(\tensoro{v}{}_{\cell r}, \tensoro{v}{}_{\dCell r}) \tensoro{\tau}{}_{\cell \theta \theta}
        =
        \int_{\cell} 2 \pi r \frac{\tensoro{u}{}_{\cell r}}{r} \tensoro{\tau}{}_{\cell \theta \theta}
        =
        \int_{\cell} 2 \pi \tensoro{u}{}_{\cell r} \tensoro{\tau}{}_{\cell \theta \theta}
        &&
        \forall \tensorii{\tau}{}_{\cell} \in \stressSpaceCell
    \end{aligned}
\end{equation}
%
%
%
In the radial and ordonal directions, \textit{i.e.} $\forall i, j \in \{ r,z \}$, the expression given in \eqref{eq_grad} is retrieved, and the component $G_{\cell ij}(\tensoro{v}{}_{\cell i}, \tensoro{v}{}_{\dCell i})$ solves
%
%
%
\begin{equation}
    \label{axi_symmetric_gradient_rz}
    \begin{aligned}
    \int_{\cell} 2 \pi r G_{\cell ij}(\tensoro{v}{}_{\cell i}, \tensoro{v}{}_{\dCell i}) \tau_{\cell ij} =
    \int_{\cell} 2 \pi r \frac{\partial \tensoro{u}{}_{\cell i}}{\partial j} \tau_{ij} -
    \int_{\dCell} 2 \pi r (u_{\dCell i} - u_{\cell i} \vert_{\dCell}) \tau_{\cell ij} \vert_{\dCell} n_{j}
    &&
    \forall \tensorii{\tau}{}_{\cell} \in \stressSpaceCell
    \end{aligned}
\end{equation}
%
%
%

\paragraph{Reconstructed higher order displacement}

For any $\tensori{d}{}_{\cell}^{k + 1} \in \discretePotentialSpaceCell$, the radial component $w^{k+1}_{\cell r}$ solves
%
%
%
\begin{equation}
    \label{axi_symmetric_potential_r}
    \begin{aligned}
        \int_{\cell} 2 \pi r (\sum_{i \in \{ r,z \}} \frac{\partial w^{k+1}_{\cell r}}{\partial i} \frac{\partial d^{k+1}_{\cell r}}{\partial i} + \frac{w^{k+1}_{\cell r}}{r} \frac{d^{k+1}_{\cell r}}{r})
        = &
        \int_{\cell} 2 \pi r (\sum_{i \in \{ r,z \}} \frac{\partial u_{\cell r}}{\partial i} \frac{\partial d^{k+1}_{\cell r}}{\partial i} + \frac{u_{\cell r}}{r} \frac{d^{k+1}_{\cell r}}{r})
        % &&
        % \forall \tensori{w}{}_{\cell} \in \mathbb{P}^{k + 1}(T, \mathbb{R}^2)
        \\
        &
        +
        \int_{\dCell} 2 \pi r \sum_{i \in \{ r,z \}} (u_{\dCell r} - u_{\cell r} \vert_{\dCell}) \frac{\partial d^{k+1}_{\cell r}}{\partial i} \vert_{\dCell} n_{i}
    \end{aligned}
\end{equation}
%
%
%
where the mean value condition is not needed on the radial component of the higher order displacement since the left hand side of the system described by \eqref{axi_symmetric_potential_r} depends directly on the displacement unknown and not only on its gradient as in \eqref{axi_symmetric_potential_z}.
The ordinate component $w^{k+1}_{\cell z}$ solves :
%
%
%
\begin{subequations}
    \label{axi_symmetric_potential_z}
        \begin{alignat}{3}
            \int_{\cell} 2 \pi r \sum_{i \in \{ r,z \}}
            \frac{\partial w^{k+1}_{\cell z}}{\partial i} \frac{\partial d^{k+1}_{\cell z}}{\partial i}
            = &
            \int_{\cell} 2 \pi r \sum_{i \in \{ r,z \}} \frac{\partial u_{\cell z}}{\partial i} \frac{\partial d^{k+1}_{\cell z}}{\partial i}
            -
            \int_{\dCell} 2 \pi r \sum_{i \in \{ r,z \}} (u_{\dCell z} - u_{\cell z} \vert_{\dCell})
            \frac{\partial d^{k+1}_{\cell z}}{\partial i} \vert_{\dCell} n_{i}
            % &&
            % \forall \tensori{w}{}_{\cell} \in \mathbb{P}^{k + 1}(T, \mathbb{R}^2)
            % \\
            % &
            % -
            % \int_{\dCell} 2 \pi r \sum_{i \in \{ r,z \}} (u_{\dCell z} - u_{\cell z} \vert_{\dCell})
            \\
            \int_{\cell} 2 \pi r w^{k+1}_{\cell z} = & \int_{\cell} 2 \pi r u_{\cell z}
        \end{alignat}
\end{subequations}

% \subsubsection{Plastic behavior in small strains}

% Dans le cadre de la thermodynamique des milieux continus, la combinsaison de l'application des deux premiers principes de la themodynamique donne lieu à l'équation de Clausius-Duhem qui postule la positivité de l'énergie de dissipation
% %
% %
% %
% \begin{equation}
%     \label{eq_clausius_duhem_0}
%     \begin{aligned}
%         \mathcal{D}
%         =
%         (\tensorii{\sigma}{}_{\cell} - \frac{\partial \mecPotential{}_{\bodyLag{}}}{\partial \dot{\tensorii{\varepsilon}}{}_{\cell}}) : \dot{\tensorii{\varepsilon}}{}_{\cell}
%         -
%         \rho \frac{\partial \mecPotential{}_{\bodyLag{}}}{\partial v_{int}} \dot{v}_{int}
%         \geq
%         0
%         % =
%         % (\tensorii{\sigma}{}_{\cell} - \rho \frac{\partial \mecPotential{}_{\bodyLag{}}}{\partial \tensorii{\varepsilon}{}_{\cell}}) : \dot{\tensorii{\varepsilon}}{}_{\cell}
%         % % -
%         % % \rho (s + \frac{\partial \mecPotential{}_{\bodyLag{}}}{\partial T}) \dot{T}
%         % -
%         % \rho \frac{\partial \mecPotential{}_{\bodyLag{}}}{\partial v_{int}} \dot{v}_{int}
%         % \geq
%         % 0
%     \end{aligned}
% \end{equation}
% %
% %
% %
% en l'absence de dépendence du problème à la température. Dans le cadre de l'hyper-élasticité qui est un processus de transformation réversible, comme évoque Section \ref{sec_model_problem}, l'ensemble $V_{int}$ des variables internes $v_{int}$ est supposé vide, de sorte que l'inégalité \eqref{eq_clausius_duhem_0} revient à l'équation d'égalité \eqref{eq_stress_def}.
% En revanche, pour des comportements dissipatifs de nature élasto-visco-plastique, on introduit un certains nombre de variables internes, qui sont liées à l'expression de l'énergie dissipée et à l'irréversibilité de la transformation.
% Pour des déformations infinitésimales, on suppose la décomposition additive de la déformation
% %
% %
% %
% \begin{equation}
%     \tensorii{\varepsilon}{}_{\cell} = \tensorii{\varepsilon}{}_{\cell}^e + \tensorii{\varepsilon}{}_{\cell}^p
% \end{equation}
% %
% %
% %
% En une partie élastique $\tensorii{\varepsilon}{}_{\cell}^e$ et une partie plastique $\tensorii{\varepsilon}{}_{\cell}^p$.
% En particulier, dans le cadre des matériaux standards généralisés, on suppose l'existence d'un potentiel également décomposable en une partie élastique et en une partie plastique tel que
% %
% %
% %
% \begin{equation}
%     \label{eq_plast_2}
%     \begin{aligned}
%         \mecPotential{}_{\bodyLag{}} = \mecPotential{}_{\bodyLag{}}^e(\tensorii{\varepsilon}{}_{\cell}^e)
%         +
%         \mecPotential{}_{\bodyLag{}}^p(
%             % \tensorii{\varepsilon}{}_{\cell}^p
%             % ,
%             v_{int})
%     \end{aligned}
% \end{equation}
% %
% %
% %
% Comme évoque Section \ref{sec_model_problem}, le potentiel d'énergie libre de Helmoltz $\mecPotential{}_{\bodyLag{}}$ dépend éventuellement d'un ensemble de variables internes $v_{int}$ dans $V_{int}$, qui a été supposé vide jusque là.
% Dans le cadre d'un comportement élasto-visco-plastique, on introduit au moins une variable interne, de manière à assurer la positivité de l'énergie dissipée. Par injection de \eqref{eq_plast_2} dans \eqref{eq_clausius_duhem_0}, il vient que le tenseur des contraintes $\tensorii{\sigma}{}_{\cell}$ est la la force duale assosciées aux défromations élastiques $\tensorii{\varepsilon}{}_{\cell}^e$. On définit également les forces thermodynamiques $V_{\cell}$ duales des variables internes $v_{int}$ telles que
% %
% %
% %
% \begin{equation}
%     \label{eq_plast_1}
%     \begin{aligned}
%         \mathcal{D}
%         =
%         \tensorii{\sigma}{}_{\cell} : \dot{\tensorii{\varepsilon}}{}_{\cell}^p
%         -
%         \rho \frac{\partial \mecPotential{}_{\bodyLag{}}}{\partial v_{int}} \dot{v}_{int}
%         =
%         \begin{Bmatrix}
%             \tensorii{\sigma}{}_{\cell}
%             \\
%             % \tensori{V}{}_{\cell}
%             V_{\cell}
%         \end{Bmatrix}
%         \cdot
%         \begin{Bmatrix}
%             \dot{\tensorii{\varepsilon}}{}_{\cell}^p
%             \\
%             \dot{v}_{int}
%         \end{Bmatrix}
%         \geq
%         0
%         &&
%         \text{with}
%         &&
%         V_{\cell} = - \rho \frac{\partial \mecPotential{}_{\bodyLag{}}}{\partial v_{int}}
%     \end{aligned}
% \end{equation}
% %
% %
% %
% Par ailleurs, le cadre des matériaux standards généralisés stipule l'existence d'un convex potential $\phi$ containing the origin, together with a threshold function $f$, that define the evolution of the generalized strains such that
% %
% %
% %
% \begin{equation}
%     \label{eq_plast_1}
%     \begin{aligned}
%         \dot{v_{int}} = \frac{\partial \phi}{\partial f} \frac{\partial f}{\partial V_T}
%     \end{aligned}
% \end{equation}

% En particulier, on introduit l'ensemble des variables internes $V_{int} = \{ \tensorii{\varepsilon}{}_{\cell}^p, p \}$ avec $p$ la déformation plastique cumulée, et le potentiel plastique 
% %
% %
% %
% \begin{equation}
%     \mecPotential{}_{\bodyLag{}}^p(
%         % \tensorii{\varepsilon}{}_{\cell}^p
%         % ,
%         p
%     )
%     % =
%     % \frac{K}{2} \tensorii{\varepsilon}{}_{\cell}^p : \tensorii{\varepsilon}{}_{\cell}^p + \frac{K}{2} p^2
%     =
%     \sigma_0 p + \frac{1}{2} H p^2 + (\sigma_{\infty} - \sigma_0)(p - \frac{1 - e^{-\delta p}}{\delta})
% \end{equation}
% %
% %
% %
% \begin{equation}
%     q
%     =
%     \sigma_0 + H p + (\sigma_{\infty} - \sigma_0)(1 - e^{-\delta p})
% \end{equation}
% %
% %
% %
% où $K$ est le module d'écrouissage cinématique, et $H$ le module d'écrouissage isotrope. Les forces thermodynamiques assosciées aux variables internes $\tensorii{\varepsilon}{}_{\cell}^p$ et $p$ sont respectivement $K \tensorii{\varepsilon}{}_{\cell}^p$ et $Hp$.
% %
% %
% %
% \begin{equation}
%     % f(\tensorii{\sigma}{}_{\cell}^p, q) = \sqrt{\frac{3}{2}} \rVert \text{dev} (\tensorii{\sigma} - K \tensorii{\varepsilon}{}^p) \lVert - \sigma_0 - H p
%     f(\tensorii{\sigma}{}_{\cell}^p, q) = \sqrt{\frac{3}{2}} \rVert \text{dev} (\tensorii{\sigma} \lVert - p
% \end{equation}