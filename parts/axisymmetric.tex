In the following section, we devise a Hybrid High order method for an axisymmetric framework. The cartesian space is expressed in cylindrical coordinates and a point $\tensori{x} \in \bodyLag$ has coordinates $\tensori{x} = (r, z, \theta)$ where $r$ denotes the radial component, $z$ the ordonal one, and $\theta$ is the angular componant describing a revolution around the axis $r = 0$. By cylindrical symmetry, the angular displacement $\tensoro{u}{}_{\theta}$ is supposed to be zero, and both components $u_r$ and $u_z$ do not depend on the angular coordinate $\theta$.
Adopting notations introduced in section ABOVE, let $\cell$ an open subset of $\bodyLag \subset \mathbb{R}^2$ in the $(r,z)$ plane with cell displacement $\tensori{u}{}_{\cell} \in \displacementSpaceCell$ and boundary displacement $\tensori{u}{}_{\dCell} \in \displacementSpaceDCell$. The partial derivatives of $\tensori{u}{}_{\cell}$ with respect to the cylindrical coordinates are given by :
%
%
%
\begin{equation}
    \begin{aligned}
        \forall i, j \in \{ r,z \}, \tensoro{u}{}_{\cell i,j} = \frac{\partial u_{\cell i}}{\partial j} && \text{and} && \tensoro{u}{}_{\cell \theta, \theta} = \frac{u_{\cell r}}{r}
    \end{aligned}
\end{equation}
%
%
%
In order to express a Hybrid High Order method in such a framework and owing to the assumtptions on the displacement and its gradient, the definition of the reconstructed gradient \eqref{eq_grad} needs be modified accordingly, and the angular componenent $\tensoro{G}{}_{\cell \theta \theta}$ does not defines by the same equation as those in the other directions. In particular, for any displacement pair $(\tensori{v}{}_{\cell}, \tensori{v}{}_{\dCell}) \in \displacementSpaceCell \times \displacementSpaceDCell$, the componenent $\tensoro{G}{}_{\cell \theta \theta}(\tensoro{v}{}_{\cell r}, \tensoro{v}{}_{\dCell r})$ solves
% Dans le contexte axisymmétrique, on suppose $u_{\theta} = v_{\theta} = 0$, de sorte que la composante $G_{\theta \theta}$ du gradient dans $\mathbb{P}^{k}(T, \mathbb{R})$ s'exprime, $\forall \tau_{\theta \theta} \in \mathbb{P}^{k}(T, \mathbb{R})$:
%
%
%
\begin{equation}
    \label{axi_symmetric_gradient_theta}
    \begin{aligned}
        \int_{\cell} 2 \pi r \tensoro{G}{}_{\cell \theta \theta}(\tensoro{v}{}_{\cell r}, \tensoro{v}{}_{\dCell r}) \tensoro{\tau}{}_{\cell \theta \theta}
        =
        \int_{\cell} 2 \pi r \frac{\tensoro{u}{}_{\cell r}}{r} \tensoro{\tau}{}_{\cell \theta \theta}
        =
        \int_{\cell} 2 \pi \tensoro{u}{}_{\cell r} \tensoro{\tau}{}_{\cell \theta \theta}
        &&
        \forall \tensorii{\tau}{}_{\cell} \in \stressSpaceCell
    \end{aligned}
\end{equation}
%
%
%
whereas in the radial and ordonal directions, \textit{i.e.} $\forall i, j \in \{ r,z \}$, the expression given in \eqref{eq_grad} is retrieved, and the component $G_{\cell ij}(\tensoro{v}{}_{\cell i}, \tensoro{v}{}_{\dCell i})$ solves :
%
%
%
\begin{equation}
    \label{axi_symmetric_gradient_rz}
    \begin{aligned}
    \int_{\cell} 2 \pi r G_{\cell ij}(\tensoro{v}{}_{\cell i}, \tensoro{v}{}_{\dCell i}) \tau_{\cell ij} =
    \int_{\cell} 2 \pi r \frac{\partial \tensoro{u}{}_{\cell i}}{\partial j} \tau_{ij} -
    \int_{\dCell} 2 \pi r (u_{\dCell i} - u_{\cell i} \vert_{\dCell}) \tau_{\cell ij} \vert_{\dCell} n_{j}
    &&
    \forall \tensorii{\tau}{}_{\cell} \in \stressSpaceCell
    \end{aligned}
\end{equation}
%
%
%
As for the reconstructed gradient, the higher order potential term needed to define the HHO jump function nedds also be reconsidered such that $\forall \tensori{w}{}_{\cell} \in \mathbb{P}^{k + 1}(T, \mathbb{R}^2)$, the radial component $w_{\cell r}$ solves
% En particulier, dans le contexte axisymmétrique, $\forall w_r \in \mathbb{P}^{k + 1}(T, \mathbb{R})$, la composante $D_{\cell r}$ dans $\mathbb{P}^{k + 1}(T, \mathbb{R})$ pour le déplacement reconstruit résoud :
%
%
%
\begin{equation}
    \label{axi_symmetric_potential_r}
    \begin{aligned}
        \int_{\cell} 2 \pi r (\sum_{i \in \{ r,z \}} \frac{\partial D_{\cell r}}{\partial i} \frac{\partial w_{\cell r}}{\partial i} + \frac{D_{\cell r}}{r} \frac{w_{\cell r}}{r})
        = &
        \int_{\cell} 2 \pi r (\sum_{i \in \{ r,z \}} \frac{\partial u_{\cell r}}{\partial i} \frac{\partial w_{\cell r}}{\partial i} + \frac{u_{\cell r}}{r} \frac{w_{\cell r}}{r})
        % &&
        % \forall \tensori{w}{}_{\cell} \in \mathbb{P}^{k + 1}(T, \mathbb{R}^2)
        \\
        &
        +
        \int_{\dCell} 2 \pi r \sum_{i \in \{ r,z \}} (u_{\dCell r} - u_{\cell r} \vert_{\dCell}) \frac{\partial w_{\cell r}}{\partial i} \vert_{\dCell} n_{i}
    \end{aligned}
\end{equation}
%
%
%
and the ordonal component $D_{\cell z}$ solves :
%
%
%
\begin{subequations}
    \label{axi_symmetric_potential_z}
        \begin{alignat}{3}
            \int_{\cell} 2 \pi r \sum_{i \in \{ r,z \}}
            \frac{\partial D_{\cell z}}{\partial i} \frac{\partial w_{\cell z}}{\partial i}
            = &
            \int_{\cell} 2 \pi r \sum_{i \in \{ r,z \}} \frac{\partial u_{\cell z}}{\partial i} \frac{\partial w_{\cell z}}{\partial i}
            -
            \int_{\dCell} 2 \pi r \sum_{i \in \{ r,z \}} (u_{\dCell z} - u_{\cell z} \vert_{\dCell})
            \frac{\partial w_{\cell z}}{\partial i} \vert_{\dCell} n_{i}
            % &&
            % \forall \tensori{w}{}_{\cell} \in \mathbb{P}^{k + 1}(T, \mathbb{R}^2)
            % \\
            % &
            % -
            % \int_{\dCell} 2 \pi r \sum_{i \in \{ r,z \}} (u_{\dCell z} - u_{\cell z} \vert_{\dCell})
            \\
            \int_{\cell} 2 \pi r D_{\cell z} = & \int_{\cell} 2 \pi r u_{\cell z}
        \end{alignat}
\end{subequations}
%
%
%
In particular, one notices that the mean value condition is not needed on the radial componenent of the higher order displacement since the left hand side of the system described by \eqref{axi_symmetric_potential_r} depends direclty on the displacement unknown and not only on itrs gradient as in \eqref{axi_symmetric_potential_z}.
%
% 
%

Moreover, since in cylindrical coordinates, all integrals depend on the radial componenent $r$, there is a singularity at $r = 0$ for boundary integrals on faces located on the symmetry axis, and from a geometrical standpoint, these faces lose a dimension; a face that is not located on the symmetry axis behaves like a shell by revolution of the $(r,z)$ plane,
whereas one attached to the axis reduces to a beam that is only allowed to move and morph in the $z$ direction.
On a more algebraic note, the problem as such is ill-defined, since building the jump function involves inverting a mass matrix in $\discreteDisplacementSpaceDCell$ to define the projector $\Pi_{\dCell}^k$.
Therefore, a face on the axis is swelled by a small radius $\varrho$ such that it becomes a cylinder with same dimensions as the others (see Figure \ref{fig_axi})
%
%
%
\begin{figure}[H]
    \centering
    \includegraphics[width=10.cm]{img/sketch_axi.png}
    \caption{schematic representation of the model problem}
    \label{fig_axi}
\end{figure}