\section{A HHO method for the axi-symmetric framework}

In the following section, we devise a Hybrid High order method for an axi-symmetric framework. In such a framework, owing to geometrical assumptions on the displacement and its gradient, the definition of the reconstructed gradient \eqref{eq_grad} and of that of the higher order displacement \eqref{eq_potential} needs be modified accordingly. Details about the definitions of these ingredients can be found in Section \ref{sec_appendix}.
Moreover, owing to 

\paragraph{Axi-symmetric framework}

The cartesian space is expressed in cylindrical coordinates and a point $\tensori{X} \in \bodyLag$ has coordinates $\tensori{X} = (r, z, \theta)$ where $r$ denotes the radial component, $z$ the ordinate one, and $\theta$ is the angular component describing a revolution around the axis $r = 0$. By cylindrical symmetry, the angular displacement $\tensoro{u}{}_{\theta}$ is supposed to be zero, and both components $u_r$ and $u_z$ do not depend on the angular coordinate $\theta$.

\paragraph{Cell discplacement gradient}

% Adopting notations introduced in Section \ref{sec_composite_demo}, let $\cell$ an open subset of $\bodyLag \subset \mathbb{R}^2$ in the $(r,z)$ plane with cell displacement $\tensori{u}{}_{\cell} \in \displacementSpaceCell$ and boundary displacement $\tensori{u}{}_{\dCell} \in \displacementSpaceDCell$.
The partial derivatives of $\tensori{u}{}_{\cell}$ with respect to the cylindrical coordinates are given by
%
%
%
\begin{equation}
    \begin{aligned}
        \forall i, j \in \{ r,z \}, \tensoro{u}{}_{\cell i,j} = \frac{\partial u_{\cell i}}{\partial j} && \text{and} && \tensoro{u}{}_{\cell \theta, \theta} = \frac{u_{\cell r}}{r}
    \end{aligned}
\end{equation}
%
%
% %

% \paragraph{Reconstructed gradient}

% % $\tensoro{G}{}_{\cell \theta \theta}$ does not define by the same equation as those in the other directions. In particular,
% For any displacement pair $(\tensori{v}{}_{\cell}^l, \tensori{v}{}_{\dCell}^k) \in \discreteDisplacementSpaceCell{} \times \discreteDisplacementSpaceDCell{}$, the component $\tensoro{G}{}_{\cell \theta \theta}(\tensoro{v}{}_{\cell r}, \tensoro{v}{}_{\dCell r})$ solves
% %
% %
% %
% \begin{equation}
%     \label{axi_symmetric_gradient_theta}
%     \begin{aligned}
%         \int_{\cell} 2 \pi r \tensoro{G}{}_{\cell \theta \theta}(\tensoro{v}{}_{\cell r}, \tensoro{v}{}_{\dCell r}) \tensoro{\tau}{}_{\cell \theta \theta}
%         =
%         \int_{\cell} 2 \pi r \frac{\tensoro{u}{}_{\cell r}}{r} \tensoro{\tau}{}_{\cell \theta \theta}
%         =
%         \int_{\cell} 2 \pi \tensoro{u}{}_{\cell r} \tensoro{\tau}{}_{\cell \theta \theta}
%         &&
%         \forall \tensorii{\tau}{}_{\cell} \in \stressSpaceCell
%     \end{aligned}
% \end{equation}
% %
% %
% %
% In the radial and ordonal directions, \textit{i.e.} $\forall i, j \in \{ r,z \}$, the expression given in \eqref{eq_grad} is retrieved, and the component $G_{\cell ij}(\tensoro{v}{}_{\cell i}, \tensoro{v}{}_{\dCell i})$ solves
% %
% %
% %
% \begin{equation}
%     \label{axi_symmetric_gradient_rz}
%     \begin{aligned}
%     \int_{\cell} 2 \pi r G_{\cell ij}(\tensoro{v}{}_{\cell i}, \tensoro{v}{}_{\dCell i}) \tau_{\cell ij} =
%     \int_{\cell} 2 \pi r \frac{\partial \tensoro{u}{}_{\cell i}}{\partial j} \tau_{ij} -
%     \int_{\dCell} 2 \pi r (u_{\dCell i} - u_{\cell i} \vert_{\dCell}) \tau_{\cell ij} \vert_{\dCell} n_{j}
%     &&
%     \forall \tensorii{\tau}{}_{\cell} \in \stressSpaceCell
%     \end{aligned}
% \end{equation}
% %
% %
% %

% \paragraph{Reconstructed higher order displacement}

% For any $\tensori{d}{}_{\cell}^{k + 1} \in \discretePotentialSpaceCell$, the radial component $w^{k+1}_{\cell r}$ solves
% %
% %
% %
% \begin{equation}
%     \label{axi_symmetric_potential_r}
%     \begin{aligned}
%         \int_{\cell} 2 \pi r (\sum_{i \in \{ r,z \}} \frac{\partial w^{k+1}_{\cell r}}{\partial i} \frac{\partial d^{k+1}_{\cell r}}{\partial i} + \frac{w^{k+1}_{\cell r}}{r} \frac{d^{k+1}_{\cell r}}{r})
%         = &
%         \int_{\cell} 2 \pi r (\sum_{i \in \{ r,z \}} \frac{\partial u_{\cell r}}{\partial i} \frac{\partial d^{k+1}_{\cell r}}{\partial i} + \frac{u_{\cell r}}{r} \frac{d^{k+1}_{\cell r}}{r})
%         % &&
%         % \forall \tensori{w}{}_{\cell} \in \mathbb{P}^{k + 1}(T, \mathbb{R}^2)
%         \\
%         &
%         +
%         \int_{\dCell} 2 \pi r \sum_{i \in \{ r,z \}} (u_{\dCell r} - u_{\cell r} \vert_{\dCell}) \frac{\partial d^{k+1}_{\cell r}}{\partial i} \vert_{\dCell} n_{i}
%     \end{aligned}
% \end{equation}
% %
% %
% %
% where the mean value condition is not needed on the radial component of the higher order displacement since the left hand side of the system described by \eqref{axi_symmetric_potential_r} depends directly on the displacement unknown and not only on its gradient as in \eqref{axi_symmetric_potential_z}.
% The ordinate component $w^{k+1}_{\cell z}$ solves :
% %
% %
% %
% \begin{subequations}
%     \label{axi_symmetric_potential_z}
%         \begin{alignat}{3}
%             \int_{\cell} 2 \pi r \sum_{i \in \{ r,z \}}
%             \frac{\partial w^{k+1}_{\cell z}}{\partial i} \frac{\partial d^{k+1}_{\cell z}}{\partial i}
%             = &
%             \int_{\cell} 2 \pi r \sum_{i \in \{ r,z \}} \frac{\partial u_{\cell z}}{\partial i} \frac{\partial d^{k+1}_{\cell z}}{\partial i}
%             -
%             \int_{\dCell} 2 \pi r \sum_{i \in \{ r,z \}} (u_{\dCell z} - u_{\cell z} \vert_{\dCell})
%             \frac{\partial d^{k+1}_{\cell z}}{\partial i} \vert_{\dCell} n_{i}
%             % &&
%             % \forall \tensori{w}{}_{\cell} \in \mathbb{P}^{k + 1}(T, \mathbb{R}^2)
%             % \\
%             % &
%             % -
%             % \int_{\dCell} 2 \pi r \sum_{i \in \{ r,z \}} (u_{\dCell z} - u_{\cell z} \vert_{\dCell})
%             \\
%             \int_{\cell} 2 \pi r w^{k+1}_{\cell z} = & \int_{\cell} 2 \pi r u_{\cell z}
%         \end{alignat}
% \end{subequations}
%
% 
%

% Quand on dsicrétsie, cette face là n'est pas exitsante et donc supprimer la face linéaire

\paragraph{Axis faces treatment}

Since in cylindrical coordinates, all integrals depend on the radial component $r$, boundary integrals vanish at $r = 0$ on the symmetry axis.
Therefore, the reconstructed gradient (and the stabilization) do not depend on a closed surface wrapping a $\cell$ located on the symmetry axis.
However, this feature is necessary to prove the robustness of the HHO method to volumetric locking (see Section \ref{sec_appendix}).
Therefore, we exclude the symmetry axis from the domain, by considering infinitely thin cylindrical faces of radius $\varrho > 0$ surrounding it (see Figure \ref{fig_axi}).


% On the symmetry axis

% Moreover, since in cylindrical coordinates, all integrals depend on the radial component $r$, there is a singularity at $r = 0$ for boundary integrals on faces located on the symmetry axis, and from a geometrical standpoint, these faces lose a dimension; a face that is not located on the symmetry axis behaves like a shell by revolution of the $(r,z)$ plane,
% whereas one attached to the axis reduces to a beam that is only allowed to move and morph in the $z$ direction.
% On a more algebraic note, the problem as such is ill-defined, since building the jump function involves inverting a mass matrix in $\discreteDisplacementSpaceDCell$ to define the projector $\Pi_{\dCell}^k$.
% Therefore, a face on the axis is swelled by a small radius $\varrho $ such that it becomes a cylinder with same dimensions as the others (see Figure \ref{fig_axi})
%
%
%
\begin{figure}[H]
    \centering
    \includegraphics[width=10.cm]{img/sketch_axi.png}
    \caption{schematic representation of the modeling of a face located on the symmetry axis}
    \label{fig_axi}
\end{figure}