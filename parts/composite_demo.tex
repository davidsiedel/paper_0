In order to introduce the discontinuous setting in which lies the Hybrid High Order method, let consider the body $\bodyLag$ to be made out of some material defined by a mechanical potential $\mecPotential{}_{\bodyLag}$. The aim of this section consists in devising the expression of the mechanical energy deriving from the HHO formulation of the mechanical model problem described in Section \ref{sec_model_problem}. Following the idea of a composite medium as introduced in \ref{sec_model_problem}, let $\cell$ an arbitrary open subset in $\bodyLag$, split into two distinct media; an open bulk medium $\Bulk \subset \cell$ with boundary $\dBulk \subset \mathbb{R}^{d-1}$, and an open interface medium $\Crown \subset \cell$ between the bulk $\Bulk$ and the boundary $\dCell$, with boundary $\dCrown = \dBulk \cup \dCell$ and of some width $\ell > 0$ that is supposed to be small compared to $h_{\cell} = \max_{(\tensori{x}{}_a, \tensori{x}{}_b) \in \cell} \lVert \tensori{x}{}_a - \tensori{x}{}_b \rVert$ the diameter of $\cell$.
%
% 
% 
\begin{figure}[H]
    \centering
    \includegraphics[width=12.cm]{img/hu_washizu.png}
    \caption{schematic representation of the model problem}
    \label{fig_02}
\end{figure}
%
% 
% 
Let the boundary $\dCell$ move with a boundary displacement field $\tensori{u}{}_{\dCell} \in V_{}(\dCell)$, where $V_{}(\dCell)$ denotes the space of kinematically admissible boundary displacements. The displacement at the boundary $\dCell$ results from the interactions of $\dCell$ with neighbouring media, \textit{i.e.} from the action of $\bodyLag \backslash \cell$ onto $\dCell$ or from some boundary condition, but let assume that the bulk $\Bulk$ is \textit{a priori} not influenced by the movments of $\dCell$; that is, it supposedly only morhphs through the action of the body load $\loadLag$, producing a displacement gradient $\tensorii{G}{}_{\Bulk} \in \gradSpaceBulk$ and a stress $\tensorii{P}{}_\Bulk \in \stressSpaceBulk$ under the mechanical potential $\mecPotential{}_{\bodyLag}$, that are free from the influence of $\dCell{}$ onto $\Bulk{}$. Hence, the bulk $\Bulk{}$ is free to move away up to a rigid body motion from the rest of the body $\bodyLag{}$ at no energetical cost, which violates the conservation laws. Therefore, in order to ensure continuity of the displacement between $\Bulk$ and $\dCell$, let $\Crown$ act as a patch, such that $\tensori{u}_{\Crown} \in \displacementSpaceCrown$ the displacement in $\Crown$ links that of $\Bulk$ to that of $\dCell$ :
%
% 
% 
\begin{subequations}
    \label{eq_conformity}
        \begin{alignat}{2}
        \tensori{u}{}_{\Crown} \vert_{\dBulk} & = \tensori{u}{}_{\Bulk} \vert_{\dBulk}
        \label{eq_conformity:eq1}
        \\
        \tensori{u}{}_{\Crown} \vert_{\dCell} & = \tensori{u}{}_{\dCell}
        \label{eq_conformity:eq2}
    \end{alignat}
\end{subequations}
% 
% 
%
In order to bind the behaviour of $\Bulk{}$ to that of its neighbourhood through $\dCell{}$, let endow the interface $\Crown$ with a mechanical potential  $\mecPotential{}_{\Crown{}}$ such that it behaves like a linear elastic material of Young modulus $\beta (\ell / h_{\cell})$ and a zero Poisson ratio :
%
% 
% 
\begin{equation}
    \label{eq_0009}
        \mecPotential{}_{\Crown} = \frac{1}{2} \beta \frac{\ell}{h_{\cell}} \nabla \tensori{u}{}_{\Crown} : \nabla \tensori{u}{}_{\Crown}
\end{equation}
%
% 
% 
where the dimensionless ratio $\ell / h_{\cell}$ balances the accumulated energy with the size of the domain $\cell$. Let then $\tensorii{G}{}_{\Crown{}} \in \gradSpaceCrown$ and $\tensorii{P}{}_{\Crown{}} \in \stressSpaceCrown$ the displacement gradient and stress in $\Crown{}$. Under such assumptions and using \eqref{eq_hu_washizu_composite}, the Hu–Washizu functional over $\cell$ writes as :
%
% 
% 
\begin{equation}
\label{eq_hu_washizu_split}
    J_{\cell}^{HW}
    % (\tensori{u}{}_{\cell}, \tensorii{G}{}_{\cell}, \tensorii{P}{}_{\cell})
    =
    \int_{\Bulk} \mecPotential_{\bodyLag{}} + (\nabla_X \tensori{u}{}_{\Bulk} - \tensorii{G}{}_{\Bulk}) : \tensorii{P}{}_{\Bulk}
    +
    \int_{\Crown} \mecPotential_{\Crown{}} + (\nabla_X \tensori{u}{}_{\Crown} - \tensorii{G}{}_{\Crown}) : \tensorii{P}{}_{\Crown}
    -
    \int_{\Bulk} \loadLag \cdot \tensori{u}{}_{\Bulk}
    -
    \int_{\Crown} \loadLag \cdot \tensori{u}{}_{\Crown}
    -
    \int_{\neumannCell} \neumannCellLoad \cdot \tensori{u}{}_{\dCell}
\end{equation}
% 
% 
%
%

\subsection{Interface description}
\label{sec_interface_description}
% 
% 
% 
Let $\tensori{\Xi}{}_{\cell}$ the homotethy of ratio $(1 + \alpha \ell)$ and center $\tensori{x}{}_{\cell}$ the centroid of $\cell$, with $-1 / \ell < \alpha < 0$ such that $\Bulk$ (respectively $\dBulk$) is the image of $\cell$ (respectively $\dCell$) by $\tensori{\Xi}{}_{\cell}$. Since $\dBulk$ is an homotethy of $\dCell$, any point $\tensori{x}{}_{\dCell} \in \dCell$ and $\tensori{x}{}_{\dBulk} = \tensori{\Xi}{}_{\cell}(\tensori{x}{}_{\dCell}) \in \dBulk$ share the same unit outward normal $\tensori{n}{}$. Assuming the interface $\Crown$ to be thin compared to the cell volume $\cell$, let linearize the displacement in $\Crown$ with respect to $\tensori{n}$, such that :
%
% 
% 
\begin{equation}
    \label{eq_crown_displacement}
    \tensori{u}{}_{\Crown} (\tensori{x})
    =
    \frac{\tensori{u}{}_{\dCell}(\tensori{x})
    -
    \tensori{u}{}_{\Bulk} \vert_{\dBulk} (\tensori{x})}{\ell} \otimes \tensori{n} \cdot (\tensori{x} - \tensori{m}{}_{\dBulk})
    +
    \tensori{u}{}_{\Bulk} \vert_{\dBulk}
\end{equation}
% 
% 
%
where $\tensori{m}{}_{\dBulk} = \min_{\tensori{x}{}_{\dBulk}} \lVert \tensori{x}{}_{\dBulk} - \tensori{x} \rVert$. Furthermore, let assume that the interface is thin enough such that $\tensorii{P}{}_{\Crown}$ is constant along the $\tensori{n}{}$ direction in $\Crown{}$. By continuity of the traction force across $\dBulk$, the following equality holds true :
%
% 
% 
\begin{equation}
    \label{eq_continuity_traction_force}
    \begin{aligned}
        (\tensorii{P}{}_{\Crown} - \tensorii{P}{}_{\Bulk} \vert_{\dBulk{}}) \cdot \tensori{n}{} =  0
        &&
        \text{in}
        &&
        \Crown{}
    \end{aligned}
\end{equation}
%
% 
% 
Using \eqref{eq_continuity_traction_force} and \eqref{eq_crown_displacement}, one can write the internale contribution in $\Crown{}$ as a term depending on the bulk and boundary displacement :
%
% 
% 
\begin{equation}
    \label{eq22}
    \begin{aligned}
        J_{\Crown{}}^{HW_\text{int}}
        := &
        \int_{\Crown{}} \mecPotential{}_{\Crown} + (\nabla \tensori{u}{}_{\Crown} - \tensorii{G}{}_{\Crown}) : \tensorii{P}{}_{\Crown}
        % -
        % \int_{\Crown{}} \loadLag \cdot \tensori{u}{}_{\Crown}
        \\
        = &
        (1 - \frac{\alpha}{2} \ell)
        \int_{\dBulk{}} \frac{\beta}{2 h_{\cell}} \lVert \tensori{u}{}_{\dCell{}} - \tensori{u}{}_{\Bulk{}} \vert_{\dBulk{}} \rVert^2
        +
        (1 - \frac{\alpha}{2} \ell)
        \int_{\dBulk} (\tensori{u}{}_{\dCell{}} - \tensori{u}{}_{\Bulk{}} \vert_{\dBulk{}}) \otimes \tensori{n}{} : \tensorii{P}{}_{\Bulk{}} \vert_{\dBulk{}}
        -
        \int_{\Crown{}} \tensorii{G}{}_{\Crown{}} : \tensorii{P}{}_{\Crown{}}
        % -
        % \int_{\Crown{}} \loadLag \cdot \tensori{u}{}_{\Crown}
    \end{aligned}
\end{equation}
%
% 
% ------------------------------------------------------- DEVELOPMENT
\textcolor{blue}{
%
\begin{development}[Interafce simplification]
%
Let $C_\Crown = \{ v \in L^2(\Crown) \ \vert \ v \cdot \tensori{n} = \text{cste} \}$ the set of $L^2$-functions which are constant along the normal axis in $\Crown$. For any function in $C_\Crown$, the following equality holds true:
%
% 
% 
\begin{equation}
    \label{eq_virtual_works0}
        \int_{\Crown} v \ dV
        =
        \int_{\dBulk{}} \int_{\epsilon = 0}^{\ell} v (1 - \alpha \epsilon) \ dS d \epsilon
        =
        \ell (1 - \frac{\alpha}{2} \ell) \int_{\dBulk{}} v \ dS
\end{equation}
%
% 
% 
Noticing that $\nabla \tensori{u}{}_{\Crown} \in C_\Crown$, one has :
%
% 
% 
\begin{equation}
    \begin{aligned}
        \int_{\Crown{}} \mecPotential{}_{\Crown}
        % = &
        % \int_{\Crown{}} \frac{1}{2} \beta \frac{\ell}{h_{\cell}} \nabla \tensori{u}{}_{\Crown} : \nabla \tensori{u}{}_{\Crown}
        % \\
        = & 
        \ell (1 - \frac{\alpha}{2} \ell)
        \int_{\dBulk{}} \frac{1}{2} \beta \frac{\ell}{h_{\cell}} \nabla \tensori{u}{}_{\Crown} : \nabla \tensori{u}{}_{\Crown}
        \\
        = & 
        \ell (1 - \frac{\alpha}{2} \ell)
        \int_{\dBulk{}} \frac{\beta}{2 \ell h_{\cell}} (\tensori{u}{}_{\dCell} - \tensori{u}{}_{\Bulk} \vert_{\dBulk}) \otimes
        \tensori{n} : (\tensori{u}{}_{\dCell} - \tensori{u}{}_{\Bulk} \vert_{\dBulk}) \otimes
        \tensori{n}
        \\
        = & 
        \ell (1 - \frac{\alpha}{2} \ell)
        \int_{\dBulk{}} \frac{\beta}{2 \ell h_{\cell}} \sum_{i,j} (\tensoro{u}{}_{\dCell}{}_{i}- \tensoro{u}{}_{\Bulk}{}_{i} \vert_{\dBulk}){}^2
        \tensoro{n}_{j}{}^2
        \\
        = & 
        \ell (1 - \frac{\alpha}{2} \ell)
        \int_{\dBulk{}} \frac{\beta}{2 \ell h_{\cell}} \sum_{j} \tensoro{n}_{j}{}^2 \sum_{i} (\tensoro{u}{}_{\dCell}{}_{i}- \tensoro{u}{}_{\Bulk}{}_{i} \vert_{\dBulk}){}^2
        \\
        = & 
        \ell (1 - \frac{\alpha}{2} \ell)
        \int_{\dBulk{}} \frac{\beta}{2 \ell h_{\cell}} \sum_{i} (\tensoro{u}{}_{\dCell}{}_{i}- \tensoro{u}{}_{\Bulk}{}_{i} \vert_{\dBulk}){}^2
        \\
        = & 
        \ell (1 - \frac{\alpha}{2} \ell)
        \int_{\dBulk{}} \frac{\beta}{2 \ell h_{\cell}} \lVert \tensori{u}{}_{\dCell} - \tensori{u}{}_{\Bulk}{} \vert_{\dBulk} \lVert {}^2
        \\
        = & 
        (1 - \frac{\alpha}{2} \ell)
        \int_{\dBulk{}} \frac{\beta}{2 h_{\cell}} \lVert \tensori{u}{}_{\dCell} - \tensori{u}{}_{\Bulk}{} \vert_{\dBulk} \lVert {}^2
    \end{aligned}
\end{equation}
%
% 
% 
Moreover, for $\tensorii{P}{}_{\Crown}$ in $C_\Crown{}$ :
%
% 
% 
\begin{equation}
    \begin{aligned}
        \int_{\Crown{}} \nabla \tensori{u}{}_{\Crown} : \tensorii{P}{}_{\Crown}
        = &
        \ell (1 - \frac{\alpha}{2} \ell)
        \int_{\dBulk{}} \nabla \tensori{u}{}_{\Crown} : \tensorii{P}{}_{\Crown}
        \\
        = &
        \ell (1 - \frac{\alpha}{2} \ell)
        \int_{\dBulk{}}
        \frac{1}{\ell}
        (\tensori{u}{}_{\dCell} - \tensori{u}{}_{\Bulk}{} \vert_{\dBulk}) \otimes \tensori{n} : \tensorii{P}{}_{\Bulk{}} \vert_{\dBulk{}}
        \\
        = &
        \ell (1 - \frac{\alpha}{2} \ell)
        \int_{\dBulk{}}
        \frac{1}{\ell}
        \sum_{i,j}
        (\tensoro{u}{}_{\dCell}{}_{i} - \tensoro{u}{}_{\Bulk}{}{}_{i} \vert_{\dBulk}) \tensoro{n}{}_{j} \tensoro{P}{}_{\Bulk{}}{}_{ij} \vert_{\dBulk{}}
        \\
        = &
        \ell (1 - \frac{\alpha}{2} \ell)
        \int_{\dBulk{}}
        \frac{1}{\ell}
        (\tensori{u}{}_{\dCell} - \tensori{u}{}_{\Bulk}{} \vert_{\dBulk}) \cdot \tensorii{P}{}_{\Bulk{}} \vert_{\dBulk{}} \cdot \tensori{n}
        \\
        = &
        (1 - \frac{\alpha}{2} \ell)
        \int_{\dBulk{}}
        (\tensori{u}{}_{\dCell} - \tensori{u}{}_{\Bulk}{} \vert_{\dBulk}) \cdot \tensorii{P}{}_{\Bulk{}} \vert_{\dBulk{}} \cdot \tensori{n}
    \end{aligned}
\end{equation}
% 
% 
% 
And Finally :
%
% 
% 
\begin{equation}
    \begin{aligned}
        J_{\Crown{}}
        =
        (1 - \frac{\alpha}{2} \ell)
        \int_{\dBulk{}} \frac{\beta}{2 h_{\cell}} \lVert \tensori{u}{}_{\dCell{}} - \tensori{u}{}_{\Bulk{}} \vert_{\dBulk{}} \rVert^2
        +
        (1 - \frac{\alpha}{2} \ell)
        \int_{\dBulk} (\tensori{u}{}_{\dCell{}} - \tensori{u}{}_{\Bulk{}} \vert_{\dBulk{}}) \cdot \tensorii{P}{}_{\Bulk{}} \vert_{\dBulk{}} \cdot \tensori{n}{}
        -
        \int_{\Crown{}} \tensorii{G}{}_{\Crown{}} : \tensorii{P}{}_{\Crown{}}
    \end{aligned}
\end{equation}
%
\end{development}
}
% ------------------------------------------------------- DEVELOPMENT
% 
%
Injecting \eqref{eq22} in \eqref{eq_hu_washizu_split} :
%
% 
% 
\begin{equation}
    \label{eq_0014}
    \begin{aligned}
        J_{\cell}
        = &
        \int_{\Bulk} \mecPotential{}_{\bodyLag{}} + (\nabla \tensori{u}{}_{\Bulk} - \tensorii{G}{}_{\Bulk}) : \tensorii{P}{}_{\Bulk}
        % \\
        % &
        +
        (1 - \frac{\alpha}{2} \ell)
        % \Biggl(
        \int_{\dBulk{}} (\tensori{u}{}_{\dCell{}} - \tensori{u}{}_{\Bulk} \vert_{\dBulk}) \cdot \tensorii{P}{}_{\Bulk} \vert_{\dBulk} \cdot \tensori{n}{}
        % \\
        % &
        \\
        &
        +
        (1 - \frac{\alpha}{2} \ell)
        \int_{\dBulk{}} \frac{\beta}{2 h_T} \lVert \tensori{u}{}_{\dCell{}} - \tensori{u}{}_{\Bulk} \vert_{\dBulk{}} \rVert^2
        % \Biggr)
        % \\
        % &
        -
        \int_{\Crown{}} \tensorii{G}{}_{\Crown{}} : \tensorii{P}{}_{\Crown{}}
        % \\
        % &
        -
        \int_{\Bulk} \loadLag \cdot \tensori{u}{}_{\Bulk}
        -
        \int_{\Crown{}} \loadLag \cdot \tensori{u}{}_{\Crown{}}
        -
        \int_{\neumannCell{}} \neumannCellLoad{} \cdot \tensori{u}{}_{\dCell{}}
    \end{aligned}
\end{equation}
%
% 
% 
Since $\ell$ is arbitrary, let $\ell \rightarrow 0$,
the interface region vanishes such that $\Crown{} = \emptyset, \Bulk{} = \cell$ and $\dBulk{} = \dCell$, and the expression of the Hu–Washizu functional over the region $\cell$ writes:
% 
% 
%
\begin{equation}
    \label{eq_0015}
    \begin{aligned}
        J_{\cell}
        = &
        \int_{\cell{}} \mecPotential{}_{\bodyLag{}} + (\nabla \tensori{u}{}_{\cell{}} - \tensorii{G}{}_{\cell{}}) : \tensorii{P}{}_{\cell}
        % \\
        % &
        + \int_{\dCell{}} (\tensori{u}{}_{\dCell} - \tensori{u}{}_{\cell} \vert_{\dCell}) \cdot \tensorii{P}{}_{\cell} \vert_{\dCell{}} \cdot \tensori{n}{}
        % \\
        % &
        + \int_{\dCell} \frac{\beta}{2 h_{\cell}} \lVert \tensori{u}{}_{\dCell{}} - \tensori{u}{}_{\cell{}} \vert_{\dCell{}} \rVert^2
        \\
        &
        -
        \int_{\cell} \loadLag{} \cdot \tensori{u}{}_{\cell{}}
        -
        \int_{\neumannCell{}} \neumannCellLoad{} \cdot \tensori{u}{}_{\dCell{}}
    \end{aligned}
\end{equation}
% 
% 
%
Assuming that the displacement is continuous at the boundary $\dCell{}$ such that $\tensori{u}{}_{\dCell{}}$ is the trace of the cell displacement $\tensori{u}{}_{\cell{}}$ on $\dCell{}$ and $\tensori{u}{}_{\dCell{}} - \tensori{u}{}_{\cell{}} \vert_{\dCell{}} = 0$, one recovers the usual expression of the Hu–Washizu integral over the element for the three variables $(\tensori{u}{}_{\cell}, \tensorii{G}{}_{\cell}, \tensorii{P}{}_{\cell})$. However, if one considers that $\tensori{u}{}_{\dCell}$ and $\tensori{u}{}_{\cell}$ are disticnt variables, \textit{i.e.} that the boundary $\dCell$ is able to move from the cell $\cell$ such that the displacement across $\dCell$ is discontinuous, $J_{\cell}$ writes as a function of the four variables $(\tensori{u}{}_{\cell}, \tensori{u}{}_{\dCell}, \tensorii{G}{}_{\cell}, \tensorii{P}{}_{\cell})$.
Such an assumtion relates to the concept of hybridization of the displacement unknown, which is at the foundation of the Hybrid Discontinuous Galerkin method and of the HHO method. Besides, replacing $\tensori{u}{}_{\dCell}$ by $\tensori{u}{}_{\cell'}$ for any neighbourging region $\cell'$ to $\cell$ defines the Discontinuous Galerkin method, where only the core unknown $\tensori{u}{}_{\cell}$ is considered, and the displacement jumps depends on neighbouring regions.
Differentiating $J_{\cell}$ over each of these variables, and introducing the explicit traction force $\tensori{\theta}{}_{\dCell} = \tensorii{P}{}_{\cell} \vert_{\dCell} \cdot \tensori{n}{} + (\beta / h_{\cell}) (\tensori{u}{}_{\dCell} - \tensori{u}{}_{\cell} \vert_{\dCell})$ one obtains the system
% 
% 
%
\begin{subequations}
    \label{eq_0017}
        \begin{alignat}{3}
            \frac{\partial J_{\cell}}{\partial \tensori{u}{}_{\cell}} \delta \tensori{u}{}_{\cell}
            = & \int_{\cell} \tensorii{P}{}_{\cell} : \nabla \delta \tensori{u}{}_{\cell}
            -
            \int_{\cell} \tensori{f}{}_V \cdot \delta \tensori{u}{}_{\cell}
            -
            \int_{\dCell{}} \tensori{\theta}{}_{\dCell} \cdot \delta \tensori{u}{}_{\cell} \vert_{\dCell}
            &&
            \ \ \ \ \ \ \ \ 
            &&
            \forall \delta \tensori{u}{}_{\cell}
            \in \displacementSpaceCell
        \label{eq_0017:eq0}
        \\
            \frac{\partial J_{\cell}}{\partial \tensori{u}{}_{\dCell}} \delta \tensori{u}{}_{\dCell}
            = &
            \int_{\neumannCell} (\tensori{\theta}{}_{\dCell} - \tensori{t}{}_{\neumannCell}) \cdot \delta \tensori{u}{}_{\dCell}
            &&
            \ \ \ \ \ \ \ \ 
            &&
            \forall \delta \tensori{u}{}_{\dCell}
            \in \displacementSpaceDCell
        \label{eq_0017:eq1}
        \\
            \frac{\partial J_{\cell}}{\partial \tensorii{G}{}_{\cell}} \delta \tensorii{G}{}_{\cell}
            = &
            \int_{\cell} (\frac{\partial \mecPotential_{\bodyLag}}{\partial \tensorii{G}{}_{\cell}} - \tensorii{P}{}_{\cell}) : \delta \tensorii{G}{}_{\cell}
            &&
            \ \ \ \ \ \ \ \ 
            &&
            \forall \delta \tensorii{G}{}_{\cell}
            \in \gradSpaceCell
        \label{eq_0017:eq2}
        \\
            \frac{\partial J_{\cell}}{\partial \tensorii{P}{}_{\cell}} \delta \tensorii{P}{}_{\cell}
            = & \int_{\cell} (\nabla \tensori{u}{}_{\cell} - \tensorii{G}{}_{\cell} ) : \delta \tensorii{P}{}_{\cell}
            +
            \int_{\dCell} (\tensori{u}{}_{\dCell} - \tensori{u}{}_{\cell} \vert_{\dCell}) \cdot \delta \tensorii{P}{}_{\cell} \vert_{\dCell} \cdot \tensori{n}{}
            &&
            \ \ \ \ \ \ \ \ 
            &&
            \forall \delta \tensorii{P}{}_{\cell}
            \in \stressSpaceCell
        \label{eq_0017:eq3}
    \end{alignat}
\end{subequations}
% 
% 
%
The set of equation \eqref{eq_0017} fully defines the framework for Discontinuous Galerkin methods.
In particular, \eqref{eq_0017:eq0} is the expression of the principle of virtual works in $\cell$, where the explicit traction force $\tensori{\theta}{}_{\dCell}$ replaces the usual expression $\tensorii{P}{}_{\cell} \cdot \tensori{n}{}$ in the external contribution, and \eqref{eq_0017:eq1} denotes a supplementary equation to the usual continuous problem as seen in \eqref{eq_hu_washizu_derivative}, to account for the continuity of the flux $\tensori{\theta}{}_{\dCell}$ across the cell boundary.
This feature consitutes one of the key assets of non-conformal method; indeed, by defining a richer flux than in the ususal continuous framework, that aslo depends on the displacement jump, one allows for the latter to act as a Lagrange multiplyier in order to fulfil the flux continuity requriement on $\dCell$. The tradeoff for this condition to hold true consists in loosening the displacement continuity condition through the introduction of the displacement jump at the boundary.
% Stability of the problem is then recovered through the interface behaviour that penalizes displacement jumps in a weak sense.
% \eqref{eq_0017:eq2} defines the stress-behaviour law relation, and \eqref{eq_0017:eq3} defines a gradient field reconstruction based on a linear problem, whose second term depends on both a body and a boundary term.
\eqref{eq_0017:eq2} accounts for the constitutive equation in a weak sense, and \eqref{eq_0017:eq3} defines the equation of an enhanced gradient gradient field, that does not reduces to the projection of $\nabla \tensori{u}{}_{\cell}$ onto $\gradSpaceCell$ as in \eqref{eq_hu_washizu_derivative:eq3}, since it is enriched by a boundary component that depends on the displacement jump, which is at the origin of the robustness of non-conformal methods to volumetric locking.
%
%
%
%
% Indeed, defining $\tensori{I}{}(\tensori{v}{})$ the interpolation operator

Instead of seeking the four fields explicitly, by noticing that minimization of \eqref{eq_0017:eq3} defines a linear problem with any displacement pair $(\tensori{v}{}_{\cell}, \tensori{v}{}_{\dCell}) \in \displacementSpaceCell \times \displacementSpaceDCell$ such that there is a unique $\tensorii{G}{}_{\cell}$ minimizing \eqref{eq_0017:eq3}, and that minimization of \eqref{eq_0017:eq2} is linear with the derivative of $\mecPotential_{\bodyLag}$ with respect to $\tensorii{G}{}_{\cell}$, one can eliminate
\eqref{eq_0017:eq2} and \eqref{eq_0017:eq3} from the system, by considering the simplifed functional
% By explicitly eliminating \eqref{eq_0017:eq2} and \eqref{eq_0017:eq3} from the system, \textit{i.e.} by considering the simplifed functional
%
%
%
\begin{equation}
    \label{eq_simple}
    \begin{aligned}
        J_{\cell}
        = &
        \int_{\cell{}} \mecPotential{}_{\bodyLag{}}
        % \\
        % &
        % + \int_{\dCell{}} (\tensori{u}{}_{\dCell} - \tensori{u}{}_{\cell} \vert_{\dCell}) \cdot \tensorii{P}{}_{\cell} \vert_{\dCell{}} \cdot \tensori{n}{}
        % \\
        % &
        + \int_{\dCell} \frac{\beta}{2 h_{\cell}} \lVert \tensori{u}{}_{\dCell{}} - \tensori{u}{}_{\cell{}} \vert_{\dCell{}} \rVert^2
        % \\
        % &
        -
        \int_{\cell} \loadLag{} \cdot \tensori{u}{}_{\cell{}}
        -
        \int_{\neumannCell{}} \neumannCellLoad{} \cdot \tensori{u}{}_{\dCell{}}
    \end{aligned}
\end{equation}
%
%
%
where \eqref{eq_0017:eq3} results in the definition of the reconstructed gradient $\tensorii{G}{}_{\cell}(\tensori{v}{}_{\cell}, \tensori{v}{}_{\dCell})$ associated with any displacement pair $(\tensori{v}{}_{\cell}, \tensori{v}{}_{\dCell}) \in \displacementSpaceCell \times \displacementSpaceDCell$ that solves
% where \eqref{eq_0017:eq2} and \eqref{eq_0017:eq3} are taken to be zero such that the reconstructed gradient $\tensorii{G}{}_{\cell}(\tensori{v}{}_{\cell}, \tensori{v}{}_{\dCell})$ associated with any displacement pair $(\tensori{v}{}_{\cell}, \tensori{v}{}_{\dCell}) \in \displacementSpaceCell \times \displacementSpaceDCell$ explicitly solves
%
%
%
\begin{equation}
    \label{eq_grad}
    \begin{aligned}
        \int_{\cell} \tensorii{G}{}_{\cell}(\tensori{v}{}_{\cell}, \tensori{v}{}_{\dCell}) : \tensorii{\tau}{}_{\cell}
        =
        \int_{\cell}  \nabla \tensori{v}{}_{\cell} : \tensorii{\tau}{}_{\cell}
        +
        \int_{\dCell} (\tensori{v}{}_{\dCell} - \tensori{v}{}_{\cell} \vert_{\dCell}) \cdot \tensorii{\tau}{}_{\cell} \vert_{\dCell} \cdot \tensori{n}{}
        &&
        \forall \tensorii{\tau}{}_{\cell} \in \stressSpaceCell
    \end{aligned}
\end{equation}
%
%
%
and the stress $\tensorii{P}{}_{\cell}(\tensorii{G}{}_{\cell}(\tensori{v}{}_{\cell}, \tensori{v}{}_{\dCell}))$ defines as the projection onto $\gradSpaceCell$ of the derivative of $\mecPotential_{\bodyLag}$ with respect to $\tensorii{G}{}_{\cell}(\tensori{u}{}_{\cell}, \tensori{u}{}_{\dCell})$ for any displacement pair $(\tensori{v}{}_{\cell}, \tensori{v}{}_{\dCell}) \in \displacementSpaceCell \times \displacementSpaceDCell$
%
%
%
\begin{equation}
    \label{eq_stress}
    \begin{aligned}
        \int_{\cell} \tensorii{P}{}_{\cell}(\tensorii{G}{}_{\cell}(\tensori{v}{}_{\cell}, \tensori{v}{}_{\dCell})) : \tensorii{\gamma}{}_{\cell}
        =
        \int_{\cell} \frac{\partial \mecPotential_{\bodyLag}}{\partial \tensorii{G}{}_{\cell}(\tensori{v}{}_{\cell}, \tensori{v}{}_{\dCell})}  : \tensorii{\gamma}{}_{\cell}
        &&
        \forall \tensorii{\gamma}{}_{\cell} \in \gradSpaceCell
    \end{aligned}
\end{equation}
%
%
%
where one notices that the equality holds in a weak sense if $\stressSpaceCell \subset \gradSpaceCell$.
% Indeed, the gradient unknown does not only defines as the projection of the gradient of $\tensori{u}{}_{\cell}$ onto $\gradSpaceCell$ as in \eqref{eq_hu_washizu_derivative:eq3}, as it is enriched by a boundary component that depends on the displacement jump, which is at the origin of the robustness of non-conformal methods to volumetric locking.
%
%
%
\textcolor{blue}{
    \begin{development}[Elliptic projection]
        %
        %
        %
        Let $\discreteDisplacementSpaceCell \subset \displacementSpaceCell$ and $U^\perp(\cell) \subset \displacementSpaceCell$ such that $\displacementSpaceCell = \discreteDisplacementSpaceCell \oplus U^\perp(\cell)$, and set $\tensori{u}{}_{\cell} = \tensori{u}{}_{\cell}^h + \tensori{u}{}_{\cell}^\perp$ with
        $\tensori{u}{}_{\cell}^h \in U^h(\cell)$ and $\tensori{u}{}_{\cell}^\perp \in U^\perp(\cell)$ the orthogonal projections of $\tensori{u}{}_{\cell}$ onto $U^h(\cell)$ and $U^\perp(\cell)$ respectively.
        Let $V^h(\dCell) \subset \displacementSpaceDCell$ and $\tensori{u}{}_{\dCell}^h \in V^h(\dCell)$ the orthogonal projection of $\tensori{u}{}_{\cell}$ onto $V^h(\dCell)$.
        The orthogonal projection of $\tensori{u}{}_{\cell}$ onto $U^h(\cell) \times V^h(\dCell)$ is then the displacement pair $(\tensori{u}{}_{\cell}^h, \tensori{u}{}_{\dCell}^h)$.
        Let $S^h(\cell) = \{ \tensorii{\tau}{} \in \stressSpaceCell \ \ \vert \ \ \nabla \cdot  \tensorii{\tau}{}_{\cell}^h \in U^h(\cell) \ \ \vert \ \  \tensorii{\tau}{}_{\cell}^h \vert_{\dCell} \cdot \tensori{n}{} \in V^h(\dCell) \}$,
        and $\tensorii{G}{}_{\cell}^h \in S^h(\cell)$ the solution of \eqref{eq_grad} in $U^h(\cell) \times V^h(\dCell)$ such that
        %
        %
        %
        \begin{equation}
            \begin{aligned}
                \int_{\cell} \tensorii{G}{}_{\cell}^h(\tensori{u}{}_{\cell}^h, \tensori{u}{}_{\dCell}^h) : \tensorii{\tau}{}_{\cell}^h
                =
                \int_{\cell} \nabla \tensori{u}{}_{\cell}^h : \tensorii{\tau}{}_{\cell}^h
                +
                \int_{\dCell} (\tensori{u}{}_{\dCell}^h - \tensori{u}{}_{\cell}^h \vert_{\dCell}) \cdot \tensorii{\tau}{}_{\cell}^h \vert_{\dCell} \cdot \tensori{n}{}
                &&
                \ \ \ \ \ \ \ \ 
                &&
                \forall \tensorii{\tau}{}_{\cell}^h \in S^h(\cell)
            \end{aligned}
        \end{equation}
        %
        %
        %
        using the fact that $\tensori{u}{}_{\dCell}^h$ is the projection of $\tensori{u}{}_{\cell}$ onto $V^h(\dCell)$ and that $\tensorii{\tau}{} \vert_{\dCell} \cdot \tensori{n}{} \in V^h(\dCell)$:
        %
        %
        %
        \begin{equation}
            \begin{aligned}
                \int_{\cell} \tensorii{G}{}_{\cell}^h(\tensori{u}{}_{\cell}^h, \tensori{u}{}_{\dCell}^h) : \tensorii{\tau}{}_{\cell}^h
                = &
                \int_{\cell} \nabla \tensori{u}{}_{\cell}^h : \tensorii{\tau}{}_{\cell}^h
                +
                \int_{\dCell} (\tensori{u}{}_{\cell} \vert_{\dCell} - \tensori{u}{}_{\cell}^h \vert_{\dCell}) \cdot \tensorii{\tau}{}_{\cell}^h \vert_{\dCell} \cdot \tensori{n}{}
                &&
                \ \ \ \ \ \ \ \ 
                &&
                \forall \tensorii{\tau}{}_{\cell}^h \in S^h(\cell)
                \\
                = &
                \int_{\cell} \nabla \tensori{u}{}_{\cell}^h : \tensorii{\tau}{}_{\cell}^h
                +
                \int_{\dCell} \tensori{u}{}_{\cell}^\perp \vert_{\dCell} \cdot \tensorii{\tau}{}_{\cell}^h \vert_{\dCell} \cdot \tensori{n}{}
                &&
                \ \ \ \ \ \ \ \ 
                &&
                \forall \tensorii{\tau}{}_{\cell}^h \in S^h(\cell)
            \end{aligned}
        \end{equation}
        %
        %
        %
        using the divergence theorem and the fact that $\nabla \cdot  \tensorii{\tau}{}_{\cell}^h \in U^h(\cell)$, one has :
        %
        %
        %
        \begin{equation}
            \begin{aligned}
                \int_{\cell} \nabla \tensori{u}{}_{\cell}^\perp :  \tensorii{\tau}{}_{\cell}^h
                = &
                % -
                % \int_{\cell} \tensori{u}{}_{\cell}^\perp \cdot \nabla \cdot \tensorii{\tau}{}
                % +
                % \int_{\dCell} \tensori{u}{}_{\cell}^\perp \vert_{\dCell} \cdot \tensorii{\tau}{} \vert_{\dCell} \cdot \tensori{n}{}
                % \\
                % = &
                \int_{\dCell} \tensori{u}{}_{\cell}^\perp \vert_{\dCell} \cdot  \tensorii{\tau}{}_{\cell}^h \vert_{\dCell} \cdot \tensori{n}{}
            \end{aligned}
        \end{equation}
        %
        %
        %
        such that :
        %
        %
        %
        \begin{equation}
            \begin{aligned}
                \int_{\cell} \tensorii{G}{}_{\cell}^h(\tensori{u}{}_{\cell}^h, \tensori{u}{}_{\dCell}^h) : \tensorii{\tau}{}_{\cell}^h
                = &
                \int_{\cell} \nabla \tensori{u}{}_{\cell}^h : \tensorii{\tau}{}_{\cell}^h
                +
                \int_{\cell} \nabla \tensori{u}{}_{\cell}^\perp : \tensorii{\tau}{}_{\cell}^h
                &&
                \ \ \ \ \ \ \ \ 
                &&
                \forall \tensorii{\tau}{}_{\cell}^h \in S^h(\cell)
                \\
                = &
                \int_{\cell} \nabla \tensori{u}{}_{\cell} : \tensorii{\tau}{}_{\cell}^h
                &&
                \ \ \ \ \ \ \ \ 
                &&
                \forall \tensorii{\tau}{}_{\cell}^h \in S^h(\cell)
            \end{aligned}
        \end{equation}
        %
        %
        %
        which proves that $\tensorii{G}{}_{\cell}^h(\tensori{u}{}_{\cell}^h, \tensori{u}{}_{\dCell}^h)$ is the orthogonal projection of $\nabla \tensori{u}{}_{\cell}$ onto $S^h(\cell)$.
        IL FAUDRAIT MONTRER QUE $S^h(\cell)$ EST SUFFISANT POUR EMPECHER LE LOCKING DANS $U^h(\cell) \times V^h(\dCell)$ ??
        ET FAIRE LE LIEN ENTRE $S^h(\cell)$ et $G^h(\cell)$.
        Trouver aussi une justification pour $\stressSpaceCell \subset \gradSpaceCell$, la contrainte est plus régulière que le gradient, ce qui semble vrai (avec les décompositions sphérique dévaitoriques par exemples, etc)
        %
        %
        %
    \end{development}
}
%
%
%
The problem in primal form amounts to find the displacement pair $(\tensori{u}{}_{\cell}, \tensori{u}{}_{\dCell}) \in \displacementSpaceCell \times \displacementSpaceDCell$ verifying $\tensori{u}{}_{\dCell} = \dirichletLag$ on $\dirichletCell$,
such that for all kinematically admissible displacements pairs $(\delta \tensori{u}{}_{T}, \delta \tensori{u}{}_{\partial T}) \in \displacementSpaceCell \times \displacementSpaceDCell$, the functional \eqref{eq_simple} is minimal :
%
%
%
\begin{equation}
    \label{eq_0018}
    \begin{aligned}
        d J_{\cell}
        % = &
        % \frac{\partial J_{\cell}}{\partial \tensori{u}{}_{\cell}} \delta \tensori{u}{}_{\cell}
        % +
        % \frac{\partial J_{\cell}}{\partial \tensori{u}{}_{\dCell}} \delta \tensori{u}{}_{\dCell}
        =
        \delta J_{\cell}^{\text{int}} - \delta J_{\cell}^{\text{ext}}
        =
        0
        % \\
        % = & \delta J_{\cell}^{\text{int}} + \delta J_{\cell}^{\text{ext}}
        % \\
        % = & 
        % \int_{T}
        % \tensorii{P}{}_{\cell}(\tensorii{G}{}_{\cell}(\tensori{u}{}_{\cell}, \tensori{u}{}_{\dCell}))
        % :
        % \tensorii{G}{}_{\cell}(\delta \tensori{u}{}_{\cell}, \delta \tensori{u}{}_{\dCell})
        % % \frac{\partial \mecPotential_{\bodyLag}}{\partial \tensorii{G}{}_T} : \delta \tensorii{G}{}_{T}
        % +
        % \int_{\partial T} (\beta / h_T)
        % (\tensori{u}{}_{\partial T} - \tensori{u}{}_{T} \vert_{\partial T})
        % % \tensori{Z}{}_{\dCell{}}
        % \cdot
        % (\delta \tensori{u}{}_{\partial T} - \delta \tensori{u}{}_{T} \vert_{\partial T})
        % % \delta \tensori{Z}{}_{\dCell{}}
        % \\
        % &
        % -
        % \int_{\partial T} \tensori{t}{}_N \cdot \delta \tensori{u}{}_{\partial T}
        % -
        % \int_{T} \tensori{f}{}_V \cdot \delta \tensori{u}{}_{T}
        % =
        % 0
    \end{aligned}
\end{equation}
%
%
%
with
%
%
%
\begin{subequations}
    \label{eq_0nonamemee}
        \begin{alignat}{3}
            \delta J_{\cell}^{\text{int}} & = 
            \int_{T}
            \tensorii{P}{}_{\cell}(\tensorii{G}{}_{\cell}(\tensori{u}{}_{\cell}, \tensori{u}{}_{\dCell}))
            :
            \tensorii{G}{}_{\cell}(\delta \tensori{u}{}_{\cell}, \delta \tensori{u}{}_{\dCell})
            % \frac{\partial \mecPotential_{\bodyLag}}{\partial \tensorii{G}{}_T} : \delta \tensorii{G}{}_{T}
            +
            \int_{\dCell} (\beta / h_{\cell})
            % (\tensori{u}{}_{\dCell} - \tensori{u}{}_{\cell} \vert_{\dCell})
            % \tensori{Z}{}_{\dCell{}}
            \tensori{Z}{}_{\dCell}(\tensori{u}{}_{\cell}, \tensori{u}{}_{\dCell})
            \cdot
            % (\delta \tensori{u}{}_{\dCell} - \delta \tensori{u}{}_{\cell} \vert_{\dCell{}})
            % \delta \tensori{Z}{}_{\dCell{}}
            \tensori{Z}{}_{\dCell}(\delta \tensori{u}{}_{\cell}, \delta \tensori{u}{}_{\dCell})
            \\
            \delta J_{\cell}^{\text{ext}} & = 
            \int_{\neumannCell} \neumannCellLoad{} \cdot \delta \tensori{u}{}_{\dCell{}}
            +
            \int_{T} \loadLag \cdot \delta \tensori{u}{}_{\cell}
    \end{alignat}
\end{subequations}
%
%
%
where we introduced the jump function $\tensori{Z}{}_{\dCell}$ :
%
%
%
\begin{equation}
    \begin{aligned}
        \tensori{Z}{}_{\dCell}(\tensori{v}{}_{\cell}, \tensori{v}{}_{\dCell}) = \tensori{v}{}_{\dCell} - \tensori{v}{}_{\cell} \vert_{\dCell}
        &&
        \forall (\tensori{v}{}_{\cell}, \tensori{v}{}_{\dCell}) \in \displacementSpaceCell \times \displacementSpaceDCell
    \end{aligned}
\end{equation}
%
%
%
In particular, one can readliy see the resemblance of \eqref{eq_0nonamemee} with the ususal formulation of the principle of virtual works, where the so called reconstructed displacement gradient $\tensorii{G}{}_{\cell}(\tensori{u}{}_{\cell}, \tensori{u}{}_{\dCell})$ plays the role of the usual displacement Lagrangian gradient $\nabla \tensori{u}{}_{\cell}$, and where an additional term corresponding to a traction energy on the boundary has been added to account for the penalization of the displacement jump on $\dCell$ through $\tensori{Z}{}_{\dCell}$. This term is discribed in the literature as the stabilization term, as it prevents rigid body motions to be solutions of the problem.