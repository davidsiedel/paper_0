\begin{equation}
    \label{hho_incremenaljjjdjkkj}
    \begin{aligned}
        \begin{Bmatrix}
            \tensorii{G}{}_{\cell}^k(\tensori{v}{}_{\cell}, \tensori{v}{}_{\dCell})
        \end{Bmatrix}
        (\tensori{x}{}_q)
        =
        \begin{bmatrix}
            B_{\cell} && B_{\dCell}
        \end{bmatrix}
        (\tensori{x}{}_q)
        \cdot
        \begin{Bmatrix}
            \tensori{v}{}_{\cell}
            \\
            \tensori{v}{}_{\dCell}
        \end{Bmatrix}
        &&
        \forall (\tensori{v}{}_{\cell}, \tensori{v}{}_{\dCell}) \in \displacementSpaceCell \times \displacementSpaceDCell
    \end{aligned}
\end{equation}
%
%
%
\begin{equation}
    \label{hho_incremenaljjjdjj}
    \begin{aligned}
        % \begin{Bmatrix}
        %     \tensori{Z}{}_{\dCell}(\tensori{v}{}_{\cell}, \tensori{v}{}_{\dCell})
        % \end{Bmatrix}
        \int_{\dCell} (\beta / h_{\cell})
        \tensori{Z}{}_{\dCell}^{HHO}(\tensori{u}{}_{\cell}^l, \tensori{u}{}_{\dCell}^k)
        % \tensori{Z}{}_{\dCell{}}
        \cdot
        \tensori{Z}{}_{\dCell}^{HHO}(\delta \tensori{u}{}_{\cell}^l, \delta \tensori{u}{}_{\dCell}^k)
        =
        \beta
        \begin{Bmatrix}
            \tensori{u}{}_{\cell}^l
            \\
            \tensori{u}{}_{\dCell}^k
        \end{Bmatrix}^{\text{t}}
        \cdot
        \begin{bmatrix}
            Z_{\cell \cell} && Z_{\cell \dCell}
            \\
            Z_{\dCell \cell} && Z_{\dCell \dCell}
        \end{bmatrix}
        \cdot
        \begin{Bmatrix}
            \delta \tensori{u}{}_{\cell}^l
            \\
            \delta \tensori{u}{}_{\dCell}^k
        \end{Bmatrix}
    \end{aligned}
\end{equation}
%
%
%
\begin{equation}
    \label{hho_incremenaljjjdjj}
    \begin{aligned}
        \begin{Bmatrix}
            F_{T}^{int}
            (
                % \delta
                \tensori{u}{}_{\cell}^{l,m}
                ,
                % \delta
                \tensori{u}{}_{\dCell}^{k,m}
            )
        \end{Bmatrix}
        =
        \sum_Q
        \begin{bmatrix}
            B_{\cell} && B_{\dCell}
        \end{bmatrix}^{\text{t}}(\tensori{x}{}_q)
        \cdot
        \begin{Bmatrix}
            \tensorii{P}{}_{\cell}^k(
                \tensorii{G}{}_{\cell}^k
                (
                    % \delta
                    \tensori{u}{}_{\cell}^{l,m}
                    ,
                    % \delta
                    \tensori{u}{}_{\dCell}^{k,m}
                )
            )
        \end{Bmatrix}(\tensori{x}{}_q)
        +
        \beta
        \begin{bmatrix}
            Z_{\cell \cell} && Z_{\cell \dCell}
            \\
            Z_{\dCell \cell} && Z_{\dCell \dCell}
        \end{bmatrix}
        \cdot
        \begin{Bmatrix}
            % \delta
            \tensori{u}{}_{\cell}^{l,m}
            \\
            % \delta
            \tensori{u}{}_{\dCell}^{k,m}
        \end{Bmatrix}
    \end{aligned}
\end{equation}
%
%
%
\begin{equation}
    \label{hho_incremenaljjjdjj}
    \begin{aligned}
        \begin{Bmatrix}
            F_{T}^{ext}
        \end{Bmatrix}
        =
        \begin{Bmatrix}
            \loadLag
            \\
            \neumannLag
        \end{Bmatrix}
    \end{aligned}
\end{equation}
%
%
%
\begin{equation}
    \label{hho_incremenaljdjjjkk}
    \begin{aligned}
        \begin{bmatrix}
            K_{\cell}^{tan}(
                % \delta
                \tensori{u}{}_{\cell}^{l,m}
                ,
                % \delta
                \tensori{u}{}_{\dCell}^{k,m}
            )
        \end{bmatrix}
        =
        \sum_Q
        \begin{bmatrix}
            B_{\cell} && B_{\dCell}
        \end{bmatrix}^{\text{t}}
        (\tensori{x}{}_q)
        \cdot
        \begin{bmatrix}
            \tensoriv{A}{}
            (
                % \delta
                \tensori{u}{}_{\cell}^{l,m}
                ,
                % \delta
                \tensori{u}{}_{\dCell}^{k,m}
            )
        \end{bmatrix}
        (\tensori{x}{}_q)
        \cdot
        \begin{bmatrix}
            B_{\cell} && B_{\dCell}
        \end{bmatrix}
        (\tensori{x}{}_q)
    \end{aligned}
\end{equation}
%
%
%
\begin{equation}
    \label{hho_incremenaljdkkjjjkk}
    \begin{aligned}
        \begin{bmatrix}
            \tensoriv{A}{}
            (
                % \delta
                \tensori{u}{}_{\cell}^{l,m}
                ,
                % \delta
                \tensori{u}{}_{\dCell}^{k,m}
            )
        \end{bmatrix}
        =
        \frac{
            \partial
            \tensorii{P}{}_{\cell}^k(
                \tensorii{G}{}_{\cell}^k
                (
                    % \delta
                    \tensori{u}{}_{\cell}^{l,m}
                    ,
                    % \delta
                    \tensori{u}{}_{\dCell}^{k,m}
                )
            )
        }
        {
            \partial
            \tensorii{G}{}_{\cell}^k
            (
                % \delta
                \tensori{u}{}_{\cell}^{l,m}
                ,
                % \delta
                \tensori{u}{}_{\dCell}^{k,m}
            )
        }
    \end{aligned}
\end{equation}
%
%
%
%
\begin{equation}
    \label{hho_incremenahhljdjjjkk}
    \begin{aligned}
        \begin{bmatrix}
            K_{\cell}^{tan}(
                % \delta
                \tensori{u}{}_{\cell}^{l,m}
                ,
                % \delta
                \tensori{u}{}_{\dCell}^{k,m}
            )
        \end{bmatrix}
        =
        \begin{bmatrix}
            K_{\cell \cell} (\tensori{u}{}_{\cell}^{l,m}, \tensori{u}{}_{\dCell}^{k,m})
            &&
            K_{\cell \dCell} (\tensori{u}{}_{\cell}^{l,m}, \tensori{u}{}_{\dCell}^{k,m})
            \\
            K_{\dCell \cell} (\tensori{u}{}_{\cell}^{l,m}, \tensori{u}{}_{\dCell}^{k,m})
            &&
            K_{\dCell \dCell} (\tensori{u}{}_{\cell}^{l,m}, \tensori{u}{}_{\dCell}^{k,m})
        \end{bmatrix}
    \end{aligned}
\end{equation}
%
%
%
\begin{equation}
    \label{hho_incremenahzhjehhljdjjjkk}
    \begin{aligned}
        \begin{bmatrix}
            K_{\cell}^{tan}(
                % \delta
                \tensori{u}{}_{\cell}^{l,m}
                ,
                % \delta
                \tensori{u}{}_{\dCell}^{k,m}
            )
        \end{bmatrix}
        \cdot
        \begin{Bmatrix}
            \delta
            \tensori{u}{}_{\cell}^{l}
            \\
            \delta
            \tensori{u}{}_{\dCell}^{k}
        \end{Bmatrix}
        =
        \begin{Bmatrix}
            F_T^{int} (\tensori{u}{}_{\cell}^{l,m}, \tensori{u}{}_{\dCell}^{k,m})
        \end{Bmatrix}
        -
        \begin{Bmatrix}
            F_T^{ext}
        \end{Bmatrix}
    \end{aligned}
\end{equation}
%
%
%
\begin{equation}
    \label{hho_incremenahzhjjjklehhljdjjjkk}
    \begin{aligned}
        \begin{bmatrix}
            K_{\cell}^{tan}(
                % \delta
                \tensori{u}{}_{\cell}^{l,m}
                ,
                % \delta
                \tensori{u}{}_{\dCell}^{k,m}
            )
        \end{bmatrix}_c
        \cdot
        \begin{Bmatrix}
            \delta
            \tensori{u}{}_{\dCell}^{k}
        \end{Bmatrix}
        =
        \begin{Bmatrix}
            F_T^{int} (\tensori{u}{}_{\cell}^{l,m}, \tensori{u}{}_{\dCell}^{k,m})
        \end{Bmatrix}_c
        -
        \begin{Bmatrix}
            F_T^{ext}
        \end{Bmatrix}_c
    \end{aligned}
\end{equation}
%
%
%
% \begin{equation}
%     \label{hho_incremenaljdjj}
%     \begin{aligned}
%         \begin{bmatrix}
%             \delta U_{T}
%             &&
%             \delta U_{\partial T}
%         \end{bmatrix}^{\text{t}}
%         \cdot
%         \begin{bmatrix}
%             B_{GT} \\ B_{G \partial T}
%         \end{bmatrix}^{\text{t}}
%         \cdot
%         \begin{bmatrix}
%             A_{GG}
%         \end{bmatrix}^{\text{t}}
%         \cdot
%         \begin{bmatrix}
%             B_{GT} && B_{G \partial T}
%         \end{bmatrix}^{\text{t}}
%         \cdot
%         \begin{bmatrix}
%             \delta U_{T}
%             \\
%             \delta U_{\partial T}
%         \end{bmatrix}
%         = -
%         \begin{bmatrix}
%             R_{T}
%             \\
%             R_{\partial T}
%         \end{bmatrix}
%         &&
%         \text{tel que}
%         &&
%         \begin{bmatrix}
%             R_{T}
%             \\
%             R_{\partial T}
%         \end{bmatrix}
%         =
%         \begin{bmatrix}
%             F_{T}^{ext}
%             \\
%             F_{\partial T}^{ext}
%         \end{bmatrix}
%         -
%         \begin{bmatrix}
%             F_{T}^{int}
%             \\
%             F_{\partial T}^{int}
%         \end{bmatrix}
%     \end{aligned}
% \end{equation}
% %
% %
% %
% La résolution du problème local (\ref{eq_0nonamemeergjj}) sous forme incrémentale s'exprime par le système matriciel (\ref{hho_incremenal}) :
% %
% %
% %
% \begin{equation}
%     \label{hho_incremenal}
%     \begin{aligned}
%         \begin{bmatrix}
%             K_{TT} && K_{T \partial T}
%             \\
%             K_{\partial T T} && K_{\partial T \partial T}
%         \end{bmatrix}
%         \cdot
%         \begin{bmatrix}
%             \delta U_{T}
%             \\
%             \delta U_{\partial T}
%         \end{bmatrix}
%         = -
%         \begin{bmatrix}
%             R_{T}
%             \\
%             R_{\partial T}
%         \end{bmatrix}
%         &&
%         \text{tel que}
%         &&
%         \begin{bmatrix}
%             R_{T}
%             \\
%             R_{\partial T}
%         \end{bmatrix}
%         =
%         \begin{bmatrix}
%             F_{T}^{ext}
%             \\
%             F_{\partial T}^{ext}
%         \end{bmatrix}
%         -
%         \begin{bmatrix}
%             F_{T}^{int}
%             \\
%             F_{\partial T}^{int}
%         \end{bmatrix}
%     \end{aligned}
% \end{equation}
% %
% où on définit l'incrément de déplacement $\delta U$ la matrice de raideur $K$, le résidu $R$, les forces internes $F^{int}$ et externes $F^{ext}$, et on décompose le système élémentaire par blocs de faces et de cellule, avec la notation $[\cdot]$ pour désigner une représentation vectorielle ou matricielle; les forces internes vérifient (\ref{hho_algebraic0}):
% %
% \begin{equation}
%     \label{hho_algebraic0}
%     \begin{aligned}
%         \begin{bmatrix}
%             F_{T}^{int}
%             \\
%             F_{\partial T}^{int}
%         \end{bmatrix}
%         % {}[F_{int}]
%         & = \sum_{(\tensori{x}{}_Q, w_Q^T)} (w_Q^T \cdot [B(\tensori{x}{}_Q)]^t \cdot [\tensorii{P}{}_T^k(\tensori{x}{}_Q)])
%         +
%         \frac{\beta_{mec}}{h_T} ([S]^t \cdot [S]) \cdot 
%         \begin{bmatrix}
%             U_{T}
%             \\
%             U_{\partial T}
%         \end{bmatrix}
%     \end{aligned}
% \end{equation}
% %
% où on note $\tensori{x}{}_Q$ (respectivement $w_Q^T$) un point (respectivement un poids) de quadrature de cellule, et $[U_T, U_{\partial T}]$ le vecteur d'inconnue élémentaire. Comme pour les éléments finis strandards, l'opérateur $[B]$ est la matrice qui à $[U_T, U_{\partial T}]$ assoscie le gradient d'inconnue $[\tensorii{G}{}_T^k]$, et le superscript $t$ désigne l'opérateur de transposition. Les forces externes vérifient (\ref{hho_algebraic1}) :
% %
% \begin{equation}
%     \label{hho_algebraic1}
%     \begin{aligned}
%         \begin{bmatrix}
%             F_{T}^{ext}
%             \\
%             F_{\partial T}^{ext}
%         \end{bmatrix}
%         & =
%         \begin{bmatrix}
%             \sum_{(\tensori{x}{}_Q, w_Q^T)} w_Q^T \cdot [\phi({\tensori{x}{}_Q})]^t \cdot \tensori{f}{}_{V}(\tensori{x}{}_Q)
%             \\
%             \sum_{(\tensori{s}{}_Q, w_Q^{\partial T})} w_Q^{\partial T} \cdot [\omega(\tensori{s}{}_Q)]^t \cdot \tensori{t}{}_{N}(\tensori{s}{}_Q)
%         \end{bmatrix}
%     \end{aligned}
% \end{equation}
% avec $\tensori{s}{}_Q$ (respectivement $w_Q^{\partial T}$) un point (respectivement poids) de quadrature de face, et $\phi$ (respectivement $\omega$) le vecteur de fonctions de bases dans $T$ (respectivement $\partial T$). Finalement la matrice de raideur vérifie (\ref{hho_algebraic2}) :
% \begin{equation}
%     \label{hho_algebraic2}
%     \begin{aligned}
%         \begin{bmatrix}
%             K_{TT} && K_{T \partial T}
%             \\
%             K_{\partial T T} && K_{\partial T \partial T}
%         \end{bmatrix}
%         & = \sum_{(\tensori{x}{}_Q, w_Q^T)} (w_Q^T \cdot [B(\tensori{x}{}_Q)]^t \cdot [\tensoriv{C}{}(\tensori{x}{}_Q)] \cdot [B(\tensori{x}{}_Q)]) + \frac{\beta_{mec}}{h_T} ([S]^t \cdot [S])
%     \end{aligned}
% \end{equation}
% %
% où on a introduit $\tensoriv{C}{}$ l'opérateur tangent cohérent.
% En particulier, on peut exprimer (\ref{hho_incremenal}) sous la forme condensée (\ref{hho_incremental_cond}) par réalsation d'un complément de Schur:

% \begin{equation}
%     \label{hho_incremental_cond}
%     [K_{cond}] \cdot [\delta U_{\partial T}] = -[R_{cond}]
% \end{equation}

\subsection{Résolution par condensation statique}

Le problème global incrémental (\ref{eq_hho2}) est alors l'assemblage des systèmes élémentaires condensés (\ref{hho_incremental_cond}), dont la résolution consiste en un algorithme de Newton sur l'incrément des inconnues de faces uniquement. Ce schéma de résolution par condensation statique dont on donne le principe Figure \ref{res_cond0} exploite la relation linéaire entre l'incrément des inconnues de cellules et celui des faces.
    
\begin{figure}[H]
\centering
\includegraphics[width=14.cm]{img/reso1.png}
\caption{Description schématique de l'algorithme de condensation statique}
\label{res_cond0}
\end{figure}

\subsection{Résolution par équilibre de cellule}

Nous proposons une alternative à l'algorithme de résolution par condensation statique, postulant une relation implicite entre l'incrément des inconnues de cellule est celui des faces et consistant à résoudre localement un système non-linéaire sur l'incrément de cellule à incrément de faces fixé, afin de vérifier l'équilibre de la cellule avec ses faces à chaque itération du problème global. Ce nouveau schéma de résolution est décrit Figure \ref{res_cond}, où on note $i$ une itération de Newton pour la résolution du problème global sur l'ensemble des inconnues de faces, et $j$ une itération de Newton pour la résolution du problème local sur les inconnues de cellule dans un élément $T$:

\begin{figure}[H]
\centering
\includegraphics[width=14.cm]{img/reso0.png}
\caption{Description schématique de l'algorithme d'équilibre de cellule}
\label{res_cond}
\end{figure}