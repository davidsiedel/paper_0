\section{Introduction to discontinuous methods through a Hu-Washizu formulation}
\label{sec_composite_demo}

Many numerical methods
consider a partition of the body into elementary subsets called \textit{cells}.
% The equilibrium of the body is then expressed in terms of the sum of the mechanical contribution in each of these cells composing it.
% The equilibrium of the interface between two cells is then of major 
For the equilibrium of the whole body to hold, each cell must be in equilibrium with its neighbors, which means that they have to interact and pass information from one another.
Communication between two cells is ensured by knowledge of the displacement field at their shared boundary.
For \textit{conformal} methods to which belongs the Lagrange (\textit{i.e.} the standard) Finite Element Method, the displacement over the whole body is continuous, which implies that the displacement at a cell boundary is directly equal to that of its neighbors, and a cell has a direct knowledge of the motion of its neighborhood.
For so-called \textit{non-conformal} methods, among which are \textit{e.g.} Discontinuous Galerkin methods, Hybrid Discontinuous Galerkin methods and Hybrid High Order methods, the displacement continuity at a cell boundary is not explicitly enforced, such that one needs to introduce supplementary ingredients in the formulation to pass information from one cell to another. A straightforward way of doing so consists in introducing an interface between cells, that acts as a membrane pulling them together. One can readily feel that the stiffer the membrane, the closer the cells, and the closer to the \textit{conformal} framework. This technique is the one at the foundation of Discontinuous Galerkin methods. By adding an intermediate structure in the membrane, called a \textit{bone}, one defines the framework of Hybrid Discontinuous Galerkin methods and Hybrid High Order methods; the membrane is split into two parts, one at each side of the bone, and communication between cells transits through the bone, via the membrane. The term hybrid expresses the fact that both cells and bones carry information about the displacement field, hence introducing the \textit{skeleton} of the body.
In the following section, we show that one recovers the full mathematical framework proper to these non-conformal method, by writing the equilibrium of a cell and its interface with its neighborhood.
%  cells must stick together in order to pass the mechanical information from one cell to another

% Let $\cell$ be
% such a cell and let $\dCell{}$ its boundary.
% % For the sake of
% % simplicity, the intersection of $\dCell{}$ and $\dBodyLag{}$
% % is first assumed empty. This special case is treated later.

% In the framework of conforming Galerkin methods, the displacement field is continuous at cells interfaces, and the equilibrium is expressed on the whole structure.
% For discontinuous Galerkin methods, the displacement continuity is broken, to ensure continuity of the flux between cells instead, such that the equilibrium of the whole structure is the sum of the equilibrium expressed within each cell.
% In this section, we show that this change of paradigm amounts to consider that a cell in the continuous framework is surrounded by an infinitely thin elastic interface.

% Let the equilibrium of a cell $\cell$
% %
% %
% %
% \begin{equation}
%     \label{eq_element}
%     L_{\cell}^{eq}
%     % (\tensori{u}{}_{\cell}, \tensorii{G}{}_{\cell}, \tensorii{P}{}_{\cell})
%     =
%     \int_{\cell} \mecPotential_{\bodyLag{}} + (\nabla_X \tensori{u}{}_{\cell} - \tensorii{G}{}_{\cell}) : \tensorii{P}{}_{\cell}
%     -
%     \int_{\cell} \loadLag \cdot \tensori{u}{}_{\cell}
%     % -
%     % \int_{\Crown} \loadLag \cdot \tensori{u}{}_{\Crown}
%     -
%     \int_{\neumannCell} \neumannCellLoad \cdot \tensori{u}{}_{\dCell}
% \end{equation}
% %
% %
% %
% One can either choose a suitable functional space such that the displacement is continuous at the interfaces, or 


% Je pense vraiment qu'il faut exprimer explicitement l'équilibre d'un élément avec le reste de la structure (tractions normales et continuité des déplacement), préciser que l'on peut imposer les continuités du déplacement au sens fort ou au sens faible et **ensuite** classifier les méthodes ainsi:
% Les éléments finis standards assurent la continuité des déplacements aux interfaces
% Les DG relient directement les éléments entre eux et pénalisent les sauts de déplacement entre les éléments
% Les HDG et HHO relient les éléments à leurs frontières et pénalisent les sauts de déplacement entre l'élément et la frontière

% Puis on indique qu'un moyen pratique de pénaliser les sauts de déplacement est d'introduire une interphase élastique de faible épaisseur et de faire tendre cette épaisseur vers 0.  Dans le cadre de ce papier, seules les méthodes HDG et HHO sont traitées (et donc pas les DG).

% Cette section montre que l'application de Hu-Hashizu à l'élément à son interphase permet de retrouver tous les ingrédients principaux des méthodes HDG et HH0: le gradient reconstruit et l'opérateur de stabilisation.

\subsection{Partie 1}

\paragraph{Element description}

In the following, the cell $\cell$ is assumed to be convex.
It is split into a core part $\Bulk \subset \cell$ with boundary $\dBulk$, and into an interface part $\Crown{} \subset \cell$ with boundary $\dCrown = \dBulk \cup \dCell$, as shown in Figure \ref{fig_02}. The interface $\Crown{}$ has some thickness $\ell > 0$ that is supposed to be small compared to $h_{\cell}$ the diameter of $\cell$.
From a geometrical standpoint, the core par of the element $\Bulk{}$ is an homotethy of $\cell$ by some ratio inferior to $1$.
%
% 
% 
\begin{figure}[H]
    \centering
    \includegraphics[width=12.cm]{img/hu_washizu.png}
    \caption{schematic representation of the composite region}
    \label{fig_02}
\end{figure}
%
%
%

\paragraph{Element behaviour}

The core of the element $\Bulk{}$ is made out of the same material that composes $\Omega$ and behaves according to the free energy potential $\mecPotential{}_{\bodyLag{}}$. The interface $\Crown{}$ is made out of a linear elastic material of Young modulus $\beta (\ell / h_{\cell})$ with a zero Poisson ratio and its behavior is defined by the free energy potential $\mecPotential{}_{\Crown{}}$ such that
%
%
%
\begin{equation}
    \label{eq_0009}
        \mecPotential{}_{\Crown} = \frac{1}{2} \beta \frac{\ell}{h_{\cell}} \nabla \tensori{u}{}_{\Crown} : \nabla \tensori{u}{}_{\Crown}
\end{equation}
%
%
%
where the dimensionless ratio $\ell / h_{\cell}$ balances the accumulated energy with the size of the domain $\cell$.

\paragraph{Element loading}

The core $\Bulk$ is subjected to the volumetric loading $\loadLag{}$, and to the contact load applied by the interface $\Crown{}$ onto $\dBulk{}$. By continuity of the traction force, the same opposite contact load acts on $\Crown{}$. The interface $\Crown{}$ is also subjected to some contact load $\neumannCellLoad{}$ acting on $\dCell{}$, that accounts for the action of the rest of the solid $\bodyLag{}$ onto $\cell$.

\paragraph{Element unknowns}

Let note $\tensori{u}{}_{\Bulk}$ the displacement field, $\tensorii{G}{}_{\Bulk}$ the displacement gradient field and $\tensorii{P}{}_{\Bulk}$ the stress field in $\Bulk{}$. Similarly, let $\tensori{u}{}_{\Crown{}}$ the displacement field, $\tensorii{G}{}_{\Crown}$ the displacement gradient field and $\tensorii{P}{}_{\Crown}$ the stress field in $\Crown{}$.
The displacement of the boundary $\dCell{}$ is denoted $\tensori{u}{}_{\dCell{}}$.
By continuity of the displacement field between $\Bulk{}$ and $\dCell$,  the displacement $\tensori{u}{}_{\Crown{}}$ verifies
%
% 
% 
\begin{subequations}
    \label{eq_conformity}
        \begin{alignat}{2}
        \tensori{u}{}_{\Crown} \vert_{\dBulk} & = \tensori{u}{}_{\Bulk} \vert_{\dBulk}
        \label{eq_conformity:eq1}
        \\
        \tensori{u}{}_{\Crown} \vert_{\dCell} & = \tensori{u}{}_{\dCell}
        \label{eq_conformity:eq2}
    \end{alignat}
\end{subequations}

\paragraph{Hu-Washizu Lagrangian of the element}

By combining both the Lagragian of the core $\Bulk{}$ and that of the interface $\Crown{}$, one obtains the total Lagragian $L_{\cell}^{HW}$ over the element such that
%
%
%
\begin{equation}
    \label{eq_hu_washizu_split}
    L_{\cell}^{HW}
    % (\tensori{u}{}_{\cell}, \tensorii{G}{}_{\cell}, \tensorii{P}{}_{\cell})
    =
    \int_{\Bulk} \mecPotential_{\bodyLag{}} + (\nabla_X \tensori{u}{}_{\Bulk} - \tensorii{G}{}_{\Bulk}) : \tensorii{P}{}_{\Bulk}
    +
    \int_{\Crown} \mecPotential_{\Crown{}} + (\nabla_X \tensori{u}{}_{\Crown} - \tensorii{G}{}_{\Crown}) : \tensorii{P}{}_{\Crown}
    -
    \int_{\Bulk} \loadLag \cdot \tensori{u}{}_{\Bulk}
    % -
    % \int_{\Crown} \loadLag \cdot \tensori{u}{}_{\Crown}
    -
    \int_{\neumannCell} \neumannCellLoad \cdot \tensori{u}{}_{\dCell}
\end{equation}

\subsection{Hypotheses}
\label{sec_assumtions}

The behavior and the kinematics having been described, let now make a number of assumptions on the expression of the fields of unknowns, exploiting the fact that the core is of negligible volume compared to the interface.

\paragraph{Displacement in the interface}

Since the interface $\Crown$ is thin compared to the cell volume $\cell$, let linearize the displacement in the interface $\Crown$ with respect to $\tensori{n}$, such that
its gradient is homogeneous in $\Crown{}$
%
% 
% 
% \begin{equation}
%     \label{eq_crown_displacement}
%     \tensori{u}{}_{\Crown} (\tensori{x})
%     =
%     \frac{\tensori{u}{}_{\dCell}(\tensori{m}{}_{\dCell})
%     -
%     \tensori{u}{}_{\Bulk} \vert_{\dBulk} (\tensori{m}{}_{\dBulk})}{\ell} \otimes \tensori{n} \cdot (\tensori{x} - \tensori{m}{}_{\dBulk})
%     +
%     \tensori{u}{}_{\Bulk} \vert_{\dBulk}(\tensori{m}{}_{\dBulk})
% \end{equation}
\begin{equation}
    \label{eq_crown_displacement}
    \nabla
    \tensori{u}{}_{\Crown}
    =
    \frac{\tensori{u}{}_{\dCell}
    -
    \tensori{u}{}_{\Bulk} \vert_{\dBulk} }{\ell} \otimes \tensori{n}
\end{equation}
% 
% 
%
That is, the displacement of the interface $\Crown{}$ linearly bridges that of the boundary $\dCell{}$ to that of the bulk $\Bulk{}$.

\paragraph{Stress in the interface}

Furthermore, let assume that $\tensorii{P}{}_{\Crown}$ is constant along the direction $\tensori{n}{}$ in $\Crown{}$. By continuity of the traction force across $\dBulk$, the following equality holds true
%
% 
% 
\begin{equation}
    \label{eq_continuity_traction_force}
    \begin{aligned}
        (\tensorii{P}{}_{\Crown} - \tensorii{P}{}_{\Bulk} \vert_{\dBulk{}}) \cdot \tensori{n}{} =  0
        &&
        \text{in}
        &&
        \Crown{}
    \end{aligned}
\end{equation}

\subsection{Partie 2}

\paragraph{Hu–Washizu avec hypothèses}

Taking into accounts the assumptions made Section \ref{sec_assumtions}, the total Lagrangian of the system can be re-written (see Section \ref{sec_appendix} for details) in the following simplified form
% 
% 
%
\begin{equation}
    \label{eq_0015}
    \begin{aligned}
        J_{\cell}^{HW}
        = &
        \int_{\cell{}} \mecPotential{}_{\bodyLag{}} + (\nabla \tensori{u}{}_{\cell{}} - \tensorii{G}{}_{\cell{}}) : \tensorii{P}{}_{\cell}
        % \\
        % &
        + \int_{\dCell{}} (\tensori{u}{}_{\dCell} - \tensori{u}{}_{\cell} \vert_{\dCell}) \cdot \tensorii{P}{}_{\cell} \vert_{\dCell{}} \cdot \tensori{n}{}
        % \\
        % &
        + \int_{\dCell} \frac{\beta}{2 h_{\cell}} \lVert \tensori{u}{}_{\dCell{}} - \tensori{u}{}_{\cell{}} \vert_{\dCell{}} \rVert^2
        \\
        &
        -
        \int_{\cell} \loadLag{} \cdot \tensori{u}{}_{\cell{}}
        -
        \int_{\neumannCell{}} \neumannCellLoad{} \cdot \tensori{u}{}_{\dCell{}}
    \end{aligned}
\end{equation}
%
%
%
where we have made the width of the interface $\ell \rightarrow 0$, such that the core part $\Bulk{}$ now identifies as $\cell$.

\paragraph{hybridization of the primal unknown}

The displacement is discontinuous across $\dCell{}$ by considering the vanishing interface, that allows for the core part $\cell{}$ to move away from the boundary $\dCell$, thus introducing a possible displacement jump on $\dCell{}$.
This assumption relates to the concept of hybridization of the displacement unknown, which is at the foundation of Hybrid Discontinuous Galerkin methods.
The displacement of the element $\cell$ hence depends on the pair $(\tensori{u}_{\cell}, \tensori{u}_{\dCell})$, where the trace of the core unknown $\tensori{u}_{\cell} \vert_{\dCell{}}$ coexists with $\tensori{u}_{\dCell}$ on $\dCell{}$.

\paragraph{Discontinuous Galerkin}

By replacing $\tensori{u}{}_{\dCell}$ by $\tensori{u}{}_{\cell'} \vert_{\dCell}$ for any neighbouring cell $\cell'$ to $\cell$ amounts to describe the framework for Discontinuous Galerkin methods, where only the core unknown $\tensori{u}{}_{\cell}$ is considered, and the displacement jump on $\dCell$ depends on the trace of neighbouring cells  displacement, instead of that only defined on the boundary.

\paragraph{Conformal Galerkin formulation}

By enforcing continuity of the displacement across $\dCell{}$ such that $\tensori{u}_{\cell} \vert_{\dCell} = \tensori{u}_{\dCell}$, one recovers the usual expression over 

The displacement is discontinuous across $\dCell{}$ by considering the vanishing interface, that allows for the core part $\cell{}$ to move away from the boundary $\dCell$, thus introducing a possible displacement jump on $\dCell{}$.
This assumption relates to the concept of hybridization of the displacement unknown, which is at the foundation of Hybrid Discontinuous Galerkin methods.
The displacement of the element $\cell$ hence depends on the pair $(\tensori{u}_{\cell}, \tensori{u}_{\dCell})$, where the trace of the core unknown $\tensori{u}_{\cell} \vert_{\dCell{}}$ coexists with $\tensori{u}_{\dCell}$ on $\dCell{}$.

\paragraph{Derivative}

La fonctionelle \eqref{eq_0015} définit le problème mixte sous forme faible, et revient à résoudre les problèmes couplés suivants

\begin{subequations}
    \label{eq_0017}
        \begin{alignat}{3}
            \frac{\partial J_{\cell}^{HW}}{\partial \tensori{u}{}_{\cell}} \delta \tensori{u}{}_{\cell}
            = & \int_{\cell} \tensorii{P}{}_{\cell} : \nabla \delta \tensori{u}{}_{\cell}
            -
            \int_{\cell} \tensori{f}{}_V \cdot \delta \tensori{u}{}_{\cell}
            -
            \int_{\dCell{}} \tensori{\theta}{}_{\dCell} \cdot \delta \tensori{u}{}_{\cell} \vert_{\dCell}
            &&
            \ \ \ \ \ \ \ \ 
            &&
            \forall \delta \tensori{u}{}_{\cell}
            \in \virtualDisplacementSpaceCell
        \label{eq_0017:eq0}
        \\
            \frac{\partial J_{\cell}^{HW}}{\partial \tensori{u}{}_{\dCell}} \delta \tensori{u}{}_{\dCell}
            = &
            \int_{\neumannCell} (\tensori{\theta}{}_{\dCell} - \tensori{t}{}_{\neumannCell}) \cdot \delta \tensori{u}{}_{\dCell}
            &&
            \ \ \ \ \ \ \ \ 
            &&
            \forall \delta \tensori{u}{}_{\dCell}
            \in \virtualDisplacementSpaceDCell
        \label{eq_0017:eq1}
        \\
            \frac{\partial J_{\cell}^{HW}}{\partial \tensorii{G}{}_{\cell}} \delta \tensorii{G}{}_{\cell}
            = &
            \int_{\cell} (\frac{\partial \mecPotential_{\bodyLag}}{\partial \tensorii{G}{}_{\cell}} - \tensorii{P}{}_{\cell}) : \delta \tensorii{G}{}_{\cell}
            &&
            \ \ \ \ \ \ \ \ 
            &&
            \forall \delta \tensorii{G}{}_{\cell}
            \in \gradSpaceCell
        \label{eq_0017:eq2}
        \\
            \frac{\partial J_{\cell}^{HW}}{\partial \tensorii{P}{}_{\cell}} \delta \tensorii{P}{}_{\cell}
            = & \int_{\cell} (\nabla \tensori{u}{}_{\cell} - \tensorii{G}{}_{\cell} ) : \delta \tensorii{P}{}_{\cell}
            +
            \int_{\dCell} (\tensori{u}{}_{\dCell} - \tensori{u}{}_{\cell} \vert_{\dCell}) \cdot \delta \tensorii{P}{}_{\cell} \vert_{\dCell} \cdot \tensori{n}{}
            &&
            \ \ \ \ \ \ \ \ 
            &&
            \forall \delta \tensorii{P}{}_{\cell}
            \in \stressSpaceCell
        \label{eq_0017:eq3}
    \end{alignat}
\end{subequations}
% 
% 
%
where we introduced the \textit{reconstructed traction force} $\tensori{\theta}{}_{\dCell} = \tensorii{P}{}_{\cell} \vert_{\dCell} \cdot \tensori{n}{} + (\beta / h_{\cell}) (\tensori{u}{}_{\dCell} - \tensori{u}{}_{\cell} \vert_{\dCell})$.
In particular, \eqref{eq_0017:eq0} is the expression of the principle of virtual works in $\cell$, where the \textit{reconstructed traction force} $\tensori{\theta}{}_{\dCell}$ replaces the usual expression $\tensorii{P}{}_{\cell} \cdot \tensori{n}{}$ in the external contribution. \eqref{eq_0017:eq1} denotes a supplementary equation to the usual continuous problem as described in \eqref{eq_hu_washizu_derivative_0}, to account for the continuity of the flux $\tensori{\theta}{}_{\dCell}$ across the cell boundary.
% This feature constitutes one of the key assets of non-conformal method; indeed, by defining a richer flux than in the usual continuous framework, that also depends on the displacement jump, one allows for the latter to act as a Lagrange multiplier in order to fulfill the flux continuity requirement on $\dCell$.
% La continuité du flux aux interfaces is indeed the tradeoff for having loosened la continuité du déplacement aux interfaces.
% Stability of the problem is then recovered through the interface behaviour that penalizes displacement jumps in a weak sense.
% \eqref{eq_0017:eq2} defines the stress-behaviour law relation, and \eqref{eq_0017:eq3} defines a gradient field reconstruction based on a linear problem, whose second term depends on both a body and a boundary term.
\eqref{eq_0017:eq2} accounts for the constitutive equation in a weak sense, and \eqref{eq_0017:eq3} defines the equation of an enhanced gradient field, that does not reduce to the projection of $\nabla \tensori{u}{}_{\cell}$ as in \eqref{eq_hu_washizu_derivative_0:eq3}, since it is enriched by a boundary component that depends on the displacement jump, which is at the origin of the robustness of non-conformal methods to volumetric locking (see Section \ref{sec_appendix}).
%
%
%
%
% Indeed, defining $\tensori{I}{}(\tensori{v}{})$ the interpolation operator

\subsection{Problem in primal form}

\paragraph{Reconstructed gradient}

Since minimization of \eqref{eq_0017:eq3} defines a linear problem with any displacement pair $(\tensori{v}{}_{\cell}, \tensori{v}{}_{\dCell})$, one can eliminate the equation from the system \eqref{eq_0017}, which defines the so-called \textit{reconstructed gradient} $\tensorii{G}{}_{\cell}(\tensori{v}{}_{\cell}, \tensori{v}{}_{\dCell})$ associated with any displacement pair $(\tensori{v}{}_{\cell}, \tensori{v}{}_{\dCell})$ that solves
%
%
%
\begin{equation}
    \label{eq_grad}
    \begin{aligned}
        \int_{\cell} \tensorii{G}{}_{\cell}(\tensori{v}{}_{\cell}, \tensori{v}{}_{\dCell}) : \tensorii{\tau}{}_{\cell}
        =
        \int_{\cell}  \nabla \tensori{v}{}_{\cell} : \tensorii{\tau}{}_{\cell}
        +
        \int_{\dCell} (\tensori{v}{}_{\dCell} - \tensori{v}{}_{\cell} \vert_{\dCell}) \cdot \tensorii{\tau}{}_{\cell} \vert_{\dCell} \cdot \tensori{n}{}
        &&
        \forall \tensorii{\tau}{}_{\cell} \in \stressSpaceCell
    \end{aligned}
\end{equation}

\paragraph{Stress tensor}

Equation \eqref{eq_0017:eq2} is also linear with the derivative of $\mecPotential_{\bodyLag}$ with respect to $\tensorii{G}{}_{\cell}$, and one can eliminate it as well from \eqref{eq_0017}, hence defining the stress tensor 
%
%
%
\begin{equation}
    \label{eq_stress}
    \begin{aligned}
        \int_{\cell} \tensorii{P}{}_{\cell} : \tensorii{\gamma}{}_{\cell}
        =
        \int_{\cell} \frac{\partial \mecPotential_{\bodyLag}}{\partial \tensorii{G}{}_{\cell}}  : \tensorii{\gamma}{}_{\cell}
        &&
        \forall \tensorii{\gamma}{}_{\cell} \in \gradSpaceCell
    \end{aligned}
\end{equation}
%
%
%
In particular, one notices that \eqref{eq_stress} holds in a strong sense if $\stressSpaceCell \subset \gradSpaceCell$.

\paragraph{Simplified form}

Now that \eqref{eq_0017:eq2} and \eqref{eq_0017:eq3} have been eliminated from the system, one considers the simplified functional \eqref{eq_simple} instead of \eqref{eq_0015}
%
%
%
\begin{equation}
    \label{eq_simple}
    \begin{aligned}
        J_{\cell}^{VW}
        = &
        \int_{\cell{}} \mecPotential{}_{\bodyLag{}}
        % \\
        % &
        % + \int_{\dCell{}} (\tensori{u}{}_{\dCell} - \tensori{u}{}_{\cell} \vert_{\dCell}) \cdot \tensorii{P}{}_{\cell} \vert_{\dCell{}} \cdot \tensori{n}{}
        % \\
        % &
        + \int_{\dCell} \frac{\beta}{2 h_{\cell}} \lVert \tensori{u}{}_{\dCell{}} - \tensori{u}{}_{\cell{}} \vert_{\dCell{}} \rVert^2
        % \\
        % &
        -
        \int_{\cell} \loadLag{} \cdot \tensori{u}{}_{\cell{}}
        -
        \int_{\neumannCell{}} \neumannCellLoad{} \cdot \tensori{u}{}_{\dCell{}}
    \end{aligned}
\end{equation}

\paragraph{Principle of virtual works}
%
%
%
The problem in primal form amounts to find the displacement pair $(\tensori{u}{}_{\cell}, \tensori{u}{}_{\dCell}) \in \hybridDisplacementSpaceCell$ verifying $\tensori{u}{}_{\dCell} = \dirichletLag$ on $\dirichletCell$,
such that for all kinematically admissible displacements pairs $(\delta \tensori{u}{}_{T}, \delta \tensori{u}{}_{\partial T}) \in \virtualHybridDisplacementSpaceCell$, the functional \eqref{eq_simple} is minimal, \textit{i.e.} such that
%
%
%
\begin{equation}
    \label{eq_0018}
    \begin{aligned}
        % d J_{\cell}^{\text{HW}}
        % = &
        % \frac{\partial J_{\cell}}{\partial \tensori{u}{}_{\cell}} \delta \tensori{u}{}_{\cell}
        % +
        % \frac{\partial J_{\cell}}{\partial \tensori{u}{}_{\dCell}} \delta \tensori{u}{}_{\dCell}
        % =
        \delta J_{\cell, \text{int}}^{VW} - \delta J_{\cell, \text{ext}}^{VW}
        =
        0
        % \\
        % = & \delta J_{\cell}^{\text{int}} + \delta J_{\cell}^{\text{ext}}
        % \\
        % = & 
        % \int_{T}
        % \tensorii{P}{}_{\cell}(\tensorii{G}{}_{\cell}(\tensori{u}{}_{\cell}, \tensori{u}{}_{\dCell}))
        % :
        % \tensorii{G}{}_{\cell}(\delta \tensori{u}{}_{\cell}, \delta \tensori{u}{}_{\dCell})
        % % \frac{\partial \mecPotential_{\bodyLag}}{\partial \tensorii{G}{}_T} : \delta \tensorii{G}{}_{T}
        % +
        % \int_{\partial T} (\beta / h_T)
        % (\tensori{u}{}_{\partial T} - \tensori{u}{}_{T} \vert_{\partial T})
        % % \tensori{Z}{}_{\dCell{}}
        % \cdot
        % (\delta \tensori{u}{}_{\partial T} - \delta \tensori{u}{}_{T} \vert_{\partial T})
        % % \delta \tensori{Z}{}_{\dCell{}}
        % \\
        % &
        % -
        % \int_{\partial T} \tensori{t}{}_N \cdot \delta \tensori{u}{}_{\partial T}
        % -
        % \int_{T} \tensori{f}{}_V \cdot \delta \tensori{u}{}_{T}
        % =
        % 0
    \end{aligned}
\end{equation}
%
%
%
with
%
%
%
\begin{subequations}
    \label{eq_0nonamemee}
        \begin{alignat}{3}
            \delta J_{\cell, \text{int}}^{VW} & = 
            \int_{T}
            \tensorii{P}{}_{\cell}(\tensorii{G}{}_{\cell}(\tensori{u}{}_{\cell}, \tensori{u}{}_{\dCell}))
            :
            \tensorii{G}{}_{\cell}(\delta \tensori{u}{}_{\cell}, \delta \tensori{u}{}_{\dCell})
            % \frac{\partial \mecPotential_{\bodyLag}}{\partial \tensorii{G}{}_T} : \delta \tensorii{G}{}_{T}
            +
            \int_{\dCell} (\beta / h_{\cell})
            % (\tensori{u}{}_{\dCell} - \tensori{u}{}_{\cell} \vert_{\dCell})
            % \tensori{Z}{}_{\dCell{}}
            \tensori{Z}{}_{\dCell}(\tensori{u}{}_{\cell}, \tensori{u}{}_{\dCell})
            \cdot
            % (\delta \tensori{u}{}_{\dCell} - \delta \tensori{u}{}_{\cell} \vert_{\dCell{}})
            % \delta \tensori{Z}{}_{\dCell{}}
            \tensori{Z}{}_{\dCell}(\delta \tensori{u}{}_{\cell}, \delta \tensori{u}{}_{\dCell})
            \\
            \delta J_{\cell, \text{ext}}^{VW} & = 
            \int_{\neumannCell} \neumannCellLoad{} \cdot \delta \tensori{u}{}_{\dCell{}}
            +
            \int_{T} \loadLag \cdot \delta \tensori{u}{}_{\cell}
    \end{alignat}
\end{subequations}
%
%
%
where we introduced the jump function $\tensori{Z}{}_{\dCell}$ :
%
%
%
\begin{equation}
    \begin{aligned}
        \tensori{Z}{}_{\dCell}(\tensori{v}{}_{\cell}, \tensori{v}{}_{\dCell}) = \tensori{v}{}_{\dCell} - \tensori{v}{}_{\cell} \vert_{\dCell}
        &&
        \forall (\tensori{v}{}_{\cell}, \tensori{v}{}_{\dCell}) \in \hybridDisplacementSpaceCell
    \end{aligned}
\end{equation}
%
%
%
In particular, one can readliy see the resemblance of \eqref{eq_0nonamemee} with
\eqref{eq_virtual_works_0},
% the ususal formulation of the principle of virtual works
where the so called \textit{reconstructed gradient} $\tensorii{G}{}_{\cell}(\tensori{u}{}_{\cell}, \tensori{u}{}_{\dCell})$ plays the role of the usual displacement Lagrangian gradient $\nabla \tensori{u}{}_{\cell}$, and where an additional \textit{stabilization term} corresponding to a traction energy on the boundary has been added to account for the penalization of the displacement jump on $\dCell$ through $\tensori{Z}{}_{\dCell}$ (or, equivalently, to account for the infinitésimale interface that lays between the bulk domain and its boundary).
Equations \eqref{eq_simple}, \eqref{eq_grad} and \eqref{eq_stress} define the mechanical problem to solve at the cell level for Hybrid Discontinuous Galerkin methods, and \eqref{eq_0018} describes the weak form of these equations.