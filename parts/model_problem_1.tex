\label{sec_model_problem}

Paragraph~\ref{sec:Hu_Washizu_functional}
introduces the classical Hu–Washizu functional to describe the
quasi-static equilibrium of a body submitted to external load and the
main notations used in this paper. For the sake of simplicity, the body
is assumed hyper-elastic in this section.

Paragraph~\ref{sec:HHO} introduces the key idea
of the HHO method, which is to divide the domain in arbitrary subdomains
connected by cohesive interfaces and to apply the Hu–Washizu functional
to each sub-domains.

\subsection{The standard Hu–Washizu functional}
\label{sec:Hu_Washizu_functional}

Let $\bodyEul$ a solid body with boundary $\dBodyEul$, that deforms in the current configuration at some time $t > 0$ under a body force $\loadEul$ acting in $\bodyEul$, a prescribed displacement $\dirichletEul$ on the Dirichlet boundary $\dirichletBoundaryEul$, and a contact load $\neumannEul{}$ on the Neumann boundary $\neumannBoundaryEul$.
At the time $t = 0$, the solid is in the reference configuration $\bodyLag$ with respective Dirichlet and Neumann boundaries $\dirichletBoundaryLag$ and $\neumannBoundaryLag$.
It is subjected to the body force $\loadLag$ in $\bodyLag{}$, the prescribed displacement $\dirichletLag$ on $\dirichletBoundaryLag{}$, and the contact load $\neumannLag{}$ on $\neumannBoundaryLag{}$.
Let $\tensori{\Phi}$ the mapping that takes a point $\tensori{x} \in \bodyLag$ from the reference configuration to its coordinates $\tensori{x}{}_t \in \bodyEul$ in the current configuration, and $\tensori{u}$ the displacement field in $\bodyLag{}$ such that $\tensori{\Phi} = \tensori{I}{}_d + \tensori{u}$ where $\tensori{I}{}_d$ denotes the identity function.
%
%
%

Let consider $\cell$ an arbitrary subset of $\bodyLag{}$. From a mechanical standpoint, $\bodyLag{}$ might be a composite material, and $\cell$ represents the area occupied by some material defined by the mechanical energy potential $\mecPotential{}_{\cell}$ that depends on the gradient of the transformation gradient $\tensorii{F}{}_{\cell}
% := \nabla \tensori{\Phi}
= \tensorii{1} + \nabla \tensori{u}{}_{\cell}
$ and on a set of state variables $V_{int}$ that relate to the positivity requirement on the dissipation (in the case of an elasto-visco-plastic behaviour for instance). In the absence of dissipative process (\textit{e.g.} for (hyper)-elasticity), there is no internal state variable and $V_{int} = \emptyset$. In the following, in order to simplify developments and without loss of generality, let consider the latter case.

%
%
%
The internal energy of the solid $\mecPotential{}$ depends on the gradient of the transformation gradient $\tensorii{F}
% := \nabla \tensori{\Phi}
= \tensorii{1} + \nabla \tensori{u}
$ and on a set of state variables $V_{int}$ that relate to the positivity requirement on the dissipation (in the case of an elasto-visco-plastic behaviour for instance). In the absence of dissipative process (\textit{e.g.} for (hyper)-elasticity), there is no internal state variable and $V_{int} = \emptyset$. In the following, in order to simplify developments and without loss of generality, let consider the latter case.
%
%
%

In order to devise a smooth transition from the Hu-Washizu formulation in the continuous case to the Hybrid discontinuous one, let work on an an arbitrary subset $\cell$ in $\bodyLag{}$. In particular, if $\cell = \bodyLag{}$, one retrieves the model problem to solve for the whole body. let consider $\cell \subset \bodyLag$ an arbitrary open subset of the solid body, with boundary $\dCell$
which is split into an eventual Dirichlet boundary $\dirichletCell \subset \dirichletBoundaryLag$ subjected to an imposed displacement $\dirichletLag$ if $\cell$ shares a boundary with $\dirichletBoundaryLag$ and into the Neumann Boundary $\neumannCell \subset \neumannBoundaryLag \cup \bodyLag$ with contact load $\neumannCellLoad$ such that
%
% 
% 
\begin{equation}
    \label{eq_contact_force}
    \begin{aligned}
        \neumannCellLoad = 
        \left\{
            \begin{array}{ll}
                \tensori{t}{}_{\bodyLag \backslash \cell \rightarrow \cell} & \mbox{on } \neumannCell \cap \bodyLag \backslash \cell
                \\
                \neumannLag & \mbox{on } \neumannCell \cap \neumannBoundaryLag
            \end{array}
        \right.
    \end{aligned}
\end{equation}
% 
% 
%
with $\tensori{t}{}_{\bodyLag \backslash \cell \rightarrow \cell}$ the contact force applied by the surrounding part of the body $\bodyLag$ onto the subset $\cell$, and $\dirichletCell \cap \neumannCell = \emptyset$.
%
% 
% 
\begin{figure}[H]
    \centering
    \includegraphics[width=7.cm]{img/setting.png}
    \caption{schematic representation of the model problem}
    \label{fig_setting}
\end{figure}
%
% 
% 
% Let $\tensori{\Phi}{}_{\cell}$ the restriction of $\tensori{\Phi}$ to ${\cell}$.
Let $\tensori{u}{}_{\cell}$ the displacement field in $\displacementSpaceCell$ the space of all kinematically admissible displacement fields in $\cell$.
% , such that $\tensori{\Phi}{}_{\cell} = \tensori{I}{}_d + \tensori{u}{}_{\cell}{}$ with $\tensori{I}{}_d$ the identity function.
The three fields Hu–Washizu principle characterizes the equilibrium of the body by the minimization of the functional $L_{\cell}^{HW}$ with respect to the displacement field $\tensori{u}{}_{\cell} \in \displacementSpaceCell$ verifying $\tensori{u}{}_{\cell} \vert_{\dirichletCell{}} = \dirichletLag$, the displacement gradient field $\tensorii{G}{}_{\cell} \in \gradSpaceCell$, and the first Piola-Kirchoff stress tensor field $\tensorii{P}{}_{\cell} \in \stressSpaceCell$, such that
%
%
%
\begin{equation}
    \label{eq_hu_washizu}
        L_{\cell}^{HW}
        % (\tensori{u}{}_{\cell}, \tensorii{G}{}_{\cell}, \tensorii{P}{}_{\cell})
        =
        \int_{\cell} \mecPotential{}_{\cell} + (\nabla \tensori{u}{}_{\cell} - \tensorii{G}{}_{\cell}) : \tensorii{P}{}_{\cell}
        -
        \int_{\cell} \loadLag \cdot \tensori{u}{}_{\cell}
        -
        \int_{\neumannCell} \neumannCellLoad \cdot \tensori{u}{}_{\cell} \vert_{\dCell}
\end{equation}
%
%
%
where $\gradSpaceCell$ (respectively $\stressSpaceCell$) denotes the space of all statically admissible strain (respectively stress) fields in $\cell$.
Deriving \eqref{eq_hu_washizu} with respect to all variables yields the weak formulation of the problem
%
% 
% 
\begin{subequations}
    \label{eq_hu_washizu_derivative}
        \begin{alignat}{3}
            \frac{\partial L_{\cell}^{HW}}{\partial \tensori{u}{}_{\cell}} \delta \tensori{u}{}_{\cell}
            = & \int_{\cell} \tensorii{P}{}_{\cell} : \nabla \delta \tensori{u}{}_{\cell}
            -
            \int_{\cell} \tensori{f}{}_V \cdot \delta \tensori{u}{}_{\cell}
            -
            \int_{\dCell} \tensori{t}{}_{\neumannCell} \cdot \delta \tensori{u}{}_{\cell} \vert_{\dCell}
            &&
            \ \ \ \ \ \ \ \ 
            &&
            \forall \delta \tensori{u}{}_{\cell}
            \in \virtualDisplacementSpaceCell
        \label{eq_hu_washizu_derivative:eq0}
        \\
            \frac{\partial L_{\cell}^{HW}}{\partial \tensorii{G}{}_{\cell}} \delta \tensorii{G}{}_{\cell}
            = &
            \int_{\cell} (\frac{\partial \mecPotential_{\cell}}{\partial \tensorii{G}{}_{\cell}} - \tensorii{P}{}_{\cell}) : \delta \tensorii{G}{}_{\cell}
            &&
            \ \ \ \ \ \ \ \ 
            &&
            \forall \delta \tensorii{G}{}_{\cell}
            \in \gradSpaceCell
        \label{eq_hu_washizu_derivative:eq2}
        \\
            \frac{\partial L_{\cell}^{HW}}{\partial \tensorii{P}{}_{\cell}} \delta \tensorii{P}{}_{\cell}
            = & \int_{\cell} (\nabla \tensori{u}{}_{\cell} - \tensorii{G}{}_{\cell} ) : \delta \tensorii{P}{}_{\cell}
            &&
            \ \ \ \ \ \ \ \ 
            &&
            \forall \delta \tensorii{P}{}_{\cell}
            \in \stressSpaceCell
        \label{eq_hu_washizu_derivative:eq3}
    \end{alignat}
\end{subequations}

% 
Let $\tensorii{G}{}_{\cell}
% \in \gradSpaceCell
$ the displacement gradient field in $\gradSpaceCell$ the space of all statically admissible displacement gradient fields in $\cell$,
and $\tensorii{F}{}_{\cell} = \tensorii{1} + \tensorii{G}{}_{\cell}$ the transformation gradient, where $\nabla$ denotes the Lagrangian nabla operator.
% and $\tensorii{F}{}_{\cell} := \nabla \tensori{\Phi}{}_{\cell} = \tensorii{1} + \nabla \tensori{u}$ the transformation gradient, where $\nabla$ denotes the Lagrangian nabla operator.
%
Let $\mecPotential{}_{\cell} (\tensorii{F}{}_{\cell}, \internaleStateVariables)$ the mechanical energy potential in $\cell$ that depends on the transformation gradient $\tensorii{F}{}_{\cell}$ and possibly on a set of internal state variables $\internaleStateVariables$.
% where $\nabla$ denotes the Lagrangian nabla operator.
%
% Let $\tensorii{P}{}_{\cell} \in \stressSpaceCell$ the first Piola-Kirchoff stress tensor deriving from the expression of the mechanical energy potential, with $\stressSpaceCell$ the space of all statically admissible stress fields in $\cell$.
Let $\tensorii{P}{}_{\cell}
% \in \stressSpaceCell
$ the first Piola-Kirchoff stress tensor in $\stressSpaceCell$ the space of all statically admissible stress fields in $\cell$, that depends on
% $\mecPotential{}_{\cell, \tensori{F}{}}$
the derivative of the mechanical energy potential $\mecPotential{}_{\cell}$ with respect to $\tensorii{F}{}_{\cell}$ (or equivalently, with respect to $\tensorii{G}{}_{\cell}$).

%
%
%
The three fields Hu–Washizu principle characterizes the equilibrium of the body by the minimization of the functional $L_{\bodyLag}^{HW}$ with respect to the displacement field $\tensori{u}{} \in U(\bodyLag)$ verifying $\tensori{u}{} \vert_{\dirichletBoundaryLag{}} = \dirichletLag$, the displacement gradient field $\tensorii{G} \in G(\bodyLag)$, and the first Piola-Kirchoff stress tensor field $\tensorii{P} \in S(\bodyLag)$, such that

Let $d \in  \{1, 2, 3\}$ the euclidean dimension of the cartesian space $\mathbb{R}{}^{d}$. Let $\bodyEul \subset \mathbb{R}{}^{d}$ a solid body with boundary $\dBodyEul \subset \mathbb{R}{}^{d - 1}$, that deforms in the current configuration at some time $t > 0$ under the body forces $\loadEul$. It is subjected to a prescribed displacement $\dirichletEul$ on the Dirichlet boundary $\dirichletBoundaryEul$, and to a contact load $\neumannEul{}$ on the Neumann boundary $\neumannBoundaryEul$
% , such that $\dBodyEul = \dirichletBoundaryEul \cup \neumannBoundaryEul$ and $\dirichletBoundaryEul{} \cap \neumannBoundaryEul = \emptyset$.

The initial configuration of the body at time $t = 0$ (see Figure \ref{fig_setting}) is denoted $\bodyLag \subset \mathbb{R}{}^{d}$ with respective Dirichlet and Neumann boundaries $\dirichletBoundaryLag$ and $\neumannBoundaryLag$. It is subjected to body forces $\loadLag$, an imposed displacement $\dirichletLag$ on $\dirichletBoundaryLag$ and contact force $\neumannLag$ on $\neumannBoundaryLag$. The transformation mapping $\tensori{\Phi}$ takes a point $\tensori{x} \in \bodyLag$ from the initial configuration to $\tensori{x}{}_t \in \bodyEul$ in the current configuration.



Let $\tensori{\Phi}{}_{\cell}$ the restriction of $\tensori{\Phi}$ to ${\cell}$, and $\tensori{u}{}_{\cell} \in \displacementSpaceCell$ the displacement field in $\cell$ such that $\tensori{\Phi}{}_{\cell} = \tensori{I}{}_d + \tensori{u}{}_{\cell}{}$ with $\tensori{I}{}_d$ the identity function, where the notation $\displacementSpaceCell$ denotes the space of all kinematically admissible displacement fields in $\cell$.
% 
Let $\tensorii{G}{}_{\cell}
% \in \gradSpaceCell
$ the displacement gradient field in $\gradSpaceCell$ the space of all statically admissible displacement gradient fields in $\cell$,
and $\tensorii{F}{}_{\cell} := \nabla \tensori{\Phi}{}_{\cell} = \tensorii{1} + \tensorii{G}{}_{\cell}$ the transformation gradient, where $\nabla$ denotes the Lagrangian nabla operator.
% and $\tensorii{F}{}_{\cell} := \nabla \tensori{\Phi}{}_{\cell} = \tensorii{1} + \nabla \tensori{u}$ the transformation gradient, where $\nabla$ denotes the Lagrangian nabla operator.
%
Let $\mecPotential{}_{\cell} (\tensorii{F}{}_{\cell}, \internaleStateVariables)$ the mechanical energy potential in $\cell$ that depends on the transformation gradient $\tensorii{F}{}_{\cell}$ and possibly on a set of internal state variables $\internaleStateVariables$.
% where $\nabla$ denotes the Lagrangian nabla operator.
%
% Let $\tensorii{P}{}_{\cell} \in \stressSpaceCell$ the first Piola-Kirchoff stress tensor deriving from the expression of the mechanical energy potential, with $\stressSpaceCell$ the space of all statically admissible stress fields in $\cell$.
Let $\tensorii{P}{}_{\cell}
% \in \stressSpaceCell
$ the first Piola-Kirchoff stress tensor in $\stressSpaceCell$ the space of all statically admissible stress fields in $\cell$, that depends on
% $\mecPotential{}_{\cell, \tensori{F}{}}$
the derivative of the mechanical energy potential $\mecPotential{}_{\cell}$ with respect to $\tensorii{F}{}_{\cell}$ (or equivalently, with respect to $\tensorii{G}{}_{\cell}$).
% % where
% % %
% % %
% % %
% % \begin{equation}
% %     \label{eq_stress_def}
% %     \begin{aligned}
% %         \mecPotential{}_{\cell, \tensori{F}{}}
% %         :=
% %         \frac{\partial \mecPotential{}_{\cell}}{\partial \tensorii{F}{}_{\cell}}
% %         =
% %         \frac{\partial \mecPotential{}_{\cell}}{\partial \tensorii{G}{}_{\cell}} 
% %         \frac{\partial \tensorii{G}{}_{\cell}}{\partial \tensorii{F}{}_{\cell}}
% %         =
% %         \frac{\partial \mecPotential{}_{\cell}}{\partial \tensorii{G}{}_{\cell}}
% %     \end{aligned}
% \end{equation}
%
%
%
% 
The equilibrium of the body $\cell$ is reached for the displacement gradient field $\tensorii{G}{}_{\cell} \in \gradSpaceCell$, the first Piola-Kirchoff stress field $\tensorii{P}{}_{\cell} \in \stressSpaceCell$ and the displacement field $\tensori{u}{}_{\cell} \in \displacementSpaceCell$ verifying $\tensori{u}{}_{\cell} \vert_{\dirichletCell} = \dirichletLag$ on $\dirichletCell$ minimizing the three field Hu–Washizu energy functional $L_{\cell}^{HW}$
%
%
%
\begin{equation}
    \label{eq_hu_washizu}
        L_{\cell}^{HW}
        % (\tensori{u}{}_{\cell}, \tensorii{G}{}_{\cell}, \tensorii{P}{}_{\cell})
        =
        \int_{\cell} \mecPotential{}_{\cell} + (\nabla \tensori{u}{}_{\cell} - \tensorii{G}{}_{\cell}) : \tensorii{P}{}_{\cell}
        -
        \int_{\cell} \loadLag \cdot \tensori{u}{}_{\cell}
        -
        \int_{\neumannCell} \neumannCellLoad \cdot \tensori{u}{}_{\cell} \vert_{\dCell}
\end{equation}
%
%
%
Deriving \eqref{eq_hu_washizu} with respect to all variables yields the weak formulation of the problem
%
% 
% 
\begin{subequations}
    \label{eq_hu_washizu_derivative}
        \begin{alignat}{3}
            \frac{\partial L_{\cell}^{HW}}{\partial \tensori{u}{}_{\cell}} \delta \tensori{u}{}_{\cell}
            = & \int_{\cell} \tensorii{P}{}_{\cell} : \nabla \delta \tensori{u}{}_{\cell}
            -
            \int_{\cell} \tensori{f}{}_V \cdot \delta \tensori{u}{}_{\cell}
            -
            \int_{\dCell} \tensori{t}{}_{\neumannCell} \cdot \delta \tensori{u}{}_{\cell} \vert_{\dCell}
            &&
            \ \ \ \ \ \ \ \ 
            &&
            \forall \delta \tensori{u}{}_{\cell}
            \in \virtualDisplacementSpaceCell
        \label{eq_hu_washizu_derivative:eq0}
        \\
            \frac{\partial L_{\cell}^{HW}}{\partial \tensorii{G}{}_{\cell}} \delta \tensorii{G}{}_{\cell}
            = &
            \int_{\cell} (\frac{\partial \mecPotential_{\cell}}{\partial \tensorii{G}{}_{\cell}} - \tensorii{P}{}_{\cell}) : \delta \tensorii{G}{}_{\cell}
            &&
            \ \ \ \ \ \ \ \ 
            &&
            \forall \delta \tensorii{G}{}_{\cell}
            \in \gradSpaceCell
        \label{eq_hu_washizu_derivative:eq2}
        \\
            \frac{\partial L_{\cell}^{HW}}{\partial \tensorii{P}{}_{\cell}} \delta \tensorii{P}{}_{\cell}
            = & \int_{\cell} (\nabla \tensori{u}{}_{\cell} - \tensorii{G}{}_{\cell} ) : \delta \tensorii{P}{}_{\cell}
            &&
            \ \ \ \ \ \ \ \ 
            &&
            \forall \delta \tensorii{P}{}_{\cell}
            \in \stressSpaceCell
        \label{eq_hu_washizu_derivative:eq3}
    \end{alignat}
\end{subequations}
%
%
%
where $\cdot \ \vert_{\dCell}$ denotes the trace operator on $\dCell$, and $\virtualDisplacementSpaceCell$ the space of all kinematically admissible virtual displacements in $\cell$.
Since the subset $\cell$ is arbitrary, \eqref{eq_hu_washizu_derivative} holds true in a distributional sense, and by applying the divergence theorem on \eqref{eq_hu_washizu_derivative:eq0}, one obtains the strong formulation of problem \eqref{eq_hu_washizu}
%
% 
% 
\begin{subequations}
\label{eq_model_problem}
    \begin{alignat}{2}
    % \tensorii{F} - \nabla \tensori{u} & = \tensorii{1} \quad && \text{in } \Omega_{0} \label{eq_model_problem:eq1}
    \tensorii{G}{}_{\cell} - \nabla \tensori{u}{}_{\cell} & = 0 \quad
    &&
    \text{in } \cell
    \label{eq_model_problem:eq1}
    \\
    % \tensorii{P} - \frac{\partial \mecPotential}{\partial \tensorii{F}} & = 0 \quad && \text{in } \Omega_{0} \label{eq_model_problem:eq2}
    \tensorii{P}{}_{\cell} - \frac{\partial \mecPotential{}_{\cell}}{\partial \tensorii{G}{}_{\cell}} & = 0 \quad
    &&
    \text{in } \cell
    \label{eq_model_problem:eq2}
    \\
    \nabla \cdot \tensorii{P}{}_{\cell} + \loadLag & = 0 \quad
    &&
    \text{in } \cell
    \label{eq_model_problem:eq3}
    \\
    \tensori{u}{}_{\cell} \vert_{\dirichletCell} & = \dirichletLag \quad
    &&
    \text{on } \dirichletCell
    \label{eq_model_problem:eq4}
    \\
    \tensorii{P}{}_{\cell} \cdot \tensori{n} & = \neumannCellLoad \quad
    &&
    \text{on } \neumannCell
    \label{eq_model_problem:eq5}
\end{alignat}
\end{subequations}
%
% 
% 
where $\tensori{n}$ denotes the unit ouward normal vector on $\dCell$.
One readily identifies \eqref{eq_model_problem:eq1} with the displacement strain relation, \eqref{eq_model_problem:eq2} with the constitutive equation, \eqref{eq_model_problem:eq3} with the conservation of momentum, \eqref{eq_model_problem:eq5} with the traction force continuity and \eqref{eq_model_problem:eq4} with Dirichlet boundary conditions.
%
%
%
Considering now that $\tensorii{G}{}_{\cell}$ and $\tensorii{P}{}_{\cell}$ are not unknowns of the problem, \textit{i.e.} by explicitly eliminating both \eqref{eq_hu_washizu_derivative:eq2} and \eqref{eq_hu_washizu_derivative:eq3} from \eqref{eq_hu_washizu_derivative}, \eqref{eq_hu_washizu} simplifies as : find the displacement field $\tensori{u}{}_{\cell} \in \displacementSpaceCell$ verifying $\tensori{u}{}_{\cell} \vert_{\dirichletCell} = \dirichletLag$ on $\dirichletCell$ that minimizes the energy functional $L_{\cell}^{VW}$
%
%
%
\begin{equation}
    \label{eq_virtual_works}
    \begin{aligned}
        L_{\cell}^{VW}
        % (\tensori{u}{}_{\cell})
        =
        \int_{\cell} \mecPotential{}_{\cell}
        -
        \int_{\cell} \loadLag \cdot \tensori{u}{}_{\cell}
        -
        \int_{\neumannCell} \neumannCellLoad \cdot \tensori{u}{}_{\cell}
    \end{aligned}
\end{equation}
%
%
%
Deriving \eqref{eq_virtual_works} with respect to the primal unknown $\tensori{u}{}_{\cell}$ yields the weak form of the problem, which is the notorious principle of virtual work
%
%
%
\begin{equation}
    \label{eq_virtual_works_1}
    \begin{aligned}
        % \frac{\partial L_{\cell}}{\partial \tensori{u}{}_{\cell}} \delta \tensori{u}{}_{\cell}
        \delta L_{\cell}^{VW}
        % =
        % \frac{\partial L_{\cell}^{VW}}{\partial \tensori{u}{}_{\cell}} \delta \tensori{u}{}_{\cell}
        & =
        \int_{\cell}
        % \frac{\partial \mecPotential{}_{\cell}}{\partial \nabla \tensori{u}{}_{\cell}} : \nabla \delta \tensori{u}{}_{\cell}
        \tensorii{P}{}_{\cell} : \nabla \delta \tensori{u}{}_{\cell}
        -
        \int_{\cell} \loadLag \cdot \delta \tensori{u}{}_{\cell}
        -
        \int_{\neumannCell} \neumannCellLoad \cdot \delta \tensori{u}{}_{\cell} \vert_{\dCell{}}
        &&
        \forall \delta \tensori{u}{}_{\cell}
        \in \virtualDisplacementSpaceCell
    \end{aligned}
\end{equation}
%
%
%
that relates to the strong problem defined by equation \eqref{eq_model_problem:eq3}, \eqref{eq_model_problem:eq4}, \eqref{eq_model_problem:eq5} only, where \eqref{eq_model_problem:eq1}, \eqref{eq_model_problem:eq2} are taken to be granted.
%
%
%
% ,and $\cdot \ \vert_{\dCell}$ is the trace operator on $\dCell$.
% The equilibrium of the body $\cell$ corresponding to problem \eqref{eq_model_problem} where equations \eqref{eq_model_problem:eq1} and \eqref{eq_model_problem:eq2} are enforced strongly is reached for the displacement field $\tensori{u}{}_{\cell} \in \displacementSpaceCell$ verifying $\tensori{u}{}_{\cell} \vert_{\dirichletCell} = \dirichletLag$ on $\dirichletCell$ and minimizing the energy functional $L_{\cell}$:
% %
% % 
% % 
% \begin{equation}
%     \label{eq_virtual_works}
%     \begin{aligned}
%         L_{\cell}
%         % (\tensori{u}{}_{\cell})
%         =
%         \int_{\cell} \mecPotential{}_{\cell}
%         -
%         \int_{\cell} \loadLag \cdot \tensori{u}{}_{\cell}
%         -
%         \int_{\neumannCell} \neumannCellLoad \cdot \tensori{u}{}_{\cell}
%     \end{aligned}
% \end{equation}
% %
% % 
% % 
% The energy functional \eqref{eq_virtual_works} of the equilibrium of $T$ depends on the single displacement unknown, and is the one at the foundation of the notorious principle of virtual works; indeed, deriving \eqref{eq_virtual_works} with respect to $\tensori{u}{}_{\cell}$ and using both \eqref{eq_model_problem:eq1} and \eqref{eq_model_problem:eq2} yields :
% %
% % 
% % 
% \begin{equation}
%     \label{eq_virtual_works_1}
%     \begin{aligned}
%         % \frac{\partial L_{\cell}}{\partial \tensori{u}{}_{\cell}} \delta \tensori{u}{}_{\cell}
%         d L_{\cell} = \frac{\partial L_{\cell}}{\partial \tensori{u}{}_{\cell}} \delta \tensori{u}{}_{\cell}
%         & =
%         \int_{\cell}
%         % \frac{\partial \mecPotential{}_{\cell}}{\partial \nabla \tensori{u}{}_{\cell}} : \nabla \delta \tensori{u}{}_{\cell}
%         \tensorii{P}{}_{\cell} : \nabla \delta \tensori{u}{}_{\cell}
%         -
%         \int_{\cell} \loadLag \cdot \delta \tensori{u}{}_{\cell}
%         -
%         \int_{\neumannCell} \neumannCellLoad \cdot \delta \tensori{u}{}_{\cell} \vert_{\dCell{}}
%     \end{aligned}
% \end{equation}
% %
% % 
% % 
% with $\displacementSpaceCell{} = H^1(\cell, \mathbb{R}^{d})$. If $\tensorii{G}{}_{\cell}$ and $\tensorii{P}{}_{\cell}$ are unknowns of the problem, one obtains the three-field Hu–Washizu functional $L_{\cell}$, for the displacement field $\tensori{u}{}_{\cell} \in \displacementSpaceCell$ verifying $\tensori{u}{}_{\cell} \vert_{\dirichletCell} = \dirichletLag$ on $\dirichletCell$ such that :
% % 
% % 
% %
% % \begin{equation}
% % \label{eq_hu_washizu}
% %     L_{\cell}
% %     % (\tensori{u}{}_{\cell}, \tensorii{G}{}_{\cell}, \tensorii{P}{}_{\cell})
% %     =
% %     \int_{\cell} \mecPotential{}_{\cell} + (\nabla \tensori{u}{}_{\cell} - \tensorii{G}{}_{\cell}) : \tensorii{P}{}_{\cell}
% %     -
% %     \int_{\cell} \loadLag \cdot \tensori{u}{}_{\cell}
% %     -
% %     \int_{\neumannCell} \neumannCellLoad \cdot \tensori{u}{}_{\cell}
% % \end{equation}
% %
% % 
% % 
% Deriving \eqref{eq_hu_washizu} with respect to all variables of the problem expresses problem \eqref{eq_model_problem} in a weak sense :
% %
% % 
% % 
% \begin{subequations}
%     \label{eq_hu_washizu_derivative}
%         \begin{alignat}{3}
%             \frac{\partial L_{\cell}}{\partial \tensori{u}{}_{\cell}} \delta \tensori{u}{}_{\cell}
%             = & \int_{\cell} \tensorii{P}{}_{\cell} : \nabla \delta \tensori{u}{}_{\cell}
%             -
%             \int_{\cell} \tensori{f}{}_V \cdot \delta \tensori{u}{}_{\cell}
%             -
%             \int_{\dCell} \tensori{t}{}_{\neumannCell} \cdot \delta \tensori{u}{}_{\cell} \vert_{\dCell}
%             &&
%             \ \ \ \ \ \ \ \ 
%             &&
%             \forall \delta \tensori{u}{}_{\cell}
%             \in \displacementSpaceCell
%         \label{eq_hu_washizu_derivative:eq0}
%         \\
%             \frac{\partial L_{\cell}}{\partial \tensorii{G}{}_{\cell}} \delta \tensorii{G}{}_{\cell}
%             = &
%             \int_{\cell} (\frac{\partial \mecPotential_{\cell}}{\partial \tensorii{G}{}_{\cell}} - \tensorii{P}{}_{\cell}) : \delta \tensorii{G}{}_{\cell}
%             &&
%             \ \ \ \ \ \ \ \ 
%             &&
%             \forall \delta \tensorii{G}{}_{\cell}
%             \in \gradSpaceCell
%         \label{eq_hu_washizu_derivative:eq2}
%         \\
%             \frac{\partial L_{\cell}}{\partial \tensorii{P}{}_{\cell}} \delta \tensorii{P}{}_{\cell}
%             = & \int_{\cell} (\nabla \tensori{u}{}_{\cell} - \tensorii{G}{}_{\cell} ) : \delta \tensorii{P}{}_{\cell}
%             &&
%             \ \ \ \ \ \ \ \ 
%             &&
%             \forall \delta \tensorii{P}{}_{\cell}
%             \in \stressSpaceCell
%         \label{eq_hu_washizu_derivative:eq3}
%     \end{alignat}
% \end{subequations}
% %
% % 
% % 
% where the two supplementary equations \eqref{eq_hu_washizu_derivative:eq2} and \eqref{eq_hu_washizu_derivative:eq3} account for the weak formulation of \eqref{eq_model_problem:eq1} and \eqref{eq_model_problem:eq2}, and $\displacementSpaceCell{} = H^1(\cell, \mathbb{R}^{d})$ and $\gradSpaceCell = \stressSpaceCell = L^2(\cell, \mathbb{R}^{d \times d})$.

%
%
%

Assuming that $\cell$ is made out of a partition of $N > 0$ distinct media $\matI \subset \cell$ with respective energy potentials $\mecPotential{}_{\matI}$, the problem writes : for each medium $\matI$, find $\tensori{u}{}_{\matI} \in \displacementSpaceMatI$ verifying
$\tensori{u}{}_{\matI} \vert_{\dirichletMatI} = \dirichletLag$ on $\dirichletMatI$,
the displacement gradient field $\tensorii{G}{}_{\matI} \in \gradSpaceMatI$ and the first Piola-Kirchoff stress field $\tensorii{P}{}_{\matI} \in \stressSpaceMatI$, that minimize the functional
% 
% 
% 
\begin{equation}
\label{eq_hu_washizu_composite}
\begin{aligned}
    L_{\cell}^{HW}
    = & \sum_{1 \leq i \leq N} \int_{\matI} \mecPotential{}_{\matI} + (\nabla \tensori{u}{}_{\matI} - \tensorii{G}{}_{\matI}) : \tensorii{P}{}_{\matI}
    -
    \int_{\matI} \loadLag \cdot \tensori{u}{}_{\matI}
    -
    \int_{\neumannMatI \cap \neumannCell} \neumannCellLoad \cdot \tensori{u}{}_{\matI}
\end{aligned}
\end{equation}
%
% 
% 
where for all $1 \leq i \neq j \leq N$, the external forces corresponding to the traction applied by $\matI$ onto $\partial T_j \cap \dMatI$ and to that of $T_j$ onto $\dMatI \cap \partial T_j$ are direclty eliminated by continuity of the traction force across $\partial T_j \cap \dMatI$.
%
%
%

If the unknown fields are continuous in $\cell$, one has :
$
\tensori{u}{}_{\matI} = \tensori{u}{}_{\cell} \vert_{\matI},
\tensorii{G}{}_{\matI} = \tensorii{G}{}_{\cell} \vert_{\matI},
\tensorii{P}{}_{\matI} = \tensorii{P}{}_{\cell} \vert_{\matI}
$,
and the problem simplifies in : find $\tensori{u}{}_{\cell} \in \displacementSpaceCell$
verifying $\tensori{u}{}_{\cell} \vert_{\dirichletCell} = \dirichletLag$ on $\dirichletCell$,
the displacement gradient field $\tensorii{G}{}_{\cell} \in \gradSpaceCell$
and the first Piola-Kirchoff stress field $\tensorii{P}{}_{\cell} \in \stressSpaceCell$ that minimize
%
% 
% 
\begin{equation}
\label{eq_hu_washizu_composite_continuous}
\begin{aligned}
    L_{\cell}^{HW}
    = & \sum_{1 \leq i \leq N} \int_{\matI} \mecPotential{}_{\matI} + (\nabla \tensori{u}{}_{\cell} - \tensorii{G}{}_{\cell}) : \tensorii{P}{}_{\cell}
    -
    \int_{\matI} \loadLag \cdot \tensori{u}{}_{\cell}
    -
    \int_{\neumannCell{}} \neumannCellLoad \cdot \tensori{u}{}_{\cell} \vert_{\dCell{}}
\end{aligned}
\end{equation}

In particular, assuming $\cell = \bodyLag$, one obtains the mechanical problem to solve for the whole body $\bodyLag$.