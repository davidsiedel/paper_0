\section{The model problem}
\label{sec_model_problem}

Paragraph~\ref{sec:Hu_Washizu_functional}
introduces the classical Hu–Washizu functional to describe the
quasi-static equilibrium of a body submitted to external load and the
main notations used in this paper. For the sake of simplicity, the body
is assumed hyper-elastic in this section.

Paragraph~\ref{sec:Hu_Washizu_functional}
introduces the classical Hu–Washizu functional to describe the
quasi-static equilibrium of a body submitted to external load and the
main notations used in this paper. For the sake of simplicity, the body
is assumed hyper-elastic in this section.

Paragraph~\ref{sec:HHO} introduces the key idea
of the HHO method, which is to divide the domain in arbitrary subdomains
connected by cohesive interfaces and to apply the Hu–Washizu functional
to each sub-domains.

\subsection{The standard Hu–Washizu lagragian}
\label{sec_Hu_Washizu_functional}

This paragraph introduces the standard Hu–Washizu three field
principle. For the sake of simplicity, and without loss of generality,
we consider the case of an hyperelastic material. Extensions to
mechanical behaviours with internal state variables is treated in
classical textbooks of computational mechanics. We will treat this
extension in the Section~\ref{sec_implementation} discussing the
numerical implementation of the Hybrid High Order method and in
Section~\ref{sec_numerical_examples} which provides several examples in
plasticity.

% \subsubsection{Description of the mechanical problem and notations}

\paragraph{Solid body}

Let us consider a solid body whose reference configuration is denoted
$\bodyLag$. At a given time $t > 0$, the body is in the current
configuration $\bodyEul$.

\paragraph{Mechanical loading}

The body is assumed to be submitted to a body force $\loadEul$ acting
in $\bodyEul$, a prescribed displacement $\dirichletEul$ on the
Dirichlet boundary $\dirichletBoundaryEul$, and a contact load
$\neumannEul{}$ on the Neumann boundary $\neumannBoundaryEul$.

\paragraph{Deformation}

The transformation mapping 
$\tensori{\Phi}$ takes a point from the reference configuration $\bodyLag$ to the current
configuration $\bodyEul$ such that
%
%
%
\begin{equation}
    \tensori{\Phi}\paren{\tensori{X}} = \tensori{x} = \tensori{X}+\tensori{u}\paren{X}
\end{equation}
%
%
%
where $\tensori{X}$, $\tensori{x}$ and $\tensori{u}$ denote respectively
the position in the reference configuration $\bodyLag$, the position
in the current configuration $\bodyEul$ and the displacement.

\paragraph{Deformation gradient, gradient of the displacement}

The deformation gradient $\tensorii{F}$ is defined as:
%
%
%
\begin{equation}
    \tensorii{F} = \nabla \tensori{\Phi} = \tensorii{I} + \tensorii{G}
\end{equation}
%
%
%
where $\nabla$ is the gradient operator in the
reference configuration and $\tensorii{G} = \nabla \tensori{u}$ denotes the gradient of the
displacement.

\paragraph{Hyperelastic material}

The body is assumed made of an hyperelastic material described by a
free energy $\mecPotential_{\bodyLag{}}$ which relates the deformation gradient
$\tensorii{F}$ and the first Piola-Kirchhoff stress tensor $\tensorii{P}$ as follows:
%
%
%
\begin{equation}
\tensorii{P}=\deriv{\mecPotential_{\bodyLag{}}}{\tensorii{F}}
\end{equation}

\paragraph{Total lagrangian}

The total Lagrangian 
$L^{VW}_{\bodyLag{}}$ of the body is defined as the stored energy minus the work of
external loadings as follows:
%
%
%
\begin{equation}
\label{eq_Lagrangian}
L^{VW}_{\bodyLag{}}
% \paren{\tensori{u}}
= \int_{\Omega}\mecPotential_{\bodyLag{}} \paren{\tensorii{F}\paren{\tensori{u}}}
- \int_{\bodyLag} \tensori{f}{}_V \cdot \delta \tensori{u}{}
- \int_{\neumannBoundaryLag} \neumannLag{} \cdot \delta \tensori{u}{}
\vert_{\neumannBoundaryLag}
\end{equation}
%
%
%
where the body forces $\tensori{f}_{V}$ and conctat tractions
$\neumannLag$ in the reference configuration have been obtained from
their counterparts $\tensori{f}_{v}$ and $\neumannEul$ thanks to the
Nanson formulae, i.e. $\tensori{f}_{V}=...$ and $\tensori{f}{}_V=...$
%
%\paragraph{Variational characterisation of the mechanical equilibrium}
%
The displacement $\tensori{u}$ satisfying
the mechanical equilibrium minimizes the Lagragian $L^{VW}_{\bodyLag{}}$:
%
%
%
\begin{equation}
\tensori{u} = \underset{\displaystyle\tensori{u}^{\star}\in U(\bodyLag)}{\argmin}\,
L^{VW}_{\bodyLag{}}\paren{\tensori{u}^{\star}}
\end{equation}
%
%
%
where $U(\bodyLag)$ denotes the set of admissible displacements.
Taking the first order variation of Lagrangian yields the principle of
virtual work:
%
%
%
\begin{equation}
    \label{eq_virtual_works_0}
    \frac{\partial L^{VW}_{\bodyLag{}}\paren{\tensori{u}}}{\partial \tensori{u}} =
    \int_{\bodyLag} \tensorii{P} : \nabla \delta \tensori{u} = 
    \int_{\bodyLag} \tensori{f}_V \cdot \delta \tensori{u} +
    \int_{\neumannBoundaryLag} \neumannLag{} \cdot \delta \tensori{u}
    \vert_{\neumannBoundaryLag}
\end{equation}

\paragraph{Hu-Washizu Lagrangian}

The Hu-Washizu Lagrangian generalizes the previous variational principle by
considering that the gradient of the displacement $\tensorii{G}$ and
the first Piola-Kirchoff $\tensorii{P}$ stress are independent
unknowns of the problem.
The Hu-Washizu Lagrangian $L^{HW}$ is then defined as follows:
%
%
%
\begin{equation}
L^{HW}\paren{\tensori{u},\tensorii{G},
  \tensorii{P}} = \int_{\bodyLag{}}
\mecPotential\paren{\tensorii{I}+\tensorii{G}} + (\nabla \tensori{u}{} -
\tensorii{G}{})\,\colon\,\tensorii{P} - \int_{\bodyLag{}} \loadLag \cdot
\tensori{u}{} - \int_{\neumannBoundaryLag{}} \neumannLag{} \cdot
\tensori{u}
\vert_{\neumannBoundaryLag}
\end{equation}
%
%
%
and the solution $\tensori{u}$, $\tensorii{G}$ and $\tensorii{P}$
minimize the Hu-Washizu Lagrangian $L^{HW}$:
%
%
%
\begin{equation}
\paren{\tensori{u},\, \tensorii{G},\, \tensorii{P}} =
\underset{\displaystyle\substack{\tensori{u}^{\star}\in U(\bodyLag),\\ \tensorii{G}^{\star},\tensorii{P}^{\star}}}{\argmin}\,
L\paren{\tensori{u}^{\star}, \tensorii{G}^{\star},\tensorii{P}^{\star}}
\end{equation}
%
%
%
The first order variation of the Hu-Washizu Lagragian with respect to
$\tensori{u}$, $\tensorii{G}$, $\tensorii{P}$ yields:
\begin{subequations}
    \label{eq_hu_washizu_derivative_0}
        \begin{alignat}{3}
           \variation{L^{HW}}{\tensori{u}}
            = & \int_{\bodyLag} \tensorii{P}\,\cdot\,\nabla \delta \tensori{u}
            -
            \int_{\bodyLag} \tensori{f}_V \cdot \delta \tensori{u}
            -
            \int_{\neumannBoundaryLag} \neumannLag \cdot \delta \tensori{u}
            \vert_{\neumannBoundaryLag}
        \label{eq_hu_washizu_derivative_0:eq0}
        \\
            \variation{L^{HW}}{\tensorii{P}}
            = & \int_{\bodyLag} \paren{\nabla \tensori{u} - \tensorii{G}}\,\cdot\,\delta \tensorii{P}
        \label{eq_hu_washizu_derivative_0:eq2}
        \\
            \variation{L^{HW}}{\tensorii{G}}
            = &
            \int_{\bodyLag} (\frac{\partial \mecPotential}{\partial \tensorii{G}} - \tensorii{P})\,\cdot\,\delta \tensorii{G}
        \label{eq_hu_washizu_derivative_0:eq3}
    \end{alignat}
\end{subequations}

Equation~\eqref{eq_hu_washizu_derivative_0:eq2}
shows that $\tensorii{G}$ is equal to $\nabla \tensori{u}$ is a weak
sense and Equation~\eqref{eq_hu_washizu_derivative_0:eq3} shows that the
First Piola Kirchhoff stress is equal to the derivative the free energy
in a weak sense.

\paragraph{Some words about the importance of the Hu-Washizu principle}

...