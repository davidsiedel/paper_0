\section{The model problem}
\label{sec_model_problem}

% Paragraph~\ref{sec:Hu_Washizu_functional}
% introduces the classical Hu–Washizu functional to describe the
% quasi-static equilibrium of a body submitted to external load and the
% main notations used in this paper. For the sake of simplicity, the body
% is assumed hyper-elastic in this section.

% Paragraph~\ref{sec:Hu_Washizu_functional}
% introduces the classical Hu–Washizu functional to describe the
% quasi-static equilibrium of a body submitted to external load and the
% main notations used in this paper. For the sake of simplicity, the body
% is assumed hyper-elastic in this section.

% Paragraph~\ref{sec:HHO} introduces the key idea
% of the HHO method, which is to divide the domain in arbitrary subdomains
% connected by cohesive interfaces and to apply the Hu–Washizu functional
% to each sub-domains.

\subsection{The standard Hu–Washizu Lagrangian}
\label{sec_Hu_Washizu_functional}

This paragraph introduces the standard Hu–Washizu three field
principle. For the sake of simplicity, and without loss of generality,
we consider the case of an hyperelastic material. Extensions to
mechanical behaviours with internal state variables is treated in
classical textbooks of computational mechanics. We will treat this
extension in the Section~\ref{sec_implementation} discussing the
numerical implementation of the Hybrid High Order method and in
Section~\ref{sec_numerical_examples} which provides several examples in
plasticity.

\subsubsection{Description of the mechanical problem and notations}

\paragraph{Solid body}

Let us consider a solid body whose reference configuration is denoted
$\bodyLag$. At a given time $t > 0$, the body is in the current
configuration $\bodyEul$.

\paragraph{Mechanical loading}

The body is assumed to be submitted to a body force $\loadEul$ acting
in $\bodyEul$, a prescribed displacement $\dirichletEul$ on the
Dirichlet boundary $\dirichletBoundaryEul$, and a contact load
$\neumannEul{}$ on the Neumann boundary $\neumannBoundaryEul$.

\paragraph{Deformation}

The transformation mapping 
$\tensori{\Phi}$ takes a point from the reference configuration $\bodyLag$ to the current
configuration $\bodyEul$ such that
%
%
%
\begin{equation}
    \tensori{\Phi}\paren{\tensori{X}} = \tensori{x} = \tensori{X}+\tensori{u}\paren{\tensori{X}}
\end{equation}
%
%
%
where $\tensori{X}$, $\tensori{x}$ and $\tensori{u}$ denote respectively
the position in the reference configuration $\bodyLag$, the position
in the current configuration $\bodyEul$ and the displacement.

\paragraph{Deformation gradient, gradient of the displacement}

The deformation gradient $\tensorii{F}$ is defined as
%
%
%
\begin{equation}
    \tensorii{F} = \nabla \tensori{\Phi} = \tensorii{I} + \tensorii{G}
\end{equation}
%
%
%
where $\nabla$ is the gradient operator in the
reference configuration and 
%
%
%
\begin{equation}
    \label{eq_grad_def}
    \tensorii{G} = \nabla \tensori{u}
\end{equation}
%
%
%
denotes the gradient of the
displacement.

\paragraph{Stress tensor}

The body is assumed made of an hyperelastic material described by a
free energy $\mecPotential_{\bodyLag{}}$ which relates the deformation gradient
$\tensorii{F}$ and the first Piola-Kirchhoff stress tensor $\tensorii{P}$ such that
%
%
%
\begin{equation}
    \label{eq_stress_def}
\tensorii{P}=\deriv{\mecPotential_{\bodyLag{}}}{\tensorii{F}}
\end{equation}

\subsubsection{Primal problem and Principle of Virtual Works}

\paragraph{Total lagrangian}

The total Lagrangian $L^{VW}_{\bodyLag{}}$ of the body is defined as
the stored energy minus the work of external loadings, as follows:

%
\begin{equation}
\label{eq_Lagrangian}
L^{VW}_{\bodyLag{}}
% \paren{\tensori{u}}
= \int_{\Omega}\mecPotential_{\bodyLag{}}
(\tensorii{F}(\tensori{u}))
% \paren{\tensorii{F}\paren{\tensori{u}}}
- \int_{\bodyLag} \tensori{f}{}_V \cdot \tensori{u}{}
- \int_{\neumannBoundaryLag} \neumannLag{} \cdot \tensori{u}{}
\vert_{\neumannBoundaryLag}
\end{equation}
%
%
%
where the body forces $\tensori{f}_{V}$ and conctat tractions
$\neumannLag$ in the reference configuration have been obtained from
their counterparts $\tensori{f}_{v}$ and $\neumannEul$ using the
Nanson formulae.
%\paragraph{Variational characterisation of the mechanical equilibrium}
%

\paragraph{Principle of Virtual Works}

The displacement $\tensori{u}$ satisfying
the mechanical equilibrium minimizes the Lagragian $L^{VW}_{\bodyLag{}}$.
%
%
%
% \begin{equation}
% \tensori{u} = \underset{\displaystyle\tensori{u}^{\star}\in U(\bodyLag)}{\argmin}\,
% L^{VW}_{\bodyLag{}}\paren{\tensori{u}^{\star}}
% \end{equation}
%
%
%
% where $U(\bodyLag)$ denotes the set of admissible displacements.
The first order variation of Lagrangian is given by:
\begin{equation}
  \label{eq_virtual_works_0}
  \frac{\partial L^{VW}_{\bodyLag{}}
    % \paren{\tensori{u}}
  }{\partial \tensori{u}} (\delta \tensori{u}) =
  \int_{\bodyLag} \tensorii{P} : \nabla \delta \tensori{u} -
  \int_{\bodyLag} \tensori{f}_V \cdot \delta \tensori{u} -
  \int_{\neumannBoundaryLag} \neumannLag{} \cdot \delta \tensori{u}
  \vert_{\neumannBoundaryLag}
\end{equation}
which must be null for the the solution displacement. The solution
displacement thus satisfies the principle of virtual work:
\[
\int_{\bodyLag} \tensorii{P} : \nabla \delta \tensori{u} =
\int_{\bodyLag} \tensori{f}_V \cdot \delta \tensori{u} +
\int_{\neumannBoundaryLag} \neumannLag{} \cdot \delta \tensori{u}
\vert_{\neumannBoundaryLag}
\quad
\forall \delta \tensori{u}{}
\]
%
%
%
% which gives the Principle of Virtual Works

\subsubsection{The Hu-Washizu Lagrangian}

The Hu-Washizu Lagrangian $L^{HW}_{\bodyLag{}}$ generalizes the
previous variational principle by considering that the gradient of the
displacement $\tensorii{G}$ and the first Piola-Kirchoff $\tensorii{P}$
stress are independent unknowns of the problem, such that:
%
%
\begin{equation}
    \label{eq_HW_0}
L^{HW}
% \paren{\tensori{u},\tensorii{G}, \tensorii{P}}
= \int_{\bodyLag{}}
\mecPotential_{\bodyLag{}} (\tensorii{I}+\tensorii{G} ) + (\nabla \tensori{u}{} -
\tensorii{G}{})\,\colon\,\tensorii{P} - \int_{\bodyLag{}} \loadLag \cdot
\tensori{u}{} - \int_{\neumannBoundaryLag{}} \neumannLag{} \cdot
\tensori{u}
\vert_{\neumannBoundaryLag}
\end{equation}

The solution $(\tensori{u}, \tensorii{G}, \tensorii{P})$ satisfying
the mechanical equilibrium minimizes the Lagragian $L^{HW}_{\bodyLag{}}$.
The first order variation of the Hu-Washizu Lagragian with respect to
$\tensori{u}, \tensorii{G}$, and $\tensorii{P}$ yields
%
%
%
\begin{subequations}
    \label{eq_hu_washizu_derivative_0}
        \begin{alignat}{3}
        \deriv{L^{HW}_{\bodyLag{}}}{\tensori{u}} (\delta \tensori{u})
        = & \int_{\bodyLag} \tensorii{P} : \nabla \delta \tensori{u}
        -
        \int_{\bodyLag} \tensori{f}_V \cdot \delta \tensori{u}
        -
        \int_{\neumannBoundaryLag} \neumannLag \cdot \delta \tensori{u}
        \vert_{\neumannBoundaryLag}
        &&
        \ \ \ \ \ \ \ \ 
        &&
        \forall \delta \tensori{u}{}
        \label{eq_hu_washizu_derivative_0:eq0}
        \\
        \deriv{L^{HW}_{\bodyLag{}}}{\tensorii{P}} (\delta \tensorii{P})
        = & \int_{\bodyLag} ( \nabla \tensori{u} - \tensorii{G} ) : \delta \tensorii{P}
        &&
        \ \ \ \ \ \ \ \ 
        &&
        \forall \delta \tensorii{P}{}
        \label{eq_hu_washizu_derivative_0:eq2}
        \\
        \deriv{L^{HW}_{\bodyLag{}}}{\tensorii{G}} (\delta \tensorii{G})
        = &
        \int_{\bodyLag} (\frac{\partial \mecPotential}{\partial \tensorii{G}} - \tensorii{P}) : \delta \tensorii{G}
        &&
        \ \ \ \ \ \ \ \ 
        &&
        \forall \delta \tensorii{G}{}
        \label{eq_hu_washizu_derivative_0:eq3}
    \end{alignat}
\end{subequations}
%
%
%
where equation \eqref{eq_hu_washizu_derivative_0:eq2} and \eqref{eq_hu_washizu_derivative_0:eq3}
account for \eqref{eq_grad_def} and \eqref{eq_stress_def} respectively in a weak
sense.

\subsection{On the use of the Hu-Washizu Lagrangian in mechanics to circumvent volumetric locking}

In the continuous framework, the Hu-Washizu functional is not relevant, since equations
\eqref{eq_hu_washizu_derivative_0:eq2} and \eqref{eq_hu_washizu_derivative_0:eq3} would hold true in a strong sense.

\paragraph{Pressure swelling formulations}

Since volumetric locking is pressure dependent phenomenon, considering for instance a decomposition of the stress and strain fields into \textit{e.g.} devatoric and spherical components, one can
express a mixed problem in terms of pressure and swelling, which is at the origin of so-called UPG methods \cite{al_akhrass_integrating_2014, simo_quasi-incompressible_1991,simo_variational_1985}. The scalar pressure and swelling unknowns replace respectively the stress and strain tensorial unknowns in \eqref{eq_HW_0}, and the pressure field is directly used in the behavior law to account for the spherical component of the stress. 

\paragraph{Enhanced assumed strains formulations}

Another approach of the use of the Hu-Washizu consists in studying the equilibrium of a single element. Such a framework falls into the scope of so-called Enhanced Assumed Strains methods \cite{simo_variational_1986,simo_class_1990}, which result for instance in the B-bar method, that consists in defining a modified derivation operator, such that one gets rid of the three-field formulation, to express the problem in terms of primal unknowns only.

\paragraph{Towards Discontinuous methods}

In the present document, we propose an introduction to so-called \textit{non-conformal} methods, through the lens of the Hu–Washizu Lagrangian.
At the origin of these methods is the Discontinuous Galerkin method, which postulates the discontinuity of the displacement across elements.
This feature that was introduced for mathematical reasons allows the method to be robust to volumetric locking.
However, its formulation takes root in a possibly dry mathematical background, and the ingredients of the method are not introduced in the literature through physical arguments.
The goal of the next section aims at introducing the whole framework of non-conformal methods, including the displacement discontinuity, through the use of the Hu–Washizu Lagragian.
Though one counts a few occurances of the use of the Hu–Washizu Lagragian in the context of discontinuous methods \cite{noels_general_2006,neunteufel_three-field_2021}, none, to our knowledge, introduce all the ingredients of the method through the 
sole Hu–Washizu Lagragian.

% and though one counts a few applications of the Hu–Washizu Lagrangian for Discontinuous Galerkin methods \cite{}, none of them exploit the 
% Its application in mechanics had not resulted in a break through, and
% so did not its variants, among which the Hybird Discontinuous Galerkin method and the Hybird High Order method.
% Though one counts a few applications of the Hu–Washizu Lagrangian for Discontinuous Galerkin methods 

% En continue, aucun intérêt. Par contre, très puissant une fois les
% bases d'approximations choisies.

% Many variants:

% - Ne considérer uniquement l'espace sphérique.

% - Gardez des champs globaux: U-P-G inconnues nodales. Variantes liées aux choix des espaces d'approximations de U, P et G (Al-Akrass).

% - Travailler par éléments: Assumed strain (c.f. Belytchko).
