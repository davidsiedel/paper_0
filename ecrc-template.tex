
% Template for Elsevier CRC journal article
% version 1.2 dated 09 May 2011

% This file (c) 2009-2011 Elsevier Ltd.  Modifications may be freely made,
% provided the edited file is saved under a different name

% This file contains modifications for Procedia Computer Science
% but may easily be adapted to other journals

% Changes since version 1.1
% - added "procedia" option compliant with ecrc.sty version 1.2a
%   (makes the layout approximately the same as the Word CRC template)
% - added example for generating copyright line in abstract

%-----------------------------------------------------------------------------------

%% This template uses the elsarticle.cls document class and the extension package ecrc.sty
%% For full documentation on usage of elsarticle.cls, consult the documentation "elsdoc.pdf"
%% Further resources available at http://www.elsevier.com/latex

%-----------------------------------------------------------------------------------

%%%%%%%%%%%%%%%%%%%%%%%%%%%%%%%%%%%%%%%%%%%%%%%%%%%%%%%%%%%%%%
%%%%%%%%%%%%%%%%%%%%%%%%%%%%%%%%%%%%%%%%%%%%%%%%%%%%%%%%%%%%%%
%%                                                          %%
%% Important note on usage                                  %%
%% -----------------------                                  %%
%% This file should normally be compiled with PDFLaTeX      %%
%% Using standard LaTeX should work but may produce clashes %%
%%                                                          %%
%%%%%%%%%%%%%%%%%%%%%%%%%%%%%%%%%%%%%%%%%%%%%%%%%%%%%%%%%%%%%%
%%%%%%%%%%%%%%%%%%%%%%%%%%%%%%%%%%%%%%%%%%%%%%%%%%%%%%%%%%%%%%

%% The '3p' and 'times' class options of elsarticle are used for Elsevier CRC
%% Add the 'procedia' option to approximate to the Word template
%\documentclass[3p,times,procedia]{elsarticle}
\documentclass[3p,times,fleqn]{elsarticle}
% \documentclass[fleqn]{article}

%% The `ecrc' package must be called to make the CRC functionality available
\usepackage{ecrc}

%% The ecrc package defines commands needed for running heads and logos.
%% For running heads, you can set the journal name, the volume, the starting page and the authors

%% set the volume if you know. Otherwise `00'
\volume{00}

%% set the starting page if not 1
\firstpage{1}

%% Give the name of the journal
\journalname{Procedia Computer Science}

%% Give the author list to appear in the running head
%% Example \runauth{C.V. Radhakrishnan et al.}
\runauth{}

%% The choice of journal logo is determined by the \jid and \jnltitlelogo commands.
%% A user-supplied logo with the name <\jid>logo.pdf will be inserted if present.
%% e.g. if \jid{yspmi} the system will look for a file yspmilogo.pdf
%% Otherwise the content of \jnltitlelogo will be set between horizontal lines as a default logo

%% Give the abbreviation of the Journal.  Contact the journal editorial office if in any doubt
\jid{procs}

%% Give a short journal name for the dummy logo (if needed)
\jnltitlelogo{Procedia Computer Science}

%% Provide the copyright line to appear in the abstract
%% Usage:
%   \CopyrightLine[<text-before-year>]{<year>}{<restt-of-the-copyright-text>}
%   \CopyrightLine[Crown copyright]{2011}{Published by Elsevier Ltd.}
%   \CopyrightLine{2011}{Elsevier Ltd. All rights reserved}
\CopyrightLine{2011}{Published by Elsevier Ltd.}

%% Hereafter the template follows `elsarticle'.
%% For more details see the existing template files elsarticle-template-harv.tex and elsarticle-template-num.tex.

%% Elsevier CRC generally uses a numbered reference style
%% For this, the conventions of elsarticle-template-num.tex should be followed (included below)
%% If using BibTeX, use the style file elsarticle-num.bst

%% End of ecrc-specific commands
%%%%%%%%%%%%%%%%%%%%%%%%%%%%%%%%%%%%%%%%%%%%%%%%%%%%%%%%%%%%%%%%%%%%%%%%%%

%% The amssymb package provides various useful mathematical symbols
\usepackage{amssymb}
%% The amsthm package provides extended theorem environments
%% \usepackage{amsthm}

%% The lineno packages adds line numbers. Start line numbering with
%% \begin{linenumbers}, end it with \end{linenumbers}. Or switch it on
%% for the whole article with \linenumbers after \end{frontmatter}.
%% \usepackage{lineno}

%% natbib.sty is loaded by default. However, natbib options can be
%% provided with \biboptions{...} command. Following options are
%% valid:

%%   round  -  round parentheses are used (default)
%%   square -  square brackets are used   [option]
%%   curly  -  curly braces are used      {option}
%%   angle  -  angle brackets are used    <option>
%%   semicolon  -  multiple citations separated by semi-colon
%%   colon  - same as semicolon, an earlier confusion
%%   comma  -  separated by comma
%%   numbers-  selects numerical citations
%%   super  -  numerical citations as superscripts
%%   sort   -  sorts multiple citations according to order in ref. list
%%   sort&compress   -  like sort, but also compresses numerical citations
%%   compress - compresses without sorting
%%
%% \biboptions{comma,round}

% \biboptions{}

% if you have landscape tables
\usepackage[figuresright]{rotating}

% put your own definitions here:
%   \newcommand{\cZ}{\cal{Z}}
%   \newtheorem{def}{Definition}[section]
%   ...

% add words to TeX's hyphenation exception list
%\hyphenation{author another created financial paper re-commend-ed Post-Script}

% declarations for front matter
\usepackage{import}
\usepackage{example}
\usepackage{graphicx}
\usepackage{amsmath}
\usepackage{float}
\usepackage{amssymb}
\usepackage{xcolor}
\usepackage{hyperref}
\usepackage{longtable}
\usepackage{notations}
\usepackage{listings}
\usepackage{multicol}

\newtheorem{theorem}{Theorem}[section]
\newtheorem{corollary}{Corollary}[theorem]
\newtheorem{lemma}[theorem]{Lemma}
\newtheorem{property}[theorem]{Property}

\begin{document}

% \begin{multicols}{0}

\begin{frontmatter}

%% Title, authors and addresses

%% use the tnoteref command within \title for footnotes;
%% use the tnotetext command for the associated footnote;
%% use the fnref command within \author or \address for footnotes;
%% use the fntext command for the associated footnote;
%% use the corref command within \author for corresponding author footnotes;
%% use the cortext command for the associated footnote;
%% use the ead command for the email address,
%% and the form \ead[url] for the home page:
%%
%% \title{Title\tnoteref{label1}}
%% \tnotetext[label1]{}
%% \author{Name\corref{cor1}\fnref{label2}}
%% \ead{email address}
%% \ead[url]{home page}
%% \fntext[label2]{}
%% \cortext[cor1]{}
%% \address{Address\fnref{label3}}
%% \fntext[label3]{}

\dochead{}
%% Use \dochead if there is an article header, e.g. \dochead{Short communication}
%% \dochead can also be used to include a conference title, if directed by the editors
%% e.g. \dochead{17th International Conference on Dynamical Processes in Excited States of Solids}

\title{}

%% use optional labels to link authors explicitly to addresses:
%% \author[label1,label2]{<author name>}
%% \address[label1]{<address>}
%% \address[label2]{<address>}

\author{}

\address{}

\begin{abstract}
%% Text of abstract
\end{abstract}

\begin{keyword}
%% keywords here, in the form: keyword \sep keyword

%% PACS codes here, in the form: \PACS code \sep code

%% MSC codes here, in the form: \MSC code \sep code
%% or \MSC[2008] code \sep code (2000 is the default)

\end{keyword}

\end{frontmatter}

%%
%% Start line numbering here if you want
%%
% \linenumbers

%% main text

\section{Introduction}
\label{sec_introduction}

The Hybird High Order method (HHO) is a discontinuous discretization
method, that takes root in the Discontinuous Galerkin method (DG). From
the physical standpoint, DG methods ensure the continuity of the flux
across interfaces, by seeking the solution element-wise, hence allowing
jumps of the potential across elements. They can be seen as a
generalization of Finite Volume methods, and are able to capture
physically relevant discontinuities without producing spurious
oscillations.

The origin of DG methods dates back to the pioneering work of
\cite{reed_triangular_1973}, where an hyperbolic formualtion is used to
solve the neutron transport equation. The first application of the
method to elliptic problems originates in \cite{babuska_finite_1973}
where Nitsche's method \cite{nitsche_uber_1970} is used to weakly impose
continuity of the flux across interfaces. \textcolor{blue} { In 2002,
  Hansbo and Larson \cite{hansbo_discontinuous_2002-1} were the first to
  consider the Nitsche's classical DG method for nearly incompressible
  elasticity. They showed, theoretically and numerically, that this
  method is free from volumetric locking. } However, the bilinear form
arising from this formulation is not symmetric. A so called interior
penalty term has been introduced in \cite{wheeler_elliptic_1978},
leading to the Symmetric Interior Penalty (SIP) DG method. A first study
of the method to linear elasticity has been devised by
\cite{riviere_optimal_2000}, where optimal error estimate has been
proved. \textcolor{blue} { \cite{lew_optimal_2004} generalized the
  Symmetric Interior Penalty method to linear elasticity. }
\textcolor{blue} {
  % In 2002, Hansbo and Larson \cite{hansbo_discontinuous_2002-1} were the first to
 % consider the Nitsche's classical DG method for nearly incompressible
  % elasticity. They showed, theoretically and numerically, that this method
 % is free from volumetric locking. % \cite{lew_optimal_2004}
  % generalized the Symmetric
  % Interior Penalty method to linear elasticity. In about the same
  period of time, DG methods were proposed for other linear problems in
  solid mechanics, such as Timoshenko beams
  \cite{celiker_locking-free_2006}, Bernoulli-Euler beam and the
  Poisson-Kirchhoff plate \cite{brenner_balancing_1999,
    engel_continuousdiscontinuous_2002} and Reissner-Mindlin plates
  \cite{arnold_family_2005}. In the mid 2000's, the first applications
  of DG methods to nonlinear elasticity problems was undertaken by
  \cite{ten_eyck_discontinuous_2006, noels_general_2006}, and in 2007,
  Ortner and Süli \cite{ortner_discontinuous_2007} carried out the a
  priori error analysis of DG methods for nonlinear elasticity.
  % This pioneering work
  % shed light on how to calculate a lower bound on the stability parameters.
 }

DG methods then sollicitated a vigourus interest, mostly in fluid dynamics \cite{shahbazi_high-order_2007, persson_discontinuous_2009} due to their local conservative property and stability in convection domniated problems. However, except some applications for instance in fracture mechanics using XFEM methods \cite{gracie_blending_2008, shen_stability_2010}, or gradient plasticity \cite{djoko_discontinuous_2007,djoko_discontinuous_2007-1} DG methods did not break through in computational solid mechanics because of their numerical cost, since nodal unknowns need be duplicated to define local basis functions in each element.

To adress this problem, in the early 2010's, \cite{cockburn_unified_2009, soon_hybridizable_2009} introduced additional faces unknowns on element interfaces for linear elastic problem, hence leading to the hybridization of DG methods, or Hybridizable Discontinuous Galerkin method (HDG). By adding supplementary boundary unknowns, the authors actually allowed to eliminate original cell unknowns by a static condensation process, in order to express the global problem on faces ones only. Extension of HDG methods to non-linear elasticity were first undertaken in \cite{soon_hybridizable_2008} and have then fueled intense reaserch works for various applications such as linear and non-linear convection-diffusion problems \cite{nguyen_implicit_2009,nguyen_implicit_2009-1,nguyen_hybridizable_2010}, incompressible stokes flows \cite{nguyen_hybridizable_2010, nguyen_implicit_2011} and non-linear mechanics \cite{nguyen_hybridizable_2012}.

In \cite{di_pietro_hybrid_2015, di_pietro_arbitrary-order_2014}, the authors introduced a higher order potential reconstruction operator in the classical HDG formulation for elliptic problems, providing a $h^{k+1} H^1$-norm convergence rate as compared to the ususal $h^k$-rate. This higher order term coined the name for the so called HHO method.
Recent developments of HHO methods in
computational mechanics include the incompressible Stokes
equations (with possibly large irrotational forces) \cite{di_pietro_discontinuous_2016}, the
incompressible Navier–Stokes equations \cite{di_pietro_hybrid_2018}, Biot’s consolidation problem \cite{boffi_nonconforming_2016}, and nonlinear elasticity with small
deformations \cite{botti_hybrid_2017}

\textcolor{red}{
    The difference
    between HHO and HDG methods is twofold: (1) the HHO
    reconstruction operator replaces the discrete HDG flux (a
    similar rewriting of an HDG method for nonlinear elastic-
    ity can be found in [29]), and, more importantly, (2) both
    HHO and HDG penalize in a least-squares sense the differ-
    ence between the discrete trace unknown and the trace of the
    discrete primal unknown (with a possibly mesh-dependent
    weight), but HHO uses a non-local operator over each mesh
    cell boundary that delivers one-order higher approximation
    than just penalizing pointwise the difference as in HDG.
    Discretization methods for linear and nonlinear elastic-
    ity have undergone a vigorous development over the last
    decade. For discontinuous Galerkin (dG) methods, we men-
    tion in particular [14,26,32] for linear elasticity, and [35,41]
    for nonlinear elasticity. HDG methods for linear elasticity
    have been coined in [38] (see also [13] for incompressible
    Stokes flows), and extensions to nonlinear elasticity can be
    found in [29,34,37]. Other recent developments in the last few
    years include, among others, Gradient Schemes for nonlinear
    elasticity with small deformations [22], the Virtual Element
    Method (VEM) for linear and nonlinear elasticity with small
    [3] and finite deformations [8,43], the (low-order) hybrid dG
    method with conforming traces for nonlinear elasticity [44],
    the hybridizable weakly conforming Galerkin method with
    nonconforming traces for linear elasticity [30], the Weak
    Galerkin method for linear elasticity [42], and the discon-
    tinuous Petrov–Galerkin method for linear elasticity [7]. 
}

Contrary to the standard (\textit{i.e.} the Lagrange) Finite Element method, non-conformal methods (among which the Hybrid High Order one) postulate the discontinuity of the displacement field across elements. Hence, each element is \textit{a priori} free to move independently from others; in order to restore a weak form of continuity on the mesh, a \textit{stabilization} term is computed, to penalize in a least square sense the displacement jump between two neighbouring cells.
The displacement jump between elements is exploited to define discrete operators in each element, that provide conservation of physcial properties.
Moreover, the Hybrid High Order method is hybrid, hence introducing faces unknowns in addition to the regular cells ones.
Morover, cell unknowns are expressed in terms of coefficients in a polynomial basis, that have no physcial meaning, as opposed to the ususal nodal unknowns of Lagrange finite elements.
This feature allows to equivalently express shape functions on any generic polygonal element, as opposed to the Lagrange Finite Element method that needs particlaur shape function for each element geometry.

All these differences with the wide spread Lagrange Finite Element method make the Hybrid High Order one more \textcolor{blue}{ununderstandable} to a computational mechanics public.
Discontinuous methods were developed by the mathematical community, such that they are put forward in the literature through a possibly arid way for the computational mechanics reader. Therefore, in the present document, we propose an introduction to these methods, based on mechanical arguments, by considering the ususal continuous framework proper to the standard Finite Element method, and using a limit case to meet the discontinous setting in which lies the HHO method.

In a second part, we propose and devise a Hybrid High Order method for axisymetrical configurations.

\section{The Hybrid High Order method}
\label{sec_1}

\subsection{Description of the model problem}
\label{sec_10}

Let $d \in  \{1, 2\}$ the euclidean dimension of the cartesian space $\mathbb{R}{}^{d}$, and $\mathcal{R}_d$ the euclidean reference frame. Let $\Omega{}_{} \subset \mathbb{R}{}^{d}$ a solid body with boundary $\partial \Omega{}_{} \subset \mathbb{R}{}^{d - 1}$, that deforms under the volumic load $\tensori{f}{}_{v}$. It is subjected to a prescribed displacement $\tensori{u}{}_{d}$ on the Dirichlet boundary $\partial \Omega{}_{d}$, and to a contact load $\tensori{t}{}_{n}$ on the Neumann boundary $\partial \Omega{}_{n}$, such that $\partial \Omega{}_{} = \partial \Omega{}_{d} \cup \partial \Omega{}_{n}$ and $\partial \Omega{}_{d} \cap \partial \Omega{}_{n} = \emptyset$.

The initial configuration of the body (see Figure \ref{fig_setting}) is denoted $\Omega{}_{0} \in \mathbb{R}{}^{d}$ with respective Dirichlet and Neumann boundaries $\partial \Omega{}_{D}$ and $\partial \Omega{}_{N}$. The transformation mapping $\tensori{\Phi}$ takes a point $\tensori{X} \in \Omega{}_{0}$ to $\tensori{x} \in \Omega{}_{}$, such that $\tensori{x} = \tensori{\Phi}(\tensori{X}) = \tensori{X} + \tensori{u}(\tensori{X})$ where $\tensori{u}$ denotes the displacement of the physical point. Let $\tensorii{F} = \nabla_X \tensori{\Phi} = \tensorii{1} + \nabla_{X} \tensori{u}$ the transformation gradient.
The mechanical problem to solve reads, find $\tensori{u}$ such that:
%
\begin{subequations}
\label{eq_0000}
    \begin{alignat}{2}
    \tensorii{F} - \nabla_{X} \tensori{u} & = \tensorii{1} \quad && \text{in } \Omega_{0} \label{eq_0000:eq1}
    \\
    \tensorii{P} - \frac{\partial \psi_{\Omega_0}}{\partial \tensorii{F}} & = 0 \quad && \text{in } \Omega_{0} \label{eq_0000:eq2}
    \\
    \nabla_{X} \cdot \tensorii{P} - \tensori{f}{}_{V} & = 0 \quad && \text{in } \Omega_{0} \label{eq_0000:eq3}
    \\
    \tensori{u} & = \tensori{u}{}_{D} \quad && \text{on } \partial_{D} \Omega_{0} \label{eq_0000:eq4}
    \\
    \tensorii{P} \cdot \tensori{n} & = \tensori{t}{}_{N} \quad && \text{on } \partial_{N} \Omega_{0} \label{eq_0000:eq5}
\end{alignat}
\end{subequations}
%
where $\psi_{\Omega_0}$ denotes the mechanical energy potential of the body $\Omega_0$, and $\tensorii{P}$ is the first Piola-Kirchoff stress tensor.
%
\begin{figure}[H]
\centering
\includegraphics[width=7.cm]{img/mech_setting.png}
\caption{schematic representation of the model problem}
\label{fig_setting}
\end{figure}
%
The equilibrium of the body $\Omega_0$ is reached for the displacement field $\tensori{u}{} \in H^1(\Omega_0, \mathbb{R}^d)$ minimizing the energy functional:
%
\begin{equation}
    \label{eq_0001}
    \begin{aligned}
        J_{\Omega_0}(\tensori{u}{}) = \int_{\Omega_0} \psi_{\Omega_0} - \int_{\Omega_0} \tensori{f}{}_V \cdot \tensori{u}{}
        -
        \int_{\partial_N \Omega_0} \tensori{t}{}_N \cdot \tensori{u}{}
    \end{aligned}
\end{equation}
%
corresponding to problem \eqref{eq_0000} where equations \eqref{eq_0000:eq1} and \eqref{eq_0000:eq2} are enforced strongly.
If \eqref{eq_0000:eq1} and \eqref{eq_0000:eq2} are considered in weak sense, one obtains the three-field Hu–Washizu functional :
%
\begin{equation}
\label{eq_0002}
    J_{\Omega_0}(\tensori{u}, \tensorii{G}, \tensorii{P}) =
    \int_{\Omega_0} \psi_{\Omega_0} + (\nabla_X \tensori{u} - \tensorii{G}) : \tensorii{P}
    -
    \int_{\Omega_0} \tensori{f}{}_V \cdot \tensori{u}
    -
    \int_{\partial_N \Omega_0} \tensori{t}{}_N \cdot \tensori{u}
\end{equation}
%
where $\tensorii{P} \in H^1_{\text{div}}(\Omega_0, \mathbb{R}^{d \times d})$, $\tensorii{F} \in L^2(\Omega_0, \mathbb{R}^{d \times d})$, and $\tensorii{G} := \tensorii{F} - \tensorii{1} \in L^2(\Omega_0, \mathbb{R}^{d \times d})$.

\subsection{Hybrid mesh}
\label{sec_1bis}

Since the Hybird High Order method relies on both cell and faces unknowns, a so called hybrid mesh is considered. It consists in a collection of cells, as is the case with the standard Finite Element method, and in the collection of the cell faces, forming the skeleton of the mesh.
Hence, let $\mathcal{T}_h(\Omega_0)$ the cell collection be a triangulation of the domain $\Omega_0$ into a set of disjoints open polyhedra with planar faces called elements (or cells) $T_i \subset \mathbb{R}^{d}, 1 \leq i \leq N_T$, where $N_T$ denotes the number of elements in the mesh, such that $\Omega_0 = \cup_{1 \leq i \leq N_T} T_i$. For each element $T_i$, let $\partial T_i \subset \mathbb{R}^{d-1}$ its boundary, composed of its faces (if $d = 3$) or edges (if $d = 2$).

Let $\mathcal{F}_h(\Omega_0)$ the skeleton of the mesh, collecting all element faces in the mesh.
A face $F \subset \mathbb{R}^{d - 1}$ is a closed subset of $\Omega_0$, and either there are two cells $T_1$ and $T_2$ such that $F = \partial T_1 \cap \partial T_2$ ($F$ is then an interior face), or there is a single cell $T$ such that $F = \partial T \cap \partial \Omega_0$ ($F$ is then an exterior face).

Let $\mathcal{F}_h^i(\Omega_0)$ denote the set of interior faces, and $\mathcal{F}_h^e(\Omega_0)$ that of exterior ones.
$\mathcal{F}_h^e(\Omega_0)$ is partitioned into $\mathcal{F}_{h,D}^e(\Omega_0) = \{ F \in \mathcal{F}_h^e(\Omega_0) \ \vert \ F \subset \partial_D \Omega_0 \}$ the set of exterior faces imposed to prescribed Dirichlet boundary conditions, and into $\mathcal{F}_{h,N}^e(\Omega_0) = \{ F \in \mathcal{F}_h^e(\Omega_0) \ \vert \ F \subset \partial_N \Omega_0 \}$ the set of exterior faces imposed to prescribed Neumann boundary conditions.

\subsection{Introducing}
\label{sec_1bis2}

Since the cell displacement field is discontinuous on the mesh, it is sought in the so called broken Sobolev space $H^1(\mathcal{T}_h(\Omega_0), \mathbb{R}^{d}) = \{ \tensori{v} \in L^2(\Omega_0, \mathbb{R}^{d}) \ \vert \  \tensori{v}{}_T \in H^1(T, \mathbb{R}^{d}) , \forall T \in \mathcal{T}_h(\Omega_0)\}$. In addition, the face displacement field over the mesh is sought in
$L^2(\mathcal{F}_h(\Omega_0), \mathbb{R}^{d}) = \{ \tensori{v} \in L^2(\partial T, \mathbb{R}^{d}), \forall T \in \mathcal{T}_h(\Omega_0)\}$, such that the global displacement field is in
$U = H^1(\mathcal{T}_h(\Omega_0), \mathbb{R}^{d}) \times L^2(\mathcal{F}_h(\Omega_0), \mathbb{R}^{d})$

\subsection{Composite Element}
\label{sec_11}

In order to introduce the discontinuous setting in which lies the Hybrid High Order method, let consider a mesh composed of composite elements, each formed by a bulk part and an interface part between the bulk and the faces of the element, of width some characteristic length.
The aim of this section consists in expressing the mechanical equilibrium of such a composite element in order to introduce the main operatos at the foundation of discontinous methods, and in meeting the discontinuous setting by making the characteristic length tend towards zero.
%
\begin{figure}[H]
    \centering
    \includegraphics[width=7.cm]{img/element_sketch.png}
\end{figure}
%

\subsubsection{Composite element geometry}
\label{sec_111}

Let consider an element $T$, an open subset of $\mathbb{R}^d$, with boundary $\partial T$, split into a thin open volumic region $\Gamma \subset T$ of width $\ell > 0$, called the interface, that is attached to the element boundary $\partial T$, and into an open bulk region $\Upsilon \subset T$, such that
%
\begin{equation*}
    \begin{aligned}
        \Upsilon \cup \Gamma = T
        &&
        \text{and}
        &&
        \overline{\Upsilon} \cap \overline{\Gamma} = \partial \Upsilon
    \end{aligned}    
\end{equation*}
%
where $\partial \Upsilon$ denotes the boundary of $\Upsilon$.
Let $\tensori{\Phi}{}_{T}$ the homotethy of ratio $(1 + \alpha \ell)$ and center $\tensori{X}{}_T$ the centroid of $T$, with $\alpha < 0$ such that $\Upsilon$ (respectively $\partial \Upsilon$) is the image of $T$ (respectively $\partial T$) by $\tensori{\Phi}{}_{T}$. Since $\partial \Upsilon$ is an homotethy of $\partial T$, any point $\tensori{X}{}_{\partial T} \in \partial T$ and $\tensori{X}{}_{\partial \Upsilon} = \tensori{\Phi}{}_T(\tensori{X}{}_{\partial T}) \in \partial \Upsilon$ share the same unit outward normal $\tensori{n}{}$.
Furthermore, let introduce the following property for integrable functions in $\Gamma$:

\begin{property}
    Let $C_\Gamma = \{ v \in L^2(\Gamma) \ \vert \ v \cdot \tensori{n} = \text{cste} \}$ the set of $L^2$-functions which are constant along the normal axis in $\Gamma$. For any function in $C_\Gamma$, the following equality holds true:
    %
    \begin{equation}
        \label{eq_00010}
            \int_{\Gamma} v \ dV
            =
            \int_{\partial T} \int_{\epsilon = 0}^{\ell} v (1 + \alpha \epsilon) \ dS d \epsilon
            =
            \ell (1 + \frac{\alpha}{2} \ell) \int_{\partial T} v \ dS
    \end{equation}
\end{property}

\subsubsection{Composite element behaviour}
\label{sec_112}

Let endow the bulk volume $\Upsilon$ with a displacement field
% $\tensori{u}_{\Upsilon} \in H^1(\Upsilon, \mathbb{R}^d)$,
$\tensori{u}_{\Upsilon}$,
such that it is \textit{a priori} free to move independently from the boudary $\partial T$, with a displacement field
% $\tensori{u}{}_{\partial T} \in L^2(\partial T, \mathbb{R}^d)$
$\tensori{u}{}_{\partial T}$.
In order to ensure continuity of the displacement and to bind the displacement of $\Upsilon$ to that of $\partial T$, let $\Gamma$ act as a patch between $\Upsilon$ and $\partial T$, such that
% $\tensori{u}_{\Gamma} \in H^1(\Gamma, \mathbb{R}^d)$
$\tensori{u}_{\Gamma}$
the displacement of $\Gamma$ links that of $\Upsilon$ to that of $\partial T$.
Furthermore, assuming the interface $\Gamma$ to be thin compared to the cell volume $T$, such that $\ell \ll h_T$ is negligeable with respect to $h_T$ the diameter of $T$, let linearize the displacement in $\Gamma$ with respect to $\tensori{n}$, such that :
%
% \begin{subequations}
%     \label{eq_0004}
%     \begin{alignat}{2}
%         \tensori{u}_{\Gamma} \vert_{\partial \Upsilon} & = \tensori{u}_{\Upsilon} \vert_{\partial \Upsilon} \label{eq_0004:eq1}
%         \\
%         \tensori{u}_{\Gamma} \vert_{\partial T} & = \tensori{u}_{\partial T} \label{eq_0004:eq2}
%     \end{alignat}
% \end{subequations}
%
% Furthermore, assuming the interface $\Gamma$ to be thin compared to the cell volume $T$, such that $\ell \ll h_T$ is negligeable with respect to $h_T$ the diameter of $T$, let linearize the displacement in $\Gamma$ with respect to $\tensori{n}$, such that :
%
\begin{equation}
    \label{eq_0005}
    \tensori{u}{}_{\Gamma}
    =
    \frac{\tensori{u}{}_{\partial T}
    -
    \tensori{u}{}_{\Upsilon} \vert_{\partial \Upsilon}}{\ell} \otimes \tensori{n} \cdot \tensori{X}
    +
    \tensori{u}{}_{\Upsilon} \vert_{\partial \Upsilon}
\end{equation}
%
% The displacement $\tensori{u}_{T} \in H^1(T, \mathbb{R}^d)$ in the whole domain $T$ is then the composition of the displacement field in $\Upsilon \cup \Gamma$ such that :
% \begin{equation}
%     \label{eq_0003bis}
%     \begin{aligned}
%         \tensori{u}{}_{T} = 
%         \left\{
%             \begin{array}{ll}
%                 \tensori{u}{}_{\Upsilon} & \mbox{in } \Upsilon
%                 \\
%                 \tensori{u}{}_{\Gamma} & \mbox{in } \Gamma
%                 % \\
%                 % \tensori{u}{}_{\partial T} & \mbox{on } \partial T
%             \end{array}
%         \right.
%     \end{aligned}
% \end{equation}
%
Let $\psi_{\Upsilon} = \psi_{\Omega_0}$ the free energy potential in the bulk $\Upsilon$ and $\psi_{\Gamma}$ that in $\Gamma$.
The interface $\Gamma$ is supposed to behave like a linear elsatic material with a Young modulus $\beta$ and a zero Poisson ratio, such that :
% $\psi_{\Upsilon}$ is arbitrary and expresses that of the solid body $\Omega_0$, such that $\psi_{\Upsilon} = \psi_{\Omega_0}$.
% Let endow the interface $\Gamma$ with a linear elastic behaviour, such that :
%
\begin{equation}
    \label{eq_0009}
        \psi_{\Gamma} = \frac{1}{2} \beta \frac{\ell}{h_T} \nabla_X \tensori{u}{}_{\Gamma} : \nabla_X \tensori{u}{}_{\Gamma}
\end{equation}
%
where
% the parameter $\beta$ is the Young modulus of the membrane, and
the dimensionless ratio $\ell / h_T$ balances the accumulated energy with the size of the element.
% Consequently, let $\psi_T$ the mechanical energy potential in $T$ such that :
% \begin{equation}
%     \begin{aligned}
%         \psi_T = 
%         \left\{
%             \begin{array}{ll}
%                 \psi_{\Omega_0} & \mbox{in } \Upsilon
%                 \\
%                 \psi_{\Gamma} & \mbox{in } \Gamma
%             \end{array}
%         \right.
%     \end{aligned}
% \end{equation}
%
Thus, the element $T$ is made out of a composite material, with a linear elastic layer $\Gamma$ and a bulk (or matrix) layer $\Upsilon$ with respective behaviours $\psi_{\Gamma}$ and $\psi_{\Omega_0}$.
Let $\tensorii{P}{}_{\Upsilon}$ the first Piola Kirchoff stress tensor in the bulk $\Upsilon$, and let $\tensorii{P}{}_{\Gamma}$ that in $\Gamma$.
Since $\Gamma$ is thin compared to $\Upsilon$, let suppose that the stress in $\Gamma$ is homogeneous along $\tensori{n}$, such that it carries the traction force from the boundary $\partial \Upsilon$ to $\partial T$.
% such that it carries the traction force from the boundary $\partial \Upsilon$ to $\partial T$.
By continuity of the traction force across $\partial \Upsilon$, the following equality holds true :
%
\begin{equation}
    \label{eq_continuity_traction_force}
    (\tensorii{P}{}_{\Gamma} - \tensorii{P}{}_{\Upsilon} \vert_{\partial \Upsilon}) \cdot \tensori{n}{} =  0
\end{equation}
%
Let $\tensori{u}{}_{T}, \psi_{T}, \tensorii{P}{}_{T}$ respectively the displacement, the free energy potential and the stress in $T$ such that :
%
\begin{equation}
    \begin{aligned}
        \tensori{u}{}_{T} = 
        \left\{
            \begin{array}{ll}
                \tensori{u}{}_{\Upsilon} & \mbox{in } \Upsilon
                \\
                \tensori{u}{}_{\Gamma} & \mbox{in } \Gamma
                % \\
                % \tensori{u}{}_{\partial T} & \mbox{on } \partial T
            \end{array}
        \right.
        &&
        \text{,}
        &&
        \psi_{T} = 
        \left\{
            \begin{array}{ll}
                \psi_{\Upsilon} & \mbox{in } \Upsilon
                \\
                \psi_{\Gamma} & \mbox{in } \Gamma
                % \\
                % \tensori{u}{}_{\partial T} & \mbox{on } \partial T
            \end{array}
        \right.
        &&
        \text{,}
        &&
        \tensorii{P}{}_{T} = 
        \left\{
            \begin{array}{ll}
                \tensorii{P}{}_{\Upsilon} & \mbox{in } \Upsilon
                \\
                \tensorii{P}{}_{\Gamma} & \mbox{in } \Gamma
                % \\
                % \tensori{u}{}_{\partial T} & \mbox{on } \partial T
            \end{array}
        \right.
    \end{aligned}
\end{equation}

\subsection{Element enregy balance}
\label{sec_11bis}

Following \eqref{eq_0002} and writing $J_T$ the Hu–Washizu functional over the composite element $T$ yields :
%
\begin{equation}
    \label{eq_000Bbis}
    \begin{aligned}
        J_{T}
        % (\tensori{u}{}_{T}, \tensorii{G}{}_{T}, \tensorii{P}{}_{T})
        =
        \int_{\Upsilon} \psi_{\Upsilon} + (\nabla_X \tensori{u}{}_{\Upsilon} - \tensorii{G}{}_{\Upsilon}) : \tensorii{P}{}_{\Upsilon}
        +
        \int_{\Gamma} \psi_{\Gamma} + (\nabla_X \tensori{u}{}_{\Gamma} - \tensorii{G}{}_{\Gamma}) : \tensorii{P}{}_{\Gamma}
        -
        \int_{\Upsilon} \tensori{f}{}_V \cdot \tensori{u}{}_{\Upsilon}
        % -
        % \int_{\Gamma} \tensori{f}{}_V \cdot \tensori{u}{}_{\Gamma}
        -
        \int_{\partial_N T} \tensori{t}{}_{\partial T} \cdot \tensori{u}{}_{\partial T}
    \end{aligned}
\end{equation}
%
where
% $\tensorii{G}{}_{T} \in L^2(T, \mathbb{R}^{d \times d})$
$\tensorii{G}{}_{\Upsilon}$ and $\tensorii{G}{}_{\Gamma}$ are the displacement gradient unknowns in respectively $\Upsilon$ and $\Gamma$.
$\tensori{t}{}_{\partial T}$ denotes the resulting contact forces applied on $\partial T$, \textit{i.e.} it is either equal to $\tensori{t}{}_{N}$ if $\partial T \subset \partial_N \Omega_0$, or to $\tensori{t}{}_{T' \rightarrow T}$ the contatc force applyed by $T'$ to $T$ for any neighbouring element $T'$ sharing a boundary with $T$.
In particular, we assumed that the volumetric load is concentrated in the core part of the element $\Upsilon$, and neglegted in the boundary part.
Using \eqref{eq_0005} and \eqref{eq_00010} for $\psi_{\Gamma} \in C_\Gamma$ one can write the expression of the mechanical energy in the membrane as a term only depending on the boundary :
%
\begin{equation}
    \label{eq_00011}
    \begin{aligned}
        \int_{\Gamma} \psi_{\Gamma}
        = &
        (1 + \frac{\alpha}{2} \ell) \int_{\partial T} \frac{\beta }{2 h_T} \lVert \tensori{u}{}_{\partial T} - \tensori{u}{}_{\Upsilon} \vert_{\partial \Upsilon} \rVert^2
    \end{aligned}
\end{equation}
%
using a similar argument for the second volumetric term, as well as \eqref{eq_continuity_traction_force} one has:
%
\begin{equation}
    \label{eq_00012}
    \begin{aligned}
        \int_{\Gamma} (\nabla_X \tensori{u}{}_{\Gamma} - \tensorii{G}{}_{\Gamma}) : \tensorii{P}{}_{\Gamma}
        =
        (1 + \frac{\alpha}{2} \ell)
        \int_{\partial T} (\tensori{u}{}_{\partial T} - \tensori{u}{}_{\Upsilon} \vert_{\partial \Upsilon}) \otimes \tensori{n}{} : \tensorii{P}{}_{\Upsilon} \vert_{\partial \Upsilon}
        -
        \int_{\Gamma} \tensorii{G}{}_{\Gamma} : \tensorii{P}{}_{\Gamma}
    \end{aligned}
\end{equation}
%
using \eqref{eq_00012} and \eqref{eq_00011} in the expression of $J_T$ yields:
%
\begin{equation}
    \label{eq_0014}
    \begin{aligned}
        J_{T}
        % (\tensori{u}{}_{\Upsilon}, \tensori{u}{}_{\partial T}, \tensorii{G}{}_{T}, \tensorii{P}{}_{T})
        = &
        \int_{\Upsilon} \psi_{\Upsilon} + (\nabla_X \tensori{u}{}_{\Upsilon} - \tensorii{G}{}_{\Upsilon}) : \tensorii{P}{}_{\Upsilon}
        % \\
        % &
        +
        (1 + \frac{\alpha}{2} \ell)
        % \Biggl(
        \int_{\partial T} (\tensori{u}{}_{\partial T} - \tensori{u}{}_{T} \vert_{\partial T}) \otimes \tensori{n}{} : \tensorii{P}{}_{\Upsilon} \vert_{\partial \Upsilon}
        % \\
        % &
        \\
        &
        +
        (1 + \frac{\alpha}{2} \ell)
        \int_{\partial T} \frac{\beta}{2 h_T} \lVert \tensori{u}{}_{\partial T} - \tensori{u}{}_{\Upsilon} \vert_{\partial \Upsilon} \rVert^2
        % \Biggr)
        % \\
        % &
        -
        \int_{\Gamma} \tensorii{G}{}_{\Gamma} : \tensorii{P}{}_{\Gamma}
        % \\
        % &
        -
        \int_{\Upsilon} \tensori{f}{}_V \cdot \tensori{u}{}_{\Upsilon} - \int_{\partial_N T} \tensori{t}{}_{\partial T} \cdot \tensori{u}{}_{\partial T}
    \end{aligned}
\end{equation}
%
Since $\ell$ is arbitrary, let $\ell \rightarrow 0$;
the interface region vanishes such that $\Gamma \rightarrow \emptyset, \Upsilon \rightarrow T$ and $\partial \Upsilon \rightarrow \partial T$.
Using a density argument, one has $\tensori{u}{}_{\Upsilon} = \tensori{u}{}_{T}, \psi{}_{\Upsilon} = \psi{}_{T}$ and $\tensorii{P}{}_{\Upsilon} = \tensorii{P}{}_{T}$, and the expression of the Hu–Washizu functional over the element $T$ writes:
%
\begin{equation}
    \label{eq_0015}
    \begin{aligned}
        J_{T}
        % (\tensori{u}{}_{T}, \tensori{u}{}_{\partial T}, \tensorii{G}{}_{T}, \tensorii{P}{}_{T})
        = &
        \int_{T} \psi_{T} + (\nabla_X \tensori{u}{}_{T} - \tensorii{G}{}_{T}) : \tensorii{P}{}_{T}
        % \\
        % &
        + \int_{\partial T} (\tensori{u}{}_{\partial T} - \tensori{u}{}_{T} \vert_{\partial T}) \otimes \tensori{n}{} : \tensorii{P}{}_{T} \vert_{\partial T}
        % \\
        % &
        + \int_{\partial T} \frac{\beta}{2 h_T} \lVert \tensori{u}{}_{\partial T} - \tensori{u}{}_{T} \vert_{\partial T} \rVert^2
        \\
        &
        -
        \int_{T} \tensori{f}{}_V \cdot \tensori{u}{}_{T}
        -
        \int_{\partial_N T} \tensori{t}{}_{\partial T} \cdot \tensori{u}{}_{\partial T}
    \end{aligned}
\end{equation}
%
Assuming that the displacement is continuous at the boundary $\partial T$ such that $\tensori{u}{}_{\partial T} - \tensori{u}{}_{T} \vert_{\partial T} = 0$
% and letting $\tensori{u}{}_{\overline{T}}$ the continuous displacement in $\overline{T}$,
one recovers the usual expression of the Hu–Washizu integral over the element for the three variables $(\tensori{u}{}_{T}, \tensorii{G}{}_{T}, \tensorii{P}{}_{T})$. However, if one considers that $\tensori{u}{}_{\partial T}$ and $\tensori{u}{}_{T}$ are disticnt variables, \textit{i.e.} that the displacement across $\partial T$ is discontinuous, the functional writes as a function of the four variables $(\tensori{u}{}_{T}, \tensori{u}{}_{\partial T}, \tensorii{G}{}_{T}, \tensorii{P}{}_{T})$.
Differentiating $J_T$ over each of these variables, and introducing the numerical flux $\tensori{\theta}{}_{\partial T} = \tensorii{P}{}_{T} \vert_{\partial T} \cdot \tensori{n}{} + (\beta / h_T) (\tensori{u}{}_{\partial T} - \tensori{u}{}_{T} \vert_{\partial T})$ one obtains the system:
%
\begin{subequations}
    \label{eq_0017}
        \begin{alignat}{3}
            \frac{\partial J_{T}}{\partial \tensori{u}{}_{T}} \delta \tensori{u}{}_{T}
            = & \int_{T} (\tensorii{P}{}_{T} : \nabla_X \delta \tensori{u}{}_{T} - \tensori{f}{}_V) \cdot \delta \tensori{u}{}_{T}
            -
            \int_{\partial T} \tensori{\theta}{}_{\partial T} \cdot \delta \tensori{u}{}_{T} \vert_{\partial T}
            &&
            \ \ \ \ \ \ \ \ 
            &&
            \forall \delta \tensori{u}{}_{T}
            % \in H^1(T, \mathbb{R}^d)
        \label{eq_0017:eq0}
        \\
            \frac{\partial J_{T}}{\partial \tensori{u}{}_{\partial T}} \delta \tensori{u}{}_{\partial T}
            = &
            \int_{\partial T} (\tensori{\theta}{}_{\partial T} - \tensori{t}{}_N) \cdot \delta \tensori{u}{}_{\partial T}
            &&
            \ \ \ \ \ \ \ \ 
            &&
            \forall \delta \tensori{u}{}_{\partial T}
            % \in H^1(\partial T, \mathbb{R}^d)
        \label{eq_0017:eq1}
        \\
            \frac{\partial J_{T}}{\partial \tensorii{G}{}_{T}} \delta \tensorii{G}{}_{T}
            = &
            \int_{T} (\frac{\partial \psi_T}{\partial \tensorii{G}{}_{T}} - \tensorii{P}{}_{T}) : \delta \tensorii{G}{}_{T}
            &&
            \ \ \ \ \ \ \ \ 
            &&
            \forall \delta \tensorii{G}{}_{T}
            % \in L^2(T, \mathbb{R}^{d \times d})
        \label{eq_0017:eq2}
        \\
            \frac{\partial J_{T}}{\partial \tensorii{P}{}_{T}} \delta \tensorii{P}{}_{T}
            = & \int_{T} (\nabla_X \tensori{u}{}_{T} - \tensorii{G}{}_{T} ) : \delta \tensorii{P}{}_{T}
            +
            \int_{\partial T} (\tensori{u}{}_{\partial T} - \tensori{u}{}_{T} \vert_{\partial T}) \otimes \tensori{n}{} : \delta \tensorii{P}{}_{T} \vert_{\partial T}
            &&
            \ \ \ \ \ \ \ \ 
            &&
            \forall \delta \tensorii{P}{}_{T}
            % \in H^1_{\text{div}}(T, \mathbb{R}^{d \times d})
        \label{eq_0017:eq3}
    \end{alignat}
\end{subequations}
%
By explicitly eliminating \eqref{eq_0017:eq2} and \eqref{eq_0017:eq3} from the system, one obtains the problem in primal form: find
% $(\tensori{u}{}_{T}, \tensori{u}{}_{\partial T}) \in H^1(T, \mathbb{R}^d) \times L^2(\partial T, \mathbb{R}^d)$,
$(\tensori{u}{}_{T}, \tensori{u}{}_{\partial T})$,
such that for all
% $(\delta \tensori{u}{}_{T}, \delta \tensori{u}{}_{\partial T}) \in H^1(T, \mathbb{R}^d) \times L^2(\partial T, \mathbb{R}^d)$
$(\delta \tensori{u}{}_{T}, \delta \tensori{u}{}_{\partial T})$
%
\begin{equation}
    \label{eq_0018}
    \begin{aligned}
        d J_{T}=
        \frac{\partial J_{T}}{\partial \tensori{u}{}_{T}} \delta \tensori{u}{}_{T}
        +
        \frac{\partial J_{T}}{\partial \tensori{u}{}_{\partial T}} \delta \tensori{u}{}_{\partial T}
        =
        0
    \end{aligned}
\end{equation}
%
injetcing \eqref{eq_0017:eq0} and \eqref{eq_0017:eq1} :
%
\begin{equation}
    \label{eq_0019}
    \begin{aligned}
        d J_{T}
        = &
        \int_{T} \tensorii{P}{}_{T} : \nabla_X \delta \tensori{u}{}_{T}
        % \\
        % &
        +
        \int_{\partial T} (\delta \tensori{u}{}_{\partial T} - \delta \tensori{u}{}_{T} \vert_{\partial T}) \otimes \tensori{n} : \tensorii{P}{}_{T} \vert_{\partial T}
        % \\
        % &
        +
        \int_{\partial T} (\beta / h_T) (\tensori{u}{}_{\partial T} - \tensori{u}{}_{T} \vert_{\partial T}) \cdot (\delta \tensori{u}{}_{\partial T} - \delta \tensori{u}{}_{T} \vert_{\partial T})
        \\
        &
        -
        \int_{\partial T} \tensori{t}{}_N \cdot \delta \tensori{u}{}_{\partial T}
        -
        \int_{T} \tensori{f}{}_V \cdot \delta \tensori{u}{}_{T}
    \end{aligned}
\end{equation}
%
using both \eqref{eq_0017:eq2} and \eqref{eq_0017:eq3} :
%
\begin{equation}
    \label{eq_0020}
    \begin{aligned}
        d J_{T}
        & =
        \int_{T} \frac{\partial \psi_T}{\partial \tensorii{G}{}_T} : \delta \tensorii{G}{}_{T}
        % \\
        % &
        +
        \int_{\partial T} (\beta / h_T) (\tensori{u}{}_{\partial T} - \tensori{u}{}_{T} \vert_{\partial T}) \cdot (\delta \tensori{u}{}_{\partial T} - \delta \tensori{u}{}_{T} \vert_{\partial T})
        % \\
        % &
        -
        \int_{\partial T} \tensori{t}{}_N \cdot \delta \tensori{u}{}_{\partial T}
        -
        \int_{T} \tensori{f}{}_V \cdot \delta \tensori{u}{}_{T}
    \end{aligned}
\end{equation}
%
where $\delta \tensorii{G}{}_{T}$ (respectively $\tensorii{G}{}_{T}$) solves \eqref{eq_0017:eq3} for the unknowns set $(\delta \tensori{u}{}_{T}, \delta \tensori{u}{}_{\partial T})$ (respectively $(\tensori{u}{}_{T}, \tensori{u}{}_{\partial T})$)

\subsection{Discretization}
\label{sec_21}

Problem \eqref{eq_0020} describes the continuous problem. The discrete problem consists in seeking the unknown couple $(\tensori{u}{}_{T}, \tensori{u}{}_{\partial T})$ in a .



Le problème (\ref{eq_hu_washizu_hho}) discrétisé consiste à chercher l'inconnue $(\tensori{u}{}_{{T}}^l, \tensori{u}{}_{\partial T}^k)$ dans l'espace des polynômes $P^l({T}, \mathbb{R}^d) \times P^k(\partial T, \mathbb{R}^d)$ d'ordre respectivement $l$ et $k$ tels que $k > 0$ avec $k - 1 \leq l \leq k + 1$, et les champs de gradients de déplacement $\tensorii{G}{}_T^k$ et de contraintes $\tensorii{P}{}_T^k$ dans $P^k({T}, \mathbb{R}^{d \times d})$. On définit la force de traction discrète $\tensori{\theta}{}_{\partial T}^{HHO} = \tensorii{P}{}_T^k \cdot \tensori{n} + ({\beta_{mec}}/{h_T}) \tensori{S}_{\partial T}^{k*}$ telle que $\tensori{S}_{\partial T}^{k*}$ est l'opérateur adjoint de l'opérateur de stabilisation $\tensori{S}_{\partial T}^{k}$ définit par:
%
\begin{equation}
    \label{eq_stabilisation}
    \tensori{S}_{\partial T}^{k}(\tensori{v}{}_{T}^l, \tensori{v}{}_{\partial T}^k) = \Pi_{\partial T}^k
    (
    \tensori{v}{}_{\partial T}^k - \tensori{v}{}_{T}^l
    - (\tensori{1}{}-\Pi_{T}^k) \tensori{D}{}_T^{k + 1}
    )
\end{equation}
%
où $\Pi_{\partial T}^k$ et $\Pi_{T}^k$ sont les projecteurs orthogonaux au sens $L^2$ sur $P^k({\partial T}, \mathbb{R}^d)$ et $P^k({T}, \mathbb{R}^d)$ respectivement, et le champ de déplacement $\tensori{D}{}_T^{k + 1} \in P^{k+1}(T, \mathbb{R}^d)$ est solution du problème (\ref{eq_potential}):
%
\begin{equation}
    \label{eq_potential}
    \begin{aligned}
    \int_T (\nabla_X \tensori{D}{}_{T}^{k+1} - \nabla_X \tensori{u}{}_{T}^l) : \nabla_X \tensori{w}{}^{k+1} & = 
    \int_{\partial T} (\tensori{u}{}_{\partial T}^k - \tensori{u}{}_{T}^l) \cdot \nabla_X \tensori{w}{}^{k+1} \tensori{n}{}
    &&
    \forall \tensori{w}{}^{k+1} \in {P}{}^{k+1}(T, \mathbb{R}^d)
    \\
    \int_T \tensori{D}{}_{T}^{k+1} & = \int_T \tensori{u}{}_{T}^{l}
    \end{aligned}
\end{equation}

D'un point de vue numérique, on calcule dans une étape de pré-traitement l'opérateur de stabilisation ${[S]} : (\tensori{v}{}_{T}^l, \tensori{v}{}_{\partial T}^k) \rightarrow \tensori{S}{}_{\partial T}^{k}$ défini par (\ref{eq_stabilisation}) et l'opérateur de dérivation ${[B]} : (\tensori{v}{}_{T}^l, \tensori{v}{}_{\partial T}^k) \rightarrow \tensorii{G}{}_{T}^{k}$ défini par la formulation discrète de (\ref{eq_hu_washizu_hho_1}), de sorte que le problème discrétisé local (\ref{eq_hu_washizu_hho}) ne dépend plus que de l'inconnue primale $(\tensori{u}{}_{T}^l, \tensori{u}{}_{\partial T}^k)$ vérifiant $\forall (\tensori{v}{}_{{T}}^l, \tensori{v}{}_{\partial T}^k) \in P^l({T}, \mathbb{R}^d) \times P^k(\partial T, \mathbb{R}^d)$:
%
\begin{equation}
    \label{eq_ptv_hho}
    \int_{T} \tensorii{P}{}_{T}^k : \tensorii{G}{}_{T}^k
    +
    \int_{\partial_T} \frac{\beta_{mec}}{h_T}
    \tensori{S}_{\partial T}^{k}(\tensori{u}_{T}^l, \tensori{u}_{\partial T}^k)
    \cdot
    \tensori{S}_{\partial T}^{k}(\tensori{v}_{T}^l, \tensori{v}_{\partial T}^k)
    =
    % \sum_{T \in \mathcal{T}(\Omega_0)}
    \int_{\Omega} \tensori{f}{}_{V} \cdot \tensori{v}{}_{T}^l
    +
    % \sum_{\partial T \in \mathcal{F}_N(\Omega_0)}
    \int_{\partial_T}\tensori{t}{}_{N} \cdot \tensori{v}{}_{\partial T}^k
\end{equation}
%
où les contraintes $\tensorii{P}{}_{T}^k$ sont calculées aux points de quadrature par intégration de la loi de comportement. Le principe des travaux virtuels discret à l'échelle de la structure vérifie donc $\forall (\tensori{v}{}_{\mathcal{T}}^l, \tensori{v}{}_{\mathcal{F}}^k) \in P^l(\mathcal{T}, \mathbb{R}^d) \times P^k(\mathcal{F}, \mathbb{R}^d)$:

\begin{equation}
    \label{eq_hho2}
    \begin{aligned}
        \sum_{T \in \mathcal{T}(\Omega_0)}
        \int_{T} \tensorii{P}{}_{T}^k : \tensorii{G}{}_{T}^k + \int_{\partial_T} \frac{\beta_{mec}}{h_T}
        \tensori{S}_{\partial T}^k(\tensori{u}_{T}^l, \tensori{u}_{\partial T}^k) \cdot
        \tensori{S}_{\partial T}^k(\tensori{v}_{T}^l, \tensori{v}_{\partial T}^k)
        = &
        \sum_{T \in \mathcal{T}(\Omega_0)}
        \int_{\Omega} \tensori{f}{}_{V} \cdot \tensori{v}{}_{T}^l
        \\
        & +
        \sum_{\partial T \in \mathcal{F}_N(\Omega_0)}
        \int_{\partial_T}\tensori{t}{}_{N} \cdot \tensori{v}{}_{\partial T}^k
    \end{aligned}
\end{equation}

\bibliographystyle{elsarticle-num}
\bibliography{bib}

%% Authors are advised to use a BibTeX database file for their reference list.
%% The provided style file elsarticle-num.bst formats references in the required Procedia style

%% For references without a BibTeX database:

% \begin{thebibliography}{00}

%% \bibitem must have the following form:
%%   \bibitem{key}...
%%

% \bibitem{}

% \end{thebibliography}

% \end{multicols}

\end{document}

%%
%% End of file `ecrc-template.tex'. 