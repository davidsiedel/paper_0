
% Template for Elsevier CRC journal article
% version 1.2 dated 09 May 2011

% This file (c) 2009-2011 Elsevier Ltd.  Modifications may be freely made,
% provided the edited file is saved under a different name

% This file contains modifications for Procedia Computer Science
% but may easily be adapted to other journals

% Changes since version 1.1
% - added "procedia" option compliant with ecrc.sty version 1.2a
%   (makes the layout approximately the same as the Word CRC template)
% - added example for generating copyright line in abstract

%-----------------------------------------------------------------------------------

%% This template uses the elsarticle.cls document class and the extension package ecrc.sty
%% For full documentation on usage of elsarticle.cls, consult the documentation "elsdoc.pdf"
%% Further resources available at http://www.elsevier.com/latex

%-----------------------------------------------------------------------------------

%%%%%%%%%%%%%%%%%%%%%%%%%%%%%%%%%%%%%%%%%%%%%%%%%%%%%%%%%%%%%%
%%%%%%%%%%%%%%%%%%%%%%%%%%%%%%%%%%%%%%%%%%%%%%%%%%%%%%%%%%%%%%
%%                                                          %%
%% Important note on usage                                  %%
%% -----------------------                                  %%
%% This file should normally be compiled with PDFLaTeX      %%
%% Using standard LaTeX should work but may produce clashes %%
%%                                                          %%
%%%%%%%%%%%%%%%%%%%%%%%%%%%%%%%%%%%%%%%%%%%%%%%%%%%%%%%%%%%%%%
%%%%%%%%%%%%%%%%%%%%%%%%%%%%%%%%%%%%%%%%%%%%%%%%%%%%%%%%%%%%%%

%% The '3p' and 'times' class options of elsarticle are used for Elsevier CRC
%% Add the 'procedia' option to approximate to the Word template
%\documentclass[3p,times,procedia]{elsarticle}
\documentclass[3p,times,fleqn]{elsarticle}
% \documentclass[fleqn]{article}

%% The `ecrc' package must be called to make the CRC functionality available
\usepackage{ecrc}

%% The ecrc package defines commands needed for running heads and logos.
%% For running heads, you can set the journal name, the volume, the starting page and the authors

%% set the volume if you know. Otherwise `00'
\volume{00}

%% set the starting page if not 1
\firstpage{1}

%% Give the name of the journal
\journalname{Procedia Computer Science}

%% Give the author list to appear in the running head
%% Example \runauth{C.V. Radhakrishnan et al.}
\runauth{}

%% The choice of journal logo is determined by the \jid and \jnltitlelogo commands.
%% A user-supplied logo with the name <\jid>logo.pdf will be inserted if present.
%% e.g. if \jid{yspmi} the system will look for a file yspmilogo.pdf
%% Otherwise the content of \jnltitlelogo will be set between horizontal lines as a default logo

%% Give the abbreviation of the Journal.  Contact the journal editorial office if in any doubt
\jid{procs}

%% Give a short journal name for the dummy logo (if needed)
\jnltitlelogo{Procedia Computer Science}

%% Provide the copyright line to appear in the abstract
%% Usage:
%   \CopyrightLine[<text-before-year>]{<year>}{<restt-of-the-copyright-text>}
%   \CopyrightLine[Crown copyright]{2011}{Published by Elsevier Ltd.}
%   \CopyrightLine{2011}{Elsevier Ltd. All rights reserved}
\CopyrightLine{2011}{Published by Elsevier Ltd.}

%% Hereafter the template follows `elsarticle'.
%% For more details see the existing template files elsarticle-template-harv.tex and elsarticle-template-num.tex.

%% Elsevier CRC generally uses a numbered reference style
%% For this, the conventions of elsarticle-template-num.tex should be followed (included below)
%% If using BibTeX, use the style file elsarticle-num.bst

%% End of ecrc-specific commands
%%%%%%%%%%%%%%%%%%%%%%%%%%%%%%%%%%%%%%%%%%%%%%%%%%%%%%%%%%%%%%%%%%%%%%%%%%

%% The amssymb package provides various useful mathematical symbols
\usepackage{amssymb}
%% The amsthm package provides extended theorem environments
%% \usepackage{amsthm}

%% The lineno packages adds line numbers. Start line numbering with
%% \begin{linenumbers}, end it with \end{linenumbers}. Or switch it on
%% for the whole article with \linenumbers after \end{frontmatter}.
%% \usepackage{lineno}

%% natbib.sty is loaded by default. However, natbib options can be
%% provided with \biboptions{...} command. Following options are
%% valid:

%%   round  -  round parentheses are used (default)
%%   square -  square brackets are used   [option]
%%   curly  -  curly braces are used      {option}
%%   angle  -  angle brackets are used    <option>
%%   semicolon  -  multiple citations separated by semi-colon
%%   colon  - same as semicolon, an earlier confusion
%%   comma  -  separated by comma
%%   numbers-  selects numerical citations
%%   super  -  numerical citations as superscripts
%%   sort   -  sorts multiple citations according to order in ref. list
%%   sort&compress   -  like sort, but also compresses numerical citations
%%   compress - compresses without sorting
%%
%% \biboptions{comma,round}

% \biboptions{}

% if you have landscape tables
\usepackage[figuresright]{rotating}

% put your own definitions here:
%   \newcommand{\cZ}{\cal{Z}}
%   \newtheorem{def}{Definition}[section]
%   ...

% add words to TeX's hyphenation exception list
%\hyphenation{author another created financial paper re-commend-ed Post-Script}

% declarations for front matter
\usepackage{import}
\usepackage{example}
\usepackage{graphicx}
\usepackage{amsmath}
\usepackage{float}
\usepackage{amssymb}
\usepackage{xcolor}
\usepackage{hyperref}
\usepackage{longtable}
\usepackage{notations}
\usepackage{listings}
\usepackage{multicol}
\usepackage{import}
\usepackage{comment}
% \usepackage{cancel}
\usepackage{ulem}

\newtheorem{theorem}{Theorem}[section]
\newtheorem{corollary}{Corollary}[theorem]
\newtheorem{lemma}[theorem]{Lemma}
\newtheorem{property}[theorem]{Property}
\newtheorem{development}[theorem]{Development}

\newcommand{\paren}[1]{\left(#1\right)}
\newcommand{\deriv}[2]{\displaystyle \frac{\displaystyle \partial #1}{\displaystyle \partial #2}}
\newcommand{\variation}[2]{\displaystyle \frac{\displaystyle \delta #1}{\displaystyle \delta #2}}
\newcommand{\argmin}{\mathrm{argmin}}

\newcommand\bodyEul{\Omega_t}
\newcommand\bodyLag{\Omega}
\newcommand\dBodyEul{\partial \Omega_t}
\newcommand\dBodyLag{\partial \Omega}
\newcommand\dirichletBoundaryEul{\partial_d \Omega_t}
\newcommand\neumannBoundaryEul{\partial_n \Omega_t}
\newcommand\dirichletBoundaryLag{\partial_D \Omega}
\newcommand\neumannBoundaryLag{\partial_N \Omega}

\newcommand\mecPotential{\psi}


\newcommand\CAset{U}

\newcommand\cell{T}
\newcommand\dCell{\partial T}
\newcommand\dirichletCell{\partial_D T}
\newcommand\neumannCell{\partial_N T}
\newcommand\neumannCellLoad{\tensori{t}{}_{\partial_N T}}

% \newcommand\matA{K}
% \newcommand\dMatA{\partial K}
% \newcommand\dirichletMatA{\partial_D K}
% \newcommand\neumannMatA{\partial_N K}
% \newcommand\neumannMatALoad{\tensori{t}{}_{\partial_N K}}

\newcommand\matI{T_i}
\newcommand\dMatI{\partial T_i}
\newcommand\dirichletMatI{\partial_D T_i}
\newcommand\neumannMatI{\partial_N T_i}
\newcommand\neumannMatILoad{\tensori{t}{}_{\partial_N T_i}}

% \newcommand\matB{H}
% \newcommand\dMatB{\partial H}
% \newcommand\dirichletMatB{\partial_D H}
% \newcommand\neumannMatB{\partial_N H}
% \newcommand\neumannMatBLoad{\tensori{t}{}_{\partial_N H}}

\newcommand\Bulk{K}
\newcommand\dBulk{\partial K}
\newcommand\Crown{I}
\newcommand\dCrown{\partial I}

\newcommand\HybridMesh{\mathcal{T}}
\newcommand\dHybridMesh{\mathcal{F}}
\newcommand\HybridMeshWhole{\bar{\mathcal{T}}}

\newcommand\displacementSpaceHybridMesh{U(\mathcal{T})}
\newcommand\displacementSpaceDHybridMesh{V(\mathcal{F})}
\newcommand\hybridDisplacementSpaceHybridMesh{U(\bar{\mathcal{T}})}

\newcommand\virtualDisplacementSpaceHybridMesh{U_0(\mathcal{T})}
\newcommand\virtualDisplacementSpaceDHybridMesh{V_0(\mathcal{F})}
\newcommand\virtualHybridDisplacementSpaceHybridMesh{U_0(\bar{\mathcal{T}})}

\newcommand\discreteDisplacementSpaceHybridMesh{U^h(\mathcal{T})}
\newcommand\discreteDisplacementSpaceDHybridMesh{V^h(\mathcal{F})}
\newcommand\discreteHybridDisplacementSpaceHybridMesh{U^h(\bar{\mathcal{T}})}
\newcommand\discreteVirtualDisplacementSpaceHybridMesh{U_0^h(\mathcal{T})}
\newcommand\discreteVirtualDisplacementSpaceDHybridMesh{V_0^h(\mathcal{F})}
\newcommand\discreteVirtualHybridDisplacementSpaceHybridMesh{U_0^h(\bar{\mathcal{T}})}

% \newcommand\bound{\partial{}^{\text{e}} T}
% \newcommand\neumannBound{\partial{}^{\text{e}}{}_N T}
% \newcommand\dirichletBound{\partial{}^{\text{e}}{}_D T}
% \newcommand\neumannBoundLoad{\tensori{t}{}_{\partial{}^{\text{e}}{}_N T}}

% \newcommand\bound{\partial T}
% \newcommand\neumannBound{\partial_N T}
% \newcommand\dirichletBound{\partial_D T}
% \newcommand\neumannBoundLoad{\tensori{t}{}_{\partial_N T}}

\newcommand\loadLag{\tensori{f}{}_V}
\newcommand\loadEul{\tensori{f}{}_v}
\newcommand\neumannLag{\tensori{t}{}_N}
\newcommand\neumannEul{\tensori{t}{}_n}
\newcommand\dirichletLag{\tensori{u}{}_D}
\newcommand\dirichletEul{\tensori{u}{}_d}

\newcommand\internaleStateVariables{v_{int}}

\newcommand\PK{\tensorii{P}}
\newcommand\G{\tensorii{G}}
\newcommand\F{\tensorii{F}}

\newcommand\gradSpaceMatI{G(\matI)}
\newcommand\stressSpaceMatI{S(\matI)}
\newcommand\displacementSpaceMatI{U(\matI)}
\newcommand\displacementSpaceDMatI{V(\dMatI)}

\newcommand\potentialSpaceCell{D(\cell)}
\newcommand\gradSpaceCell{G(\cell)}
\newcommand\stressSpaceCell{S(\cell)}
\newcommand\displacementSpaceCell{U(\cell)}
\newcommand\displacementSpaceDCell{V(\dCell)}
\newcommand\virtualDisplacementSpaceCell{U_0(\cell)}
\newcommand\virtualDisplacementSpaceDCell{V_0(\dCell)}
\newcommand\hybridDisplacementSpaceCell{U(\bar{\cell})}
\newcommand\virtualHybridDisplacementSpaceCell{U_0(\bar{\cell})}

\newcommand\discretePotentialSpaceCell{D^h(\cell)}
\newcommand\discreteGradSpaceCell{G^h(\cell)}
\newcommand\discreteStressSpaceCell{S^h(\cell)}
\newcommand\discreteDisplacementSpaceCell{U^h(\cell)}
\newcommand\discreteDisplacementSpaceDCell{V^h(\dCell)}
\newcommand\discreteVirtualDisplacementSpaceCell{U_0^h(\cell)}
\newcommand\discreteVirtualDisplacementSpaceDCell{V_0^h(\dCell)}
\newcommand\discreteHybridDisplacementSpaceCell{U^h(\bar{\cell})}
\newcommand\discreteVirtualHybridDisplacementSpaceCell{U_0^h(\bar{\cell})}

\newcommand\gradSpaceBulk{G(\Bulk)}
\newcommand\stressSpaceBulk{S(\Bulk)}
\newcommand\displacementSpaceBulk{U(\Bulk)}
\newcommand\displacementSpaceDBulk{V(\dBulk)}

\newcommand\gradSpaceCrown{G(\Crown)}
\newcommand\stressSpaceCrown{S(\Crown)}
\newcommand\displacementSpaceCrown{U(\Crown)}
\newcommand\displacementSpaceDCrown{V(\dCrown)}

\newcommand\cellQuadrature{Q_T}
\newcommand{\commentTH}[1]{
  \textcolor{blue}{\textbf{Commentaire T.H.}~: #1}
}

\begin{document}

\includecomment{comment_TH}
% \begin{multicols}{0}

\begin{frontmatter}

%% Title, authors and addresses

%% use the tnoteref command within \title for footnotes;
%% use the tnotetext command for the associated footnote;
%% use the fnref command within \author or \address for footnotes;
%% use the fntext command for the associated footnote;
%% use the corref command within \author for corresponding author footnotes;
%% use the cortext command for the associated footnote;
%% use the ead command for the email address,
%% and the form \ead[url] for the home page:
%%
%% \title{Title\tnoteref{label1}}
%% \tnotetext[label1]{}
%% \author{Name\corref{cor1}\fnref{label2}}
%% \ead{email address}
%% \ead[url]{home page}
%% \fntext[label2]{}
%% \cortext[cor1]{}
%% \address{Address\fnref{label3}}
%% \fntext[label3]{}

\dochead{}
%% Use \dochead if there is an article header, e.g. \dochead{Short communication}
%% \dochead can also be used to include a conference title, if directed by the editors
%% e.g. \dochead{17th International Conference on Dynamical Processes in Excited States of Solids}

\title{}

%% use optional labels to link authors explicitly to addresses:
%% \author[label1,label2]{<author name>}
%% \address[label1]{<address>}
%% \address[label2]{<address>}

\author{}

\address{}

\begin{abstract}
%% Text of abstract
The Hybrid High Order (HHO) method is a powerful
discretization method which has only been recently applied to
non linear computational mechanics.

The Hybrid High Order method divides the domain of interest
in cells of arbitrary polyhedral shape, whose boundaries form
the skeleton of the mesh, and introduces two kinds of degrees of freedom: the
displacements in cells and the displacements of the skeleton.

Most introductory materials to the HHO method is focused on
mathematical aspects. While they are
important, an approach based on physical considerations would help
spreading this method to the computational mechanics and engineering
communities.

This paper derives Hybrid High Order method from the
classical Hu–Washizu functional.

Practical implementation of the method is discussed in
depth using notations closed to the ones used in standard finite
elements textbooks, highlighting the use of polyhedral cells and the
use of approximation spaces based on polynomials of arbitrary
orders.

From the point of view of numerical performances, the
elimination of the cell degrees of freedom is mandatory to reduce
the size of the stiffness matrix. The standard static condensation,
is presented, as well as a novel strategy called "cell equilibrium".
Advantages and disadvantages of both strategies are discussed.

The resolution of axi-symmetrical problems, which has seldom,
if ever, been discussed in the literature, is then presented.

Numerical examples prove the robustness of the method
with regards to volumetric locking

\end{abstract}

\begin{keyword}
%% keywords here, in the form: keyword \sep keyword

%% PACS codes here, in the form: \PACS code \sep code

%% MSC codes here, in the form: \MSC code \sep code
%% or \MSC[2008] code \sep code (2000 is the default)
Computational mechanics \sep Hybrid High Order
method \sep Static condensation \sep Cell equilibrium algorithm \sep
Volumetric locking \sep Axi-symmetric modelling hypothesis
\end{keyword}

\end{frontmatter}

%%
%% Start line numbering here if you want
%%
% \linenumbers

%% main text
  
\tableofcontents

\section{Introduction}

% The origin of DG methods dates back to the pioneering work of
% \cite{reed_triangular_1973}, where an hyperbolic formualtion is used to
% solve the neutron transport equation.

The Hybird High Order method (HHO) is a discontinuous discretization
method, that takes root in the Discontinuous Galerkin method (DG). From
the physical standpoint, DG methods ensure the continuity of the flux
across interfaces, by seeking the solution element-wise, hence allowing
jumps of the potential across elements. They can be seen as a
generalization of Finite Volume methods, and are able to capture
physically relevant discontinuities without producing spurious
oscillations.

The origin of DG methods dates back to the pioneering work of
\cite{reed_triangular_1973}, where an hyperbolic formualtion is used to
solve the neutron transport equation. The first application of the
method to elliptic problems originates in \cite{babuska_finite_1973}
where Nitsche's method \cite{nitsche_uber_1970} is used to weakly impose
continuity of the flux across interfaces. \textcolor{blue} { In 2002,
  Hansbo and Larson \cite{hansbo_discontinuous_2002-1} were the first to
  consider the Nitsche's classical DG method for nearly incompressible
  elasticity. They showed, theoretically and numerically, that this
  method is free from volumetric locking. } However, the bilinear form
arising from this formulation is not symmetric. A so called interior
penalty term has been introduced in \cite{wheeler_elliptic_1978},
leading to the Symmetric Interior Penalty (SIP) DG method. A first study
of the method to linear elasticity has been devised by
\cite{riviere_optimal_2000}, where optimal error estimate has been
proved. \textcolor{blue} { \cite{lew_optimal_2004} generalized the
  Symmetric Interior Penalty method to linear elasticity. }
\textcolor{blue} {
  % In 2002, Hansbo and Larson \cite{hansbo_discontinuous_2002-1} were the first to
 % consider the Nitsche's classical DG method for nearly incompressible
  % elasticity. They showed, theoretically and numerically, that this method
 % is free from volumetric locking. % \cite{lew_optimal_2004}
  % generalized the Symmetric
  % Interior Penalty method to linear elasticity. In about the same
  period of time, DG methods were proposed for other linear problems in
  solid mechanics, such as Timoshenko beams
  \cite{celiker_locking-free_2006}, Bernoulli-Euler beam and the
  Poisson-Kirchhoff plate \cite{brenner_balancing_1999,
    engel_continuousdiscontinuous_2002} and Reissner-Mindlin plates
  \cite{arnold_family_2005}. In the mid 2000's, the first applications
  of DG methods to nonlinear elasticity problems was undertaken by
  \cite{ten_eyck_discontinuous_2006, noels_general_2006}, and in 2007,
  Ortner and Süli \cite{ortner_discontinuous_2007} carried out the a
  priori error analysis of DG methods for nonlinear elasticity.
  % This pioneering work
  % shed light on how to calculate a lower bound on the stability parameters.
 }

DG methods then sollicitated a vigourus interest, mostly in fluid dynamics \cite{shahbazi_high-order_2007, persson_discontinuous_2009} due to their local conservative property and stability in convection domniated problems. However, except some applications for instance in fracture mechanics using XFEM methods \cite{gracie_blending_2008, shen_stability_2010}, or gradient plasticity \cite{djoko_discontinuous_2007,djoko_discontinuous_2007-1} DG methods did not break through in computational solid mechanics because of their numerical cost, since nodal unknowns need be duplicated to define local basis functions in each element.

To adress this problem, in the early 2010's, \cite{cockburn_unified_2009, soon_hybridizable_2009} introduced additional faces unknowns on element interfaces for linear elastic problem, hence leading to the hybridization of DG methods, or Hybridizable Discontinuous Galerkin method (HDG). By adding supplementary boundary unknowns, the authors actually allowed to eliminate original cell unknowns by a static condensation process, in order to express the global problem on faces ones only. Extension of HDG methods to non-linear elasticity were first undertaken in \cite{soon_hybridizable_2008} and have then fueled intense reaserch works for various applications such as linear and non-linear convection-diffusion problems \cite{nguyen_implicit_2009,nguyen_implicit_2009-1,nguyen_hybridizable_2010}, incompressible stokes flows \cite{nguyen_hybridizable_2010, nguyen_implicit_2011} and non-linear mechanics \cite{nguyen_hybridizable_2012}.

In \cite{di_pietro_hybrid_2015, di_pietro_arbitrary-order_2014}, the authors introduced a higher order potential reconstruction operator in the classical HDG formulation for elliptic problems, providing a $h^{k+1} H^1$-norm convergence rate as compared to the ususal $h^k$-rate. This higher order term coined the name for the so called HHO method.
Recent developments of HHO methods in
computational mechanics include the incompressible Stokes
equations (with possibly large irrotational forces) \cite{di_pietro_discontinuous_2016}, the
incompressible Navier–Stokes equations \cite{di_pietro_hybrid_2018}, Biot’s consolidation problem \cite{boffi_nonconforming_2016}, and nonlinear elasticity with small
deformations \cite{botti_hybrid_2017}
\section{The model problem}
\label{sec_model_problem}

Paragraph~\ref{sec:Hu_Washizu_functional}
introduces the classical Hu–Washizu functional to describe the
quasi-static equilibrium of a body submitted to external load and the
main notations used in this paper. For the sake of simplicity, the body
is assumed hyper-elastic in this section.

Paragraph~\ref{sec:Hu_Washizu_functional}
introduces the classical Hu–Washizu functional to describe the
quasi-static equilibrium of a body submitted to external load and the
main notations used in this paper. For the sake of simplicity, the body
is assumed hyper-elastic in this section.

Paragraph~\ref{sec:HHO} introduces the key idea
of the HHO method, which is to divide the domain in arbitrary subdomains
connected by cohesive interfaces and to apply the Hu–Washizu functional
to each sub-domains.

\subsection{The standard Hu–Washizu lagragian}
\label{sec_Hu_Washizu_functional}

This paragraph introduces the standard Hu–Washizu three field
principle. For the sake of simplicity, and without loss of generality,
we consider the case of an hyperelastic material. Extensions to
mechanical behaviours with internal state variables is treated in
classical textbooks of computational mechanics. We will treat this
extension in the Section~\ref{sec_implementation} discussing the
numerical implementation of the Hybrid High Order method and in
Section~\ref{sec_numerical_examples} which provides several examples in
plasticity.

% \subsubsection{Description of the mechanical problem and notations}

\paragraph{Solid body}

Let us consider a solid body whose reference configuration is denoted
$\bodyLag$. At a given time $t > 0$, the body is in the current
configuration $\bodyEul$.

\paragraph{Mechanical loading}

The body is assumed to be submitted to a body force $\loadEul$ acting
in $\bodyEul$, a prescribed displacement $\dirichletEul$ on the
Dirichlet boundary $\dirichletBoundaryEul$, and a contact load
$\neumannEul{}$ on the Neumann boundary $\neumannBoundaryEul$.

\paragraph{Deformation}

The transformation mapping 
$\tensori{\Phi}$ takes a point from the reference configuration $\bodyLag$ to the current
configuration $\bodyEul$ such that
%
%
%
\begin{equation}
    \tensori{\Phi}\paren{\tensori{X}} = \tensori{x} = \tensori{X}+\tensori{u}\paren{X}
\end{equation}
%
%
%
where $\tensori{X}$, $\tensori{x}$ and $\tensori{u}$ denote respectively
the position in the reference configuration $\bodyLag$, the position
in the current configuration $\bodyEul$ and the displacement.

\paragraph{Deformation gradient, gradient of the displacement}

The deformation gradient $\tensorii{F}$ is defined as:
%
%
%
\begin{equation}
    \tensorii{F} = \nabla \tensori{\Phi} = \tensorii{I} + \tensorii{G}
\end{equation}
%
%
%
where $\nabla$ is the gradient operator in the
reference configuration and $\tensorii{G} = \nabla \tensori{u}$ denotes the gradient of the
displacement.

\paragraph{Hyperelastic material}

The body is assumed made of an hyperelastic material described by a
free energy $\mecPotential_{\bodyLag{}}$ which relates the deformation gradient
$\tensorii{F}$ and the first Piola-Kirchhoff stress tensor $\tensorii{P}$ as follows:
%
%
%
\begin{equation}
\tensorii{P}=\deriv{\mecPotential_{\bodyLag{}}}{\tensorii{F}}
\end{equation}

\paragraph{Total lagrangian}

The total Lagrangian 
$L^{VW}_{\bodyLag{}}$ of the body is defined as the stored energy minus the work of
external loadings as follows:
%
%
%
\begin{equation}
\label{eq_Lagrangian}
L^{VW}_{\bodyLag{}}
% \paren{\tensori{u}}
= \int_{\Omega}\mecPotential_{\bodyLag{}} \paren{\tensorii{F}\paren{\tensori{u}}}
- \int_{\bodyLag} \tensori{f}{}_V \cdot \delta \tensori{u}{}
- \int_{\neumannBoundaryLag} \neumannLag{} \cdot \delta \tensori{u}{}
\vert_{\neumannBoundaryLag}
\end{equation}
%
%
%
where the body forces $\tensori{f}_{V}$ and conctat tractions
$\neumannLag$ in the reference configuration have been obtained from
their counterparts $\tensori{f}_{v}$ and $\neumannEul$ thanks to the
Nanson formulae, i.e. $\tensori{f}_{V}=...$ and $\tensori{f}{}_V=...$
%
%\paragraph{Variational characterisation of the mechanical equilibrium}
%
The displacement $\tensori{u}$ satisfying
the mechanical equilibrium minimizes the Lagragian $L^{VW}_{\bodyLag{}}$:
%
%
%
\begin{equation}
\tensori{u} = \underset{\displaystyle\tensori{u}^{\star}\in U(\bodyLag)}{\argmin}\,
L^{VW}_{\bodyLag{}}\paren{\tensori{u}^{\star}}
\end{equation}
%
%
%
where $U(\bodyLag)$ denotes the set of admissible displacements.
Taking the first order variation of Lagrangian yields the principle of
virtual work:
%
%
%
\begin{equation}
    \label{eq_virtual_works_0}
    \frac{\partial L^{VW}_{\bodyLag{}}\paren{\tensori{u}}}{\partial \tensori{u}} =
    \int_{\bodyLag} \tensorii{P} : \nabla \delta \tensori{u} = 
    \int_{\bodyLag} \tensori{f}_V \cdot \delta \tensori{u} +
    \int_{\neumannBoundaryLag} \neumannLag{} \cdot \delta \tensori{u}
    \vert_{\neumannBoundaryLag}
\end{equation}

\paragraph{Hu-Washizu Lagrangian}

The Hu-Washizu Lagrangian generalizes the previous variational principle by
considering that the gradient of the displacement $\tensorii{G}$ and
the first Piola-Kirchoff $\tensorii{P}$ stress are independent
unknowns of the problem.
The Hu-Washizu Lagrangian $L^{HW}$ is then defined as follows:
%
%
%
\begin{equation}
L^{HW}\paren{\tensori{u},\tensorii{G},
  \tensorii{P}} = \int_{\bodyLag{}}
\mecPotential\paren{\tensorii{I}+\tensorii{G}} + (\nabla \tensori{u}{} -
\tensorii{G}{})\,\colon\,\tensorii{P} - \int_{\bodyLag{}} \loadLag \cdot
\tensori{u}{} - \int_{\neumannBoundaryLag{}} \neumannLag{} \cdot
\tensori{u}
\vert_{\neumannBoundaryLag}
\end{equation}
%
%
%
and the solution $\tensori{u}$, $\tensorii{G}$ and $\tensorii{P}$
minimize the Hu-Washizu Lagrangian $L^{HW}$:
%
%
%
\begin{equation}
\paren{\tensori{u},\, \tensorii{G},\, \tensorii{P}} =
\underset{\displaystyle\substack{\tensori{u}^{\star}\in U(\bodyLag),\\ \tensorii{G}^{\star},\tensorii{P}^{\star}}}{\argmin}\,
L\paren{\tensori{u}^{\star}, \tensorii{G}^{\star},\tensorii{P}^{\star}}
\end{equation}
%
%
%
The first order variation of the Hu-Washizu Lagragian with respect to
$\tensori{u}$, $\tensorii{G}$, $\tensorii{P}$ yields:
\begin{subequations}
    \label{eq_hu_washizu_derivative_0}
        \begin{alignat}{3}
           \variation{L^{HW}}{\tensori{u}}
            = & \int_{\bodyLag} \tensorii{P}\,\cdot\,\nabla \delta \tensori{u}
            -
            \int_{\bodyLag} \tensori{f}_V \cdot \delta \tensori{u}
            -
            \int_{\neumannBoundaryLag} \neumannLag \cdot \delta \tensori{u}
            \vert_{\neumannBoundaryLag}
        \label{eq_hu_washizu_derivative_0:eq0}
        \\
            \variation{L^{HW}}{\tensorii{P}}
            = & \int_{\bodyLag} \paren{\nabla \tensori{u} - \tensorii{G}}\,\cdot\,\delta \tensorii{P}
        \label{eq_hu_washizu_derivative_0:eq2}
        \\
            \variation{L^{HW}}{\tensorii{G}}
            = &
            \int_{\bodyLag} (\frac{\partial \mecPotential}{\partial \tensorii{G}} - \tensorii{P})\,\cdot\,\delta \tensorii{G}
        \label{eq_hu_washizu_derivative_0:eq3}
    \end{alignat}
\end{subequations}

Equation~\eqref{eq_hu_washizu_derivative_0:eq2}
shows that $\tensorii{G}$ is equal to $\nabla \tensori{u}$ is a weak
sense and Equation~\eqref{eq_hu_washizu_derivative_0:eq3} shows that the
First Piola Kirchhoff stress is equal to the derivative the free energy
in a weak sense.

\paragraph{Some words about the importance of the Hu-Washizu principle}

...
\section{Introduction to discontinuous methods through a Hu-Washizu formulation}
\label{sec_composite_demo}

Many numerical methods
consider a partition of the body into elementary subsets called \textit{cells}.
% The equilibrium of the body is then expressed in terms of the sum of the mechanical contribution in each of these cells composing it.
% The equilibrium of the interface between two cells is then of major 
For the equilibrium of the whole body to hold, each cell must be in equilibrium with its neighbors, which means that they have to interact and pass information from one another.
Communication between two cells is ensured by knowledge of the displacement field at their shared boundary.
For \textit{conformal} methods to which belongs the Lagrange (\textit{i.e.} the standard) Finite Element Method, the displacement over the whole body is continuous, which implies that the displacement at a cell boundary is directly equal to that of its neighbors, and a cell has a direct knowledge of the motion of its neighborhood.
For so-called \textit{non-conformal} methods, among which are \textit{e.g.} Discontinuous Galerkin methods, Hybrid Discontinuous Galerkin methods and Hybrid High Order methods, the displacement continuity at a cell boundary is not explicitly enforced, such that one needs to introduce supplementary ingredients in the formulation to pass information from one cell to another. A straightforward way of doing so consists in introducing an interface between cells, that acts as a membrane pulling them together. One can readily feel that the stiffer the membrane, the closer the cells, and the closer to the \textit{conformal} framework. This technique is the one at the foundation of Discontinuous Galerkin methods. By adding an intermediate structure in the membrane, called a \textit{bone}, one defines the framework of Hybrid Discontinuous Galerkin methods and Hybrid High Order methods; the membrane is split into two parts, one at each side of the bone, and communication between cells transits through the bone, via the membrane. The term hybrid expresses the fact that both cells and bones carry information about the displacement field, hence introducing the \textit{skeleton} of the body.
In the following section, we show that one recovers the full mathematical framework proper to these non-conformal method, by writing the equilibrium of a cell and its interface with its neighborhood.
%  cells must stick together in order to pass the mechanical information from one cell to another

% Let $\cell$ be
% such a cell and let $\dCell{}$ its boundary.
% % For the sake of
% % simplicity, the intersection of $\dCell{}$ and $\dBodyLag{}$
% % is first assumed empty. This special case is treated later.

% In the framework of conforming Galerkin methods, the displacement field is continuous at cells interfaces, and the equilibrium is expressed on the whole structure.
% For discontinuous Galerkin methods, the displacement continuity is broken, to ensure continuity of the flux between cells instead, such that the equilibrium of the whole structure is the sum of the equilibrium expressed within each cell.
% In this section, we show that this change of paradigm amounts to consider that a cell in the continuous framework is surrounded by an infinitely thin elastic interface.

% Let the equilibrium of a cell $\cell$
% %
% %
% %
% \begin{equation}
%     \label{eq_element}
%     L_{\cell}^{eq}
%     % (\tensori{u}{}_{\cell}, \tensorii{G}{}_{\cell}, \tensorii{P}{}_{\cell})
%     =
%     \int_{\cell} \mecPotential_{\bodyLag{}} + (\nabla_X \tensori{u}{}_{\cell} - \tensorii{G}{}_{\cell}) : \tensorii{P}{}_{\cell}
%     -
%     \int_{\cell} \loadLag \cdot \tensori{u}{}_{\cell}
%     % -
%     % \int_{\Crown} \loadLag \cdot \tensori{u}{}_{\Crown}
%     -
%     \int_{\neumannCell} \neumannCellLoad \cdot \tensori{u}{}_{\dCell}
% \end{equation}
% %
% %
% %
% One can either choose a suitable functional space such that the displacement is continuous at the interfaces, or 


% Je pense vraiment qu'il faut exprimer explicitement l'équilibre d'un élément avec le reste de la structure (tractions normales et continuité des déplacement), préciser que l'on peut imposer les continuités du déplacement au sens fort ou au sens faible et **ensuite** classifier les méthodes ainsi:
% Les éléments finis standards assurent la continuité des déplacements aux interfaces
% Les DG relient directement les éléments entre eux et pénalisent les sauts de déplacement entre les éléments
% Les HDG et HHO relient les éléments à leurs frontières et pénalisent les sauts de déplacement entre l'élément et la frontière

% Puis on indique qu'un moyen pratique de pénaliser les sauts de déplacement est d'introduire une interphase élastique de faible épaisseur et de faire tendre cette épaisseur vers 0.  Dans le cadre de ce papier, seules les méthodes HDG et HHO sont traitées (et donc pas les DG).

% Cette section montre que l'application de Hu-Hashizu à l'élément à son interphase permet de retrouver tous les ingrédients principaux des méthodes HDG et HH0: le gradient reconstruit et l'opérateur de stabilisation.

\subsection{Partie 1}

\paragraph{Element description}

In the following, the cell $\cell$ is assumed to be convex.
It is split into a core part $\Bulk \subset \cell$ with boundary $\dBulk$, and into an interface part $\Crown{} \subset \cell$ with boundary $\dCrown = \dBulk \cup \dCell$, as shown in Figure \ref{fig_02}. The interface $\Crown{}$ has some thickness $\ell > 0$ that is supposed to be small compared to $h_{\cell}$ the diameter of $\cell$.
From a geometrical standpoint, the core par of the element $\Bulk{}$ is an homotethy of $\cell$ by some ratio inferior to $1$.
%
% 
% 
\begin{figure}[H]
    \centering
    \includegraphics[width=12.cm]{img/hu_washizu.png}
    \caption{schematic representation of the composite region}
    \label{fig_02}
\end{figure}
%
%
%

\paragraph{Element behaviour}

The core of the element $\Bulk{}$ is made out of the same material that composes $\Omega$ and behaves according to the free energy potential $\mecPotential{}_{\bodyLag{}}$. The interface $\Crown{}$ is made out of a linear elastic material of Young modulus $\beta (\ell / h_{\cell})$ with a zero Poisson ratio and its behavior is defined by the free energy potential $\mecPotential{}_{\Crown{}}$ such that
%
%
%
\begin{equation}
    \label{eq_0009}
        \mecPotential{}_{\Crown} = \frac{1}{2} \beta \frac{\ell}{h_{\cell}} \nabla \tensori{u}{}_{\Crown} : \nabla \tensori{u}{}_{\Crown}
\end{equation}
%
%
%
where the dimensionless ratio $\ell / h_{\cell}$ balances the accumulated energy with the size of the domain $\cell$.

\paragraph{Element loading}

The core $\Bulk$ is subjected to the volumetric loading $\loadLag{}$, and to the contact load applied by the interface $\Crown{}$ onto $\dBulk{}$. By continuity of the traction force, the same opposite contact load acts on $\Crown{}$. The interface $\Crown{}$ is also subjected to some contact load $\neumannCellLoad{}$ acting on $\dCell{}$, that accounts for the action of the rest of the solid $\bodyLag{}$ onto $\cell$.

\paragraph{Element unknowns}

Let note $\tensori{u}{}_{\Bulk}$ the displacement field, $\tensorii{G}{}_{\Bulk}$ the displacement gradient field and $\tensorii{P}{}_{\Bulk}$ the stress field in $\Bulk{}$. Similarly, let $\tensori{u}{}_{\Crown{}}$ the displacement field, $\tensorii{G}{}_{\Crown}$ the displacement gradient field and $\tensorii{P}{}_{\Crown}$ the stress field in $\Crown{}$.
The displacement of the boundary $\dCell{}$ is denoted $\tensori{u}{}_{\dCell{}}$.
By continuity of the displacement field between $\Bulk{}$ and $\dCell$,  the displacement $\tensori{u}{}_{\Crown{}}$ verifies
%
% 
% 
\begin{subequations}
    \label{eq_conformity}
        \begin{alignat}{2}
        \tensori{u}{}_{\Crown} \vert_{\dBulk} & = \tensori{u}{}_{\Bulk} \vert_{\dBulk}
        \label{eq_conformity:eq1}
        \\
        \tensori{u}{}_{\Crown} \vert_{\dCell} & = \tensori{u}{}_{\dCell}
        \label{eq_conformity:eq2}
    \end{alignat}
\end{subequations}

\paragraph{Hu-Washizu Lagrangian of the element}

By combining both the Lagragian of the core $\Bulk{}$ and that of the interface $\Crown{}$, one obtains the total Lagragian $L_{\cell}^{HW}$ over the element such that
%
%
%
\begin{equation}
    \label{eq_hu_washizu_split}
    L_{\cell}^{HW}
    % (\tensori{u}{}_{\cell}, \tensorii{G}{}_{\cell}, \tensorii{P}{}_{\cell})
    =
    \int_{\Bulk} \mecPotential_{\bodyLag{}} + (\nabla_X \tensori{u}{}_{\Bulk} - \tensorii{G}{}_{\Bulk}) : \tensorii{P}{}_{\Bulk}
    +
    \int_{\Crown} \mecPotential_{\Crown{}} + (\nabla_X \tensori{u}{}_{\Crown} - \tensorii{G}{}_{\Crown}) : \tensorii{P}{}_{\Crown}
    -
    \int_{\Bulk} \loadLag \cdot \tensori{u}{}_{\Bulk}
    % -
    % \int_{\Crown} \loadLag \cdot \tensori{u}{}_{\Crown}
    -
    \int_{\neumannCell} \neumannCellLoad \cdot \tensori{u}{}_{\dCell}
\end{equation}

\subsection{Hypotheses}
\label{sec_assumtions}

The behavior and the kinematics having been described, let now make a number of assumptions on the expression of the fields of unknowns, exploiting the fact that the core is of negligible volume compared to the interface.

\paragraph{Displacement in the interface}

Since the interface $\Crown$ is thin compared to the cell volume $\cell$, let linearize the displacement in the interface $\Crown$ with respect to $\tensori{n}$, such that
its gradient is homogeneous in $\Crown{}$
%
% 
% 
% \begin{equation}
%     \label{eq_crown_displacement}
%     \tensori{u}{}_{\Crown} (\tensori{x})
%     =
%     \frac{\tensori{u}{}_{\dCell}(\tensori{m}{}_{\dCell})
%     -
%     \tensori{u}{}_{\Bulk} \vert_{\dBulk} (\tensori{m}{}_{\dBulk})}{\ell} \otimes \tensori{n} \cdot (\tensori{x} - \tensori{m}{}_{\dBulk})
%     +
%     \tensori{u}{}_{\Bulk} \vert_{\dBulk}(\tensori{m}{}_{\dBulk})
% \end{equation}
\begin{equation}
    \label{eq_crown_displacement}
    \nabla
    \tensori{u}{}_{\Crown}
    =
    \frac{\tensori{u}{}_{\dCell}
    -
    \tensori{u}{}_{\Bulk} \vert_{\dBulk} }{\ell} \otimes \tensori{n}
\end{equation}
% 
% 
%
That is, the displacement of the interface $\Crown{}$ linearly bridges that of the boundary $\dCell{}$ to that of the bulk $\Bulk{}$.

\paragraph{Stress in the interface}

Furthermore, let assume that $\tensorii{P}{}_{\Crown}$ is constant along the direction $\tensori{n}{}$ in $\Crown{}$. By continuity of the traction force across $\dBulk$, the following equality holds true
%
% 
% 
\begin{equation}
    \label{eq_continuity_traction_force}
    \begin{aligned}
        (\tensorii{P}{}_{\Crown} - \tensorii{P}{}_{\Bulk} \vert_{\dBulk{}}) \cdot \tensori{n}{} =  0
        &&
        \text{in}
        &&
        \Crown{}
    \end{aligned}
\end{equation}

\subsection{Partie 2}

\paragraph{Hu–Washizu avec hypothèses}

Taking into accounts the assumptions made Section \ref{sec_assumtions}, the total Lagrangian of the system can be re-written (see Section \ref{sec_appendix} for details) in the following simplified form
% 
% 
%
\begin{equation}
    \label{eq_0015}
    \begin{aligned}
        J_{\cell}^{HW}
        = &
        \int_{\cell{}} \mecPotential{}_{\bodyLag{}} + (\nabla \tensori{u}{}_{\cell{}} - \tensorii{G}{}_{\cell{}}) : \tensorii{P}{}_{\cell}
        % \\
        % &
        + \int_{\dCell{}} (\tensori{u}{}_{\dCell} - \tensori{u}{}_{\cell} \vert_{\dCell}) \cdot \tensorii{P}{}_{\cell} \vert_{\dCell{}} \cdot \tensori{n}{}
        % \\
        % &
        + \int_{\dCell} \frac{\beta}{2 h_{\cell}} \lVert \tensori{u}{}_{\dCell{}} - \tensori{u}{}_{\cell{}} \vert_{\dCell{}} \rVert^2
        \\
        &
        -
        \int_{\cell} \loadLag{} \cdot \tensori{u}{}_{\cell{}}
        -
        \int_{\neumannCell{}} \neumannCellLoad{} \cdot \tensori{u}{}_{\dCell{}}
    \end{aligned}
\end{equation}
%
%
%
where we have made the width of the interface $\ell \rightarrow 0$, such that the core part $\Bulk{}$ now identifies as $\cell$.

\paragraph{hybridization of the primal unknown}

The displacement is discontinuous across $\dCell{}$ by considering the vanishing interface, that allows for the core part $\cell{}$ to move away from the boundary $\dCell$, thus introducing a possible displacement jump on $\dCell{}$.
This assumption relates to the concept of hybridization of the displacement unknown, which is at the foundation of Hybrid Discontinuous Galerkin methods.
The displacement of the element $\cell$ hence depends on the pair $(\tensori{u}_{\cell}, \tensori{u}_{\dCell})$, where the trace of the core unknown $\tensori{u}_{\cell} \vert_{\dCell{}}$ coexists with $\tensori{u}_{\dCell}$ on $\dCell{}$.

\paragraph{Discontinuous Galerkin}

By replacing $\tensori{u}{}_{\dCell}$ by $\tensori{u}{}_{\cell'} \vert_{\dCell}$ for any neighbouring cell $\cell'$ to $\cell$ amounts to describe the framework for Discontinuous Galerkin methods, where only the core unknown $\tensori{u}{}_{\cell}$ is considered, and the displacement jump on $\dCell$ depends on the trace of neighbouring cells  displacement, instead of that only defined on the boundary.

\paragraph{Conformal Galerkin formulation}

By enforcing continuity of the displacement across $\dCell{}$ such that $\tensori{u}_{\cell} \vert_{\dCell} = \tensori{u}_{\dCell}$, one recovers the usual expression over 

The displacement is discontinuous across $\dCell{}$ by considering the vanishing interface, that allows for the core part $\cell{}$ to move away from the boundary $\dCell$, thus introducing a possible displacement jump on $\dCell{}$.
This assumption relates to the concept of hybridization of the displacement unknown, which is at the foundation of Hybrid Discontinuous Galerkin methods.
The displacement of the element $\cell$ hence depends on the pair $(\tensori{u}_{\cell}, \tensori{u}_{\dCell})$, where the trace of the core unknown $\tensori{u}_{\cell} \vert_{\dCell{}}$ coexists with $\tensori{u}_{\dCell}$ on $\dCell{}$.

\paragraph{Derivative}

La fonctionelle \eqref{eq_0015} définit le problème mixte sous forme faible, et revient à résoudre les problèmes couplés suivants

\begin{subequations}
    \label{eq_0017}
        \begin{alignat}{3}
            \frac{\partial J_{\cell}^{HW}}{\partial \tensori{u}{}_{\cell}} \delta \tensori{u}{}_{\cell}
            = & \int_{\cell} \tensorii{P}{}_{\cell} : \nabla \delta \tensori{u}{}_{\cell}
            -
            \int_{\cell} \tensori{f}{}_V \cdot \delta \tensori{u}{}_{\cell}
            -
            \int_{\dCell{}} \tensori{\theta}{}_{\dCell} \cdot \delta \tensori{u}{}_{\cell} \vert_{\dCell}
            &&
            \ \ \ \ \ \ \ \ 
            &&
            \forall \delta \tensori{u}{}_{\cell}
            \in \virtualDisplacementSpaceCell
        \label{eq_0017:eq0}
        \\
            \frac{\partial J_{\cell}^{HW}}{\partial \tensori{u}{}_{\dCell}} \delta \tensori{u}{}_{\dCell}
            = &
            \int_{\neumannCell} (\tensori{\theta}{}_{\dCell} - \tensori{t}{}_{\neumannCell}) \cdot \delta \tensori{u}{}_{\dCell}
            &&
            \ \ \ \ \ \ \ \ 
            &&
            \forall \delta \tensori{u}{}_{\dCell}
            \in \virtualDisplacementSpaceDCell
        \label{eq_0017:eq1}
        \\
            \frac{\partial J_{\cell}^{HW}}{\partial \tensorii{G}{}_{\cell}} \delta \tensorii{G}{}_{\cell}
            = &
            \int_{\cell} (\frac{\partial \mecPotential_{\bodyLag}}{\partial \tensorii{G}{}_{\cell}} - \tensorii{P}{}_{\cell}) : \delta \tensorii{G}{}_{\cell}
            &&
            \ \ \ \ \ \ \ \ 
            &&
            \forall \delta \tensorii{G}{}_{\cell}
            \in \gradSpaceCell
        \label{eq_0017:eq2}
        \\
            \frac{\partial J_{\cell}^{HW}}{\partial \tensorii{P}{}_{\cell}} \delta \tensorii{P}{}_{\cell}
            = & \int_{\cell} (\nabla \tensori{u}{}_{\cell} - \tensorii{G}{}_{\cell} ) : \delta \tensorii{P}{}_{\cell}
            +
            \int_{\dCell} (\tensori{u}{}_{\dCell} - \tensori{u}{}_{\cell} \vert_{\dCell}) \cdot \delta \tensorii{P}{}_{\cell} \vert_{\dCell} \cdot \tensori{n}{}
            &&
            \ \ \ \ \ \ \ \ 
            &&
            \forall \delta \tensorii{P}{}_{\cell}
            \in \stressSpaceCell
        \label{eq_0017:eq3}
    \end{alignat}
\end{subequations}
% 
% 
%
where we introduced the \textit{reconstructed traction force} $\tensori{\theta}{}_{\dCell} = \tensorii{P}{}_{\cell} \vert_{\dCell} \cdot \tensori{n}{} + (\beta / h_{\cell}) (\tensori{u}{}_{\dCell} - \tensori{u}{}_{\cell} \vert_{\dCell})$.
In particular, \eqref{eq_0017:eq0} is the expression of the principle of virtual works in $\cell$, where the \textit{reconstructed traction force} $\tensori{\theta}{}_{\dCell}$ replaces the usual expression $\tensorii{P}{}_{\cell} \cdot \tensori{n}{}$ in the external contribution. \eqref{eq_0017:eq1} denotes a supplementary equation to the usual continuous problem as described in \eqref{eq_hu_washizu_derivative_0}, to account for the continuity of the flux $\tensori{\theta}{}_{\dCell}$ across the cell boundary.
% This feature constitutes one of the key assets of non-conformal method; indeed, by defining a richer flux than in the usual continuous framework, that also depends on the displacement jump, one allows for the latter to act as a Lagrange multiplier in order to fulfill the flux continuity requirement on $\dCell$.
% La continuité du flux aux interfaces is indeed the tradeoff for having loosened la continuité du déplacement aux interfaces.
% Stability of the problem is then recovered through the interface behaviour that penalizes displacement jumps in a weak sense.
% \eqref{eq_0017:eq2} defines the stress-behaviour law relation, and \eqref{eq_0017:eq3} defines a gradient field reconstruction based on a linear problem, whose second term depends on both a body and a boundary term.
\eqref{eq_0017:eq2} accounts for the constitutive equation in a weak sense, and \eqref{eq_0017:eq3} defines the equation of an enhanced gradient field, that does not reduce to the projection of $\nabla \tensori{u}{}_{\cell}$ as in \eqref{eq_hu_washizu_derivative_0:eq3}, since it is enriched by a boundary component that depends on the displacement jump, which is at the origin of the robustness of non-conformal methods to volumetric locking (see Section \ref{sec_appendix}).
%
%
%
%
% Indeed, defining $\tensori{I}{}(\tensori{v}{})$ the interpolation operator

\subsection{Problem in primal form}

\paragraph{Reconstructed gradient}

Since minimization of \eqref{eq_0017:eq3} defines a linear problem with any displacement pair $(\tensori{v}{}_{\cell}, \tensori{v}{}_{\dCell})$, one can eliminate the equation from the system \eqref{eq_0017}, which defines the so-called \textit{reconstructed gradient} $\tensorii{G}{}_{\cell}(\tensori{v}{}_{\cell}, \tensori{v}{}_{\dCell})$ associated with any displacement pair $(\tensori{v}{}_{\cell}, \tensori{v}{}_{\dCell})$ that solves
%
%
%
\begin{equation}
    \label{eq_grad}
    \begin{aligned}
        \int_{\cell} \tensorii{G}{}_{\cell}(\tensori{v}{}_{\cell}, \tensori{v}{}_{\dCell}) : \tensorii{\tau}{}_{\cell}
        =
        \int_{\cell}  \nabla \tensori{v}{}_{\cell} : \tensorii{\tau}{}_{\cell}
        +
        \int_{\dCell} (\tensori{v}{}_{\dCell} - \tensori{v}{}_{\cell} \vert_{\dCell}) \cdot \tensorii{\tau}{}_{\cell} \vert_{\dCell} \cdot \tensori{n}{}
        &&
        \forall \tensorii{\tau}{}_{\cell} \in \stressSpaceCell
    \end{aligned}
\end{equation}

\paragraph{Stress tensor}

Equation \eqref{eq_0017:eq2} is also linear with the derivative of $\mecPotential_{\bodyLag}$ with respect to $\tensorii{G}{}_{\cell}$, and one can eliminate it as well from \eqref{eq_0017}, hence defining the stress tensor 
%
%
%
\begin{equation}
    \label{eq_stress}
    \begin{aligned}
        \int_{\cell} \tensorii{P}{}_{\cell} : \tensorii{\gamma}{}_{\cell}
        =
        \int_{\cell} \frac{\partial \mecPotential_{\bodyLag}}{\partial \tensorii{G}{}_{\cell}}  : \tensorii{\gamma}{}_{\cell}
        &&
        \forall \tensorii{\gamma}{}_{\cell} \in \gradSpaceCell
    \end{aligned}
\end{equation}
%
%
%
In particular, one notices that \eqref{eq_stress} holds in a strong sense if $\stressSpaceCell \subset \gradSpaceCell$.

\paragraph{Simplified form}

Now that \eqref{eq_0017:eq2} and \eqref{eq_0017:eq3} have been eliminated from the system, one considers the simplified functional \eqref{eq_simple} instead of \eqref{eq_0015}
%
%
%
\begin{equation}
    \label{eq_simple}
    \begin{aligned}
        J_{\cell}^{VW}
        = &
        \int_{\cell{}} \mecPotential{}_{\bodyLag{}}
        % \\
        % &
        % + \int_{\dCell{}} (\tensori{u}{}_{\dCell} - \tensori{u}{}_{\cell} \vert_{\dCell}) \cdot \tensorii{P}{}_{\cell} \vert_{\dCell{}} \cdot \tensori{n}{}
        % \\
        % &
        + \int_{\dCell} \frac{\beta}{2 h_{\cell}} \lVert \tensori{u}{}_{\dCell{}} - \tensori{u}{}_{\cell{}} \vert_{\dCell{}} \rVert^2
        % \\
        % &
        -
        \int_{\cell} \loadLag{} \cdot \tensori{u}{}_{\cell{}}
        -
        \int_{\neumannCell{}} \neumannCellLoad{} \cdot \tensori{u}{}_{\dCell{}}
    \end{aligned}
\end{equation}

\paragraph{Principle of virtual works}
%
%
%
The problem in primal form amounts to find the displacement pair $(\tensori{u}{}_{\cell}, \tensori{u}{}_{\dCell}) \in \hybridDisplacementSpaceCell$ verifying $\tensori{u}{}_{\dCell} = \dirichletLag$ on $\dirichletCell$,
such that for all kinematically admissible displacements pairs $(\delta \tensori{u}{}_{T}, \delta \tensori{u}{}_{\partial T}) \in \virtualHybridDisplacementSpaceCell$, the functional \eqref{eq_simple} is minimal, \textit{i.e.} such that
%
%
%
\begin{equation}
    \label{eq_0018}
    \begin{aligned}
        % d J_{\cell}^{\text{HW}}
        % = &
        % \frac{\partial J_{\cell}}{\partial \tensori{u}{}_{\cell}} \delta \tensori{u}{}_{\cell}
        % +
        % \frac{\partial J_{\cell}}{\partial \tensori{u}{}_{\dCell}} \delta \tensori{u}{}_{\dCell}
        % =
        \delta J_{\cell, \text{int}}^{VW} - \delta J_{\cell, \text{ext}}^{VW}
        =
        0
        % \\
        % = & \delta J_{\cell}^{\text{int}} + \delta J_{\cell}^{\text{ext}}
        % \\
        % = & 
        % \int_{T}
        % \tensorii{P}{}_{\cell}(\tensorii{G}{}_{\cell}(\tensori{u}{}_{\cell}, \tensori{u}{}_{\dCell}))
        % :
        % \tensorii{G}{}_{\cell}(\delta \tensori{u}{}_{\cell}, \delta \tensori{u}{}_{\dCell})
        % % \frac{\partial \mecPotential_{\bodyLag}}{\partial \tensorii{G}{}_T} : \delta \tensorii{G}{}_{T}
        % +
        % \int_{\partial T} (\beta / h_T)
        % (\tensori{u}{}_{\partial T} - \tensori{u}{}_{T} \vert_{\partial T})
        % % \tensori{Z}{}_{\dCell{}}
        % \cdot
        % (\delta \tensori{u}{}_{\partial T} - \delta \tensori{u}{}_{T} \vert_{\partial T})
        % % \delta \tensori{Z}{}_{\dCell{}}
        % \\
        % &
        % -
        % \int_{\partial T} \tensori{t}{}_N \cdot \delta \tensori{u}{}_{\partial T}
        % -
        % \int_{T} \tensori{f}{}_V \cdot \delta \tensori{u}{}_{T}
        % =
        % 0
    \end{aligned}
\end{equation}
%
%
%
with
%
%
%
\begin{subequations}
    \label{eq_0nonamemee}
        \begin{alignat}{3}
            \delta J_{\cell, \text{int}}^{VW} & = 
            \int_{T}
            \tensorii{P}{}_{\cell}(\tensorii{G}{}_{\cell}(\tensori{u}{}_{\cell}, \tensori{u}{}_{\dCell}))
            :
            \tensorii{G}{}_{\cell}(\delta \tensori{u}{}_{\cell}, \delta \tensori{u}{}_{\dCell})
            % \frac{\partial \mecPotential_{\bodyLag}}{\partial \tensorii{G}{}_T} : \delta \tensorii{G}{}_{T}
            +
            \int_{\dCell} (\beta / h_{\cell})
            % (\tensori{u}{}_{\dCell} - \tensori{u}{}_{\cell} \vert_{\dCell})
            % \tensori{Z}{}_{\dCell{}}
            \tensori{Z}{}_{\dCell}(\tensori{u}{}_{\cell}, \tensori{u}{}_{\dCell})
            \cdot
            % (\delta \tensori{u}{}_{\dCell} - \delta \tensori{u}{}_{\cell} \vert_{\dCell{}})
            % \delta \tensori{Z}{}_{\dCell{}}
            \tensori{Z}{}_{\dCell}(\delta \tensori{u}{}_{\cell}, \delta \tensori{u}{}_{\dCell})
            \\
            \delta J_{\cell, \text{ext}}^{VW} & = 
            \int_{\neumannCell} \neumannCellLoad{} \cdot \delta \tensori{u}{}_{\dCell{}}
            +
            \int_{T} \loadLag \cdot \delta \tensori{u}{}_{\cell}
    \end{alignat}
\end{subequations}
%
%
%
where we introduced the jump function $\tensori{Z}{}_{\dCell}$ :
%
%
%
\begin{equation}
    \begin{aligned}
        \tensori{Z}{}_{\dCell}(\tensori{v}{}_{\cell}, \tensori{v}{}_{\dCell}) = \tensori{v}{}_{\dCell} - \tensori{v}{}_{\cell} \vert_{\dCell}
        &&
        \forall (\tensori{v}{}_{\cell}, \tensori{v}{}_{\dCell}) \in \hybridDisplacementSpaceCell
    \end{aligned}
\end{equation}
%
%
%
In particular, one can readliy see the resemblance of \eqref{eq_0nonamemee} with
\eqref{eq_virtual_works_0},
% the ususal formulation of the principle of virtual works
where the so called \textit{reconstructed gradient} $\tensorii{G}{}_{\cell}(\tensori{u}{}_{\cell}, \tensori{u}{}_{\dCell})$ plays the role of the usual displacement Lagrangian gradient $\nabla \tensori{u}{}_{\cell}$, and where an additional \textit{stabilization term} corresponding to a traction energy on the boundary has been added to account for the penalization of the displacement jump on $\dCell$ through $\tensori{Z}{}_{\dCell}$ (or, equivalently, to account for the infinitésimale interface that lays between the bulk domain and its boundary).
Equations \eqref{eq_simple}, \eqref{eq_grad} and \eqref{eq_stress} define the mechanical problem to solve at the cell level for Hybrid Discontinuous Galerkin methods, and \eqref{eq_0018} describes the weak form of these equations.
\section{Discretization}

\paragraph{Mesh}

One defines the collection of all cells in the mesh as
$\HybridMesh(\bodyLag) = \{ \matI \subset \bodyLag \ \vert \ 1 \leq i \leq N_{\cell} \}$, where $N_T$ denotes the total number of cells.

\paragraph{Face}

The boundary $\dCell{}$ of each element is decomposed in faces, such thata face $F$ is a subset of $\bodyLag$, and either there are two cells $\cell$ and $\cell'$ such that $F = \dCell \cap \dCell'$ ($F$ is then an interior face), or there is a single cell $\cell$ such that $F = \dCell \cap \partial \Omega$ ($F$ is then an exterior face).

\paragraph{Faces sets}

% Let $\dHybridMesh{}^i(\bodyLag)$ denote the set of interior faces, and $\dHybridMesh{}^e(\bodyLag)$ that of exterior ones.
% $\dHybridMesh{}^e(\bodyLag)$ is partitioned into $\dHybridMesh{}_{D}^e(\bodyLag) = \{ F \in \dHybridMesh{}^e(\bodyLag) \ \vert \ F \subset \partial_D \bodyLag \}$ the set of exterior faces imposed to prescribed Dirichlet boundary conditions, and into $\dHybridMesh{}_{N}^e(\bodyLag) = \{ F \in \dHybridMesh{}^e(\bodyLag) \ \vert \ F \subset \partial_N \bodyLag \}$ the set of exterior faces imposed to prescribed Neumann boundary conditions.
For any element $\cell \in \mathcal{T}$, let $\mathcal{F}(\cell) = \{ F \in \dHybridMesh \ \vert \ F \subset \dCell \}$ the set of faces composing the boundary of $\cell$.
Let finally $\dHybridMesh(\bodyLag) = \{ F_i \subset \bodyLag \ \vert \ 1 \leq i \leq N_{F} \}$ the skeleton of the mesh, collecting all element faces $F_i$ in the mesh, where $N_{F}$ denotes the number of faces.

\paragraph{Hybird mesh}

The composition of both $\mathcal{T}(\bodyLag)$ and $\dHybridMesh{}(\bodyLag)$ forms the hybrid mesh
% $\overline{\mathcal{T}}({\bodyLag}) = \{T \subset \bodyLag, F \subset \bodyLag \ \vert \ T \in \mathcal{T}(\bodyLag) \ \vert \ F \in \mathcal{F}(\bodyLag) \}$.
$\HybridMeshWhole({\bodyLag}) = \{ \mathcal{T}(\bodyLag), \mathcal{F}(\bodyLag) \}$.
%
%
%

\paragraph{Global unknown}

% Contrary to the standard finite element method that consists in seeking a global solution in a regular enough space over the whole mesh, we consider here a much broader space that consists in $\displacementSpaceHybridMesh = \prod_{\cell \in \HybridMesh(\bodyLag)} \displacementSpaceCell$ the collection of all regular enough displacements element-wise.
% Similarly, we consider $\displacementSpaceDHybridMesh = \prod_{F \in \dHybridMesh(\bodyLag)} V(F)$ the space of all regular enough displacements face-wise for the skeleton unknown, such that the global solution space $\hybridDisplacementSpaceHybridMesh = \displacementSpaceHybridMesh \times \displacementSpaceDHybridMesh$ for the whole problem is simply the assembly of all element and face spaces in the mesh.

Let the global unknown $(\tensori{v}{}_{\HybridMesh}, \tensori{v}{}_{\dHybridMesh})$ a displacement pair such that for each $\cell \in \HybridMesh(\bodyLag), \tensori{v}{}_{\HybridMesh} = \tensori{v}{}_{\cell}$ in $\cell$ and for each $F \in \dHybridMesh(\bodyLag), \tensori{v}{}_{\dHybridMesh} = \tensori{v}{}_{F}$ on $F$, where $\tensori{v}{}_{\cell} \in \displacementSpaceCell$ and $\tensori{v}{}_{F} \in \displacementSpaceDCell$ denote a cell and a face displacement field respectively.

% \newcommand\virtualDisplacementSpaceHybridMesh{U_0(\mathcal{T})}
% \newcommand\virtualDisplacementSpaceDHybridMesh{V_0(\mathcal{F})}
% \newcommand\virtualHybridDisplacementSpaceHybridMesh{U_0(\bar{\mathcal{T}})}
% The global unknown is then sought in the space $\displacementSpaceHybridMesh \times \displacementSpaceDHybridMesh$
% %
% %
% %
% \begin{equation}
%     \label{eq_space_def}
%     \begin{aligned}
%         \displacementSpaceHybridMesh = \prod_{\cell \in \HybridMesh(\bodyLag)} \displacementSpaceCell
%         &&
%         \text{and}
%         &&
%         \displacementSpaceDHybridMesh = \prod_{F \in \dHybridMesh(\bodyLag)} V(F)
%     \end{aligned}
% \end{equation}


\paragraph{Global weak form}


% , where both cells and faces are considered as part of the support for defining
The weak form of the global mechanical problem of $\bodyLag$ reads : find the global displacement unknown pair $(\tensori{u}{}_{\HybridMesh}, \tensori{u}{}_{\dHybridMesh})$ verifying $\tensori{u}{}_{\dHybridMesh} \vert_{\dirichletBoundaryLag} = \dirichletLag$ on $\dirichletBoundaryLag$ such that
$\forall (\delta \tensori{u}{}_{\HybridMesh}, \delta \tensori{u}{}_{\dHybridMesh}) \in \virtualHybridDisplacementSpaceHybridMesh$
%
%
%
\begin{equation}
    \label{eq_0018kdk}
    \begin{aligned}
        \delta J_{\HybridMesh, \text{int}}^{VW} - \delta J_{\HybridMesh, \text{ext}}^{HW}
        =
        0
    \end{aligned}
\end{equation}
%
%
%
with
%
%
%
\begin{subequations}
    \label{eq_0nonamemeerg}
        \begin{alignat}{3}
            \delta J_{\HybridMesh, \text{int}}^{VW} & = 
            \sum_{\cell \in \HybridMesh(\bodyLag)}
            \int_{\cell}
            \tensorii{P}{}_{\cell}(\tensorii{G}{}_{\cell}(\tensori{u}{}_{\cell}, \tensori{u}{}_{\dCell}))
            :
            \tensorii{G}{}_{\cell}(\delta \tensori{u}{}_{\cell}, \delta \tensori{u}{}_{\dCell})
            % \frac{\partial \mecPotential_{\bodyLag}}{\partial \tensorii{G}{}_\cell} : \delta \tensorii{G}{}_{\cell}
            +
            \int_{\dCell} (\beta / h_{\cell})
            % (\tensori{u}{}_{\dCell} - \tensori{u}{}_{\cell} \vert_{\dCell})
            % \tensori{Z}{}_{\dCell{}}
            \tensori{Z}{}_{\dCell}(\tensori{u}{}_{\cell}, \tensori{u}{}_{\dCell})
            \cdot
            % (\delta \tensori{u}{}_{\dCell} - \delta \tensori{u}{}_{\cell} \vert_{\dCell{}})
            % \delta \tensori{Z}{}_{\dCell{}}
            \tensori{Z}{}_{\dCell}(\delta \tensori{u}{}_{\cell}, \delta \tensori{u}{}_{\dCell})
            \\
            \delta J_{\HybridMesh, \text{ext}}^{HW} & = 
            \sum_{F \in \dHybridMesh{}_{N}^e(\bodyLag)}
            \int_{F} \neumannLag \cdot \delta \tensori{u}{}_{F}
            +
            \sum_{\cell \in \HybridMesh(\bodyLag)}
            \int_{\cell} \loadLag \cdot \delta \tensori{u}{}_{\cell}
    \end{alignat}
\end{subequations}
%
%
%
% where $\displacementSpaceHybridMesh$ (respectively $\displacementSpaceDHybridMesh$) denotes the space of all globally kinematically admissible cell (respectively face) displacement fields,
% and
where for each element $\cell \in \HybridMesh(\bodyLag)$, the boundary displacement field $\tensori{v}{}_{\dCell}$ is such that $\tensori{v}{}_{\dCell} = \tensori{v}{}_{F}$ on $F$ for every $F \in \mathcal{F}(\cell)$

% Contrary to the standard finite element method that consists in seeking a global solution in a regular enough space over the whole mesh, we consider here a much broader space that consists in the collection of all regular enough displacements element-wise, such that displacement jumps are actually possible across elements.
% Similarily, we consider the space of all regular enough displacements face-wise for the skeleton unknown, such that the global unknown space for the whole problem is simply the assembly of all element and face spaces in the mesh :
% %
% %
% %
% \begin{equation}
%     \label{eq_space_def}
%     \begin{aligned}
%         \displacementSpaceHybridMesh = \prod_{\cell \in \HybridMesh(\bodyLag)} \displacementSpaceCell
%         &&
%         \text{and}
%         &&
%         \displacementSpaceDHybridMesh = \prod_{F \in \dHybridMesh(\bodyLag)} V(F)
%     \end{aligned}
% \end{equation}
% %
% %
% %
% Similarily, for each element $\cell \in \mathcal{T}$, the displacement space of its boundary $\displacementSpaceDCell$ is the collection of the face displacement spaces to which it is connected such that $\displacementSpaceDCell = \prod_{F \in \mathcal{F}(\cell)} V(F)$.

\paragraph{Discrete functional space}

A polynomial approximation of the global solution is then sought in a subspace of $\displacementSpaceHybridMesh$, and for each element $\cell \in \HybridMesh(\bodyLag)$, we denote $\discreteDisplacementSpaceCell \subset \displacementSpaceCell$ the approximation displacement space in the cell, and $\discreteDisplacementSpaceDCell \subset \displacementSpaceDCell$ that on the boundary. Similarly, let $\discreteGradSpaceCell \subset \gradSpaceCell$ the space used to build the discrete reconstructed gradient and $\discreteStressSpaceCell \subset \stressSpaceCell$ that chosen for the discrete stress such that
%
%
%
\begin{equation*}
    \begin{aligned}
        \discreteDisplacementSpaceCell & = P^l(\cell, \mathbb{R}^{d})
        \\
        \discreteDisplacementSpaceDCell & = P^k(\dCell, \mathbb{R}^{d})
        \\
        \discreteGradSpaceCell & = P^k(\cell, \mathbb{R}^{d \times d})
        \\
        \discreteStressSpaceCell & = P^k(\cell, \mathbb{R}^{d \times d})
    \end{aligned}
\end{equation*}
%
%
%
where the cell displacement polynomial order $l$ might be chosen different from the face displacement order $k$ such that $k - 1 \leq l \leq k + 1$.
% The jump function $\tensori{Z}{}_{\dCell}$ can be tweaked in order to act on the convergence order of the approximated solution
% LES STABILIZATIONS SONT A LORIOGINE DE LORDRE DE CONV
% Since two polynomial orders are available to define the displacement approximation,
Accounting for the possible different polynomial order between the cell and faces, one can specify a discrete jump function in a natural way such that it delivers the displacement difference point-wise for any displacement pair $(\tensori{v}{}_{\cell}^l, \tensori{v}{}_{\dCell}^k) \in \discreteHybridDisplacementSpaceCell$
%
%
%
\begin{equation}
    \begin{aligned}
        \tensori{Z}{}_{\dCell}^{HDG}(\tensori{v}{}_{\cell}^l, \tensori{v}{}_{\dCell}^k) = \Pi^k_{\dCell{}} (
            \tensori{v}{}_{\dCell}^k - \tensori{v}{}_{\cell}^l \vert_{\dCell}
        )
    \end{aligned}
\end{equation}
%
%
%
where $\discreteHybridDisplacementSpaceCell = \discreteDisplacementSpaceCell \times \discreteDisplacementSpaceDCell$ and $\Pi^k_{\dCell{}}$ denotes the orthogonal projector onto $\discreteDisplacementSpaceDCell$.
This straightforward discrete jump function is at the origin of Hybrid Discontinuous Galerkin methods, and grants a convergence of order $k$ in the energy norm.
A richer discrete jump function $\tensori{Z}{}_{\dCell}^{HHO}$ providing a convergence of order $k + 1$ in the energy norm was introduced in \cite{di_pietro_discontinuous-skeletal_2015}, hence giving the Hybrid High Order method its name, such that
%
%
%
\begin{equation}
    \label{eq_hho_stabilization_vector}
    \begin{aligned}
        \tensori{Z}{}_{\dCell}^{HHO}(\tensori{v}{}_{\cell}^l, \tensori{v}{}_{\dCell}^k) = \Pi^k_{\dCell{}} (
            \tensori{v}{}_{\dCell}^k - \tensori{v}{}_{\cell}^l \vert_{\dCell}
            -
            (
                (\tensoro{I}{}_{\cell}^{k + 1} - \Pi_{\cell}^k) (
                    \tensori{w}{}_\cell^{k + 1}
                )
            ) \vert_{\dCell{}}
        )
    \end{aligned}
\end{equation}
%
%
%
where $\Pi_{\cell}^k$ is the projector onto $P^{k}(\cell, \mathbb{R}^{d})$, $\tensoro{I}{}_{\cell}^{k + 1}$ is the identity function in $\discretePotentialSpaceCell = P^{k + 1}(\cell, \mathbb{R}^{d})$, and $\tensori{w}{}_{\cell}^{k+1} \in \discretePotentialSpaceCell$ denotes a higher order discrete displacement solving the following linear problem for any displacement pair $(\tensori{v}{}_{\cell}^l, \tensori{v}{}_{\dCell}^k) \in \discreteHybridDisplacementSpaceCell$
%
%
%
\begin{subequations}
    \label{eq_potential}
        \begin{alignat}{3}
            \int_\cell \nabla \tensori{w}{}_{\cell}^{k+1} : \nabla \tensori{d}{}_{\cell}^{k+1}
            & =
            \int_\cell \nabla \tensori{v}{}_{\cell}^l : \nabla \tensori{d}{}_{\cell}^{k+1}
            +
            \int_{\dCell} (\tensori{v}{}_{\dCell}^k - \tensori{v}{}_{\cell}^l) \cdot \nabla \tensori{d}{}_{\cell}^{k+1} \cdot \tensori{n}{}
            \ \ \ \ \ \ \ \ 
            &&
            \forall \tensori{d}{}_{\cell}^{k+1} \in \discretePotentialSpaceCell
            \label{eq_potential:eq0}
            \\
            \int_\cell \tensori{w}{}_{\cell}^{k+1} & = \int_\cell \tensori{v}{}_{\cell}^{l}
            \label{eq_potential:eq1}
    \end{alignat}
\end{subequations}
%
%
%
%

% With obvious notations,
Let 
$\discreteDisplacementSpaceHybridMesh = \prod_{\cell \in \HybridMesh(\bodyLag)} \discreteDisplacementSpaceCell$ the global discrete cell displacement space,
$\discreteDisplacementSpaceDHybridMesh = \prod_{F \in \dHybridMesh(\bodyLag)} V^h(F)$ the global discrete face displacement space, and
$\discreteHybridDisplacementSpaceHybridMesh = \discreteDisplacementSpaceHybridMesh \times \discreteDisplacementSpaceDHybridMesh$ the global unknown approximation space.
Let $\discreteVirtualDisplacementSpaceHybridMesh$ and $\discreteVirtualDisplacementSpaceDHybridMesh$ the respective discrete mesh and skeleton virtual displacement spaces, and
$\discreteVirtualHybridDisplacementSpaceHybridMesh = \discreteVirtualDisplacementSpaceHybridMesh \times \discreteVirtualDisplacementSpaceDHybridMesh$ the discrete virtual global displacement space.
% Let $(\tensori{v}{}_{\HybridMesh}^l, \tensori{v}{}_{\dHybridMesh}^k) \in \discreteHybridDisplacementSpaceHybridMesh$ denote a global displacement pair such that $\forall \cell \in \HybridMesh(\bodyLag), \tensori{v}{}_{\HybridMesh}^l = \tensori{v}{}_{\cell}^l$ in $\cell$ and $\forall F \in \HybridMesh(\bodyLag), \tensori{v}{}_{\dHybridMesh}^k = \tensori{v}{}_{F}^k$ in $F$.

The global mechanical problem in discrete weak form for the Hybrid High Order method finally writes : find the global displacement unknown pair $(\tensori{u}{}_{\HybridMesh}^l, \tensori{u}{}_{\dHybridMesh}^k) \in \discreteHybridDisplacementSpaceHybridMesh$ verifying $\tensori{u}{}_{\dHybridMesh}^k \vert_{\dirichletBoundaryLag} = \dirichletLag$ on $\dirichletBoundaryLag$ such that $\forall (\delta \tensori{u}{}_{\HybridMesh}^l, \delta \tensori{u}{}_{\dHybridMesh}^k) \in \discreteVirtualHybridDisplacementSpaceHybridMesh$
%
%
%
\begin{equation}
    \label{eq_0018kdk}
    \begin{aligned}
        \delta J_{\HybridMesh, \text{int}}^{HHO} - \delta J_{\HybridMesh, \text{ext}}^{HHO}
        =
        0
    \end{aligned}
\end{equation}
%
%
%
with
%
%
%
\begin{subequations}
    \label{eq_0nonamemeergjj}
        \begin{alignat}{3}
            \delta J_{\HybridMesh, \text{int}}^{HHO} & = 
            \sum_{\cell \in \HybridMesh(\bodyLag)}
            \int_{\cell}
            \tensorii{P}{}_{\cell}^k(\tensorii{G}{}_{\cell}^k(\tensori{u}{}_{\cell}^l, \tensori{u}{}_{\dCell}^k))
            :
            \tensorii{G}{}_{\cell}^k(\delta \tensori{u}{}_{\cell}^l, \delta \tensori{u}{}_{\dCell}^k)
            % \frac{\partial \mecPotential_{\bodyLag}}{\partial \tensorii{G}{}_\cell} : \delta \tensorii{G}{}_{\cell}
            +
            \int_{\dCell} (\beta / h_{\cell})
            \tensori{Z}{}_{\dCell}^{HHO}(\tensori{u}{}_{\cell}^l, \tensori{u}{}_{\dCell}^k)
            % \tensori{Z}{}_{\dCell{}}
            \cdot
            \tensori{Z}{}_{\dCell}^{HHO}(\delta \tensori{u}{}_{\cell}^l, \delta \tensori{u}{}_{\dCell}^k)
            % \delta \tensori{Z}{}_{\dCell{}}
            \\
            \delta J_{\HybridMesh, \text{ext}}^{HHO} & = 
            \sum_{F \in \dHybridMesh{}_{N}^e(\bodyLag)}
            \int_{F} \neumannLag \cdot \delta \tensori{u}{}_{F}^k
            +
            \sum_{\cell \in \HybridMesh(\bodyLag)}
            \int_{\cell} \loadLag \cdot \delta \tensori{u}{}_{\cell}^l
    \end{alignat}
\end{subequations}
%
%
%
with the discrete reconstructed gradient $\tensorii{G}{}_{\cell}^k(\tensori{v}{}_{\cell}^l, \tensori{v}{}_{\dCell}^k) \in \discreteGradSpaceCell$ solving $\forall (\tensori{v}{}_{\cell}^l, \tensori{v}{}_{\dCell}^k) \in \discreteHybridDisplacementSpaceCell$
%
%
%
\begin{equation}
    \label{eq_discrete_grad}
    \begin{aligned}
        \int_{\cell} \tensorii{G}{}_{\cell}^k(\tensori{v}{}_{\cell}^l, \tensori{v}{}_{\dCell}^k) : \tensorii{\tau}{}_{\cell}^k
        =
        \int_{\cell}  \nabla \tensori{v}{}_{\cell}^l : \tensorii{\tau}{}_{\cell}^k
        +
        \int_{\dCell} (\tensori{v}{}_{\dCell}^k - \tensori{v}{}_{\cell}^l \vert_{\dCell}) \cdot \tensorii{\tau}{}_{\cell}^k \vert_{\dCell} \cdot \tensori{n}{}
        &&
        \forall \tensorii{\tau}{}_{\cell}^k \in \discreteStressSpaceCell
    \end{aligned}
\end{equation}
\section{Implementation}
\label{sec_implementation}

In this section, we specify the underlying matricial implementation of problem \eqref{eq_0nonamemeergjj}. In the following, the expression $\{ \cdot \}$ denotes a real-valued vector, and the notation $[\cdot]$ a real-valued matrix.

\paragraph{Quadrature}

As is customary with finite element methods, integrals are evaluated numerically by means of a quadrature rule on an element shape. Hence, let $\cellQuadrature$ a quadrature rule for the cell $\cell$ of order at least $2k$. A quadrature point is denoted $\tensori{X}{}_q$ and a quadrature weight $w_q$.

\subsection{Reconstructed gradient and stabilization operators}

\paragraph{Reconstructed gradient operator}

From an algebraic standpoint, \eqref{eq_discrete_grad} defines a linear problem
% with respect to the pair $(\tensori{u}{}_{\cell}^l, \tensori{u}{}_{\cell}^k)$
consisting in inverting a mass matrix in $\discreteGradSpaceCell{}$. One can thus defines 
% $\begin{bmatrix}
%     B_{\cell} && B_{\dCell}
% \end{bmatrix}
% (\tensori{x}{}_q)
% $
$
[B_{\cell}]
$
the discrete gradient operator acting on the pair $(\tensori{v}{}_{\cell}^l, \tensori{v}{}_{\cell}^k)$ at a quadrature point $\tensori{x}{}_q \in \cellQuadrature$ to evaluate the discrete gradient $\tensorii{G}{}_{\cell}^k(\tensori{v}{}_{\cell}^l, \tensori{v}{}_{\dCell}^k)$ such that
% Hence, the algebraic realization of \eqref{eq_discrete_grad}
%
%
%
\begin{equation}
    \label{eq_discrete_gradient_vector}
    \begin{aligned}
        \begin{Bmatrix}
            \tensorii{G}{}_{\cell}^k(\tensori{v}{}_{\cell}^l, \tensori{v}{}_{\dCell}^k)
        \end{Bmatrix}
        (\tensori{X}{}_q)
        =
        \begin{bmatrix}
            B_{\cell}
            % &&
            % B_{\dCell}
        \end{bmatrix}
        (\tensori{x}{}_q)
        \cdot
        \begin{Bmatrix}
            \tensori{v}{}_{\cell}^l
            \\
            \tensori{v}{}_{\dCell}^k
        \end{Bmatrix}
        &&
        \forall (\tensori{v}{}_{\cell}^l, \tensori{v}{}_{\dCell}^k) \in \discreteHybridDisplacementSpaceCell{}
    \end{aligned}
\end{equation}
%
%
%
% where we have decomposed the expression of $[B_{\cell}]$ into a cell block $B_{\cell}$ and a boundary block $B_{\dCell}$, to emphasize the dependence of the problem on both unknowns.
% Once this offline computation step is performed, the 
where $[B_{\cell}]$ is composed by a cell block $B_{\cell}$ and a boundary block $B_{\dCell}$.
% to emphasize the dependence of the problem on both unknowns.


\paragraph{Stabilization operator}

Similarly, the algebraic realization of \eqref{eq_hho_stabilization_vector} amounts to compute the stabilization operator $[Z_{\cell}]$ such that 
%
%
%
\begin{equation}
    \label{eq_discrete_stabilization_vector}
    \begin{aligned}
        \begin{Bmatrix}
            \tensori{Z}{}_{\dCell}^{HHO}(\tensori{v}{}_{\cell}^l, \tensori{v}{}_{\dCell}^k)
        \end{Bmatrix}
        =
        \begin{bmatrix}
            Z_{\cell}
            % &&
            % Z_{\dCell}
        \end{bmatrix}
        \cdot
        \begin{Bmatrix}
            \tensori{v}{}_{\cell}^l
            \\
            \tensori{v}{}_{\dCell}^k
        \end{Bmatrix}
        &&
        \forall (\tensori{v}{}_{\cell}^l, \tensori{v}{}_{\dCell}^k) \in \discreteHybridDisplacementSpaceCell{}
    \end{aligned}
\end{equation}
%
%
%
as for $[B_{\cell}]$, the operator $[Z_{\cell}]$ is composed by a cell block $Z_{\cell}$ and a boundary block $Z_{\dCell}$.

\paragraph{Offline computation}

Since \eqref{eq_discrete_grad} and \eqref{eq_hho_stabilization_vector} depend on the geometry of the element $\cell$ only, one can compute the operators $[B_{\cell}]$ and $[Z_{\cell}]$ for each element once and for all in an offline pre-computation step by working in the reference configuration. Once this offline step is performed, the algebraic form of the problem resembles closely to the standard finite element one, where the operator $[B_{\cell}]$ replaces the usual shape function gradient operator, and the stabilization operator 
$[Z_{\cell}]$ is incorporated in the expression of the tangent matrix and in that of internal forces.

%
%
%
% \begin{equation}
%     \label{hho_incremenaljjjdjj}
%     \begin{aligned}
%         % \begin{Bmatrix}
%         %     \tensori{Z}{}_{\dCell}(\tensori{v}{}_{\cell}, \tensori{v}{}_{\dCell})
%         % \end{Bmatrix}
%         \int_{\dCell} (\beta / h_{\cell})
%         \tensori{Z}{}_{\dCell}^{HHO}(\tensori{u}{}_{\cell}^l, \tensori{u}{}_{\dCell}^k)
%         % \tensori{Z}{}_{\dCell{}}
%         \cdot
%         \tensori{Z}{}_{\dCell}^{HHO}(\delta \tensori{u}{}_{\cell}^l, \delta \tensori{u}{}_{\dCell}^k)
%         =
%         \beta
%         \begin{Bmatrix}
%             \tensori{u}{}_{\cell}^l
%             \\
%             \tensori{u}{}_{\dCell}^k
%         \end{Bmatrix}^{\text{trans}}
%         \cdot
%         \begin{bmatrix}
%             Z_{\cell \cell} && Z_{\cell \dCell}
%             \\
%             Z_{\dCell \cell} && Z_{\dCell \dCell}
%         \end{bmatrix}
%         \cdot
%         \begin{Bmatrix}
%             \delta \tensori{u}{}_{\cell}^l
%             \\
%             \delta \tensori{u}{}_{\dCell}^k
%         \end{Bmatrix}
%     \end{aligned}
% \end{equation}

\subsection{Iterative method}

\paragraph{Notations}

In the following, let $(\tensori{u}{}_{\cell}^{l, {m,n}}, \tensori{u}{}_{\dCell}^{k, {m,n}})$ denote the displacement pair value at some pseudo time step $m$ and some iteration $n$. The initial value of the displacement at time step $m = 0$ and iteration $n = 0$ is set to zero, and at a new pseudo time step $m+1$, the displacement at the first iteration $n = 0$ takes the value of the displacement of the last iteration of the previous time step.

\paragraph{Internal forces}

In such a context, the internal forces vector $\{ F_{T}^{int} (\tensori{u}{}_{\cell}^{l,m,n}, \tensori{u}{}_{\dCell}^{k,m,n}) \}$ writes
%
%
%
\begin{equation}
    \label{hho_incremenaljjjdjj}
    \begin{aligned}
        \begin{Bmatrix}
            F_{T}^{int}
            (
                % \delta
                \tensori{u}{}_{\cell}^{l,m,n}
                ,
                % \delta
                \tensori{u}{}_{\dCell}^{k,m,n}
            )
        \end{Bmatrix}
        = &
        \sum_{\tensori{X}{}_q \in \cellQuadrature{}}
        \begin{bmatrix}
            B_{\cell}
            % &&
            % B_{\dCell}
        \end{bmatrix}^{\text{trans}}(\tensori{X}{}_q)
        \cdot
        \begin{Bmatrix}
            \tensorii{P}{}_{\cell}^k(
                \tensorii{G}{}_{\cell}^k
                (
                    % \delta
                    \tensori{u}{}_{\cell}^{l,m,n}
                    ,
                    % \delta
                    \tensori{u}{}_{\dCell}^{k,m,n}
                )
            )
        \end{Bmatrix}(\tensori{X}{}_q)
        % \\
        % &
        +
        \frac{\beta}{h_T}
        \begin{bmatrix}
            Z_{\cell}
            % &&
            % Z_{\dCell}
        \end{bmatrix}^{\text{trans}}
        \cdot
        \begin{bmatrix}
            Z_{\cell}
            % && 
            % Z_{\dCell}
        \end{bmatrix}
        % \begin{bmatrix}
        %     Z_{\cell \cell} && Z_{\cell \dCell}
        %     \\
        %     Z_{\dCell \cell} && Z_{\dCell \dCell}
        % \end{bmatrix}
        \cdot
        \begin{Bmatrix}
            % \delta
            \tensori{u}{}_{\cell}^{l,m,n}
            \\
            % \delta
            \tensori{u}{}_{\dCell}^{k,m,n}
        \end{Bmatrix}
    \end{aligned}
\end{equation}
%
%
%
where the superscript $[\cdot]^{\text{trans}}$ denotes the transpose operation, and
$\{ \tensorii{P}{}_{\cell}^k(\tensorii{G}{}_{\cell}^k (\tensori{u}{}_{\cell}^{l,m,n}, \tensori{u}{}_{\dCell}^{k,m,n})) \}$ is the stress components vector, computed by integration of the behavior law at each quadrature point $\tensori{X}{}_q$ from the values of the displacement gradient $\tensorii{G}{}_{\cell}^k (\tensori{u}{}_{\cell}^{l,m,n}, \tensori{u}{}_{\dCell}^{k,m,n})$.
% The parameter $\beta / h_T$ is the so-called stabilization parameter, that is the stiffness of the interface introduced in \eqref{eq_0009}.

\paragraph{External forces}

The external forces vector is, as is customary with the standard finite element method, the evaluation of the given bulk and boundary loads at respective cell and face quadrature points tested against the respective cell and face shape functions, and is denoted
%
%
%
\begin{equation}
    \label{hho_incremenaljjjdjj}
    \begin{aligned}
        \begin{Bmatrix}
            F_{T}^{ext}
        \end{Bmatrix}
        =
        \begin{Bmatrix}
            \loadLag
            \\
            \neumannLag
        \end{Bmatrix}
    \end{aligned}
\end{equation}
%
%
%

\paragraph{Tangent matrix}

The tangent matrix $[K_{\cell}^{tan}(\tensori{u}{}_{\cell}^{l,m,n}, \tensori{u}{}_{\dCell}^{k,m,n})]$ is the sum of the usual product of the displacement gradients by the tangent operator $\tensoriv{A}{}(\tensori{u}{}_{\cell}^{l,m,n}, \tensori{u}{}_{\dCell}^{k,m,n})$ and of an additional stabilization term such that
%
%
%
\begin{equation}
    \label{hho_incremenaljdjjjkk}
    \begin{aligned}
        \begin{bmatrix}
            K_{\cell}^{tan}(
                % \delta
                \tensori{u}{}_{\cell}^{l,m,n}
                ,
                % \delta
                \tensori{u}{}_{\dCell}^{k,m,n}
            )
        \end{bmatrix}
        = &
        \sum_{\tensori{X}{}_q \in \cellQuadrature{}}
        \begin{bmatrix}
            B_{\cell}
            % &&
            % B_{\dCell}
        \end{bmatrix}^{\text{trans}}
        (\tensori{X}{}_q)
        \cdot
        \begin{bmatrix}
            \tensoriv{A}{}
            (
                % \delta
                \tensori{u}{}_{\cell}^{l,m,n}
                ,
                % \delta
                \tensori{u}{}_{\dCell}^{k,m,n}
            )
        \end{bmatrix}
        (\tensori{X}{}_q)
        \cdot
        \begin{bmatrix}
            B_{\cell}
            % &&
            % B_{\dCell}
        \end{bmatrix}
        (\tensori{X}{}_q)
        % \\
        % &
        +
        \frac{\beta}{h_T}
        \begin{bmatrix}
            Z_{\cell}
            % &&
            % Z_{\dCell}
        \end{bmatrix}^{\text{trans}}
        \cdot
        \begin{bmatrix}
            Z_{\cell}
            % &&
            % Z_{\dCell}
        \end{bmatrix}
    \end{aligned}
\end{equation}
%
%
%
where $\tensoriv{A}{}(\tensori{u}{}_{\cell}^{l,m,n}, \tensori{u}{}_{\dCell}^{k,m,n})$ is the derivative of the stress with respect to the displacement gradient
%
%
%
\begin{equation}
    \label{hho_incremenaljdkkjjjkk}
    \begin{aligned}
        % \begin{bmatrix}
            \tensoriv{A}{}
            (
                % \delta
                \tensori{u}{}_{\cell}^{l,m,n}
                ,
                % \delta
                \tensori{u}{}_{\dCell}^{k,m,n}
            )
        % \end{bmatrix}
        =
        \frac{
            \partial
            \tensorii{P}{}_{\cell}^k
            % (
            %     \tensorii{G}{}_{\cell}^k
            %     (
            %         % \delta
            %         \tensori{u}{}_{\cell}^{l,m,n}
            %         ,
            %         % \delta
            %         \tensori{u}{}_{\dCell}^{k,m,n}
            %     )
            % )
        }
        {
            \partial
            \tensorii{G}{}_{\cell}^k
            % (
            %     % \delta
            %     \tensori{u}{}_{\cell}^{l,m,n}
            %     ,
            %     % \delta
            %     \tensori{u}{}_{\dCell}^{k,m,n}
            % )
        }
    \end{aligned}
\end{equation}

\paragraph{Newton method}
%
%
%
Following the iterative Newton method, the algebraic system to solve at the element level consists in finding the displacement increment $(\delta \tensori{u}{}_{\cell}^{l}, \delta \tensori{u}{}_{\dCell}^{k})$ that solves
%
%
%
\begin{equation}
    \label{eq_element_system}
    \begin{aligned}
        -
        \begin{bmatrix}
            K_{\cell}^{tan}(
                % \delta
                \tensori{u}{}_{\cell}^{l,m,n}
                ,
                % \delta
                \tensori{u}{}_{\dCell}^{k,m,n}
            )
        \end{bmatrix}
        \cdot
        \begin{Bmatrix}
            \delta
            \tensori{u}{}_{\cell}^{l}
            \\
            \delta
            \tensori{u}{}_{\dCell}^{k}
        \end{Bmatrix}
        =
        \begin{Bmatrix}
            R_{\cell}^{m,n}
        \end{Bmatrix}
        &&
        \text{with}
        &&
        \begin{Bmatrix}
            R_{\cell}^{m,n}
        \end{Bmatrix}
        =
        \begin{Bmatrix}
            F_T^{int} (\tensori{u}{}_{\cell}^{l,m,n}, \tensori{u}{}_{\dCell}^{k,m,n})
        \end{Bmatrix}
        -
        \begin{Bmatrix}
            F_T^{ext}
        \end{Bmatrix}
    \end{aligned}
\end{equation}
%
%
%
such that the displacement at the next iteration is incremented by the displacement increment $(\delta \tensori{u}{}_{\cell}^{l}, \delta \tensori{u}{}_{\dCell}^{k})$
%
%
%
\begin{equation}
    \label{eq_adding_increment}
    \begin{Bmatrix}
        \tensori{u}{}_{\cell}^{l,m,n+1}
        \\
        \tensori{u}{}_{\dCell}^{k,m,n+1}
    \end{Bmatrix}
    =
    \begin{Bmatrix}
        \tensori{u}{}_{\cell}^{l,m,n}
        \\
        \tensori{u}{}_{\dCell}^{k,m,n}
    \end{Bmatrix}
    +
    \begin{Bmatrix}
        \delta
        \tensori{u}{}_{\cell}^{l}
        \\
        \delta
        \tensori{u}{}_{\dCell}^{k}
    \end{Bmatrix}
\end{equation}

\subsection{Static condensation}

\paragraph{Static condensation}
%
%
%
Since both $[B_{\cell}]$ and $[Z_{\cell}]$ are expressed in terms of cell and boundary blocks, so does the tangent matrix which can be decomposed into four coupled cell-boundary blocks with the notation
%
%
%
\begin{equation}
    \label{hho_incremenahhljdjjjkk}
    \begin{aligned}
        \begin{bmatrix}
            K_{\cell}^{tan}(
                % \delta
                \tensori{u}{}_{\cell}^{l,m,n}
                ,
                % \delta
                \tensori{u}{}_{\dCell}^{k,m,n}
            )
        \end{bmatrix}
        =
        \begin{bmatrix}
            K_{\cell \cell} (\tensori{u}{}_{\cell}^{l,m,n}, \tensori{u}{}_{\dCell}^{k,m,n})
            &&
            K_{\cell \dCell} (\tensori{u}{}_{\cell}^{l,m,n}, \tensori{u}{}_{\dCell}^{k,m,n})
            \\
            K_{\dCell \cell} (\tensori{u}{}_{\cell}^{l,m,n}, \tensori{u}{}_{\dCell}^{k,m,n})
            &&
            K_{\dCell \dCell} (\tensori{u}{}_{\cell}^{l,m,n}, \tensori{u}{}_{\dCell}^{k,m,n})
        \end{bmatrix}
    \end{aligned}
\end{equation}
%
%
%
Moreover, since $\tensoriv{A}{}(\tensori{u}{}_{\cell}^{l,m,n}, \tensori{u}{}_{\dCell}^{k,m,n})$ is definite symmetric (and positive until an eventual loss of coercivity for \textit{e.g.} high plastic deformations), the cell block $K_{\cell \cell}
% (\tensori{u}{}_{\cell}^{l,m,n}, \tensori{u}{}_{\dCell}^{k,m,n})
$ is invertible and one can condensate it through a Schur complement step in order to eliminate the cell unknown, such that \eqref{eq_element_system} expresses only in terms of boundary increment unknowns
%
%
%
\begin{equation}
    \label{hho_incremenahzhjjjklehhljdjjjkkh}
    \begin{aligned}
        -
        \begin{bmatrix}
            K_{\cell}^{tan}(
                % \delta
                \tensori{u}{}_{\cell}^{l,m,n}
                ,
                % \delta
                \tensori{u}{}_{\dCell}^{k,m,n}
            )
        \end{bmatrix}_{\text{cond}}
        \cdot
        \begin{Bmatrix}
            \delta
            \tensori{u}{}_{\dCell}^{k}
        \end{Bmatrix}
        =
        \begin{Bmatrix}
            F_T^{int} (\tensori{u}{}_{\cell}^{l,m,n}, \tensori{u}{}_{\dCell}^{k,m,n})
        \end{Bmatrix}_{\text{cond}}
        -
        \begin{Bmatrix}
            F_T^{ext}
        \end{Bmatrix}_{\text{cond}}
        =
        \begin{Bmatrix}
            R_{\cell}^{m,n}
        \end{Bmatrix}_{\text{cond}}
    \end{aligned}
\end{equation}
%
%
%
with
%
%
%
\begin{equation}
    \label{hho_incremenahzhjjjklehhljdjjjkk}
    \begin{aligned}
        \begin{bmatrix}
            K_{\cell}^{tan}
            % (\tensori{u}{}_{\cell}^{l,m,n}, \tensori{u}{}_{\dCell}^{k,m,n})
        \end{bmatrix}_{\text{cond}}
        = &
        \begin{bmatrix}
            K_{\dCell \dCell}
            % (\tensori{u}{}_{\cell}^{l,m,n}, \tensori{u}{}_{\dCell}^{k,m,n})
        \end{bmatrix}
        -
        \begin{bmatrix}
            K_{\dCell \cell}
            % (\tensori{u}{}_{\cell}^{l,m,n}, \tensori{u}{}_{\dCell}^{k,m,n})
        \end{bmatrix}
        \cdot
        \begin{bmatrix}
            K_{\cell \cell}
            % (\tensori{u}{}_{\cell}^{l,m,n}, \tensori{u}{}_{\dCell}^{k,m,n})
        \end{bmatrix}^{-1}
        \cdot
        \begin{bmatrix}
            K_{\cell \dCell}
            % (\tensori{u}{}_{\cell}^{l,m,n}, \tensori{u}{}_{\dCell}^{k,m,n})
        \end{bmatrix}
        &&
        \text{and}
        &&
        \begin{Bmatrix}
            R_{\cell}
            % (\tensori{u}{}_{\cell}^{l,m,n}, \tensori{u}{}_{\dCell}^{k,m,n})
        \end{Bmatrix}_{\text{cond}}
        = &
        \begin{Bmatrix}
            R_{\dCell}
            % (\tensori{u}{}_{\cell}^{l,m,n}, \tensori{u}{}_{\dCell}^{k,m,n})
        \end{Bmatrix}
        -
        \begin{bmatrix}
            K_{\dCell \cell}
            % (\tensori{u}{}_{\cell}^{l,m,n}, \tensori{u}{}_{\dCell}^{k,m,n})
        \end{bmatrix}
        \cdot
        \begin{bmatrix}
            K_{\cell \cell}
            % (\tensori{u}{}_{\cell}^{l,m,n}, \tensori{u}{}_{\dCell}^{k,m,n})
        \end{bmatrix}^{-1}
        \cdot
        \begin{Bmatrix}
            R_{\cell}
            % (\tensori{u}{}_{\cell}^{l,m,n}, \tensori{u}{}_{\dCell}^{k,m,n})
        \end{Bmatrix}
    \end{aligned}
\end{equation}
%
%
%
and the incremental cell displacement expresses linearly with the respect to the boundary one such that
%
%
%
\begin{equation}
    \label{hho_incremenahzhjjjklehhljdjjjkkhh}
    \begin{aligned}
        \begin{Bmatrix}
            \delta
            \tensori{u}{}_{\cell}^{l}
        \end{Bmatrix}
        =
        \begin{bmatrix}
            K_{\cell \cell}
            % (\tensori{u}{}_{\cell}^{l,m,n}, \tensori{u}{}_{\dCell}^{k,m,n})
        \end{bmatrix}^{-1}
        (
            -
            \begin{Bmatrix}
                R_{\cell}
                % (\tensori{u}{}_{\cell}^{l,m,n}, \tensori{u}{}_{\dCell}^{k,m,n})
            \end{Bmatrix}
            -
            \begin{bmatrix}
                K_{\cell \dCell}
                % (\tensori{u}{}_{\cell}^{l,m,n}, \tensori{u}{}_{\dCell}^{k,m,n})
            \end{bmatrix}
            \cdot
            \begin{Bmatrix}
                \delta
                \tensori{u}{}_{\dCell}^{k}
            \end{Bmatrix}
        )
    \end{aligned}
\end{equation}
%
%
%

\subsection{Algorithmic aspects}

\paragraph{Linear static condensation algorithm}
\label{par_static_cond}

At a given pseudo-time step $m$ and iteration $n$, the element displacement unknown is incremented by the element displacement increment.
The reconstructed gradient field is then computed, and is used to integrate the behaviour law, which provides the stress and tangent operator values at quadrature points.
The internal forces, external forces and tangent matrix are then computed, and condensated on the element faces. The resulting system is assembled on the global system matrix, and a new value of the increment is computed by inverting the global matrix.
A schematic representation this procedure is given in Figure \ref{res_cond0}
% Le problème global incrémental (\ref{eq_hho2}) est alors l'assemblage des systèmes élémentaires condensés (\ref{hho_incremental_cond}), dont la résolution consiste en un algorithme de Newton sur l'incrément des inconnues de faces uniquement. Ce schéma de résolution par condensation statique dont on donne le principe Figure \ref{res_cond0} exploite la relation linéaire entre l'incrément des inconnues de cellules et celui des faces.
    
\begin{figure}[H]
\centering
\includegraphics[width=14.cm]{img/reso1.png}
\caption{
    schematic description of the Linear static condensation algorithm
}
\label{res_cond0}
\end{figure}

\paragraph{Cell equlibrium algorithm}

We propose an alternative to the static condensation solving algorithm, postulating an implicit relation between the increment of the cell unknowns and the increment of the faces, that consists in solving locally a nonlinear system on the cell increment at a fixed face increment. This non-linear local procedure adds up to the algorithm described above, to ensure the equilibrium of the cell with its faces at each iteration of the global problem. This new solution scheme is described in Figure \ref{res_cond}, where we note $i$ a Newton iteration for solving the global problem on the set of face unknowns, and $j$ a Newton iteration for solving the local problem on the cell unknowns in an element.

% Nous proposons une alternative à l'algorithme de résolution par condensation statique, postulant une relation implicite entre l'incrément des inconnues de cellule est celui des faces et consistant à résoudre localement un système non-linéaire sur l'incrément de cellule à incrément de faces fixé, afin de vérifier l'équilibre de la cellule avec ses faces à chaque itération du problème global. Ce nouveau schéma de résolution est décrit Figure \ref{res_cond}, où on note $i$ une itération de Newton pour la résolution du problème global sur l'ensemble des inconnues de faces, et $j$ une itération de Newton pour la résolution du problème local sur les inconnues de cellule dans un élément $T$:

\begin{figure}[H]
\centering
\includegraphics[width=14.cm]{img/reso0.png}
\caption{
    schematic description of the Linear Cell equlibrium algorithm
}
\label{res_cond}
\end{figure}

\paragraph{Comparison of both algorithms}

The Linear static condensation algorithm is the one described in the literature \cite{di_pietro_discontinuous-skeletal_2015,cockburn_algorithm_2019,abbas_hybrid_2019-1,abbas_hybrid_2018} to condensate cell unknowns on faces. This procedure needs not iterate at the cell level to acomodate the cell increment, and is hence, faster. However, one needs to store matrices used during the condensation step from one iteration to another in order do decondensate the cell increment.

The Cell equlibrium algorithm, needs iterate at the cell level. It is hence is more costly, and requires to integrate the behaviour law more times that does the Linear static condensation algorithm.
However, it allows to consider extending the present non-linear algorithm to \textit{e.g.} constrained resolution algorithm, to solve inequality constrained problems, as encountered in multi-field plasticity for instance.


% We show in Section \ref{sec_numerical_examples} that this algorithm gives identical results to those obtains with the Linear static condensation algorithm.
% In the following section, we devise a Hybrid High order method for an axisymmetric framework. The cartesian space is expressed in cylindrical coordinates and a point $\tensori{x} \in \bodyLag$ has coordinates $\tensori{x} = (r, z, \theta)$ where $r$ denotes the radial component, $z$ the ordonal one, and $\theta$ is the angular componant describing a revolution around the axis $r = 0$. By cylindrical symmetry, the angular displacement $\tensoro{u}{}_{\theta}$ is supposed to be zero, and both components $u_r$ and $u_z$ do not depend on the angular coordinate $\theta$.
Adopting notations introduced in section ABOVE, let $\cell$ an open subset of $\bodyLag \subset \mathbb{R}^2$ in the $(r,z)$ plane with cell displacement $\tensori{u}{}_{\cell} \in \displacementSpaceCell$ and boundary displacement $\tensori{u}{}_{\dCell} \in \displacementSpaceDCell$. The partial derivatives of $\tensori{u}{}_{\cell}$ with respect to the cylindrical coordinates are given by :
%
%
%
\begin{equation}
    \begin{aligned}
        \forall i, j \in \{ r,z \}, \tensoro{u}{}_{\cell i,j} = \frac{\partial u_{\cell i}}{\partial j} && \text{and} && \tensoro{u}{}_{\cell \theta, \theta} = \frac{u_{\cell r}}{r}
    \end{aligned}
\end{equation}
%
%
%
In order to express a Hybrid High Order method in such a framework and owing to the assumtptions on the displacement and its gradient, the definition of the reconstructed gradient \eqref{eq_grad} needs be modified accordingly, and the angular componenent $\tensoro{G}{}_{\cell \theta \theta}$ does not defines by the same equation as those in the other directions. In particular, for any displacement pair $(\tensori{v}{}_{\cell}, \tensori{v}{}_{\dCell}) \in \displacementSpaceCell \times \displacementSpaceDCell$, the componenent $\tensoro{G}{}_{\cell \theta \theta}(\tensoro{v}{}_{\cell r}, \tensoro{v}{}_{\dCell r})$ solves
% Dans le contexte axisymmétrique, on suppose $u_{\theta} = v_{\theta} = 0$, de sorte que la composante $G_{\theta \theta}$ du gradient dans $\mathbb{P}^{k}(T, \mathbb{R})$ s'exprime, $\forall \tau_{\theta \theta} \in \mathbb{P}^{k}(T, \mathbb{R})$:
%
%
%
\begin{equation}
    \label{axi_symmetric_gradient_theta}
    \begin{aligned}
        \int_{\cell} 2 \pi r \tensoro{G}{}_{\cell \theta \theta}(\tensoro{v}{}_{\cell r}, \tensoro{v}{}_{\dCell r}) \tensoro{\tau}{}_{\cell \theta \theta}
        =
        \int_{\cell} 2 \pi r \frac{\tensoro{u}{}_{\cell r}}{r} \tensoro{\tau}{}_{\cell \theta \theta}
        =
        \int_{\cell} 2 \pi \tensoro{u}{}_{\cell r} \tensoro{\tau}{}_{\cell \theta \theta}
        &&
        \forall \tensorii{\tau}{}_{\cell} \in \stressSpaceCell
    \end{aligned}
\end{equation}
%
%
%
whereas in the radial and ordonal directions, \textit{i.e.} $\forall i, j \in \{ r,z \}$, the expression given in \eqref{eq_grad} is retrieved, and the component $G_{\cell ij}(\tensoro{v}{}_{\cell i}, \tensoro{v}{}_{\dCell i})$ solves :
%
%
%
\begin{equation}
    \label{axi_symmetric_gradient_rz}
    \begin{aligned}
    \int_{\cell} 2 \pi r G_{\cell ij}(\tensoro{v}{}_{\cell i}, \tensoro{v}{}_{\dCell i}) \tau_{\cell ij} =
    \int_{\cell} 2 \pi r \frac{\partial \tensoro{u}{}_{\cell i}}{\partial j} \tau_{ij} -
    \int_{\dCell} 2 \pi r (u_{\dCell i} - u_{\cell i} \vert_{\dCell}) \tau_{\cell ij} \vert_{\dCell} n_{j}
    &&
    \forall \tensorii{\tau}{}_{\cell} \in \stressSpaceCell
    \end{aligned}
\end{equation}
%
%
%
As for the reconstructed gradient, the higher order potential term needed to define the HHO jump function nedds also be reconsidered such that $\forall \tensori{w}{}_{\cell} \in \mathbb{P}^{k + 1}(T, \mathbb{R}^2)$, the radial component $w_{\cell r}$ solves
% En particulier, dans le contexte axisymmétrique, $\forall w_r \in \mathbb{P}^{k + 1}(T, \mathbb{R})$, la composante $D_{\cell r}$ dans $\mathbb{P}^{k + 1}(T, \mathbb{R})$ pour le déplacement reconstruit résoud :
%
%
%
\begin{equation}
    \label{axi_symmetric_potential_r}
    \begin{aligned}
        \int_{\cell} 2 \pi r (\sum_{i \in \{ r,z \}} \frac{\partial D_{\cell r}}{\partial i} \frac{\partial w_{\cell r}}{\partial i} + \frac{D_{\cell r}}{r} \frac{w_{\cell r}}{r})
        = &
        \int_{\cell} 2 \pi r (\sum_{i \in \{ r,z \}} \frac{\partial u_{\cell r}}{\partial i} \frac{\partial w_{\cell r}}{\partial i} + \frac{u_{\cell r}}{r} \frac{w_{\cell r}}{r})
        % &&
        % \forall \tensori{w}{}_{\cell} \in \mathbb{P}^{k + 1}(T, \mathbb{R}^2)
        \\
        &
        +
        \int_{\dCell} 2 \pi r \sum_{i \in \{ r,z \}} (u_{\dCell r} - u_{\cell r} \vert_{\dCell}) \frac{\partial w_{\cell r}}{\partial i} \vert_{\dCell} n_{i}
    \end{aligned}
\end{equation}
%
%
%
and the ordonal component $D_{\cell z}$ solves :
%
%
%
\begin{subequations}
    \label{axi_symmetric_potential_z}
        \begin{alignat}{3}
            \int_{\cell} 2 \pi r \sum_{i \in \{ r,z \}}
            \frac{\partial D_{\cell z}}{\partial i} \frac{\partial w_{\cell z}}{\partial i}
            = &
            \int_{\cell} 2 \pi r \sum_{i \in \{ r,z \}} \frac{\partial u_{\cell z}}{\partial i} \frac{\partial w_{\cell z}}{\partial i}
            -
            \int_{\dCell} 2 \pi r \sum_{i \in \{ r,z \}} (u_{\dCell z} - u_{\cell z} \vert_{\dCell})
            \frac{\partial w_{\cell z}}{\partial i} \vert_{\dCell} n_{i}
            % &&
            % \forall \tensori{w}{}_{\cell} \in \mathbb{P}^{k + 1}(T, \mathbb{R}^2)
            % \\
            % &
            % -
            % \int_{\dCell} 2 \pi r \sum_{i \in \{ r,z \}} (u_{\dCell z} - u_{\cell z} \vert_{\dCell})
            \\
            \int_{\cell} 2 \pi r D_{\cell z} = & \int_{\cell} 2 \pi r u_{\cell z}
        \end{alignat}
\end{subequations}
%
%
%
In particular, one notices that the mean value condition is not needed on the radial componenent of the higher order displacement since the left hand side of the system described by \eqref{axi_symmetric_potential_r} depends direclty on the displacement unknown and not only on itrs gradient as in \eqref{axi_symmetric_potential_z}.
%
% 
%

Moreover, since in cylindrical coordinates, all integrals depend on the radial componenent $r$, there is a singularity at $r = 0$ for boundary integrals on faces located on the symmetry axis, and from a geometrical standpoint, these faces lose a dimension; a face that is not located on the symmetry axis behaves like a shell by revolution of the $(r,z)$ plane,
whereas one attached to the axis reduces to a beam that is only allowed to move and morph in the $z$ direction.
On a more algebraic note, the problem as such is ill-defined, since building the jump function involves inverting a mass matrix in $\discreteDisplacementSpaceDCell$ to define the projector $\Pi_{\dCell}^k$.
Therefore, a face on the axis is swelled by a small radius $\varrho$ such that it becomes a cylinder with same dimensions as the others (see Figure \ref{fig_axi})
%
%
%
\begin{figure}[H]
    \centering
    \includegraphics[width=10.cm]{img/sketch_axi.png}
    \caption{schematic representation of the model problem}
    \label{fig_axi}
\end{figure}
The goal of this section is to evaluate the proposed HHO method on two and three-dimensional benchmarks from the
literature: (i) a necking of a 2D rectangular bar subjected to uniaxial extension, (ii) a quasi-incompressible sphere under internal pressure.
We compare
our results to the analytical solution whenever available or to numerical results obtained using the industrial open-source
FEM software code aster. In this case, we consider a linear, respectively, quadratic, cG formulation, referred to as
Q1, respectively, T2 or Q2, when full integration is used, or, Q2-RI when reduced integration is used, depending on the
mesh, and a three-field mixed formulation in which the unknowns are the displacement, the pressure, and the volumetric
strain fields referred to as UPG 6 ; in the UPG method, the displacement field is quadratic, whereas both the pressure and
the volumetric strain fields are linear. The conforming Q1, T2, and Q2 methods with full integration, contrary to the
Q2-RI method with reduced integration in most of the situations, are known to present volumetric locking due to plastic
incompressibility, whereas the UPG method is known to be robust but costly. Numerical results obtained using the UPG
method are used as a reference solution whenever an analytical solution is not available.
The nonlinear isotropic plasticity model with a von Mises yield criterion described in Section ABOVE is used for the test
cases. For the first three test cases, strain-hardening plasticity is considered with the following material parameters: Young
modulus E = 206.9 GPa, Poisson ratio nu = 0.29, hardening parameter H = 129.2 MPa, initial yield stress sig = 450 MPa,
infinite yield stress sigunf = 715 MPa, and saturation parameter $\delta$ = 16.93. For the fourth case, perfect plasticity is considered
with the following material parameters: Young modulus E = 28.85 MPa, Poisson ratio nu = 0.499, hardening
parameter H = 0 MPa, initial and infinite yield stresses 6 MPa, and saturation parameter 0. Moreover, for the two-dimensional test cases (i) and (ii), we assume additionally a plane strain condition. In the numerical
experiments reported in this section, the stabilization parameter is taken to be 1, and all the quadratures
use positive weights. In particular, for the HHO method, we employ a quadrature of order k Q = 2k for the behavior cell
integration. We employ the notation HHO(k, l) when using face polynomials of order k and cell polynomials of order l.
In Section 5, we perform further numerical investigations to test other aspects of HHO methods such as the support of
general meshes with possibly nonconforming interfaces, the possibility of considering the lowest-order case k = 0, and
the dependence on the stabilization parameter $\beta$.

\subsection{Necking of a 2D rectangular bar}

In this first benchmark, we consider a 2D rectangular bar with an initial imperfection. The bar is subjected to uniaxial
extension. This example has been studied previously by many authors as a necking problem 3,5,7,8,22 and can be used to
test the robustness of the different methods. The bar has a length of 53.334 mm and a variable width from an initial width
value of 12.826 mm at the top to a width of 12.595 mm at the center of the bar to create a geometric imperfection. A vertical
displacement u y = 5mm is imposed at both ends, as shown in Figure 2A. For symmetry reasons, only one-quarter of the
bar is discretized, and the mesh is composed of 400 quadrangles (see Figure 2B). The load-displacement curve is plotted
in Figure 2C. We observe that, except for Q1, all the other methods give very similar results. Moreover, the equivalent
plastic strain p, respectively, the trace of the Cauchy stress tensor sigma, are shown in Figure 3, respectively, in Figure 4, at
the quadrature points on the final configuration. A sign of locking is the presence of strong oscillations in the trace of
the Cauchy stress tensor sigma. We notice that the cG formulations Q1 and Q2 lock, contrary to the HHO, Q2-RI, and UPG
methods that deliver similar results. We remark, however, that the results for HHO(1;1), HHO(1;2), and Q2-RI are slightly
less smooth than for HHO(2;2), HHO(2;3), and UPG. The reason is that, on a fixed mesh, the three former methods have
less quadrature points than the three latter ones (see Table 1) (HHO(2;2), HHO(2;3), and UPG have the same number of
quadrature points). Therefore, the stress is evaluated using less points in HHO(1;1), HHO(1;2), and Q2-RI. It is sufficient
to refine the mesh or to increase the order of the quadrature by two in HHO(1;1) and HHO(1;2) to retrieve similar results
to those for the three other methods (not shown for brevity).
%
%
%
\begin{figure}[H]
    \centering
    \includegraphics[width=10.cm]{img_calcs/ssna_concat.png}
    \caption{schematic representation of the model problem}
    \label{fig_ssnaall}
\end{figure}

\subsection{Quasi-incompressible sphere under internal pressure}

This last benchmark 6 consists of a quasi-incompressible sphere under internal pressure for which an analytical solution
is known when the entire sphere has reached a plastic state. This benchmark is particularly challenging compared to
the previous ones since we consider here perfect plasticity. The sphere has an inner radius R in = 0.8 mm and an outer
radius R out = 1 mm. An internal radial pressure P is imposed. For symmetry reasons, only one-eighth of the sphere is
discretized, and the mesh is composed of 1580 tetrahedra (see Figure 10A). The simulation is performed until the limit
load corresponding to an internal pressure 2.54 MPa is reached.
The equivalent plastic strain p is plotted for
HHO(1;2) in Figure 10B, and the trace of the Cauchy stress tensor simga is compared for HHO, UPG, and T2 methods in
Figure 11 at all the quadrature points on the final configuration for the limit load. We notice that the quadratic element T2
locks, whereas HHO and UPG do not present any sign of locking and produce results that are very close to the analytical
solution. However, the trace of the Cauchy stress tensor simga is slightly more dispersed around the analytical solution for
HHO(2;2) and HHO(2;3) than for HHO(1;1) and HHO(1;2) near the outer boundary. For this test case, we do not expect
that HHO(2;2) and HHO(2;3) will deliver more accurate solutions than HHO(1;1) and HHO(1;2) since the geometry is
discretized using tetrahedra with planar faces.
We next investigate the influence of the quadrature order k Q on the accuracy of the solution. The trace of the Cauchy
stress tensor simga is compared for HHO(1;1), HHO(2;2), and UPG methods in Figure 12 at all the quadrature points on the
final configuration for the limit load, and for a quadrature order k Q higher than the one employed in Figure 11 (HHO(1;2)
and HHO(2;3) give similar results and are not shown for brevity). We remark that, when we increase the quadrature
order, UPG locks for quasi-incompressible finite deformations, whereas HHO does not lock, and the results are (only) a
bit more dispersed around the analytical solution. Moreover, HHO(2;2) is less sensitive than HHO(1;1) to the choice of
the quadrature order k Q . Note that this problem is not present for HHO methods with small deformations. Furthermore,
this sensitivity to the quadrature order seems to be absent for finite deformations when the elastic deformations are
compressible (the plastic deformations are still incompressible). To illustrate this claim, we perform the same simulations
as before but for a compressible material. The Poisson ratio is taken now as nu = 0.3 (recall that we used nu = 0.499
in the quasi-incompressible case), whereas the other material parameters are unchanged. Unfortunately, an analytical
solution is no longer available in the compressible case. We compare again the trace of the Cauchy stress tensor simga for
HHO(1;1), HHO(2;2), and UPG methods in Figure 13 at all the quadrature points on the final configuration and for
different quadrature orders k Q . We observe a quite marginal dependence on the quadrature order for HHO methods (as in
the quasi-incompressible case), whereas the UPG method still locks if the order of the quadrature is increased. Moreover,
in the compressible case, HHO(2;2) gives a more accurate solution than HHO(1;1).
\section{Numerical examples for the algorithm}
\label{sec_numerical_algorithm}

In this section, we evaluate the response of the cell equilibrium algorithm.

\paragraph{Specimen and loading}

We consider a the cook membrane specimen that is subjected to uniaxial
traction.
% This example has been studied previously by many authors as a necking problem 3,5,7,8,22 and can be used to
% test the robustness of the different methods.
The membrane has a width of $48$ mm and a height of $60$ mm.
A vertical traction $t_y = 1000$ N/m is imposed at the top, as shown in Figure \ref{fig_ssnaallmesh}.
The HHO computation is compared with a standard Q1 and Q2 (\textit{i.e.} linear and quadratic approximations)

\paragraph{Behaviour law}

The same behavior law as that in \ref{sec_swelling_sphere} is considered for the present test case. 
However, the finite strain hypothesis is chosen, based on a logarithmic decomposition of the stress \cite{miehe_anisotropic_2002}.

\paragraph{Material parameters}

Materials parameters are taken as
$\sigma_0 = 450$ MPa, $\sigma_{\infty} = 715$ MPa with a saturation parameter $\delta = 16.93$. The Young modulus is $E = 206.9$ GPa, and the Poisson ratio is $\nu = 0.29$.

% \begin{figure}[H]
%     \centering
%     \includegraphics[width=12.cm]{img_calcs/sphere_mesh.png}
%     \caption{the swelling sphere test case. Geometry, loadings, final displacement along the radius of the sphere, and final equivalent plastic strain map at quadrature points}
%     \label{fig_sphereall}
% \end{figure}

% In this section, we evaluate the proposed axi-symmetric HHO method on classical test cases taken from the literature to emphasize robustness to volumetric locking.
% We consider both the small and large strains framework, and for elasto-plastic behaviors.
% The first test case is that of a elasto-perfect plastic swelling sphere. The second one consists in the necking of a notched bar.
% In this section, we denote by HHO($k,l$) the HHO element of order $k$ on faces, and order $l$ in the cell.

\begin{figure}[H]
    \centering
    \includegraphics[width=12.cm]{img_calcs/cook_comp.png}
    \caption{Hydrostatic pressure map one the reference configuration at the limit load}
    \label{fig_sphereall}
\end{figure}

\paragraph{Algorithm performance}

We compare the performance of 

\paragraph{Prediction decondensation step}

Using a decondensation setp for the cell algorithm

\begin{figure}[H]
    \centering
    \includegraphics[width=12.cm]{img_calcs/algo_comp.png}
    \caption{Comparison in terms of performance for different algorithms}
    \label{fig_sphereall}
\end{figure}

\appendix

\section{Appendix}
\label{sec_appendix}

\subsection{From the continuous Hu-Washizu Lagrangian to the HDG one}

Let $\tensori{\Psi}(\tensori{X})$ the linear mapping consisting in a change of euclidean frame such that
%
%
%
\begin{equation}
    \tensori{\Psi} : \tensori{X} \mapsto \tensori{x} = \tensorii{Q}{} \tensori{X} + \tensori{c}
\end{equation}

%
% 
% 
\begin{equation}
    \label{eq22}
    \begin{aligned}
        L_{\Crown{}, \text{int}}^{HW}
        := &
        \int_{\Crown{}} \mecPotential{}_{\Crown} + (\nabla \tensori{u}{}_{\Crown} - \tensorii{G}{}_{\Crown}) : \tensorii{P}{}_{\Crown}
        % -
        % \int_{\Crown{}} \loadLag \cdot \tensori{u}{}_{\Crown}
        \\
        = &
        (1 - \frac{\alpha}{2} \ell)
        \int_{\dBulk{}} \frac{\beta}{2 h_{\cell}} \lVert \tensori{u}{}_{\dCell{}} - \tensori{u}{}_{\Bulk{}} \vert_{\dBulk{}} \rVert^2
        +
        (1 - \frac{\alpha}{2} \ell)
        \int_{\dBulk} (\tensori{u}{}_{\dCell{}} - \tensori{u}{}_{\Bulk{}} \vert_{\dBulk{}}) \cdot \tensorii{P}{}_{\Bulk{}} \vert_{\dBulk{}} \cdot \tensori{n}{}
        -
        \int_{\Crown{}} \tensorii{G}{}_{\Crown{}} : \tensorii{P}{}_{\Crown{}}
        % -
        % \int_{\Crown{}} \loadLag \cdot \tensori{u}{}_{\Crown}
    \end{aligned}
\end{equation}
%
% 
%
The development of \eqref{eq22} is given in Appendix. Injecting \eqref{eq22} in \eqref{eq_hu_washizu_split} yields
%
% 
% 
\begin{equation}
    \label{eq_0014}
    \begin{aligned}
        L_{\cell}^{HW}
        = &
        \int_{\Bulk} \mecPotential{}_{\bodyLag{}} + (\nabla \tensori{u}{}_{\Bulk} - \tensorii{G}{}_{\Bulk}) : \tensorii{P}{}_{\Bulk}
        % \\
        % &
        +
        (1 - \frac{\alpha}{2} \ell)
        % \Biggl(
        \int_{\dBulk{}} (\tensori{u}{}_{\dCell{}} - \tensori{u}{}_{\Bulk} \vert_{\dBulk}) \cdot \tensorii{P}{}_{\Bulk} \vert_{\dBulk} \cdot \tensori{n}{}
        % \\
        % &
        \\
        &
        +
        (1 - \frac{\alpha}{2} \ell)
        \int_{\dBulk{}} \frac{\beta}{2 h_T} \lVert \tensori{u}{}_{\dCell{}} - \tensori{u}{}_{\Bulk} \vert_{\dBulk{}} \rVert^2
        % \Biggr)
        % \\
        % &
        -
        \int_{\Crown{}} \tensorii{G}{}_{\Crown{}} : \tensorii{P}{}_{\Crown{}}
        % \\
        % &
        -
        \int_{\Bulk} \loadLag \cdot \tensori{u}{}_{\Bulk}
        -
        \int_{\Crown{}} \loadLag \cdot \tensori{u}{}_{\Crown{}}
        -
        \int_{\neumannCell{}} \neumannCellLoad{} \cdot \tensori{u}{}_{\dCell{}}
    \end{aligned}
\end{equation}
%
% 
% ------------------------------------------------------- DEVELOPMENT
\textcolor{blue}{
%
\begin{development}[Interafce simplification]
%
Let $C_\Crown = \{ v \in L^2(\Crown) \ \vert \ v \cdot \tensori{n} = \text{cste} \}$ the set of $L^2$-functions which are constant along the normal axis in $\Crown$. For any function in $C_\Crown$, the following equality holds true:
%
% 
% 
\begin{equation}
    \label{eq_virtual_works0}
        \int_{\Crown} v \ dV
        =
        \int_{\dBulk{}} \int_{\epsilon = 0}^{\ell} v (1 - \alpha \epsilon) \ dS d \epsilon
        =
        \ell (1 - \frac{\alpha}{2} \ell) \int_{\dBulk{}} v \ dS
\end{equation}
%
% 
% 
Noticing that $\nabla \tensori{u}{}_{\Crown} \in C_\Crown$, one has :
%
% 
% 
\begin{equation}
    \begin{aligned}
        \int_{\Crown{}} \mecPotential{}_{\Crown}
        % = &
        % \int_{\Crown{}} \frac{1}{2} \beta \frac{\ell}{h_{\cell}} \nabla \tensori{u}{}_{\Crown} : \nabla \tensori{u}{}_{\Crown}
        % \\
        = & 
        \ell (1 - \frac{\alpha}{2} \ell)
        \int_{\dBulk{}} \frac{1}{2} \beta \frac{\ell}{h_{\cell}} \nabla \tensori{u}{}_{\Crown} : \nabla \tensori{u}{}_{\Crown}
        \\
        = & 
        \ell (1 - \frac{\alpha}{2} \ell)
        \int_{\dBulk{}} \frac{\beta}{2 \ell h_{\cell}} (\tensori{u}{}_{\dCell} - \tensori{u}{}_{\Bulk} \vert_{\dBulk}) \otimes
        \tensori{n} : (\tensori{u}{}_{\dCell} - \tensori{u}{}_{\Bulk} \vert_{\dBulk}) \otimes
        \tensori{n}
        \\
        = & 
        \ell (1 - \frac{\alpha}{2} \ell)
        \int_{\dBulk{}} \frac{\beta}{2 \ell h_{\cell}} \sum_{i,j} (\tensoro{u}{}_{\dCell}{}_{i}- \tensoro{u}{}_{\Bulk}{}_{i} \vert_{\dBulk}){}^2
        \tensoro{n}_{j}{}^2
        \\
        = & 
        \ell (1 - \frac{\alpha}{2} \ell)
        \int_{\dBulk{}} \frac{\beta}{2 \ell h_{\cell}} \sum_{j} \tensoro{n}_{j}{}^2 \sum_{i} (\tensoro{u}{}_{\dCell}{}_{i}- \tensoro{u}{}_{\Bulk}{}_{i} \vert_{\dBulk}){}^2
        \\
        = & 
        \ell (1 - \frac{\alpha}{2} \ell)
        \int_{\dBulk{}} \frac{\beta}{2 \ell h_{\cell}} \sum_{i} (\tensoro{u}{}_{\dCell}{}_{i}- \tensoro{u}{}_{\Bulk}{}_{i} \vert_{\dBulk}){}^2
        \\
        = & 
        \ell (1 - \frac{\alpha}{2} \ell)
        \int_{\dBulk{}} \frac{\beta}{2 \ell h_{\cell}} \lVert \tensori{u}{}_{\dCell} - \tensori{u}{}_{\Bulk}{} \vert_{\dBulk} \lVert {}^2
        \\
        = & 
        (1 - \frac{\alpha}{2} \ell)
        \int_{\dBulk{}} \frac{\beta}{2 h_{\cell}} \lVert \tensori{u}{}_{\dCell} - \tensori{u}{}_{\Bulk}{} \vert_{\dBulk} \lVert {}^2
    \end{aligned}
\end{equation}
%
% 
% 
Moreover, for $\tensorii{P}{}_{\Crown}$ in $C_\Crown{}$ :
%
% 
% 
\begin{equation}
    \begin{aligned}
        \int_{\Crown{}} \nabla \tensori{u}{}_{\Crown} : \tensorii{P}{}_{\Crown}
        = &
        \ell (1 - \frac{\alpha}{2} \ell)
        \int_{\dBulk{}} \nabla \tensori{u}{}_{\Crown} : \tensorii{P}{}_{\Crown}
        \\
        = &
        \ell (1 - \frac{\alpha}{2} \ell)
        \int_{\dBulk{}}
        \frac{1}{\ell}
        (\tensori{u}{}_{\dCell} - \tensori{u}{}_{\Bulk}{} \vert_{\dBulk}) \otimes \tensori{n} : \tensorii{P}{}_{\Bulk{}} \vert_{\dBulk{}}
        \\
        = &
        \ell (1 - \frac{\alpha}{2} \ell)
        \int_{\dBulk{}}
        \frac{1}{\ell}
        \sum_{i,j}
        (\tensoro{u}{}_{\dCell}{}_{i} - \tensoro{u}{}_{\Bulk}{}{}_{i} \vert_{\dBulk}) \tensoro{n}{}_{j} \tensoro{P}{}_{\Bulk{}}{}_{ij} \vert_{\dBulk{}}
        \\
        = &
        \ell (1 - \frac{\alpha}{2} \ell)
        \int_{\dBulk{}}
        \frac{1}{\ell}
        (\tensori{u}{}_{\dCell} - \tensori{u}{}_{\Bulk}{} \vert_{\dBulk}) \cdot \tensorii{P}{}_{\Bulk{}} \vert_{\dBulk{}} \cdot \tensori{n}
        \\
        = &
        (1 - \frac{\alpha}{2} \ell)
        \int_{\dBulk{}}
        (\tensori{u}{}_{\dCell} - \tensori{u}{}_{\Bulk}{} \vert_{\dBulk}) \cdot \tensorii{P}{}_{\Bulk{}} \vert_{\dBulk{}} \cdot \tensori{n}
    \end{aligned}
\end{equation}
% 
% 
% 
And Finally :
%
% 
% 
\begin{equation}
    \begin{aligned}
        L_{\Crown{}, \text{int}}^{HW}
        =
        (1 - \frac{\alpha}{2} \ell)
        \int_{\dBulk{}} \frac{\beta}{2 h_{\cell}} \lVert \tensori{u}{}_{\dCell{}} - \tensori{u}{}_{\Bulk{}} \vert_{\dBulk{}} \rVert^2
        +
        (1 - \frac{\alpha}{2} \ell)
        \int_{\dBulk} (\tensori{u}{}_{\dCell{}} - \tensori{u}{}_{\Bulk{}} \vert_{\dBulk{}}) \cdot \tensorii{P}{}_{\Bulk{}} \vert_{\dBulk{}} \cdot \tensori{n}{}
        -
        \int_{\Crown{}} \tensorii{G}{}_{\Crown{}} : \tensorii{P}{}_{\Crown{}}
    \end{aligned}
\end{equation}
%
\end{development}
}
% ------------------------------------------------------- DEVELOPMENT
%
%
%
\textcolor{blue}{
    \begin{development}[Elliptic projection]
        %
        %
        %
        Let $\discreteDisplacementSpaceCell \subset \displacementSpaceCell$ and $U^\perp(\cell) \subset \displacementSpaceCell$ such that $\displacementSpaceCell = \discreteDisplacementSpaceCell \oplus U^\perp(\cell)$, and set $\tensori{u}{}_{\cell} = \tensori{u}{}_{\cell}^h + \tensori{u}{}_{\cell}^\perp$ with
        $\tensori{u}{}_{\cell}^h \in U^h(\cell)$ and $\tensori{u}{}_{\cell}^\perp \in U^\perp(\cell)$ the orthogonal projections of $\tensori{u}{}_{\cell}$ onto $U^h(\cell)$ and $U^\perp(\cell)$ respectively.
        Let $V^h(\dCell) \subset \displacementSpaceDCell$ and $\tensori{u}{}_{\dCell}^h \in V^h(\dCell)$ the orthogonal projection of $\tensori{u}{}_{\cell}$ onto $V^h(\dCell)$.
        The orthogonal projection of $\tensori{u}{}_{\cell}$ onto $\discreteHybridDisplacementSpaceCell = U^h(\cell) \times V^h(\dCell)$ is then the displacement pair $(\tensori{u}{}_{\cell}^h, \tensori{u}{}_{\dCell}^h)$.
        Let $S^h(\cell) = \{ \tensorii{\tau}{}_{\cell}^h \in \stressSpaceCell \ \ \vert \ \ \nabla \cdot  \tensorii{\tau}{}_{\cell}^h \in U^h(\cell) \ \ \vert \ \  \tensorii{\tau}{}_{\cell}^h \vert_{\dCell} \cdot \tensori{n}{} \in V^h(\dCell) \}$,
        and $\tensorii{G}{}_{\cell}^h \in S^h(\cell)$ the solution of \eqref{eq_grad} for $(\tensori{u}{}_{\cell}^h, \tensori{u}{}_{\dCell}^h)$ such that
        %
        %
        %
        \begin{equation}
            \begin{aligned}
                \int_{\cell} \tensorii{G}{}_{\cell}^h(\tensori{u}{}_{\cell}^h, \tensori{u}{}_{\dCell}^h) : \tensorii{\tau}{}_{\cell}^h
                =
                \int_{\cell} \nabla \tensori{u}{}_{\cell}^h : \tensorii{\tau}{}_{\cell}^h
                +
                \int_{\dCell} (\tensori{u}{}_{\dCell}^h - \tensori{u}{}_{\cell}^h \vert_{\dCell}) \cdot \tensorii{\tau}{}_{\cell}^h \vert_{\dCell} \cdot \tensori{n}{}
                &&
                \ \ \ \ \ \ \ \ 
                &&
                \forall \tensorii{\tau}{}_{\cell}^h \in S^h(\cell)
            \end{aligned}
        \end{equation}
        %
        %
        %
        using the fact that $\tensori{u}{}_{\dCell}^h$ is the projection of $\tensori{u}{}_{\cell}$ onto $V^h(\dCell)$ and that $\tensorii{\tau}{} \vert_{\dCell} \cdot \tensori{n}{} \in V^h(\dCell)$:
        %
        %
        %
        \begin{equation}
            \begin{aligned}
                \int_{\cell} \tensorii{G}{}_{\cell}^h(\tensori{u}{}_{\cell}^h, \tensori{u}{}_{\dCell}^h) : \tensorii{\tau}{}_{\cell}^h
                = &
                \int_{\cell} \nabla \tensori{u}{}_{\cell}^h : \tensorii{\tau}{}_{\cell}^h
                +
                \int_{\dCell} (\tensori{u}{}_{\cell} \vert_{\dCell} - \tensori{u}{}_{\cell}^h \vert_{\dCell}) \cdot \tensorii{\tau}{}_{\cell}^h \vert_{\dCell} \cdot \tensori{n}{}
                &&
                \ \ \ \ \ \ \ \ 
                &&
                \forall \tensorii{\tau}{}_{\cell}^h \in S^h(\cell)
                \\
                = &
                \int_{\cell} \nabla \tensori{u}{}_{\cell}^h : \tensorii{\tau}{}_{\cell}^h
                +
                \int_{\dCell} \tensori{u}{}_{\cell}^\perp \vert_{\dCell} \cdot \tensorii{\tau}{}_{\cell}^h \vert_{\dCell} \cdot \tensori{n}{}
                &&
                \ \ \ \ \ \ \ \ 
                &&
                \forall \tensorii{\tau}{}_{\cell}^h \in S^h(\cell)
            \end{aligned}
        \end{equation}
        %
        %
        %
        using the divergence theorem and the fact that $\nabla \cdot  \tensorii{\tau}{}_{\cell}^h \in U^h(\cell)$, one has :
        %
        %
        %
        \begin{equation}
            \begin{aligned}
                \int_{\cell} \nabla \tensori{u}{}_{\cell}^\perp :  \tensorii{\tau}{}_{\cell}^h
                = &
                % -
                % \int_{\cell} \tensori{u}{}_{\cell}^\perp \cdot \nabla \cdot \tensorii{\tau}{}
                % +
                % \int_{\dCell} \tensori{u}{}_{\cell}^\perp \vert_{\dCell} \cdot \tensorii{\tau}{} \vert_{\dCell} \cdot \tensori{n}{}
                % \\
                % = &
                \int_{\dCell} \tensori{u}{}_{\cell}^\perp \vert_{\dCell} \cdot  \tensorii{\tau}{}_{\cell}^h \vert_{\dCell} \cdot \tensori{n}{}
            \end{aligned}
        \end{equation}
        %
        %
        %
        such that :
        %
        %
        %
        \begin{equation}
            \begin{aligned}
                \int_{\cell} \tensorii{G}{}_{\cell}^h(\tensori{u}{}_{\cell}^h, \tensori{u}{}_{\dCell}^h) : \tensorii{\tau}{}_{\cell}^h
                = &
                \int_{\cell} \nabla \tensori{u}{}_{\cell}^h : \tensorii{\tau}{}_{\cell}^h
                +
                \int_{\cell} \nabla \tensori{u}{}_{\cell}^\perp : \tensorii{\tau}{}_{\cell}^h
                &&
                \ \ \ \ \ \ \ \ 
                &&
                \forall \tensorii{\tau}{}_{\cell}^h \in S^h(\cell)
                \\
                = &
                \int_{\cell} \nabla \tensori{u}{}_{\cell} : \tensorii{\tau}{}_{\cell}^h
                &&
                \ \ \ \ \ \ \ \ 
                &&
                \forall \tensorii{\tau}{}_{\cell}^h \in S^h(\cell)
            \end{aligned}
        \end{equation}
        %
        %
        %
        which states that $\tensorii{G}{}_{\cell}^h(\tensori{u}{}_{\cell}^h, \tensori{u}{}_{\dCell}^h)$ is the orthogonal projection of $\nabla \tensori{u}{}_{\cell}$ onto $S^h(\cell)$.
        % IL FAUDRAIT MONTRER QUE $S^h(\cell)$ EST SUFFISANT POUR EMPECHER LE LOCKING DANS $U^h(\cell) \times V^h(\dCell)$ ??
        % ET FAIRE LE LIEN ENTRE $S^h(\cell)$ et $G^h(\cell)$.
        % Trouver aussi une justification pour $\stressSpaceCell \subset \gradSpaceCell$, la contrainte est plus régulière que le gradient, ce qui semble vrai (avec les décompositions sphérique dévaitoriques par exemples, etc)
        %
        %
        %
    \end{development}
}

\paragraph{Reconstructed gradient}

% $\tensoro{G}{}_{\cell \theta \theta}$ does not define by the same equation as those in the other directions. In particular,
For any displacement pair $(\tensori{v}{}_{\cell}^l, \tensori{v}{}_{\dCell}^k) \in \discreteDisplacementSpaceCell{} \times \discreteDisplacementSpaceDCell{}$, the component $\tensoro{G}{}_{\cell \theta \theta}(\tensoro{v}{}_{\cell r}, \tensoro{v}{}_{\dCell r})$ solves
%
%
%
\begin{equation}
    \label{axi_symmetric_gradient_theta}
    \begin{aligned}
        \int_{\cell} 2 \pi r \tensoro{G}{}_{\cell \theta \theta}(\tensoro{v}{}_{\cell r}, \tensoro{v}{}_{\dCell r}) \tensoro{\tau}{}_{\cell \theta \theta}
        =
        \int_{\cell} 2 \pi r \frac{\tensoro{u}{}_{\cell r}}{r} \tensoro{\tau}{}_{\cell \theta \theta}
        =
        \int_{\cell} 2 \pi \tensoro{u}{}_{\cell r} \tensoro{\tau}{}_{\cell \theta \theta}
        &&
        \forall \tensorii{\tau}{}_{\cell} \in \stressSpaceCell
    \end{aligned}
\end{equation}
%
%
%
In the radial and ordonal directions, \textit{i.e.} $\forall i, j \in \{ r,z \}$, the expression given in \eqref{eq_grad} is retrieved, and the component $G_{\cell ij}(\tensoro{v}{}_{\cell i}, \tensoro{v}{}_{\dCell i})$ solves
%
%
%
\begin{equation}
    \label{axi_symmetric_gradient_rz}
    \begin{aligned}
    \int_{\cell} 2 \pi r G_{\cell ij}(\tensoro{v}{}_{\cell i}, \tensoro{v}{}_{\dCell i}) \tau_{\cell ij} =
    \int_{\cell} 2 \pi r \frac{\partial \tensoro{u}{}_{\cell i}}{\partial j} \tau_{ij} -
    \int_{\dCell} 2 \pi r (u_{\dCell i} - u_{\cell i} \vert_{\dCell}) \tau_{\cell ij} \vert_{\dCell} n_{j}
    &&
    \forall \tensorii{\tau}{}_{\cell} \in \stressSpaceCell
    \end{aligned}
\end{equation}
%
%
%

\paragraph{Reconstructed higher order displacement}

For any $\tensori{d}{}_{\cell}^{k + 1} \in \discretePotentialSpaceCell$, the radial component $w^{k+1}_{\cell r}$ solves
%
%
%
\begin{equation}
    \label{axi_symmetric_potential_r}
    \begin{aligned}
        \int_{\cell} 2 \pi r (\sum_{i \in \{ r,z \}} \frac{\partial w^{k+1}_{\cell r}}{\partial i} \frac{\partial d^{k+1}_{\cell r}}{\partial i} + \frac{w^{k+1}_{\cell r}}{r} \frac{d^{k+1}_{\cell r}}{r})
        = &
        \int_{\cell} 2 \pi r (\sum_{i \in \{ r,z \}} \frac{\partial u_{\cell r}}{\partial i} \frac{\partial d^{k+1}_{\cell r}}{\partial i} + \frac{u_{\cell r}}{r} \frac{d^{k+1}_{\cell r}}{r})
        % &&
        % \forall \tensori{w}{}_{\cell} \in \mathbb{P}^{k + 1}(T, \mathbb{R}^2)
        \\
        &
        +
        \int_{\dCell} 2 \pi r \sum_{i \in \{ r,z \}} (u_{\dCell r} - u_{\cell r} \vert_{\dCell}) \frac{\partial d^{k+1}_{\cell r}}{\partial i} \vert_{\dCell} n_{i}
    \end{aligned}
\end{equation}
%
%
%
where the mean value condition is not needed on the radial component of the higher order displacement since the left hand side of the system described by \eqref{axi_symmetric_potential_r} depends directly on the displacement unknown and not only on its gradient as in \eqref{axi_symmetric_potential_z}.
The ordinate component $w^{k+1}_{\cell z}$ solves :
%
%
%
\begin{subequations}
    \label{axi_symmetric_potential_z}
        \begin{alignat}{3}
            \int_{\cell} 2 \pi r \sum_{i \in \{ r,z \}}
            \frac{\partial w^{k+1}_{\cell z}}{\partial i} \frac{\partial d^{k+1}_{\cell z}}{\partial i}
            = &
            \int_{\cell} 2 \pi r \sum_{i \in \{ r,z \}} \frac{\partial u_{\cell z}}{\partial i} \frac{\partial d^{k+1}_{\cell z}}{\partial i}
            -
            \int_{\dCell} 2 \pi r \sum_{i \in \{ r,z \}} (u_{\dCell z} - u_{\cell z} \vert_{\dCell})
            \frac{\partial d^{k+1}_{\cell z}}{\partial i} \vert_{\dCell} n_{i}
            % &&
            % \forall \tensori{w}{}_{\cell} \in \mathbb{P}^{k + 1}(T, \mathbb{R}^2)
            % \\
            % &
            % -
            % \int_{\dCell} 2 \pi r \sum_{i \in \{ r,z \}} (u_{\dCell z} - u_{\cell z} \vert_{\dCell})
            \\
            \int_{\cell} 2 \pi r w^{k+1}_{\cell z} = & \int_{\cell} 2 \pi r u_{\cell z}
        \end{alignat}
\end{subequations}

% \subsubsection{Plastic behavior in small strains}

% Dans le cadre de la thermodynamique des milieux continus, la combinsaison de l'application des deux premiers principes de la themodynamique donne lieu à l'équation de Clausius-Duhem qui postule la positivité de l'énergie de dissipation
% %
% %
% %
% \begin{equation}
%     \label{eq_clausius_duhem_0}
%     \begin{aligned}
%         \mathcal{D}
%         =
%         (\tensorii{\sigma}{}_{\cell} - \frac{\partial \mecPotential{}_{\bodyLag{}}}{\partial \dot{\tensorii{\varepsilon}}{}_{\cell}}) : \dot{\tensorii{\varepsilon}}{}_{\cell}
%         -
%         \rho \frac{\partial \mecPotential{}_{\bodyLag{}}}{\partial v_{int}} \dot{v}_{int}
%         \geq
%         0
%         % =
%         % (\tensorii{\sigma}{}_{\cell} - \rho \frac{\partial \mecPotential{}_{\bodyLag{}}}{\partial \tensorii{\varepsilon}{}_{\cell}}) : \dot{\tensorii{\varepsilon}}{}_{\cell}
%         % % -
%         % % \rho (s + \frac{\partial \mecPotential{}_{\bodyLag{}}}{\partial T}) \dot{T}
%         % -
%         % \rho \frac{\partial \mecPotential{}_{\bodyLag{}}}{\partial v_{int}} \dot{v}_{int}
%         % \geq
%         % 0
%     \end{aligned}
% \end{equation}
% %
% %
% %
% en l'absence de dépendence du problème à la température. Dans le cadre de l'hyper-élasticité qui est un processus de transformation réversible, comme évoque Section \ref{sec_model_problem}, l'ensemble $V_{int}$ des variables internes $v_{int}$ est supposé vide, de sorte que l'inégalité \eqref{eq_clausius_duhem_0} revient à l'équation d'égalité \eqref{eq_stress_def}.
% En revanche, pour des comportements dissipatifs de nature élasto-visco-plastique, on introduit un certains nombre de variables internes, qui sont liées à l'expression de l'énergie dissipée et à l'irréversibilité de la transformation.
% Pour des déformations infinitésimales, on suppose la décomposition additive de la déformation
% %
% %
% %
% \begin{equation}
%     \tensorii{\varepsilon}{}_{\cell} = \tensorii{\varepsilon}{}_{\cell}^e + \tensorii{\varepsilon}{}_{\cell}^p
% \end{equation}
% %
% %
% %
% En une partie élastique $\tensorii{\varepsilon}{}_{\cell}^e$ et une partie plastique $\tensorii{\varepsilon}{}_{\cell}^p$.
% En particulier, dans le cadre des matériaux standards généralisés, on suppose l'existence d'un potentiel également décomposable en une partie élastique et en une partie plastique tel que
% %
% %
% %
% \begin{equation}
%     \label{eq_plast_2}
%     \begin{aligned}
%         \mecPotential{}_{\bodyLag{}} = \mecPotential{}_{\bodyLag{}}^e(\tensorii{\varepsilon}{}_{\cell}^e)
%         +
%         \mecPotential{}_{\bodyLag{}}^p(
%             % \tensorii{\varepsilon}{}_{\cell}^p
%             % ,
%             v_{int})
%     \end{aligned}
% \end{equation}
% %
% %
% %
% Comme évoque Section \ref{sec_model_problem}, le potentiel d'énergie libre de Helmoltz $\mecPotential{}_{\bodyLag{}}$ dépend éventuellement d'un ensemble de variables internes $v_{int}$ dans $V_{int}$, qui a été supposé vide jusque là.
% Dans le cadre d'un comportement élasto-visco-plastique, on introduit au moins une variable interne, de manière à assurer la positivité de l'énergie dissipée. Par injection de \eqref{eq_plast_2} dans \eqref{eq_clausius_duhem_0}, il vient que le tenseur des contraintes $\tensorii{\sigma}{}_{\cell}$ est la la force duale assosciées aux défromations élastiques $\tensorii{\varepsilon}{}_{\cell}^e$. On définit également les forces thermodynamiques $V_{\cell}$ duales des variables internes $v_{int}$ telles que
% %
% %
% %
% \begin{equation}
%     \label{eq_plast_1}
%     \begin{aligned}
%         \mathcal{D}
%         =
%         \tensorii{\sigma}{}_{\cell} : \dot{\tensorii{\varepsilon}}{}_{\cell}^p
%         -
%         \rho \frac{\partial \mecPotential{}_{\bodyLag{}}}{\partial v_{int}} \dot{v}_{int}
%         =
%         \begin{Bmatrix}
%             \tensorii{\sigma}{}_{\cell}
%             \\
%             % \tensori{V}{}_{\cell}
%             V_{\cell}
%         \end{Bmatrix}
%         \cdot
%         \begin{Bmatrix}
%             \dot{\tensorii{\varepsilon}}{}_{\cell}^p
%             \\
%             \dot{v}_{int}
%         \end{Bmatrix}
%         \geq
%         0
%         &&
%         \text{with}
%         &&
%         V_{\cell} = - \rho \frac{\partial \mecPotential{}_{\bodyLag{}}}{\partial v_{int}}
%     \end{aligned}
% \end{equation}
% %
% %
% %
% Par ailleurs, le cadre des matériaux standards généralisés stipule l'existence d'un convex potential $\phi$ containing the origin, together with a threshold function $f$, that define the evolution of the generalized strains such that
% %
% %
% %
% \begin{equation}
%     \label{eq_plast_1}
%     \begin{aligned}
%         \dot{v_{int}} = \frac{\partial \phi}{\partial f} \frac{\partial f}{\partial V_T}
%     \end{aligned}
% \end{equation}

% En particulier, on introduit l'ensemble des variables internes $V_{int} = \{ \tensorii{\varepsilon}{}_{\cell}^p, p \}$ avec $p$ la déformation plastique cumulée, et le potentiel plastique 
% %
% %
% %
% \begin{equation}
%     \mecPotential{}_{\bodyLag{}}^p(
%         % \tensorii{\varepsilon}{}_{\cell}^p
%         % ,
%         p
%     )
%     % =
%     % \frac{K}{2} \tensorii{\varepsilon}{}_{\cell}^p : \tensorii{\varepsilon}{}_{\cell}^p + \frac{K}{2} p^2
%     =
%     \sigma_0 p + \frac{1}{2} H p^2 + (\sigma_{\infty} - \sigma_0)(p - \frac{1 - e^{-\delta p}}{\delta})
% \end{equation}
% %
% %
% %
% \begin{equation}
%     q
%     =
%     \sigma_0 + H p + (\sigma_{\infty} - \sigma_0)(1 - e^{-\delta p})
% \end{equation}
% %
% %
% %
% où $K$ est le module d'écrouissage cinématique, et $H$ le module d'écrouissage isotrope. Les forces thermodynamiques assosciées aux variables internes $\tensorii{\varepsilon}{}_{\cell}^p$ et $p$ sont respectivement $K \tensorii{\varepsilon}{}_{\cell}^p$ et $Hp$.
% %
% %
% %
% \begin{equation}
%     % f(\tensorii{\sigma}{}_{\cell}^p, q) = \sqrt{\frac{3}{2}} \rVert \text{dev} (\tensorii{\sigma} - K \tensorii{\varepsilon}{}^p) \lVert - \sigma_0 - H p
%     f(\tensorii{\sigma}{}_{\cell}^p, q) = \sqrt{\frac{3}{2}} \rVert \text{dev} (\tensorii{\sigma} \lVert - p
% \end{equation}


\bibliographystyle{elsarticle-num}
\bibliography{bib}

%% Authors are advised to use a BibTeX database file for their reference list.
%% The provided style file elsarticle-num.bst formats references in the required Procedia style

%% For references without a BibTeX database:

% \begin{thebibliography}{00}

%% \bibitem must have the following form:
%%   \bibitem{key}...
%%

% \bibitem{}

% \end{thebibliography}

% \end{multicols}

\end{document}

%%
%% End of file `ecrc-template.tex'. 